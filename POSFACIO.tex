\newcommand{\subtitulo}[1]{\NoCaseChange{\textnormal{\break\Large\itshape#1}}}
\chapter*{Posfácio\smallskip\subtitulo{A instabilidade do mundo\\de Max Blecher}}
\markboth{Posfácio}{}
\addcontentsline{toc}{chapter}{Posfácio, \emph{por Fábio Zuker}}

\begin{flushright}
\textsc{fábio zuker}
\end{flushright}


“Que esplêndido, que sublime é ser louco!”, diz o narrador, não em voz alta ou para alguém, mas para si mesmo, em um fluxo de pensamentos que o acompanha logo após sair de uma sessão matinal do cinema. Ele estranha as mudanças e continuidades do mundo fora das telas. O simples anoitecer, que acontecera enquanto ele estava absorto na escuridão do filme, lhe causa uma melancolia. Parece-lhe difícil acompanhar que, embora a abrupta transição do dia que se fez noite enquanto estava no cinema em nada afetou a continuidade da vida, que prossegue em seu desenrolar usual.

Não há uma palavra sobre o filme que acabara de assistir, mas uma divagação acerca do automatismo do mundo e das pessoas. Elas, tal qual a noite que invariavelmente chega a uma determinada hora para encerrar o dia, performavam mecanicamente “uma espécie de triste obrigação de sempre continuar”. 

É então que, caminhando pelas ruas ele se depara com aquela que denomina “a louca da cidade”. Sujismunda, cabelos desgrenhados, sem dentes, carregando troços que lhe haviam sido doados, sua figura contrastava com a racionalização irrefletida da repetição de quem, ao seu redor, seguia desempenhando seus papéis, tal como previsto, ou mesmo exigido, pelos ditames sociais. O narrador, cuja construção desde o início da obra parece ser um alter-ego do próprio Max Blecher, inveja essa concretude da presença da “louca”. Ela exibe seu sexo sem pudor àqueles que passavam pela rua. 

Há aqui uma inversão: não é a “louca” que está fazendo uma cena. Ela está sendo ela mesma, e goza da liberdade de não seguir um roteiro pré-determinado. São os demais transeuntes que a observam, que permanecem presos aos seus papéis.  O narrador estabelece com a louca uma profunda identificação, ainda que não haja menção alguma à uma que ela sequer o tenha percebido.

Ainda assim, para ele, somente a louca o entende. Sua loucura possui uma verdade, que os outros são incapazes de acessar: “as pessoas ao meu redor pareciam pobres criaturas dignas de pena pela seriedade com que continuamente se ocupavam, acreditando, ingênuas, naquilo que faziam e sentiam.”. Mas não o narrador. Este outro eu do jovem Blecher se afasta desta existência simulada, triste, deixando-se levar pelas mais variadas situações inauditas, excitantes e angustiantes que as ruas de uma cidade lhe oferecem. 

“Quem nunca foi tomado por esse sentimento está condenado a jamais sentir a verdadeira amplitude do mundo”. E é esta verdadeira amplitude do mundo que Blecher oferece às leitoras e leitores de Acontecimentos na Irrealidade Imediata. Um mundo cujos detalhes possuem maior densidade, capazes de perturbar o seu próprio ordenamento usual. E que, desafiando as próprias leis da física, são ao mesmo tempo mais voláteis, suficientemente mais instáveis para atravessar o próprio corpo daquele que narra, como se o contorno que separam os objetos dos seres do mundo, fosse colocado em suspensão, para abrir espaço para outra narrativa. Talvez mais próxima à da “louca”, do que daqueles que caminhavam ao seu redor, e para os quais o mundo oferecia menos inquietações capazes de alterar o seu passo a passo.


Fragmentos na cidade

Acontecimentos na Irrealidade Imediata é uma obra composta por um amálgama calendoscópico de fragmentos da infância e juventude do narrador, que transcorrem, majoritariamente, ao longo de um verão. Este é um relato incompleto, construído por cenas nas quais se sucedem a distensão temporal das deambulações pelas ruas; os encontros eróticos e sexuais; a irrupção de desejos variados, impossíveis de serem contidos; as crises de saúde, que não raro se convertem em desmaios; as reflexões sobre o modo particular como objetos se imiscuem em seu próprio corpo, tornando-o algo alheio a si mesmo. A melhor palavra que encontro para definir as sensações que o narrador experimenta é um assombro diante da ordem do mundo, que se faz particularmente presente em momentos nos quais o narrador se depara com fotografias (inclusive com a sua), imagens, encenações ou outras formas de representação.

Lançado pela em 1936, Acontecimentos na Irrealidade Imediata é uma obra difícil de ser classificada - e por isso, por não entrar facilmente em nenhuma caixinha, fascinante. Ela é um híbrido de autoficção e relato memorialista. São experiências desconcertantes, cheias de vida e melancolia. Aqui, não se trata de uma obra acerca das doçuras e alegrias da juventude, que escapam a uma pessoa moribunda capaz apenas de aceder a elementos parciais de um verão marcante. Em Blecher, existe a busca por uma escrita de um corpo permeável, cujas categorias básicas de percepção da realidade, tempo e espaço, se confundem.

A escrita é fragmentada, emulando os próprios mecanismos desconexos da memória, de episódios da infância e juventude, em que excertos narrativos por vezes se concatenam, por vezes levam a outros e subitamente são interrompidos, passando para o próximo, sem que se saiba ao certo a cronologia. Essa sucessão de cenas parece a única estratégia narrativa possível para dar forma à fluidez dos acontecimentos que cabem aqui ser contados. Estamos, assim, nas antípodas de uma narrativa grandiloquente, que anseie algum tipo de perspectiva totalizante. 

Mas há um terceiro elemento, que de alguma forma subverte os traços de autoficção e do relato memorialista, que torna esta obra singular: trata-se de um refinado experimento literário surrealista. Próximo ao movimento de André Breton, com quem frequentemente trocava cartas, Blecher chegou a publicar na revista Le surréalisme au Service de la révolution (O surrealismo à serviço da revolução, em tradução livre). Ecoam em Acontecimentos na Irrealidade Imediata romances surrealistas, como O Camponês de Paris (1926), de Louis Aragon e Nadja (1928), do próprio Breton. Ambos são escritas do caminhar, cujas narrativas se constroem no ato de deambular pela cidade e seus imponderáveis.

Cabe aqui evitar qualquer tipo de anacronismo que possamos ter, com a banalidade com a qual hoje consideramos uma caminhada pela cidade. A intensidade da experiência urbana era algo novo, e objeto de reflexão por parte de intelectuais: Georg Simmel identifica no surgimento do blasé uma forma de autoproteção aos estímulos incessantes da metrópole; Walter Benjamin, em seu fascinante e multifacetado livro Passagens, convida o leitor para flanar pela Paris moderna, seus bulevares, galerias, moda, iluminação e demais formas culturais que ali emergem, como radicalmente distintas de tudo aquilo que até então caracterizaria a vida.

Caminhar, andar pela cidade, perder-se, como uma experiência vertiginosa, uma experiência moderna para a qual se exige preparo, é como defrontar-se com um enigma que exige ser decifrado, sob o risco de se terminar devorado. A experiência moderna, indissociada da carnificina da Primeira Guerra Mundial e da iminente degeneração nazifascista, demandou dos surrealistas outras categorias de pensamento, outras formas de experimentação narrativa - avessas a um real demasiado fixo. E não havia cenário melhor do que a cidade, para desdobrar estas busca por outras formas de expressão, ávidas por enterrar no passado formas artísticas devotadas a reis poderosos, religiosos influentes ou burgueses endinheirados.


Um corpo poroso

É folheando um livro de anatomia que o narrador encontra a fotografia de um modelo de cera que reproduz o interior de uma orelha. “Todos os canais, seios e buracos eram de matéria plena, formando sua imagem positiva”. Sua impressão diante da fotografia é tão desmedida, que ele quase desmaia.

Estamos diante de uma situação modelo de Acontecimentos na Irrealidade Imediata, e que se repete ao longo do livro, sempre implicando um novo estranhamento. A partir de um pequeno acontecimento, por mais comezinho e ordinário que possa nos parecer, Blecher nos conduz a uma série de reflexões marcadas por esta perplexidade quanto ao ordenamento do real. A partir da imagem do modelo de cera da orelha, o narrador percebe o potencial de existência de uma outra realidade, ao avesso: “tudo o que é furado se tornasse cheio, e os relevos atuais se transformassem em vácuos de forma idêntica, sem qualquer conteúdo”. Uma realidade que tivesse como modelo os fósseis, cujos contornos apenas existem na medida em que deixaram marcas esculpidas em pedras.

“Num mundo como esse, as pessoas cessariam de ser excrescências multicoloridas e carnosas, cheias de órgãos complexos e putrescíveis, tornando-se vácuos puros, flutuantes como bolhas de ar dentro d’água, atravessando a matéria quente e mole do universo pleno.” Esta sensação de ser vazio, sem órgãos, como invólucros transparentes lhe vem um insight, como quando após refletir sobre algum problema que há muito nos angustia, finalmente conseguimos observá-lo por outro ângulo - aquele mínimo afastar-se de uma neurose que nos absorvia, e dificilmente compartilhável com alguém que não acompanhou este processo.

Mas é isso que parece fazer o narrador. Seu fascínio com a fotografia da orelha de cera lhe permite melhor compreender as aflições e anseios de sua adolescência, quando o sobrepeso do mundo tornava a própria estabilidade das coisas mais opressora, “como se as pessoas e as casas em derredor de repente houvessem se amalgamado numa massa compacta e uniforme de uma única matéria, na qual eu existia como um simples vácuo sem finalidade, locomovendo-me para lá e para cá”.

Blecher narra a experiência de um corpo poroso, atravessado por feixes de elementos, sensações, objetos e desejos - abruptos, como que dotados de uma vida própria, que incontroláveis, excedem à própria razão do narrador que parece organizá-los, e a eles cede, com prazer e angústia. Talvez deixando de ser um daqueles transeuntes que caminhavam ao redor da “louca” que na rua se divertia em mostrar sua genitália, para aproximar-se dela

O tema da exclusão, de um certo apartar-se da realidade, uma estranheza, ou melhor, um mal-estar no mundo, perpassa o livro. Este é um tópico presente na obra de alguns escritores judeus que viveram na Europa na mesma época que Blecher, como Franz Kafka e Bruno Schulz, dos quais frequentemente o autor é aproximado. Também em Léxico Familiar (1963), de Natalia Ginzburg, uma geração mais nova que Blecher, há essa sensação de pertencimento de uma família judia assimilada à sociedade italiana e, simultaneamente, de distanciamento, de não pertencer. Já em Marcel Proust, também ele de origem judaica e uma geração mais velha que Blecher, com quem frequentemente o escritor é comparado, o afastamento ocorre no plano das memórias que irrompem, tornando o presente algo que se escapa entre os dedos (um maintenant que nunca ocorre), e pelas suas meditações acerca da incongruência entre imagens e as pessoas que estas pretendem representar.

Mas em Blecher, este afastamento não parece ser nem da ordem do comedimento ou da clausura, como em Kafka e Schulz, nem das tensões explicitamente políticas e do antissemitismo que se acirra com as Leis Raciais de 1938 na Itália fascista de Ginzburg. Em Blecher, a experiência de estar no mundo com um corpo doente é sui generis, e implica momentos de crises que o obrigam a deitar-se, a desmaios, e o levam ações transgressoras, como chafurdar-se em um lamaçal, e nele adormecer até o anoitecer.

Há aqui algo de Antonin Artaud que em Para dar um fim no Juízo de Deus (1947). Em sua peça-transmissão radiofônica, Artaud coloca em cheque a pretensão e superioridade de julgamento das clínicas manicomiais pelas quais passou durante anos. Para questioná-las, faz uma crítica da própria capacidade divina de julgar, e reivindica para si um "corpo sem órgãos", expressão posteriormente consagrada pela teorização que lhe dá Gilles Deleuze e Félix Guattari, como um corpo que, livre da composição de seus órgãos, abre-se para um experimento de si. "Se quiserem, podem meter-me numa camisa de força, mas não existe coisa mais inútil que um órgão. Quando tiverem conseguido um corpo sem órgãos, então o terão liberado dos seus automatismos e devolvido sua verdadeira liberdade," escreve Artaud.

A aproximação entre Blecher e o Artaud quanto a este corpo vazado, cujas margens com o mundo são borradas, salienta este aspecto fundamental da obra de Blecher: a escrita emerge dessa experiência da doença e de sucessivas internações para tratamento. No caso de Blecher, e diferente de Artaud, são tratamentos não-manicomiais, mas de uma tuberculose óssea chamada Mal de Pott, para a qual não havia cura na época.

Blecher nasceu em 1909 em Botoșani (Romênia), filho de bem-sucedidos comerciantes judeus do ramo de porcelana. Fez seus estudos em Roman, na Moldava romena, onde provavelmente é narrada Acontecimentos na Irrealidade Imediata. É a mesma cidade onde Blecher viria a falecer em 1938, aos 28 anos de idade, após cerca de dez anos doente perambulando entre hospitais europeus. A Roman de sua infância e adolescência está separada da Roman de sua morte por dez anos de intensa atividade artística e viagens.

Durante o liceu, começa a tomar contato com a literatura francesa. E de acordo com especialistas em sua obra, já aos 16 anos começa a apresentar os primeiros sinais de sua enfermidade. Em 1928, terminado o liceu (equivalente ao colegial no Brasil), e diante do crescente antisemitismo na Romênia, Blecher se une a uma legião de jovens de todo o mundo seduzidos pelo ambiente cultural moderno da França, e decide matricular-se no curso de medicina da Universidade de Rouen. Mas é obrigado a abandoná-lo pouco tempo depois, quando lhe é diagnosticada a pneumonia óssea. A partir de então, a sua vida é uma sequência de internações em sanatórios da França, da Suíça e de sua Romênia natal.
	
Essa década de doença e internações é também a mais rica de seus escritos e correspondências, das quais a crítica salienta com frequência as cartas trocadas com o líder do movimento surrealista, André Breton. Foi também um momento de recrudescimento do antissemitismo romeno com o movimento nacionalista e fascista Guarda de Ferro. É neste ambiente político instável e com o avançar das dores causadas pela doença, que Blecher escreve, entre 1934 e 1938, a parte mais significativa de sua obra: Corpo Transparente, uma coleção de poemas e Corações Cicatrizados, romance, (ambos traduzidos ao português por Fernando Klabin, que também assina esta tradução), e  A Toca Iluminada (publicação póstuma) além de, evidentemente, Acontecimentos na Irrealidade Imediata.

Atravessado por forças contrapostas, como melancolias, desespero e situações de desmaios e perdas de consciência, descobertas sexuais e impulsos violentos que acabam por nunca se concretizar, Acontecimentos na Irrealidade Imediata é uma obra sem meias palavras. O que faz de sua escrita tão potente é o modo como se combinam o detalhamento e expressão de um fluxo de pensamentos profundos acerca de um constante mal-estar no mundo com a crueza direta dos acontecimentos. Tal como o narrador que a partir de acontecimentos que cruzam seu caminho sem pedir licença é arrastado a um turbilhão sem fim de pensamentos, é impossível não ser arrebatado pela escrita Max Blecher.