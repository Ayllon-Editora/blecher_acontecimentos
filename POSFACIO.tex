%\pagestyle{posf}  

\chapter*{Posfácio\smallskip\subtitulo{O instável mundo de Max Blecher}}
\addtocontents{toc}{\medskip}
\addcontentsline{toc}{chapter}{Posfácio, \textit{por Fábio Zuker}}

\begin{flushright}
\textsc{fábio zuker}
\end{flushright}

\noindent{}\textit{Quem nunca foi tomado por esse sentimento está condenado a jamais sentir a verdadeira amplitude do mundo}, diz o narrador desta história. Um mundo  denso, cujos detalhes são capazes de perturbar sua ordem usual --- mas suficientemente voláteis para atravessar o corpo daquele que narra. Como se o contorno que separa os objetos dos seres fosse suspendido ao abrir espaço para um outro tipo de discurso: mais próximo à \textit{loucura}, de um estado de espírito diferente do roteiro pré-estabelecido, através do qual são oferecidas inquietações capazes de alterar a própria rota.

Publicado em 1936, \textit{Acontecimentos na irrealidade imediata} é de difícil classificação. O livro reúne experiências desconcertantes, cheias ao mesmo tempo de energia e de melancolia. Em Blecher, a busca é pela escrita de um \textit{corpo permeável}, na qual a percepção da realidade, do tempo e do espaço se confundem. 

\textls[-5]{A escrita fragmentada emula os mecanismos desconexos da memória, no qual excertos narrativos podem tanto se concatenar entre si quanto levar a outros tantos --- cenários por vezes interrompidos, quem sabe até fantasiosos, sem um fio condutor racional. Uma sucessão de cenas que cabe como estratégia fantástica à descrição dos fluidos \textit{acontecimentos}, nas antípodas de uma narrativa grandiloquente que anseie por uma perspectiva totalizante.}\looseness=-1

\textls[-15]{Composto por um amálgama caleidoscópico de fragmentos da infância e juventude do narrador, a história transcorre em sua maior parte ao longo de um verão. O relato é construído através de distensões temporais, deambulações pelas ruas, encontros eróticos e sexuais, irrupção de desejos impossíveis de serem contidos, crises de saúde que não raro se convertem em desmaios, além de objetos que se imiscuem no próprio corpo e tornam-no alheio a si mesmo. Uma experiência de assombro constante, particularmente presente ao se deparar com fotografias --- inclusive com a sua própria ---, imagens, encenações ou outras formas de representação do mundo.}\looseness=-1

\section{Que sublime é ser louco}

\textls[-15]{\textit{Que esplêndido, que sublime é ser louco!}, diz o narrador a si mesmo, em um fluxo de pensamentos que o acompanha ao sair de uma sessão vespertina de cinema. Ele estranha o mundo fora das telas. O anoitecer, que acontecera enquanto estava absorto na escuridão do filme, lhe causa melancolia. Parece-lhe difícil que a abrupta transição do dia que se fez noite enquanto estava no cinema em nada afetou a continuidade da vida, que prossegue em seu desenrolar usual.}\looseness=-1

Não há uma palavra sobre o filme que acabara de assistir, mas uma divagação acerca do automatismo do mundo e das pessoas. Tal qual a noite que invariavelmente chega a uma determinada hora para encerrar o dia, os elementos performam mecanicamente \textit{uma espécie de triste obrigação de sempre continuar}.

\textls[-15]{E é caminhando pelas ruas que se depara com aquela que denomina a \textit{louca da cidade}. Suja, cabelos desgrenhados, sem dentes, carregando troços que lhe haviam sido doados, sua figura contrasta com a racionalização irrefletida da repetição de quem, ao seu redor, segue desempenhando o próprio papel, tal como previsto --- ou exigido --- pelos ditames sociais. O narrador, que parece ser um alter-ego do próprio Max Blecher, inveja a presença da \textit{louca}, que exibe seu sexo sem pudor àqueles que passavam pela rua.}\looseness=-1

Mas há uma inversão: não é a \textit{louca} que performa uma cena, pois goza da liberdade de não seguir um roteiro pré-determinado. São os demais transeuntes que a observam, e permanecem presos aos seus papéis. O narrador estabelece com ela, que sequer o percebeu, uma profunda identificação. Sua loucura possui uma verdade que os outros são incapazes de acessar: \textit{as pessoas ao meu redor pareciam pobres criaturas dignas de pena pela seriedade com que continuamente se ocupavam, acreditando, ingênuas, naquilo que faziam e sentiam}. Mas não o narrador. Este \textit{outro eu} do jovem Blecher se afasta desta existência simulada, triste, deixando-se levar pelas mais variadas situações inauditas, excitantes e angustiantes que as ruas de uma cidade lhe oferecem. 

\section{O surrealismo e a cidade}

Existe um elemento que subverte os traços de autoficção ou relato memorialista e torna esta obra singular: um refinado experimento literário surrealista. Relativamente próximo ao movimento de André Breton, com quem inclusive trocou cartas, Blecher chegou a publicar textos na revista \textit{O surrealismo a serviço da revolução}.\footnote{Em francês, \textit{Le surréalisme au Service de la révolution}.} Ecoam em \textit{Acontecimentos na irrealidade imediata} formas surrealistas, como a dos romances \textit{O camponês de Paris}, escrito em 1926 por Louis Aragon, ou \textit{Nadja}, do próprio Breton, escrito em 1928 --- cujas narrativas se constroem no ato da caminhada pela cidade e de seus elementos imponderáveis.

\textls[-20]{Cabe aqui evitar qualquer tipo de anacronismo em relação ao que consideramos hoje uma caminhada pela cidade. A intensidade da experiência urbana era algo novo, e objeto de reflexão intelectual. Georg Simmel identifica no surgimento do \textit{blasé} uma forma de autoproteção aos estímulos incessantes da metrópole; Walter Benjamin, em seu fascinante e multifacetado \textit{Passagens}, convida o leitor a flanar pela Paris moderna, seus \textit{boulevards}, galerias, moda, iluminação e demais formas culturais que ali emergiram, como radicalmente distintas de tudo o que até então caracterizaria a vida.}\looseness=-1

Andar pela cidade, perder-se na paisagem urbana: uma experiência vertiginosa e moderna. É exigido preparo, sob o risco de defrontar-se com caminhos ainda desconhecidos. Entre a carnificina da Primeira Guerra Mundial e a iminente degeneração nazifascista, os surrealistas interpelavam a modernidade a partir da busca por outras categorias de pensamento e formas de experimentação avessas a um real demasiado fixo. Não havia cenário melhor do que a cidade para buscar outras formas de expressão, ávidas por enterrar no passado formas artísticas devotadas a reis poderosos, religiosos influentes ou burgueses endinheirados.

\section{Um corpo poroso}

É folheando um livro de anatomia que o narrador encontra a fotografia de um modelo de cera, que reproduz o interior de uma orelha. \textit{Todos os canais, seios e buracos eram de matéria plena, formando sua imagem positiva} --- sua impressão diante da fotografia é tão desmedida, que ele quase desmaia. É a partir da imagem do modelo de cera da orelha que o narrador percebe o potencial de existência de uma outra realidade, ao avesso. \textit{Tudo o que é furado se tornasse cheio, e os relevos atuais se transformassem em vácuos de forma idêntica, sem qualquer conteúdo}. Uma realidade que tivesse como modelo os fósseis, cujos contornos apenas existem na medida em que deixaram marcas esculpidas em pedras.

\begin{quote}
Num mundo como esse, as pessoas cessariam de ser excrescências multicoloridas e carnosas, cheias de órgãos complexos e putrescíveis, tornando-se vácuos puros, flutuantes como bolhas de ar dentro d'água, atravessando a matéria quente e mole do universo pleno.
\end{quote}

\textls[10]{A sensação de ser vazio, sem órgãos, como invólucros transparentes, lhe traz um \textit{insight} --- como quando após refletir sobre um problema que há muito nos angustia, conseguimos finalmente observá-lo através de outro ângulo. Aquele mínimo afastamento de uma neurose que nos absorvia, dificilmente compartilhada com alguém que não nós mesmos.}

Seu fascínio com a fotografia do modelo de cera lhe permite compreender as aflições e anseios de sua adolescência, quando o sobrepeso do mundo tornava a própria estabilidade opressora, \textit{como se as pessoas e as casas em derredor de repente houvessem se amalgamado numa massa compacta e uniforme de uma única matéria, na qual eu existia como um simples vácuo sem finalidade, locomovendo-me para lá e para cá}.

Blecher narra a experiência de um \textit{corpo poroso}, atravessado por elementos, sensações, objetos e desejos, como que dotados de vida própria. Incontroláveis, excedem à própria razão do narrador que parece em vão tentar organizá-los: mas a eles cede, com prazer e angústia, deixando de ser um dos transeuntes que caminhavam na rua ao redor da \textit{louca} para aproximar-se dela.

\section{A política da exclusão}

\textls[-15]{O tema da exclusão e de um certo afastamento da realidade --- uma estranheza, ou melhor, mal-estar no mundo --- perpassa o livro. É um tópico comum na obra de outros escritores judeus que viveram na Europa na mesma época de Blecher, como Franz Kafka ou Bruno Schulz, dos quais o autor frequentemente é aproximado.}\looseness=-1

No livro \textit{Léxico familiar} de Natalia Ginzburg, publicado em 1963 --- uma geração mais nova do que a de Blecher ---, o pertencimento de uma família judia assimilada à sociedade italiana também implica, simultaneamente, na distância de não pertencer. Já na obra de Marcel Proust --- também de origem judaica ---, com quem Blecher é frequentemente comparado, o afastamento ocorre no plano das memórias que irrompem, o que torna o presente algo que escapa entre os dedos.

Mas em Blecher o \textit{afastamento da realidade} não parece ser da ordem do comedimento ou clausura, como em Kafka e Schulz, nem das tensões políticas e  antissemitismo que se acirra com as leis raciais de 1938 na Itália fascista de Ginzburg. A experiência do escritor romeno de estar no mundo com um corpo doente é \textit{sui generis}, e implica momentos de crise que o obrigam a deitar-se, a ponto de desmaiar, e o levam a ações transgressoras, como chafurdar-se em um lamaçal, e nele adormecer até o anoitecer.

\textls[7]{Há algo do Antonin Artaud que escreveu \textit{Para dar um fim no juízo de Deus}, publicado em 1947. Em sua peça-transmissão radiofônica, Artaud coloca em xeque a pretensão e superioridade de julgamento das clínicas manicomiais pelas quais passou durante anos. Para questioná-las, faz uma crítica da própria capacidade divina de julgar, e reivindica para si um \textit{corpo sem órgãos}, expressão posteriormente consagrada como conceito teórico por Gilles Deleuze e Félix Guattari. Um corpo que, livre da composição de seus órgãos, abre-se para um experimento de si. \textit{Se quiserem, podem meter-me numa camisa de força, mas não existe coisa mais inútil que um órgão. Quando tiverem conseguido um corpo sem órgãos, então o terão liberado dos seus automatismos e devolvido sua verdadeira liberdade}, escreve Artaud.}

A aproximação do corpo vazado por Blecher e Artaud, cujas margens em relação ao mundo são borradas, salienta o aspecto fundamental da obra do escritor romeno: a escrita emerge de experimentar a doença e de sucessivas internações para tratá-la. No caso de Blecher, diferente de Artaud, não são tratamentos manicomiais, mas de uma tuberculose óssea chamada \textit{Mal de Pott}, para a qual não havia cura na época.

\section{O jovem Blecher}

\textls[-10]{Blecher nasceu em 1909 em Botoșani, na Romênia, filho de bem-sucedidos comerciantes judeus do ramo da porcelana. Fez seus estudos em Roman, na Moldávia romena, onde provavelmente é narrado \textit{Acontecimentos na irrealidade imediata}. É a mesma cidade onde faleceria em 1938, aos 28 anos de idade, após cerca de dez anos doente perambulando entre hospitais europeus --- e 12 anos após a apresentação dos primeiros sinais de sua enfermidade. A Roman de sua infância e adolescência está separada da Roman de sua morte por um período de intensa atividade artística e viagens}.\looseness=-1

%\footnote{Etapa escolar equivalente ao colegial.}
\textls[5]{Durante o liceu, toma contato com a literatura francesa. E ao terminar os estudos em 1928, frente ao crescente antissemitismo na Romênia, Blecher se une a uma legião mundial de jovens seduzidos pelo ambiente cultural moderno da França, e matricula-se no curso de medicina da Universidade de Rouen.}\footnote{\textls[-15]{A cidade de Rouen é localizada na região histórica da Normandia, noroeste da França.}\looseness=-1} \textls[5]{É entretanto obrigado a abandoná-lo pouco tempo depois, quando é diagnosticado com tuberculose óssea. A partir de então, sua vida se torna uma sequência de internações em sanatórios da França, da Suíça e de sua Romênia natal.}
	
\textls[-10]{A década de doença e internações é também a mais rica entre escritos e correspondências, em especial as frequentes cartas trocadas com André Breton, o líder do movimento surrealista francês. Foi também um momento de ainda maior recrudescimento do antissemitismo romeno, por conta do movimento nacionalista e fascista da Guarda de Ferro. É neste ambiente político instável e com o avançar das dores causadas pela doença que Blecher escreve, entre 1934 e 1938, a parte mais significativa de sua obra: \textit{Corpo transparente}, um livro de poemas, \textit{Corações cicatrizados}, um romance,}\looseness=-1\footnote{Ambos os livros são traduzidos para o português por Fernando Klabin, que assina também a presente tradução.} \textls[-10]{e \textit{A toca iluminada}, uma publicação póstuma --- além de, evidentemente, \textit{Acontecimentos na irrealidade imediata}.}\looseness=-1

\textls[10]{Atravessado por forças contrapostas, como melancolia, desespero, desmaios, perda de consciência, descobertas sexuais e impulsos violentos que acabam por nunca se concretizar, \textit{Acontecimentos na irrealidade imediata} é uma obra sem meias palavras. O que faz de sua escrita tão potente é o modo como se combinam o detalhamento e expressão de um fluxo de pensamentos profundos acerca de um constante mal-estar no mundo, unido à crueza dos \textit{acontecimentos}. Tal como o narrador, que a partir de situações que cruzam seu caminho sem pedir licença é arrastado a um turbilhão sem fim de pensamentos, é impossível não ser arrebatado pela escrita Max Blecher}.