\textbf{Acontecimentos na irrealidade imediata} foi originalmente publicado em 1936, e é composto por um amálgama caleidoscópico de situações que cruzam o caminho do narrador, um personagem desajustado ao mundo. Esses acontecimentos arrastam-no a um turbilhão de pensamentos e ações atravessados por forças contrapostas, no qual as percepções misturadas da realidade, do tempo e do espaço dão lugar a um tipo diferente de discurso, que oferece inquietações fragmentadas ao invés de uma ordem racional.

\textbf{Max Blecher} nasceu em 1909 em Botoșani, Romênia, filho de bem-sucedidos comerciantes judeus do ramo da porcelana. Cursou o liceu em Roman, onde tomou seu primeiro contato com a literatura francesa. Em 1928 matriculou-se no curso de medicina da Universidade de Rouen, na França, mas foi obrigado a abandoná-lo pouco tempo depois por conta de sua saúde. Volta então para Roman, onde faleceria em 1938, dez anos após uma sequência de internações hospitalares. A década de internações é a mais produtiva entre escritos e correspondências, em especial as frequentes cartas trocadas com o líder do movimento surrealista André Breton. É neste ambiente instável que Blecher escreve, entre 1934 e 1938, \textit{Corpo transparente}, \textit{Corações cicatrizados} e \textit{Acontecimentos na irrealidade imediata}, além de \textit{A toca iluminada}, uma publicação póstuma.

\textbf{Fernando Klabin} nasceu em São Paulo e formou"-se em Ciência Política pela Universidade de Bucareste, onde foi agraciado com a Ordem do Mérito Cultural da Romênia no grau de Oficial, em 2016. Além de traduzir do romeno, exerce atividades ocasionais como fotógrafo, escritor, ator e artista plástico.

\textbf{Fábio Zuker} é antropólogo, jornalista e ensaísta. É autor de \textit{Em Rota de Fuga: ensaios sobre escrita, medo e violência} (2020) e \textit{Vida e morte de uma baleia-minke no interior do Pará e outras histórias da Amazônia} (2019).

