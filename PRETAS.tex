\textbf{Max Blecher} (1909--1938) \textls[10]{nasceu em Botoșani, Romênia, filho de bem-sucedidos comerciantes judeus do ramo da porcelana. Cursou o liceu em Roman, e em 1928 matriculou-se no curso de medicina da Universidade de Rouen, na França, mas foi obrigado a abandoná-lo pouco tempo depois por conta de sua saúde. Volta então para Roman, onde faleceria em 1938, dez anos após uma sequência de internações hospitalares. A década de internações lhe rendeu muitos escritos e correspondências, como por exemplo as cartas trocadas com André Breton, líder do movimento surrealista francês, e os livros \textit{Corpo transparente}, \textit{Corações cicatrizados} e \textit{Acontecimentos na irrealidade imediata}, além de \textit{A toca iluminada}, uma publicação póstuma.}

\textbf{Acontecimentos na irrealidade imediata} (1936) \textls[10]{foi originalmente publicado em 1936, e é composto por um amálgama caleidoscópico de situações que cruzam o caminho do narrador, um personagem desajustado ao mundo. Esses acontecimentos arrastam-no a um turbilhão de pensamentos e ações atravessados por forças contrapostas, no qual as percepções misturadas da realidade, do tempo e do espaço dão lugar a um tipo diferente de discurso, que oferece inquietações fragmentadas ao invés de uma ordem racional.}

\textbf{Fernando Klabin} \textls[10]{nasceu em São Paulo e formou"-se em Ciência Política pela Universidade de Bucareste, onde foi agraciado com a Ordem do Mérito Cultural da Romênia no grau de Oficial, em 2016. Além de tradutor, exerce atividades ocasionais como fotógrafo, escritor, ator e artista plástico.}

\textbf{Fábio Zuker} \textls[15]{é antropólogo, jornalista e ensaísta. Autor de \textit{Em Rota de Fuga: ensaios sobre escrita, medo e violência} (Hedra, 2020) e \textit{Vida e morte de uma baleia-minke no interior do Pará e outras histórias da Amazônia} (Publication Studio São Paulo, 2019).}


