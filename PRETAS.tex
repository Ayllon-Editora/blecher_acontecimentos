\textbf{Mihail Sebastian} (1907--1945) foi um dos maiores intelectuais romenos do século passado. Além de romancista, atuou como dramaturgo, jornalista e ensaísta. Foi influenciado pela literatura francesa e pela rebeldia das vanguardas artísticas europeias das décadas de 1920 e 1930, assim como os compatriotas Emil Cioran, Eugène Ionesco e Mircea Eliade. A despeito da notoriedade que construiu no meio artístico, não teve o reconhecimento de seus contemporâneos por ser judeu, e passou a ser excluído e execrado desse círculo. Morreu tragicamente em 1945, aos 38 anos, atropelado por um caminhão militar soviético. A publicação de sua obra traz de volta à atenção do público um dos mais importantes autores do cenário literário romeno.

\textls[10]{\textbf{Mulheres} (1933), romance de estreia de Mihail Sebastian, é sua segunda obra publicada, após a novela curta \textit{Fragmentos de um diário encontrado} (1932). A história acontece em quatro partes, sutilmente articuladas pelas memórias que flutuam entre elas. O livro acompanha a vida de Ștefan Valeriu, desde a juventude em um \textit{resort} nos Alpes até a vida adulta já estabelecida entre Paris e Bucareste. Na contramão do ideal romântico e burguês do amor, \textit{Mulheres} explora temas como o vazio, as contradições e os desencontros que caracterizam o sentimento amoroso e, mais amplamente, as relações humanas. Renée, Marthe, Odette, Emilie, Maria e Arabela não são apenas os nomes das mulheres com quem o protagonista se relaciona: elas também simbolizam retratos da Europa entre guerras. E, em última análise, \textit{fazem e definem} Valeriu, na medida em que ele, talvez, não tenha contornos próprios definidos. Ou, ao menos, eles não nos são revelados. \textit{Mulheres} é, enfim, um hino ao amor em todas as suas formas --- muitas vezes imprudentes, outras gloriosas e sobretudo efêmeras.}

\textbf{Fernando Klabin} nasceu em São Paulo e formou"-se em Ciência Política pela Universidade de Bucareste, onde foi agraciado com a Ordem do Mérito Cultural da Romênia no grau de Oficial, em 2016. Além de tradutor, exerce atividades ocasionais como fotógrafo, escritor, ator e artista plástico.

\textls[10]{\textbf{Mirella Botaro} é professora de Estudos Brasileiros na Universidade Sorbonne Nouvelle e traduz obras literárias do francês para o português. É vencedora do prêmio \textit{\textsc{gis} Études Africaines en France}.}



