
\chapter*{}
\thispagestyle{empty}

\vfill
\begin{flushright}
\textit{I pant, I sink, I tremble, I expire}\\
\textsc{p.\,b.\,shelley}
\end{flushright}

\part[Acontecimentos na irrealidade imediata]{Acontecimentos na\break
 irrealidade imediata}

\chapter*{}

\section{i}

\letra{A}{o} fitar por muito tempo um ponto fixo na parede, às vezes acabo não
 sabendo mais quem sou nem onde estou. Então, sinto claramente falta da minha
 identidade como se tivesse me tornado, de repente, um estrangeiro perfeito.
 Esse personagem abstrato e minha pessoa real disputam em pé de igualdade a
 minha convicção.

\textls[-15]{No instante seguinte minha identidade se recompõe, como naqueles
 cartões estereoscópicos em que as duas imagens por engano às vezes não
 coincidem e só quando o operador as ajusta, sobrepondo-as, surge a ilusão de
 relevo. Nessas ocasiões, o quarto se me apresenta com um frescor inédito.
 Ele retorna à sua consistência anterior e seus objetos pousam nos devidos
 lugares, como um torrão de terra esfarelado numa garrafa cheia d'água, que
 vai se assentando em camadas de elementos diferentes, bem definidas e de
 cores variadas. Os elementos do quarto se estratificam em seu próprio
 contorno e no colorido da antiga lembrança que tenho deles.}\looseness=-1

A sensação de distanciamento e de solidão nos momentos em que minha pessoa
cotidiana se dissolve em inconsistência é diferente de quaisquer outras
sensações. Quando dura muito, ela se transforma em medo, em pavor de não
conseguir nunca mais me reencontrar. Ao longe, persiste uma silhueta insegura
de mim mesmo, rodeada por um grande halo de luz, à maneira dos objetos
visíveis através da névoa.

A terrível pergunta \textit{quem realmente sou} pulsa no meu âmago como um
corpo perfeitamente novo, que cresceu dentro de mim com pele e órgãos que me
são completamente desconhecidos. Uma lucidez mais profunda e mais essencial
que a do cérebro exige uma resposta. Tudo o que é capaz de se agitar no meu
corpo se agita, debate e revolta com mais força e de modo mais elementar do
que na vida cotidiana. Tudo implora uma solução.

Por vezes reconheço o quarto assim como ele é, como se eu fechasse e abrisse
os olhos: a cada vez o quarto é mais claro --- assim como uma paisagem vista
pela luneta, cada vez mais organizada à medida que, ajustando as distâncias,
percorremos todos os véus de imagens intermediárias.

\textls[-15]{Finalmente reconheço-me a mim mesmo e reencontro o quarto. É uma sensação de
leve embriaguez. O quarto parece extraordinariamente condensado em sua
matéria, e eu implacavelmente de volta à superfície das coisas: quanto mais
profunda a onda de imprecisão, mais alta é a sua crista; nunca, em nenhuma
outra circunstância além de tais momentos, me parece mais evidente que cada
objeto deve ocupar o lugar que ocupa e que eu devo ser quem sou.}\looseness=-1

Atormentar-me em insegurança não tem então nenhum motivo; é um simples
arrependimento de não ter encontrado nada em sua profundidade. Apenas me
surpreende que uma total falta de significado tenha podido estar tão ligada à
minha matéria íntima. Agora que me reencontrei e tento expressar minha
sensação, ela se apresenta diante de mim perfeitamente impessoal: um simples
exagero da minha identidade, que brotou como um câncer a partir de sua
própria substância. Um tentáculo de medusa que se estendeu além da medida e
que ansiou exasperadamente em meio às ondas, até enfim voltar para baixo da
ventosa de gelatina. Em alguns momentos de desassossego, percorri assim todas
as certezas e incertezas da minha existência, até retornar definitiva e
dolorosamente à minha solidão.

\textls[10]{Agora, trata-se de uma solidão mais pura e mais patética. A sensação de
distanciamento do mundo é mais clara e mais íntima: uma melancolia límpida e
suave, como um sonho resgatado no meio da madrugada.}

Só ela ainda consegue me fazer recordar um pouco do mistério e da magia meio
triste das minhas \textit{crises} de infância.

Só nesse desaparecimento súbito de identidade é que reencontro minhas quedas
no espaço maldito de outrora e só nos momentos de imediata lucidez que se
seguem ao retorno à superfície é que o mundo se me apresenta na atmosfera
incomum de inutilidade e anacronismo que se formava ao meu redor quando meus
transes alucinantes logravam derrubar-me.


\section{ii}

\letra{S}{empre} os mesmos lugares da rua, da casa ou do jardim provocavam
 minhas \textit{crises}. Sempre que eu adentrava esses espaços, o mesmo
 desmaio e a mesma tontura me envolviam. Verdadeiras armadilhas invisíveis,
 espalhadas aqui e ali pela cidade, absolutamente semelhantes ao ar que as
 rodeava --- esperavam com ferocidade que eu me tornasse vítima da atmosfera
 especial que encerravam. Bastava um passo, um só passo para adentrar tal
 espaço maldito, e a \textit{crise} era inevitável.

\textls[-10]{Um desses espaços se encontrava no parque municipal, numa pequena clareira ao
fim de uma alameda, por onde nunca ninguém passeava. Os arbustos de roseira
brava e as acácias anãs que o rodeavam permitiam uma única abertura na
direção de uma paisagem desoladora de um campo deserto. Não havia no mundo
lugar mais ermo e triste. O silêncio se depunha densamente sobre as folhas
empoeiradas durante o abafado calor do verão. De vez em quando, ouviam-se
ecos do trompete do regimento. Aqueles longos chamados no deserto
eram \textit{dilacerantemente} tristes\ldots{} Ao longe, o ar inflamado pelo
sol tremulava diáfano como vapores transparentes bailando por sobre um
líquido em ebulição.}\looseness=-1

O lugar era selvagem e isolado; sua solidão parecia interminável. Ali, o calor
do dia era mais fatigante e o ar que eu respirava, mais pesado. Os arbustos
cobertos de poeira queimavam-se amarelados sob o sol, numa atmosfera de
perfeita solidão. Uma sensação bizarra de inutilidade pairava naquela
clareira que ficava \textit{em algum lugar do mundo}, aonde eu mesmo chegara
sem motivo, numa certa tarde de verão também sem sentido. Uma tarde que se
perdera caótica no calor do sol, por entre arbustos ancorados no
espaço \textit{em algum lugar do mundo}. Então eu sentia, mais profunda e
dolorosamente, que nada havia a fazer neste mundo senão perambular por
parques --- por clareiras cobertas de poeira e fustigadas pelo sol,
abandonadas e selvagens. Era uma perambulação que, por fim, dilacerava meu
coração.


\section{iii} 

\letra{O}{utro} lugar amaldiçoado ficava bem do lado oposto da cidade, entre
 as margens altas e ocas do rio, onde eu costumava brincar com meus
 companheiros.

A margem, num determinado ponto, se transformara num barranco. No alto ficavam
as instalações de uma fábrica de óleo de semente de girassol. As cascas das
sementes eram despejadas entre as dobras do barranco e, com o passar do
tempo, o monte atingira tal altura que se formara uma ladeira de cascas secas
que ia de cima até as margens do rio. Meus companheiros desciam até o rio por
essa ladeira, com prudência, segurando um na mão do outro enquanto seus
passos afundavam no tapete em putrefação.

\textls[-5]{As paredes altas do barranco, tanto de uma parte como de outra da
 ladeira, eram abruptas e cheias de fantásticas irregularidades. A chuva
 esculpira longas tranças de rachaduras finas como arabescos, porém pavorosas
 como chagas mal cicatrizadas. Eram verdadeiros farrapos feitos a partir da
 carne do barro, feridas abertas, tenebrosas. Por entre essas paredes que me
 impressionavam desmesuradamente, eu também teria que descer até o rio.}

\textls[-22]{Já de longe e muito antes de chegar à margem do rio, o cheiro das cascas
apodrecidas invadiam minhas narinas. Ele me preparava para a \textit
{crise}, como uma espécie de breve período de incubação; era um cheiro
desagradável e ao mesmo tempo suave. Assim como as \textit{crises}.}\looseness=-1

Meu sentido olfativo, em algum lugar dentro de mim, se partia em dois, de
maneira que as emanações do cheiro de putrefação atingiam regiões de
sensações diferentes. O cheiro gelatinoso da decomposição das cascas, embora
concomitante, se distinguia bem, separando-se do perfume agradável, quente e
familiar de amendoim torrado. Esse perfume, tão logo o sentia, em poucos
instantes me transformava, circulando amplamente por todas as minhas fibras
internas como se as dissolvesse para substitui-las por uma matéria mais
vaporosa e mais incerta. A partir daquele momento, eu não podia evitar mais
nada. Brotava no meu peito um desmaio agradável e estonteante que apressava
meus passos em direção da margem, em direção da minha derrota definitiva.

Eu descia ao rio numa corrida insana, por cima do monte de cascas. O ar se
opunha ao meu movimento com uma densidade afiada e dura como o gume de uma
faca. O espaço ocupado pelo mundo se precipitava caótico num buraco imenso
com insondáveis poderes de atração.

Meus colegas assistiam aterrados à minha fuga doida. A margem de cascalho lá
embaixo era bastante estreita, de maneira que o menor passo em falso me
atiraria ao rio, num lugar em que o turbilhão na superfície da água anunciava
grandes profundezas.

\textls[10]{Eu, contudo, não sabia muito bem o que estava fazendo. Já ao longo do rio, na
mesma correria, desviava do monte de cascas e corria pela margem a jusante
até um determinado lugar onde a beira apresentava uma cavidade.}

No fundo da cavidade formara-se uma pequena gruta, uma caverna sombreada e
refrescante como um quartinho escavado na rocha. Eu costumava entrar ali e
cair no chão, transpirado, morto de cansaço, tremendo dos pés à cabeça.

Assim que me revigorava um pouco, encontrava ao meu redor o cenário íntimo e
indizivelmente agradável da gruta, com uma mina que brotava lentamente da
rocha, escorrendo pelo chão e formando uma piscina de água cristalina no meio
do cascalho, sobre a qual eu me inclinava para observar, sem jamais ficar
saciado, as maravilhosas rendas de musgo verde do fundo, vermes agarrados a
fragmentos de madeira, pedaços velhos de ferro enferrujados e cobertos de
lodo, animais e as mais variadas coisas fantasticamente belas que moram no
fundo da água.

\section{iv} 

\letra{A}{lém} \textls[15]{desses dois lugares malditos, o resto da cidade se perdia numa
 massa de uniforme banalidade, com casas que podiam ser confundidas umas com
 as outras, com árvores imóveis ao exaspero, com cachorros, terrenos baldios
 e muita poeira.}\looseness=-1

Em quartos fechados, porém, as \textit{crises} ocorriam com maior facilidade e
frequência. Em geral, eu não suportava ficar sozinho num aposento
desconhecido. Era só esperar, e em poucos instantes chegava o desmaio suave e
terrível. O próprio quarto se preparava para ele: as paredes punham-se a
emanar uma intimidade quente e hospitaleira, que escorria por todos os móveis
e objetos. O quarto de repente tornava-se sublime, e eu me sentia
extremamente feliz ali. Mas isso não passava de mais uma dissimulação
da \textit{crise}; uma perversidade suave e delicada. No instante seguinte à
beatitude, tudo se revirava e se confundia. Fitava de olhos abertos tudo o
que havia ao meu redor, mas os objetos perdiam seu sentido comum: uma nova
existência os animava.

\textls[10]{Como se houvessem sido subitamente desempacotados de papéis finos e
transparentes em que se encontravam envoltos até então, seu aspecto se
tornava inefavelmente novo. Pareciam destinados a uma utilização nova,
superior e fantástica, que eu em vão lograria encontrar.}

\textls[10]{Mas não era só isso: os objetos se deixavam tomar por um verdadeiro frenesi de
liberdade. Tornavam-se independentes uns dos outros, uma independência que
não significava um simples isolamento, mas uma exaltação extática.}

\textls[-5]{O entusiasmo de existir numa nova auréola envolvia a mim também:
 uma forte aderência me prendia a eles, com anastomoses invisíveis que me
 tornavam mais um objeto do quarto, da mesma maneira que um órgão
 transplantado em carne viva se integra ao corpo desconhecido por meio de
 sutis trocas de substâncias.}

Uma vez, durante uma \textit{crise}, o sol lançara sobre a parede uma pequena
cachoeira de raios, como uma torrente irreal de ouro marmorizada com ondas
luminosas. Eu via também, do outro lado da janela, o canto de uma estante com
tomos grossos encadernados em couro, e esses detalhes reais que eu percebia à
distância do desmaio lograram atordoar-me e derrubar-me como uma derradeira
inalação de clorofórmio. O que era mais comum e mais conhecido naqueles
objetos me perturbava ainda mais. O costume de vê-los tantas vezes fizera
provavelmente com que sua pele exterior puísse, por isso às vezes eles
surgiam diante de mim esfolados e cobertos de sangue: vivos, indizivelmente
vivos.

O momento supremo da \textit{crise} se consumava numa flutuação agradável e
dolorosa, que não era deste mundo. Ao menor ruído de passos, o quarto
rapidamente voltava ao seu aspecto inicial. Ocorria então entre as suas
paredes uma redução instantânea, uma diminuição extremamente pequena de sua
exaltação, quase imperceptível; isso me convencia de que uma finíssima crosta
separava a certeza em que eu vivia do mundo das incertezas.

\textls[7]{Dava-me por mim no quarto arquiconhecido, transpirado, exausto e preenchido
pela sensação de inutilidade das coisas que me rodeavam. Percebia nelas novos
detalhes, assim como ocorre ao descobrirmos algo que nunca havíamos reparado
num objeto utilizado diariamente, anos a fio.}

\textls[-25]{O quarto conservava vagamente a lembrança da catástrofe, como o cheiro de
enxofre que paira no local de uma explosão. Fitava os livros encadernados no
armário com portas envidraçadas e, em sua imobilidade, percebia, não sei
como, um ar pérfido de mistério e cumplicidade. Os objetos ao meu redor
jamais abdicavam de uma certa atitude secreta, ferozmente mantida em sua
severa imobilidade.}\looseness=-1


\section{v} 

\letra{A}{s} \textls[-10]{palavras cotidianas não têm valor em determinadas profundezas da
 alma. Tento definir com exatidão as minhas \textit{crises}, mas só encontro
 imagens. A palavra mágica que vier a exprimi-las deverá tomar algo
 emprestado das essências de outras sensibilidades da vida, destilando-se
 delas como uma nova fragrância criada a partir de uma sábia composição de
 perfumes.}\looseness=-1

Para existir, ela deverá incluir algo da estupefação que me abarca quando
observo uma pessoa na realidade e depois acompanho-lhe com atenção os gestos
no espelho; algo do desequilíbrio das quedas durante o sonho com o seu sibilo
de pavor que percorre a espinha dorsal num momento inesquecível; ou algo da
névoa e da transparência habitadas por bizarros elementos decorativos em
bolas de cristal.


\section{vi} 

\letra{I}{nvejava} as pessoas ao meu redor, hermeticamente fechadas em suas
 roupas e isoladas da tirania dos objetos. Viviam prisioneiras em seus
 sobretudos e casacos, nada do lado de fora era capaz de aterrorizá-las e
 vencê-las, nada penetrava em suas prisões maravilhosas. Entre mim e o mundo
 não havia separação. Tudo o que me rodeava me invadia da cabeça aos pés,
 como se minha pele houvesse sido metralhada. A atenção, aliás muito
 distraída, com que eu observava em derredor não era um simples ato de
 vontade. Todos os tentáculos do mundo se prolongavam, de maneira natural,
 dentro de mim; eu era atravessado pelos milhares de braços da hidra. Tinha
 de constatar, ao exaspero, que vivia no mundo que via. Nada havia a fazer
 contra isso.

As \textit{crises} pertenciam tanto a mim quanto aos lugares em que ocorriam.
É verdade que alguns desses lugares continham uma certa maldade \textit
{pessoal}, mas todos os outros se encontravam em transe muito antes da minha
chegada. Assim eram, por exemplo, alguns aposentos em que sentia que
minhas \textit{crises} se cristalizavam a partir da melancolia da imobilidade
e de sua ilimitada solidão.

\textls[10]{Como uma espécie de equidade, porém, entre mim e o mundo (uma equidade que me
 imergia ainda mais irremediavelmente na uniformidade da matéria bruta), a
 convicção de que os objetos podiam ser inofensivos tornara-se igual ao terror
 que por vezes eles me infundiam. Sua inocência se originava de uma falta
 universal de forças. Sentia vagamente que nada neste mundo podia ir até o
 fim, nada podia se realizar. A ferocidade dos objetos também se exauria.
 Dessa maneira nasceu em mim a ideia da imperfeição de quaisquer manifestações
 neste mundo, mesmo as sobrenaturais.}\looseness=-1

\textls[-15]{Num diálogo interior que, creio, nunca terminava, eu desafiava às vezes os
poderes maléficos ao meu redor, da mesma maneira como, outras vezes, vilmente
os adulava. Executava ritos que, embora estranhos, não eram desprovidos de
sentido. Ao sair de casa para andar por diferentes ruas, eu voltava sempre
refazendo meus passos, a fim de não desenhar, com meu trajeto, um círculo que
contivesse casas e árvores. Nesse sentido, meu andar se parecia com um
carretel de linha que, uma vez desenrolado, se eu não o enrolasse na hora, no
mesmo caminho, os objetos reunidos no laço do meu andar ficariam para sempre,
irremediável e profundamente, ligados a mim. Se quando chovesse eu evitava
encostar nas pedras sob a torrente de água, é porque não queria adicionar
nada à ação da água nem intervir no exercício de sua força elementar.}\looseness=-1

\textls[-15]{O fogo purificava tudo. Eu tinha sempre uma caixa de fósforos no bolso. Quando
estava muito triste, acendia um fósforo e passava a palma das mãos pelo fogo,
primeiro uma, depois a outra.}\looseness=-1

\textls[15]{Havia nisso tudo uma espécie de melancolia de existir e uma espécie de tortura
organizada normalmente nos limites de minha vida de criança.}

Com o tempo, as \textit{crises} desapareceram por si sós, não sem deixar em
mim, para sempre, sua forte lembrança.

\textls[15]{Ao entrar na adolescência, não sofri mais \textit{crises}, mas aquele estado
crepuscular que as precedia e o sentimento de profunda inutilidade do mundo
que as sucedia tornaram-se, de certa forma, o meu estado natural.}

\textls[10]{A inutilidade preenchera as reentrâncias do mundo como um líquido que havia se
propagado em todas as direções, e o céu acima de mim, o céu eternamente
correto, absurdo e indefinido, se imbuíra da própria cor do desespero. }

Por essa inutilidade que me rodeia e debaixo desse céu eternamente amaldiçoado
ainda hoje eu passeio.


\section{vii} 

\letra{P}{or} \textls[10]{causa das minhas \textit{crises}, levaram-me para ser examinado
 por um médico, que pronunciou uma palavra estranha: \textit
 {paludismo}; fiquei surpreso com o fato de que minhas ansiedades tão íntimas
 e secretas pudessem ter um nome e, ainda por cima, um nome tão bizarro. O
 doutor prescreveu-me quinino: outro termo admirável. Era-me impossível
 compreender como poderiam curar-se eles, os espaços doentes, com o quinino
 que eu tomaria. Contudo, o que me perturbou demasiado foi o próprio médico.
 Durante muito tempo após a consulta, ele continuou existindo e se agitando
 em minha memória com gestos miúdos e automáticos cujo inesgotável mecanismo
 eu não lograva interromper.}\looseness=-1

\textls[-5]{Era um homem de baixa estatura, com cabeça em forma de ovo. A extremidade
afilada do ovo se prolongava com uma barbicha preta sempre agitada. Seus
olhos pequenos e aveludados, seus gestos breves e sua boca que vinha para
frente o faziam parecer com um rato. Foi tão forte essa impressão desde o
primeiro momento, que pareceu-me perfeitamente natural, quando começou a
falar, ouvi-lo alongando bastante e sonoramente cada \textit{r}, como se,
enquanto falasse, estivesse sempre a roer algo escondido.}\looseness=-1

\textls[-25]{O quinino receitado reforçou também minha convicção de que o médico tinha um
quê de rato. A prova dessa convicção realizou-se de uma maneira tão estranha
e é ligada a fatos tão importantes de minha infância, que o acontecimento
merece, creio eu, ser contado à parte.}\looseness=-1


\section{viii} 

\letra{P}{erto} \textls[7]{da nossa casa, havia uma loja de máquinas de costura aonde eu
 ia todos os dias e onde ficava horas a fio. Seu proprietário era um jovem,
 Eugen, que acabara de prestar o serviço militar e encontrara uma atividade
 na cidade abrindo aquele negócio. Ele tinha uma irmã um ano mais nova:
 Clara. Moravam juntos na periferia e, de dia, tomavam conta do negócio; não
 tinham conhecidos nem parentes.}

\textls[10]{A loja era um simples espaço particular, alugado pela primeira vez para fins
comerciais.}

\textls[10]{As paredes ainda mantinham a pintura de salão, com guirlandas
 violetas de lilases e retângulos irregulares e desbotados dos lugares onde
 quadros haviam estado pendurados. No centro do teto restou um lustre de
 bronze com um \textit{plafonnier} de maiólica de um vermelho escuro, coberto
 nas beiradas por folhas verdes de acanto em relevo, de faiança. Era um
 objeto todo ornamentado, antigo, fora de moda mas imponente --- tinha um quê
 de monumento funerário ou de general veterano ostentando seu velho uniforme
 durante o desfile. As máquinas de costura estavam perfeitamente alinhadas em
 três fileiras, com duas largas alamedas entre elas. Eugen não deixava de
 borrifar o assoalho a cada manhã com uma lata velha de conservas com um furo
 no fundo. O fio d'água que escorria era bastante fino, e Eugen o manipulava
 com destreza, desenhando no chão espirais e oitos como quem quer mostrar o
 que sabe fazer. Por vezes ele até assinava e escrevia a data. A pintura da
 parede reclamava ostensivamente semelhantes delicadezas.}

\textls[10]{No fundo da loja, um biombo de tábuas dividia uma espécie de cabine do resto
do aposento; uma cortina verde cobria a entrada. Lá ficavam o tempo todo
Eugen e Clara, inclusive no almoço, para não terem de fechar a loja durante o
dia. Eles a chamavam de ``cabine dos artistas''; um dia ouvi Eugen dizendo:
``É uma verdadeira \textit{cabine de artista}. Quando vou para a loja e falo
por meia hora para conseguir vender uma máquina de costura, não estou
interpretando uma comédia?''.}

\textls[10]{E acrescentou, com um tom de maior erudição: ``A vida, em geral, é puro
teatro''.}

\textls[-13]{Do outro lado da cortina, Eugen tocava violino. Mantinha as
 partituras em cima da mesa e ficava arqueado sobre elas, decifrando com
 paciência o emaranhado das pautas como se desembaraçasse um novelo de linha
 cheio de nós, a fim de retirar dele um fio único e aguçado --- o fio da peça
 musical. Durante a tarde inteira, uma pequena lâmpada de querosene que
 preenchia o aposento com uma luz mortiça ardia em cima de uma arca,
 distorcendo a figura do violinista ao projetar na parede uma sombra
 gigantesca.}\looseness=-1

\textls[10]{Eu ia lá com tanta frequência que, com o tempo, tornei-me uma espécie de
móvel, um prolongamento do velho sofá de lona sobre o qual eu ficava imóvel,
um objeto com que ninguém se preocupava e que a ninguém perturbava.}

No fundo da cabine, todas as tardes Clara se arrumava. Guardava os vestidos
num armariozinho e ficava se admirando num espelho rachado apoiado na
lâmpada, em cima da arca. Era um espelho tão velho que já havia perdido o
brilho em alguns pontos, de maneira que, através das manchas transparentes,
do lado de trás surgiam objetos reais que se misturavam às imagens
refletidas, como numa fotografia feita com clichês sobrepostos.

Por vezes, ela se desvestia quase por completo e esfregava as axilas com água
de Colônia, erguendo os braços ou mesmo os seios --- quando enfiava a mão
entre a combinação e o corpo --- sem qualquer pejo. A combinação era curta e,
quando se inclinava, eu podia ver suas pernas inteiras, muito bonitas,
apertadas dentro de meias bem esticadas. Parecia-se absolutamente com uma
mulher seminua que eu vira num cartão-postal pornográfico que um vagabundo me
mostrara no parque certa vez.

Gerava em mim o mesmo langor confuso de uma imagem obscena, uma espécie de
vácuo no peito ao mesmo tempo que eu sentia no púbis as garras de um
exorbitante apetite sexual. Na \textit{cabine} eu ficava sentado sempre no
mesmo lugar do sofá, atrás de Eugen, esperando Clara terminar de se arrumar.
Em seguida, ela ia para a loja, passando entre mim e seu irmão por um espaço
tão estreito que era obrigada a roçar as coxas nos meus joelhos.

\textls[15]{Todos os dias eu aguardava esse momento com a mesma impaciência e o mesmo
desespero. Ele dependia de uma variedade de minúsculas circunstâncias que eu
analisava e perseguia com uma sensibilidade exasperada e extraordinariamente
aguçada. Bastava Eugen ter sede, não ter vontade de tocar, ou chegar um
cliente na loja para ele sair do lugar junto à mesa e haver espaço suficiente
para Clara passar longe de mim.}

Quando eu ia lá à tarde e me aproximava da porta da loja, longas e vibrantes
antenas emergiam de mim, passando a explorar o ambiente para captar o som do
violino; ao ouvir Eugen tocando, uma profunda calma me dominava. Entrava o
mais silenciosamente possível e anunciava o meu nome em voz alta, ainda na
soleira da porta, para que não pensasse que houvesse chegado um cliente e
interrompesse, assim, a música: pois naquele instante seria possível que a
inércia e o feitiço da melodia cessassem bruscamente, determinando que Eugen
deixasse o violino de lado e não tocasse mais naquela tarde. Isso,
entretanto, não esgotava a possibilidade de acontecimentos desfavoráveis.
Tantas outras coisas ainda ocorriam na cabine\ldots{} Enquanto Clara se
arrumava, eu ficava atento aos mínimos ruídos e observava os menores
movimentos, temendo que qualquer um deles pudesse produzir o desastre da
tarde. Podia acontecer, por exemplo, de Eugen tossir debilmente, engolir um
pouco de saliva e de repente dizer que estava com sede e que iria à
confeitaria comprar um doce; fatos insignificantes, como essa tosse, eram
capazes de gerar uma monstruosa e infindável tarde perdida. O dia inteiro,
portanto, perdia toda a sua importância. E, à noite, deitado na cama, ao
invés de pensar com todas as minúcias (e detendo-me durante alguns minutos em
cada detalhe para poder \textit{vê-lo} e me lembrar dele melhor) no instante
em que os meus joelhos tocaram as meias de Clara --- um pensamento para ser
cinzelado, esculpido e acariciado! ---, eu me debatia febril entre os
lençóis, incapaz de dormir, aguardando impaciente a chegada do dia seguinte.

\textls[-15]{Um dia, sucedeu algo perfeitamente insólito. No início, o acontecimento
parecia avançar na direção de um desastre, mas se concluiu com uma inesperada
surpresa, de maneira tão súbita e com um gesto tão desprezível, que toda a
minha alegria ulterior repousou sobre ele como uma torre de objetos ecléticos
habilmente equilibrados por um malabarista. Com um só passo, Clara modificou
completamente o conteúdo de minhas visitas, imbuindo-as de outro sentido e de
novas emoções, como na aula de química, quando pude constatar, num
experimento, como um único pedacinho de cristal, ao ser mergulhado no
recipiente com líquido vermelho, transformava-o na hora num líquido
surpreendentemente verde.}\looseness=-1

\textls[10]{Eu estava no sofá, no lugar de costume, esperando com a mesma impaciência de
sempre, quando a porta se abriu e alguém entrou na loja. Eugen saiu
imediatamente da cabine. Tudo parecia perdido. Clara continuou se arrumando
com indiferença, enquanto a conversa na loja prolongava-se sem fim. Ainda era
possível, contudo, que Eugen retornasse antes que sua irmã lograsse acabar de
se vestir.}

Acompanhava com o coração na mão o decurso de ambos os eventos, a toalete de
Clara e a conversa da loja, pensando que poderiam se desenrolar paralelamente
um ao outro até Clara entrar na loja ou, pelo contrário, poderiam se
encontrar justo na cabine como em alguns filmes do cinema, quando duas
locomotivas se apressam uma na direção da outra com uma velocidade louca,
chocando-se ou passando uma ao lado da outra a depender da intervenção ou
não, no último instante, de uma mão misteriosa que virasse a agulha do
trilho. Nesses momentos de espera, eu sentia com clareza como a conversa
continuava o seu caminho e como, numa via paralela, Clara continuava se
empoando\ldots{}

\textls[15]{Tratei de corrigir a fatalidade, estendendo bem os joelhos na direção da mesa.
Para interceptar as pernas de Clara, eu tinha de ficar na borda do sofá, numa
posição que, se não era estranha, era no mínimo cômica.}

Acho que, através do espelho, Clara olhava para mim e sorria.

\textls[-20]{Em seguida, ela terminou de retocar o contorno dos lábios com carmim e passou
pela última vez o pompom nas faces. O perfume que se espalhou na cabine me
embriagou de apetite e desespero. Ao passar a meu lado, aconteceu aquilo que
eu menos esperava: ela roçou suas coxas nos meus joelhos como das outras
vezes (ou talvez mais forte? Não, trata-se de uma ilusão, com certeza), com
um ar indiferente, como se nada estivesse acontecendo entre nós.}\looseness=-1

A cumplicidade do vício é mais profunda e mais direta do que qualquer tipo de
entendimento verbal. Ela atravessa instantaneamente o corpo como uma melodia
interior, transformando por completo os pensamentos, a carne e o sangue.

\textls[-15]{Na fração de segundo em que as pernas de Clara me tocaram,
 inflaram-se dentro de mim novas expectativas e novas esperanças.}

Com Clara compreendi tudo desde o primeiro dia, desde o primeiro instante; foi
minha primeira aventura sexual completa e normal. Uma aventura plena de
tormentos e expectativas, plena de ansiedades e de ranger de dentes, algo que
teria se parecido com amor, não houvesse sido a simples continuidade de uma
dolorosa impaciência. Na mesma medida em que eu era impulsivo e atrevido,
Clara era calma e caprichosa; tinha um modo violento de me provocar e uma
espécie de alegria canina ao me ver sofrer --- alegria essa que sempre
precedia o ato sexual e dele fazia parte.

\textls[10]{Na primeira vez que aconteceu entre nós o que eu esperava havia
 tanto tempo, sua provocação foi de uma simplicidade tão elementar (e quase
 brutal) que aquela frase pobre por ela então pronunciada e aquele verbo
 anônimo por ela utilizado mantêm ainda hoje dentro de mim algo da virulência
 de outrora. Basta pensar mais intensamente neles para que minha atual
 indiferença pareça corroída por um ácido e a frase recupere sua violência,
 assim como foi naquele então.}


\section{ix} 

\letra{E}{ugen} \textls[-10]{estava fora para resolver coisas na cidade. Ficamos ambos na
 loja, calados. Clara, em seu vestido diurno, de pernas cruzadas detrás da
 vitrine, tricotava muito atenta. Haviam-se passado algumas semanas desde o
 acontecimento da cabine, tendo-se criado bruscamente entre nós não sei que
 espécie de severa frieza, uma tensão secreta que se traduzia por uma extrema
 indiferença da sua parte. Ficávamos um diante do outro horas inteiras sem
 dizermos uma única palavra; esse silêncio, contudo, pairava como a ameaça de
 uma explosão, um acordo secreto perfeito. Faltava-me a palavra misteriosa
 que fosse capaz de romper o invólucro das convenções; a cada fim de tarde,
 eu imaginava dezenas de projetos que, no dia seguinte, esbarravam nos
 obstáculos mais elementares; o tricô que não podia ser interrompido, a falta
 de uma luz mais favorável, o silêncio da loja ou aquelas três fileiras de
 máquinas de costura, alinhadas corretamente demais para permitir qualquer
 tipo de modificação importante na loja, mesmo que de ordem sentimental. Eu
 mantinha o tempo todo os maxilares retesados; havia um silêncio terrível, um
 silêncio que, no meu interior, tinha o caráter e o contorno de um grito.}\looseness=-1

Foi Clara quem o interrompeu. Falou quase aos sussurros, sem erguer os olhos
do tricô:

--- Se você tivesse vindo hoje mais cedo, poderíamos ter feito, pois Eugen
    saiu logo depois do almoço.

Até então, jamais se infiltrara em nosso silêncio nem mesmo a sombra de uma
alusão sexual, mas eis que agora, a partir de umas poucas palavras, jorrava
entre nós uma nova realidade, tão milagrosa e extraordinária como uma estátua
de mármore que brotasse do assoalho em meio às máquinas de costura.

\textls[-0]{No mesmo instante pus-me ao lado de Clara, peguei na sua mão e a acariciei
ardorosamente. Beijei-lhe a mão. Ela a arrancou de mim.}\looseness=-1

--- Ei, largue-me --- disse ela, enervada.

--- Por favor, Clara, venha\ldots{}

\textls[-30]{--- Agora é tarde demais, o Eugen vai voltar, largue-me, largue-me.}\looseness=-1

Eu a tocava febrilmente nos ombros, nos seios, nas pernas.

--- Largue-me --- protestava Clara.

--- Venha agora, ainda temos tempo --- implorava.

--- Onde?

--- Na cabine\ldots{} vamos\ldots{} ali podemos ficar tranquilos.

\textls[-20]{Ao dizer \textit{tranquilos}, meu peito preencheu-se de uma esperança
fervente. Beijei-lhe de novo a mão e a puxei com força da cadeira. Ela se
deixou levar de mau grado, arrastando os pés pelo assoalho.}\looseness=-1

A partir daquele dia, os \textit{hábitos} da tarde se modificaram: tudo ainda
girava em torno de Eugen, Clara e das mesmas sonatas; agora, porém, o som do
violino se tornara insuportável para mim e minha impaciência espreitava o
momento em que Eugen deveria ir embora. Na mesma cabine, minhas ansiedades
tornaram-se outras, como se eu jogasse um jogo novo em cima de um tabuleiro
com linhas traçadas para um outro jogo. 

\textls[-20]{A verdadeira espera começava depois que Eugen saía. Era uma espera mais
difícil e mais insuportável do que a que havia se desenrolado até então; o
silêncio da loja se transformava num bloco de gelo.}\looseness=-1

\textls[-10]{Clara se sentava junto à vitrine e tricotava: todos os dias era assim
o \textit{começo} e, sem começo, nossa aventura não tinha como acontecer. Às
vezes, Eugen saía enquanto Clara ainda estava quase nua na cabine. Sempre
achei que isso poderia acelerar os fatos, mas estava enganado. Clara não
admitia outro começo se não o da loja. Tinha de esperar inutilmente que ela
se vestisse e se colocasse atrás da vitrine para abrir, na primeira página, o
livro da tarde.}\looseness=-1

\textls[-20]{Sentava-me numa cadeirinha à sua frente e começava a falar-lhe, a pedir-lhe, a
implorar-lhe durante muito tempo. Sabia que era em vão. Clara só raramente
concordava e, mesmo assim, lançava mão de certa malícia só para não me
conceder uma permissão completa.}\looseness=-1

--- Vou pegar um remédio na cabine, estou com uma dor de cabeça terrível, por
    favor não venha atrás de mim.

\textls[10]{Eu jurava e em seguida ia atrás dela. Iniciava-se na cabine uma verdadeira
luta na qual, de maneira evidente, as forças de Clara estavam dispostas a
ceder. Ela caía de uma vez no sofá, como se houvesse tropeçado em algo.
Depois, punha as mãos embaixo da cabeça e fechava os olhos como se estivesse
dormindo. Era-me impossível mover seu corpo um só centímetro que fosse; assim
como estava, de lado, eu tinha de arrancar-lhe o vestido por debaixo das
coxas e unir-me a ela. Clara não esboçava nenhuma oposição aos meus gestos,
mas também não me facilitava nada. Ficava imóvel e indiferente como um tronco
de madeira, e só o seu calor íntimo e secreto revelava-me que estava
prestando atenção e que \textit{sabia}.}


\section{x} 

\letra{M}{ais} \textls[15]{ou menos por essa época eu fora examinado pelo médico que me
 prescrevera quinino. A confirmação de minha impressão de que havia nele um
 quê de rato ocorreu na cabine e, como disse, de uma maneira completamente
 absurda e inesperada.} 

Um dia, enquanto eu estava colado a Clara, arrancando-lhe o vestido com gestos
febris, senti algo estranho se movendo na cabine e --- mais por um instinto
obscuro, porém muito aguçado, do extremo prazer do qual me aproximava e que
não admitia nenhuma presença estrangeira, do que por meus verdadeiros
sentidos --- adivinhei que uma criatura nos observava.

Assustado, virei a cabeça e vislumbrei, em cima da arca, atrás da caixinha de
pó de arroz, um rato. Ele parou justo ao lado do espelho na borda da arca e
ficou me olhando fixamente com seus olhinhos negros nos quais a luz da
lâmpada destilava duas gotas brilhantes de ouro que me flechavam
profundamente. Durante alguns segundos ele me fitou nos olhos com tanta
acuidade que eu sentia o olhar daqueles dois pontos vítreos penetrando-me até
o fundo do cérebro. Parecia estar pensando numa invectiva contra mim ou
somente num reproche. De súbito, porém, a fascinação desmoronou e o rato
pôs-se a correr, desaparecendo atrás da arca. Tinha certeza de que o médico
viera me espionar.

\textls[10]{No mesmo fim de tarde, ao tomar o quinino, minha suposição foi reforçada por
um raciocínio perfeitamente ilógico, embora válido para mim: o quinino era
amargo; por outro lado, o médico vira na cabine o prazer que Clara por vezes
me oferecia; por conseguinte, também para o estabelecimento de um equilíbrio
justo, ele me prescrevera o medicamento mais desagradável que podia existir.
Era capaz de ouvi-lo, roendo o seu juízo para si mesmo: ``Quanto \textit
{maiorrr} o \textit{prazerrr}, mais \textit{amarrrgo} há de \textit
{serrr} o \textit{rrremédio}!''.}

Alguns meses depois da consulta, o médico foi encontrado morto no sótão de sua
casa; dera um tiro na cabeça. Minha primeira pergunta, ao saber da sinistra
notícia, foi:

--- Havia ratos naquele sótão?

Essa certeza era-me necessária.

\textls[10]{Pois, para que o médico morresse de verdade, era absolutamente necessário que
um bando de ratos se arremessasse sobre o cadáver, o carcomesse e dele
extraísse a matéria rateira que o médico lhes tomara emprestado durante a
vida para o exercício de sua existência ilegal de \textit{ser humano}.}


\section{xi} 

\letra{Q}{uando} \textls[-10]{conheci Clara, eu tinha, salvo engano, doze anos. Não importa o
 quão longe eu remexa em minhas recordações: até o recôndito da minha
 infância, encontro-as sempre relacionadas ao conhecimento sexual. Ele se me
 apresenta tão nostálgico e puro como a vivência da noite, do medo ou das
 primeiras amizades; em nada diferente de outras melancolias e outras
 expectativas, como por exemplo a tediosa espera de me tornar \textit
 {grande}, que eu era capaz de comprovar de maneira concreta sempre que dava
 a mão a uma pessoa mais velha, tentando delimitar a diferença de peso e
 tamanho da minha mãozinha, perdida entre os dedos nodosos, na palma enorme
 de quem a apertava.}\looseness=-1

Em nenhum momento da infância ignorei a diferença existente entre homens e
mulheres. Talvez tenha existido um período em que todos os seres vivos
confundiam-se numa única transparência de movimentos e inércias; não guardo,
contudo, nenhuma lembrança exata disso. O segredo sexual sempre foi aparente.
Tão aparente quanto um objeto, uma mesa, uma cadeira.

Porém, ao investigar com atenção minhas mais remotas recordações, sua \textit
{falta de atualidade} é-me revelada pela compreensão errônea do ato sexual.
Imaginava equivocadamente os órgãos femininos e o ato em si como algo mais
pomposo e mais estranho do que como o conheci com Clara. Quaisquer que fossem
as interpretações --- erradas, mas que, com o tempo, tornavam-se cada vez
mais justas ---, pairava, inefável, um ar de mistério e amargor, que aos
poucos se adensava como um quadro que o pintor realiza a partir de um esboço
informe.


\section{xii} 

\letra{V}{ejo-me} \textls[-20]{muito pequeno, vestindo um camisão até os calcanhares,
 chorando desesperado junto à soleira de uma porta, num quintal ensolarado
 cujo portão dá para uma feira deserta, uma feira vespertina, quente e
 triste, com cachorros dormindo esticados e pessoas deitadas à sombra de
 barracas de verduras.}\looseness=-1

\textls[-20]{No ar, um odor pungente de legumes podres, algumas moscas grandes, violetas,
zumbindo ruidosamente ao meu redor, sorvendo as lágrimas que caem sobre
minhas mãos e voando em círculos frenéticos à luz densa e ebuliente do
quintal. Levanto-me e urino com cuidado sobre a poeira. Ávida, a terra
absorve o líquido e, no seu lugar, resta uma mancha escurecida como a sombra
de um objeto inexistente. Esfrego o rosto com o camisão e lambo as lágrimas
dos cantos da boca, saboreando seu gosto salgado. Torno a sentar-me na
soleira e me sinto extremamente infeliz. Levara uma surra.}\looseness=-1

No meu quarto, meu pai acabara de me dar umas palmas no traseiro despido. Não
sei muito bem por quê. Ponho-me a pensar. Eu estava deitado na cama, ao lado
de uma menina da minha idade; haviam-nos colocado ali para dormir, enquanto
nossos pais tinham ido passear. Não percebi quando retornaram e não sei o que
eu estava fazendo exatamente com a menina por debaixo das cobertas. Só sei
que, na hora em que meu pai ergueu de supetão a colcha, ela tinha acabado de
aceitar. Meu pai ficou vermelho de raiva e me bateu. Isso é tudo.

Após ter chorado e secado os olhos ao pé da porta, sob o sol, desenho agora
círculos e linhas com o dedo na poeira, troco de lugar e vou para debaixo da
sombra, fico de pernas cruzadas em cima de uma pedra e começo a me sentir
melhor. Uma menina vem buscar água no quintal e faz girar a roda enferrujada
da bomba. Atento, ouço o rangido da velha roda de ferro, observo a água
jorrando no balde como um suntuoso rabo prateado de cavalo, olho para as
pernas grandes e sujas da menina, bocejo por não ter dormido nada e, vez ou
outra, tento apanhar uma mosca. A vida simples recomeçando depois do choro.
No quintal, o sol continua derramando seu ardor tirânico. É minha primeira
aventura sexual e minha mais antiga recordação de infância.

De agora em diante, os instintos obscuros incham, crescem, deformam-se e
adentram em seus limites naturais. O que deveria ter constituído uma
amplificação e uma fascinação sempre crescente foi para mim uma série de
abdicações e de cruéis reduções à banalidade; a evolução da infância para a
adolescência significou uma contínua decadência do mundo e, à medida que as
coisas se organizavam ao meu redor, seu aspecto inefável desaparecia, como
uma superfície lustrosa que se torna baça.

\textls[15]{Extática, milagrosa, a figura de Walter mantém até hoje o seu brilho
fascinante.}

\textls[-20]{Quando o conheci, ele estava à sombra de uma acácia, sentado num tronco de
árvore, lendo um fascículo de Buffalo Bill.  A luz clara da manhã se deixava
filtrar pelas folhas verdes, cerradas num farfalhar de sombras muito frescas.
Sua roupa não era nada comum: estava vestido com uma túnica bordô com botões
de osso esculpido, calças de camurça e, nos pés sem meias, sandálias
trançadas com finas tiras de couro branco. Por vezes, ao tentar reviver por
um momento a sensação extraordinária daquele encontro, fito longamente a capa
velha e amarelada de um fascículo de Buffalo Bill. A presença real de Walter,
contudo, com sua túnica vermelha na atmosfera esverdeada da sombra da acácia,
tinha um outro impacto.}\looseness=-1

\textls[10]{A sua primeira reação foi colocar-se de pé, dando uma espécie de salto
elástico como o de um animal. Fizemos amizade no ato. Conversamos um pouco e,
de repente, fez-me uma proposta estupefaciente: comer flores de acácia. Era a
primeira vez que eu encontrava alguém que comia flores. Walter subiu na
árvore e, em pouco tempo, reuniu um buquê gigantesco. Em seguida, desceu e me
mostrou como a flor deveria ser delicadamente desprendida da corola para se
sorver só a ponta. Eu também tentei; a flor rompeu-se um pouco sob os dentes
com um estalido muito prazeroso, esparramando pela boca um perfume suave e
refrescante que eu jamais experimentara.}

Por alguns momentos ficamos calados, comendo flores de acácia. De repente, ele
segurou com força a minha mão: 

--- Quer ver a \textit{sede} do nosso bando?

\textls[-20]{Saíam faíscas do olhar de Walter. Fiquei com um pouco de medo.}\looseness=-1

--- Quer ou não quer? --- perguntou-me de novo. 

Hesitei por um segundo. 

--- Quero --- respondi-lhe com uma voz que não era mais minha e com um apetite
    de risco que irrompeu subitamente dentro de mim e que eu bem sentia não
    me pertencer. Walter me levou, segurando-me pela mão, até um terreno
    baldio depois de atravessarmos o portãozinho do fundo do quintal. A grama
    e as ervas daninhas por lá cresciam à vontade. À nossa passagem, as
    urtigas queimavam-me as pernas e, com a mão, eu tinha de afastar os
    caules grossos de cicuta e bardana. No fundo do terreno baldio, chegamos
    perto de um muro em ruínas. Diante dele havia um fosso e um buraco
    profundo. Walter pulou para dentro dele e me chamou para o acompanhar; o
    buraco atravessava o muro e, uma vez ali, entramos numa adega
    abandonada.

\textls[-20]{Os degraus estavam apodrecidos e cheios de grama, as paredes estavam
infiltradas com umidade, e a escuridão à nossa frente era absoluta. Walter
apertava minha mão com força e me puxava com ele. Em pouco tempo, já havíamos
descido dez degraus. Ali paramos.}\looseness=-1

--- Temos que nos deter por aqui --- disse ele ---, não podemos mais avançar.
    Lá no fundo moram criaturas de ferro, com mãos e cabeças de ferro,
    nascidas do interior da terra. Elas estão lá imóveis e, caso nos peguem
    no escuro, nos estrangulam.

Virei a cabeça e, desesperado, olhei para a abertura da adega acima de nós,
com uma luz que vinha de um mundo simples e claro, onde não existiam
criaturas de ferro e onde se viam, a uma grande distância, plantas, pessoas e
casas.

\textls[15]{Walter apareceu de repente com uma tábua, sobre a qual nos sentamos. Ficamos
calados de novo por alguns minutos. O ambiente da adega era gostoso e fresco,
o ar tinha um aroma pesado de umidade; lá eu seria capaz de ficar horas a
fio, isolado, longe das ruas abrasadoras, bem como da cidade triste e
enfadonha. Sentia-me bem, cercado por paredes frias, debaixo da terra que
fervia ao sol. O zumbido inútil da tarde adentrava como um eco distante pela
abertura da adega.}

\textls[20]{--- É para cá que trazemos as meninas que pegamos --- disse Walter.}

Compreendi vagamente sobre o que deveria se tratar. A adega adquiriu uma
atração insuspeitada.

--- E o que vocês fazem com elas?

Walter deu risada.

--- Mas você não sabe? Fazemos o que todos os homens fazem com as mulheres,
    deitamo-nos ao lado delas e\ldots{} com a pena\ldots{}

\textls[20]{--- Com a pena? Que tipo de pena? O que vocês fazem com as meninas?}\looseness=-1

Walter deu risada de novo.

\textls[-20]{--- Quantos anos você tem? Você não sabe o que os homens fazem com as
    mulheres? Você não tem uma pena? Olhe aqui a minha.}\looseness=-1

Tirou do bolso da túnica uma pequena pena preta de pássaro.

\textls[5]{Naquele momento, senti-me abarcado por uma daquelas minhas \textit{crises}. É
possível que, se Walter não houvesse tirado a pena do bolso, eu tivesse
suportado até o fim aquele ar de isolamento absoluto e desolador da adega.
Mas, numa instante, aquele isolamento ganhou um significado doloroso e
profundo. Só agora eu percebia o quanto a adega ficava longe da cidade e de
suas ruas empoeiradas. Era como se eu mesmo houvesse me distanciado de mim,
na solidão de uma profundeza subterrânea por debaixo de um dia qualquer de
verão. A pena preta e brilhante que Walter me mostrava significava que nada
mais existia no meu mundo conhecido. Tudo adentrava num desvanecimento em que
ela luzia estranhamente, no meio daquele espaço misterioso com uma relva
úmida, naquela escuridão que aspirava a luz como uma boca fria, ávida e
escancarada.}\looseness=-1

--- Ei, o que é que você tem? --- perguntou Walter. --- Deixe eu lhe dizer
    como é que a gente faz com a pena\ldots{}

\textls[-23]{O céu lá fora, visto pela abertura da adega, tornava-se cada vez mais branco e
vaporoso. As palavras ricocheteavam pelas paredes e me invadiam molemente
como se atravessassem uma criatura fluida.}\looseness=-1

Walter continuava falando comigo. Ele estava porém tão longe de mim e tão
aéreo que parecia apenas um clarão no escuro, uma mancha de neblina
agitando-se na sombra.

--- Primeiro você acaricia a menina com a pena --- ouvi como se dentro de um
    sonho --- e, depois, sempre com a pena, acaricia a você mesmo\ldots
    {} Você tem que saber dessas coisas\ldots{}

De repente, Walter se aproximou de mim e começou a me chacoalhar para me
arrancar do sono. Fui me recobrando aos poucos. Quando abri completamente os
olhos, Walter estava inclinado sobre o meu púbis, com a boca grudada com
força ao sexo. Era-me impossível compreender o que estava acontecendo.

Walter se levantou.

--- Está vendo, isso lhe fez bem\ldots{} É assim que os índios guerreiros
    acordam os feridos; no nosso bando, conhecemos todos as mágicas e curas
    indígenas.

Acordei zonzo e exausto. Walter fugiu correndo e desapareceu. Subi os degraus
com cuidado.

\textls[-20]{Nos dias seguintes, procurei-o por toda a parte --- em vão. Só me restava
encontrá-lo na adega, mas, quando fui até lá, o terreno baldio me pareceu
completamente diferente. Só se viam montanhas de lixo, animais mortos e
coisas em putrefação que produziam um cheiro horrível sob o calor do sol. Com
Walter, eu não vira nada disso. Desisti de ir até a adega e, assim, nunca
mais me encontrei com Walter.}\looseness=-1

\pagebreak


\section{xiii} 

\letra{A}{rranjei} \textls[-5]{uma pena que eu mantinha no bolso no maior segredo,
 embrulhada numa folha de jornal. Às vezes me parecia que eu mesmo inventara
 toda aquela história da pena, e que Walter jamais existira. De vez em quando
 eu desembrulhava a pena do jornal e a fitava longamente: seu mistério era
 impenetrável. Roçava meu rosto com seu brilho mole e sedoso, e essa carícia
 me causava arrepios como se uma pessoa invisível, porém real, me tocasse com
 a ponta dos dedos. A primeira vez que a utilizei foi numa noite serena, em
 circunstâncias extraordinárias.}\looseness=-1

Eu gostava de ficar fora até tarde. Naquela noite, caía uma tempestade pesada
e aflitiva. Todo o calor do dia se condensara numa atmosfera esmagadora,
debaixo de um céu negro, arranhado por relâmpagos. Estava sentado junto à
porta de uma casa, admirando o jogo das luzes elétricas nos muros da ruela. O
vento fazia a lâmpada que iluminava a rua balançar, e os círculos
concêntricos do globo, sombreando as paredes, oscilavam como uma água agitada
dentro de uma vasilha. Compridas echarpes de poeira se formavam a partir do
chão e se erguiam em serpentinas.

\textls[10]{De repente, numa lufada de vento, tive a impressão de que uma estátua branca
de mármore se alçava no ar. Havia naquele momento uma certeza, incontrolável
como toda certeza. O bloco de pedra branca se distanciava rapidamente para
cima, subindo oblíquo como um balão que escapa da mão de uma criança. Em
poucos instantes, a estátua se transformou numa simples mancha branca no céu,
do tamanho do meu punho. Agora eu podia ver distintamente duas pessoas
brancas, de mãos dadas, deslizando pelo céu como dois esquiadores.}

Naquele momento, uma menina surgiu na minha frente. Eu devo ter ficado
boquiaberto e de olhos arregalados para cima, pois ela me perguntou,
admirada, o que eu via no céu.

\textls[15]{--- Olhe\ldots{} uma estátua voadora\ldots{} olhe rápido\ldots{} logo vai se
    desfazer\ldots{}}

\textls[-25]{Após olhar atentamente, franzindo as sobrancelhas, a menina confessou não
estar vendo nada. Era a filha de vizinhos, gorducha, de bochechas coradas
como borrachas e mãos sempre úmidas. Até aquele fim de tarde eu só havia
trocado umas poucas palavras com ela. Em pé diante de mim, a menina de
repente começou a dar risada:}\looseness=-1

--- Eu sei por que você me enganou\ldots{} disse ela\ldots{} sei muito bem o
    que você quer\ldots{}

Começou a se distanciar de mim, pulando numa só perna. Levantei-me e corri
atrás dela; chamei-a para dentro de uma passagem obscura e ela veio sem se
opor. Ali eu levantei seu vestido. Ela se deixou tocar docilmente,
segurando-se nos meus ombros. Talvez estivesse mais surpresa com o que
acontecia do que consciente do despudor do nosso ato.

\textls[-15]{A consequência mais surpreendente daquele evento ocorreu alguns dias depois,
no meio de uma praça. Uns pedreiros despejavam cal num caixote. Estava
olhando para a cal que fervia quando, de repente, ouvi alguém me chamando
pelo nome aos berros e gritando: ``Então quer dizer que com a pena\ldots
{} com a pena\ldots{} não é?''. Era um rapaz de uns vinte anos, robusto,
ruivo e insuportável. Acho que morava numa das casas da passagem escura. Vi-o
gritando na minha direção por um só momento, do outro lado do caixote, saindo
fantasmagoricamente de entre os vapores da cal como uma aparição infernal que
falava em meio a raios e labaredas de fogo.}\looseness=-1

\textls[-10]{Talvez ele me houvesse dito outra coisa, e minha imaginação dotara suas
palavras com o significado daquilo que me preocupava naqueles dias; não podia
acreditar que ele realmente vira alguma coisa no breu compacto da passagem.
Pensando melhor, porém, veio-me à mente que talvez a passagem não fosse tão
obscura como me parecera e que tudo houvesse sido visível (talvez tenhamos
estado até debaixo da luz)\ldots{} várias suposições que reforçavam minha
convicção de que, durante o ato sexual, eu era possuído por um sonho que me
turvava a visão e os sentidos. Impus-me mais prudência. Quem sabe de que
aberrações à luz do dia eu seria capaz, sob o poder da excitação e por ele
possuído como por um sono pesado em que me movimentava inconscientemente?}\looseness=-1


\section{xiv} 

\letra{U}{ma} \textls[-10]{lembrança estreitamente ligada à pena vem-me à mente: um
 livrinho preto muito inquietante. Eu o encontrara, no meio de outros, em
 cima de uma mesa, e o folheara com grande interesse. Era um romance banal,
 Frida, de André Theuriet, numa edição ricamente ilustrada com vários
 desenhos. Em cada um deles havia a imagem de um garoto de cachos loiros, com
 roupa de veludo e uma menina gorducha com vestidinhos de babados. O garoto
 parecia-se com Walter. As crianças nos desenhos ora apareciam juntas, ora
 separadas; compreendia-se bem que eles se encontravam sobretudo em
 esconderijos num parque ou debaixo de muros em ruínas. O que faziam juntos?
 Era o que eu gostaria de saber. Teria o garoto, como eu, uma pena guardada
 no bolso? Pelos desenhos isso não se entrevia e, ademais, nem tive tempo de
 ler o romance. Alguns dias depois, o livrinho preto desapareceu sem deixar
 vestígios. Comecei a procurá-lo por toda a parte. Perguntei nas livrarias,
 mas parece que ninguém ouvira falar dele. Devia ser um livro envolto em
 muitos segredos, visto que não podia ser encontrado em lugar algum.}\looseness=-1

Certo dia, tomei coragem e entrei na sala de uma biblioteca pública. Um senhor
alto, pálido e de óculos que lhe tremiam levemente estava sentado numa
cadeira no fundo da sala e me observou entrando. Não podia mais voltar atrás.
Eu tinha de avançar até a mesa e, ali, pronunciar com clareza, diante do
senhor míope, a sensacional palavra \textit{Fri-da}, como uma confissão de
todos os meus vícios ocultos. Aproximei-me da escrivaninha e murmurei, com
voz fraca, o nome do livro. Os óculos do bibliotecário começaram a tremer
ainda mais por cima do nariz, fechou os olhos como se consultasse sua memória
e me informou ``não ter ouvido falar'' dele. O tremor dos óculos pareceu-me,
contudo, trair uma agitação interior; agora, eu tinha certeza de que Frida
abrangia as mais misteriosas e espetaculares revelações.

Muitos anos mais tarde, reencontrei o livro na estante de uma livraria. Não
era mais aquele livrinho encadernado em tecido preto, mas uma brochura
humilde e miserável de capa amarelada. Num instante quis comprá-lo, mas mudei
de ideia e o coloquei de volta na estante. De maneira que mantenho até hoje
intacta, dentro de mim, a imagem do livrinho preto em que se encerra um pouco
do perfume autêntico da minha infância.


\section{xv} 

\letra{E}{m} objetos pequenos e insignificantes: uma pena preta de pássaro, um
 livrinho banal, uma fotografia velha com personagens frágeis e anacrônicos,
 que parecem sofrer de uma grave doença interna, um singelo cinzeiro de
 faiança verde que imita a forma de uma folha de carvalho, sempre cheirando a
 cinza velha; na simples e elementar lembrança dos óculos de lentes grossas
 do velho Samuel Weber: em tais ornamentos miúdos e objetos domésticos,
 reencontro toda a melancolia da minha infância e aquela nostalgia essencial
 da inutilidade do mundo, que me envolvia por todos os lados como um mar de
 ondas empedernidas. A matéria bruta, mediante sua massa profunda e pesada de
 poeira, pedra, céu ou água, ou mediante sua forma mais incompreensível ---
 flores de papel, espelhos, bolinhas de gude com suas enigmáticas espirais
 interiores, ou estátuas coloridas ---, sempre me manteve fechado numa prisão
 que dolorosamente se chocava contra suas paredes e perpetuava dentro de mim,
 sem qualquer sentido, a estranha aventura de ser homem.

Para onde quer que eu direcionasse o pensamento, ele sempre esbarrava com
objetos e imobilidades que se apresentavam como muros diante dos quais eu
tinha de cair ajoelhado. Pensava, aterrorizado com sua diversidade, nas
infinitas formas da matéria, revolvendo-me noites inteiras, agitado com a
série de objetos que desfilavam sem fim na minha memória, como escadas
mecânicas a desdobrar milhares e milhares de degraus.

Por vezes, no intuito de bloquear a onda de coisas e de cores que me inundava
o cérebro, eu imaginava a evolução de um só contorno, ou de um só objeto.

Imaginava, por exemplo --- como um repertório correto do mundo ---, o
encadeamento de todas as sombras sobre a terra, um mundo estranho e
fantástico dormindo aos pés da vida. Um homem negro, deitado na relva como um
lenço, com pernas delgadas escorridas como água, com braços de ferro escuro,
passeando por entre árvores horizontais com ramos chorosos.

\textls[10]{As sombras dos navios deslizando pelos mares, sombras instáveis e aquáticas
como tristezas que vêm e que vão, escorregando por sobre a espuma.}

As sombras das aves que voam, como pássaros pretos vindos das profundezas da
terra, de um aquário sombrio.

E a sombra solitária, perdida em algum lugar do espaço, do nosso redondo
planeta\ldots{}

Noutra ocasião pus-me a pensar nas cavernas e buracos, nos desfiladeiros das
montanhas, com sua altura vertiginosa, até a caverna elástica e quente, a
inefável caverna sexual. Não sei onde eu arranjara uma lanterninha elétrica
com a qual, durante a noite, na cama, enlouquecido pela falta de sono e pelos
objetos que, incessantes, preenchiam o quarto, eu me metia debaixo das
cobertas e investigava, com um cuidado tenso, como uma pesquisa íntima sem
objetivo, as pregas do lençol e os pequenos vales que se formavam entre elas.
Eu tinha necessidade de uma tal ocupação precisa e miúda para conseguir me
tranquilizar um pouco. Certa vez meu pai me encontrou, à meia-noite, numa
exploração por debaixo dos travesseiros e tomou de mim a lanterna. Mas não me
disse nada, nem brigou comigo. Creio que a descoberta lhe foi tão estranha
que ele foi incapaz de encontrar vocabulário ou moralidade que pudesse ser
aplicada ao caso.

\textls[-10]{Alguns anos mais tarde, encontrei num livro de anatomia a fotografia de um
modelo de cera do interior da orelha. Todos os canais, seios e buracos eram
de matéria plena, formando sua imagem positiva. Essa fotografia me
impressionou tanto que quase desmaiei. No mesmo instante percebi que o mundo
poderia existir numa realidade mais verdadeira, numa estrutura positiva das
suas cavernas, de maneira que tudo o que é furado se tornasse cheio, e os
relevos atuais se transformassem em vácuos de forma idêntica, sem qualquer
conteúdo, como aqueles fósseis delicados e bizarros que reproduzem na pedra
os vestígios de uma concha ou de uma folha macerada ao longo das eras,
deixando apenas profundamente esculpidas as impressões delicadas de seu
contorno.}\looseness=-1

\textls[-10]{Num mundo como esse, as pessoas cessariam de ser excrescências multicoloridas
e carnosas, cheias de órgãos complexos e putrescíveis, tornando-se vácuos
puros, flutuantes como bolhas de ar dentro d'água, atravessando a matéria
quente e mole do universo pleno. Essa era, aliás, a sensação íntima e
dolorosa que eu muitas vezes tinha na adolescência sempre que, durante minhas
longas e infindáveis vagabundagens, subitamente despertava em meio a um
terrível isolamento, como se as pessoas e as casas em derredor de repente
houvessem se amalgamado numa massa compacta e uniforme de uma única matéria,
na qual eu existia como um simples vácuo sem finalidade, locomovendo-me para
lá e para cá.}\looseness=-1


\section{xvi} 

\letra{E}{m} conjunto, os objetos formavam cenários. A impressão de espetáculo
 me perseguia por toda a parte, com o sentimento de que tudo se desenrolava
 em meio a uma representação factícia e triste. Embora escapasse por vezes da
 visão enfadonha e opaca de um mundo incolor, deparava-me sempre com o seu
 aspecto teatral, enfático e antiquado.

\textls[7]{No contexto desse espetáculo geral, havia outros espetáculos assombrosos que
 me atraíam mais, posto que a sua artificialidade e os atores que neles se
 apresentavam pareciam realmente compreender o sentido de mistificação do
 mundo. Só eles sabiam que, num universo espetacular e decorativo, a vida
 deveria ser representada com falsidade e ornamentação. Tais espetáculos eram
 o cinema e o \textit{panopticum}.} 

\textls[-10]{Ó, sala de cinema B., longa e soturna como um submarino naufragado! As portas
da entrada eram recobertas de espelhos de cristal que refletiam parte da rua.
De maneira que, já na entrada, havia um espetáculo gratuito, antes do da
sala, constituído por essa tela surpreendente em que a rua surgia numa luz de
sonho, esverdeada, com pessoas e charretes movendo-se sonâmbulas em suas
águas.}\looseness=-1

\textls[-10]{Na sala reinava um cheiro fétido e ácido de banheiro público. O chão era
cimentado e as poltronas, ao se mexerem, soltavam rangidos que lembravam
breves gritos de desespero. Diante da tela, uma galeria de vagabundos e
cafajestes devoravam sementes e comentavam o filme em voz alta. As legendas
eram escandidas por dezenas de vozes em uníssono, como se fossem exercícios
de uma escola para adultos. Bem debaixo da tela agitava-se um trio composto
por pianista, violinista e um velho judeu que tocava seu contrabaixo sem
parar. O velho tinha ainda a função de emitir diferentes ruídos que deveriam
corresponder às ações que transcorriam na tela. Ele costumava berrar
``cocorocó'' sempre que aparecia o galo da empresa cinematográfica no início
do filme e ainda me lembro de que, uma vez, quando a vida de Jesus estava
sendo representada, ele se pôs a bater a caixa do contrabaixo freneticamente
com o arco no momento da ressurreição, a fim de imitar os relâmpagos
celestiais.}\looseness=-1

\textls[10]{Eu vivia os episódios do filme com uma intensidade extraordinária,
integrando-me à ação como um verdadeiro personagem da trama. Muitas vezes
acontecia de o filme absorver tanto a minha atenção que eu me flagrava
passeando pelos parques da tela, apoiado na balaustrada das varandas
italianas por onde evoluía, pateticamente, Francesca Bertini,  com os cabelos
soltos e os braços agitados como echarpes.}

\textls[-10]{Definitivamente, não há nenhuma diferença bem estabelecida entre a nossa
pessoa real e as nossas diferentes personagens interiores imaginárias. Ao
acender das luzes durante a pausa, a sala revelava um ar que vinha de longe.
Havia no ambiente algo de precário e artificial, muito mais incerto e efêmero
do que o espetáculo na tela. Eu fechava os olhos e esperava até o ruído
mecânico do aparelho anunciar-me que o filme continuaria; reencontrava então
a sala na escuridão e todas as pessoas ao meu redor, indiretamente iluminadas
pela tela, pálidas e transfiguradas como uma galeria de estátuas de mármore
num museu ao luar da meia-noite.}\looseness=-1

\textls[7]{Num determinado momento, o cinema começou a pegar fogo. A película se soltou e
ardeu imediatamente, de modo que, durante alguns segundos, as chamas do
incêndio apareceram na tela como uma espécie de aviso condescendente de que o
cinema estava em chamas e, ao mesmo tempo, como uma continuação lógica do
rolo do aparelho no sentido de apresentar as \textit{atualidades} e cuja
missão ele assim cumpria, num excesso de zelo, representando a derradeira e a
mais palpitante delas: a do seu próprio incêndio. Eclodiram de todas as
partes clamores e gritos curtos de ``Fogo! Fogo!'' como disparos de revólver.
O alarido que prorrompeu da sala foi de tal magnitude que parecia que os
espectadores, até então calados na escuridão, não haviam feito outra coisa
senão amontoar dentro de si berros e brados, como plácidos e inofensivos
condensadores que explodem assim que o limite de sua capacidade de carga é
ultrapassada.}

Em poucos minutos, e antes que metade da sala houvesse sido evacuada,
o \textit{incêndio} foi apagado. Os espectadores porém continuavam gritando
como se tivessem de gastar, uma vez desencadeada, uma determinada quantidade
de energia. Uma senhorita, com o rosto empoado como gesso, berrava com
estridência olhando fixamente para os meus olhos, sem que se movesse ou
fizesse qualquer passo na direção da saída. Um cafajeste musculoso,
convencido da utilidade de suas forças em tais circunstâncias, mas sem saber
para onde direcioná-las, agarrava uma a uma as cadeiras de madeira e as
arremessava contra a tela. Ouviu-se de repente um grande estrondo: uma delas
atingira em cheio o contrabaixo do velho músico. O cinema era cheio de
surpresas.


\section{xvii} 

\letra{N}{o} \textls[-5]{verão, eu entrava cedo na matinê e saía tarde, ao anoitecer. A
 luz do lado de fora se modificara, o dia se extinguia. Constatava que, na
 minha ausência, ocorrera no mundo um acontecimento imenso e essencial, uma
 espécie de triste obrigação de sempre continuar --- como o anoitecer, por
 exemplo --- um trabalho rotineiro, diáfano e espetacular. Entrava dessa
 maneira, de novo, no seio de uma certeza que, através de seu rigor diário,
 parecia-me de uma melancolia sem fim. Em tal mundo, submetido aos efeitos
 mais teatrais e obrigado, a cada entardecer, a representar um pôr do sol
 correto, as pessoas ao meu redor pareciam pobres criaturas dignas de pena
 pela seriedade com que continuamente se ocupavam, acreditando, ingênuas,
 naquilo que faziam e sentiam. Havia uma única criatura na cidade que
 compreendia essas coisas e pela qual eu nutria uma admiração plena de
 respeito: a louca da cidade. Só ela, em meio às pessoas rígidas e recheadas
 de preconceitos e convenções até a ponta dos cabelos, só ela mantivera a
 liberdade de gritar e de dançar na rua quando quisesse. Corroída de
 imundície, ela andava esfarrapada pelas ruas, desdentada, com o cabelo ruivo
 desgrenhado, segurando nos braços, com ternura materna, um cofrinho velho
 cheio de cascas de pão e diversos objetos retirados do lixo.}\looseness=-1

\textls[-15]{Exibia o sexo aos transeuntes com um gesto que, se fosse
 utilizado com outro objetivo, seria considerado \textit{pleno de estilo e
 elegância}. Que esplêndido, que sublime é ser louco!, dizia para mim mesmo,
 constatando, com um inimaginável desgosto, quantos costumes familiares
 arraigados e estúpidos e que esmagadora educação racional me separava da
 liberdade extrema da vida de um louco.}\looseness=-1

Quem nunca foi tomado por esse sentimento está condenado a jamais sentir a
verdadeira amplitude do mundo.


\section{xviii} 

\letra{A}{}\textls[-10]{impressão geral e basicamente espetacular se transformava em
 autêntico terror assim que eu entrava no \textit{panopticum} com figuras de
 cera.  Tratava-se de pavor misturado a uma vaga espécie de prazer e, de
 qualquer modo, àquela bizarra sensação --- que por vezes nos toma --- de já
 termos vivido num determinado cenário do passado. Se um dia porventura
 surgisse em mim o instinto de um objetivo na vida e se esse ímpeto estivesse
 ligado a algo realmente profundo, essencial e irremediável dentro de mim,
 creio que meu corpo deveria então se transformar numa estátua de cera de
 um \textit{panopticum} e minha vida, numa simples e interminável
 contemplação das vitrines do cosmorama.}\looseness=-1

\textls[-7]{À luz triste dos candeeiros a carbureto, sentia realmente viver minha própria
vida de maneira única e inimitável. Todas as minhas ações cotidianas poderiam
ser embaralhadas como num jogo de cartas, eu não fazia questão de nenhuma
delas; a irresponsabilidade das pessoas pelos seus mais conscientes gestos
era fato de evidência gritante. Que importância havia se eu ou outra pessoa
os cometesse, a partir do momento em que a diversidade do mundo os engolia na
uniforme monotonia de sempre? No \textit{panopticum}, e só no \textit
{panopticum}, não havia nenhuma contradição entre o que eu fazia e o que
acontecia. As personagens de cera constituíam a única coisa autêntica do
mundo; só elas falsificavam a vida ostensivamente, participando, com sua
estranha e artificial imobilidade, da verdadeira atmosfera do mundo. O
uniforme crivado de balas e manchado de sangue de um certo arquiduque
austríaco, de feições tristes e amareladas, era infinitamente mais trágico
que qualquer morte verdadeira. Numa caixa de cristal jazia uma mulher vestida
em rendas negras, com uma face pálida e lustrosa. Uma rosa assombrosamente
rubra estava fixada entre os seios, a peruca loira começava a descolar na
orla da testa, enquanto o róseo do arrebique palpitava nas narinas e os olhos
azuis, límpidos como vidro, fitavam-me, imóveis. Não era possível que a
mulher de cera não tivesse um significado profundo e inquietante,
desconhecido por todos. Quanto mais a contemplava, mais clara parecia se
tornar a compreensão que eu tinha dela, persistindo dentro de mim vagamente,
como uma palavra que eu quisesse lembrar mas da qual eu podia captar apenas
um ritmo longínquo.}\looseness=-1


\section{xix} 

\letra{S}{empre} \textls[-10]{tive uma estranha atração por acessórios femininos e por
 objetos artificiais com ornamentos baratos. Um amigo meu colecionava as mais
 diversas coisas que encontrava. Numa caixa de mogno, ele escondia uma faixa
 de seda preta com uma renda finíssima nas bordas, costuradas com um paetê
 cintilante de \textit{strass}. Havia sido decerto cortada de algum antigo
 vestido de baile; em alguns pontos, a seda começara a embolorar. Para que me
 permitisse vê-la, eu lhe dava selos e até mesmo dinheiro. Ele então me
 levava até uma salinha antiquada, enquanto seus pais dormiam, e me mostrava.
 Eu segurava o pedaço de seda na mão, emudecido de estupefação e de prazer.
 Meu amigo ficava na soleira da porta, de vigia; depois de alguns minutos ele
 voltava, pegava a seda da minha mão, guardava-a na caixa e dizia: ``Pronto,
 acabou, chega'', exatamente como Clara às vezes me dizia, nos momentos em
 que as tergiversações da cabine se estendiam demais.}\looseness=-1

\textls[-15]{Outro objeto que me inquietou muito quando o vi pela primeira vez foi um anel
cigano. Creio que seja o anel mais fantástico que um homem pode inventar para
adornar a mão de uma mulher.}\looseness=-1

Extraordinários ornamentos de mascarada de pássaros, animais e flores, todos
no intuito de desempenhar um papel sexual, a cauda estilizada e ultramoderna
da ave do paraíso, as penas oxidadas do pavão, a renda histérica das pétalas
das petúnias, o azul inverossímil das faces do mandril são apenas pálidas
tentativas de ornamentação sexual se comparados ao estonteante anel cigano.
Era um objeto sublime de folha-de-flandres, fino, grotesco e hediondo.
Sobretudo hediondo: atacava o amor nas regiões mais básicas e sombrias. Um
verdadeiro berro sexual.

\textls[10]{É claro que o artista que o concebeu inspirou-se das visões do \textit
{panopticum}. A pedra do anel, de um simples pedaço de vidro derretido até a
grossura de uma lente, lembrava perfeitamente as lupas dos cosmoramas pelas
quais contemplávamos, extremamente aumentados, navios naufragados, batalhas
contra os turcos e regicídios. Podia-se ver no anel um buquê de flores
cinzelado em latão ou chumbo, colorido com todas as tintas gritantes dos
quadros do \textit{panopticum}.}

O violeta dos mortos por asfixia ao lado do vermelho pornográfico das
cintas-ligas femininas, o palor plúmbeo das ondas enfurecidas no interior de
uma luz macabra, como a semiobscuridade dos jazigos com tampas de vidro. Tudo
era rodeado por folhinhas de cobre e sinais misteriosos. Alucinante.

\textls[-15]{Impressiona-me igualmente tudo o que é imitação. Flores artificiais, por
exemplo, e coroas mortuárias, sobretudo coroas mortuárias, esquecidas e
empoeiradas em suas caixas ovais de vidro na igreja do cemitério, cingindo
com uma delicadeza passadista velhos nomes anônimos, mergulhados numa
eternidade sem ressonância.}\looseness=-1

Fotografias recortadas com as quais as crianças brincam, estátuas baratas de
quermesse. Com o tempo, essas estátuas perdem a cabeça ou uma das mãos até
que sua proprietária, ao consertá-la, rodeia seu pescoço delicado com
escrófulas brancas de gesso. O bronze do resto da estátua impregna-se então
do significado de um sofrimento trágico mas nobre. E há ainda os Cristos em
tamanho natural das igrejas católicas. Os vitrais arremessam sobre o altar os
últimos reflexos de um crepúsculo vermelho de sol, enquanto os lírios, nesse
momento do dia, exalam aos pés do Cristo a plenitude do seu perfume lúgubre e
pesado. Nessa atmosfera plena de sangue aéreo e vertigem odorante, um jovem
pálido toca ao órgão os últimos acordes de uma melodia desesperada.

\textls[10]{Tudo isso emigrou do \textit{panopticum} para a vida. No cosmorama da
quermesse reconheço o lugar comum a todas essas nostalgias espalhadas pelo
mundo, as quais, uma vez reunidas, constituem a sua própria essência.}

Permanece vivo dentro de mim um único e supremo desejo: assistir ao incêndio
de um \textit{panopticum}; contemplar o derretimento lento e escabroso dos
corpos de cera, observar petrificado as belas pernas amareladas da noiva da
caixa de vidro contorcendo-se no ar, no que lhe prenderia entre as coxas uma
chama verdadeira que lhe queimasse o sexo.


\section{xx} 

\letra{A}{lém} do \textit{panopticum}, a quermesse do mês de agosto trazia-me
 muitas outras tristezas e exaltações. Seu amplo espetáculo se inchava como
 uma sinfonia, desde o prelúdio dos cosmoramas isolados, que chegavam bem
 antes de todos e davam o tom geral da quermesse como notas desgarradas e
 prolongadas que anunciam, no início das peças de concerto, o tema de toda a
 composição, até o final grandioso, rimbombando de berros, estalidos e
 fanfarras, no grande dia, a que se segue o imenso silêncio do campo que
 ficou deserto.

\textls[10]{Os cosmoramas que chegavam mais cedo compreendiam, em essência, a quermesse
inteira, representando-a com exatidão. Era suficiente que o primeiro deles se
instalasse para que todo o colorido, todo o brilho e todo o cheiro de
carbureto da quermesse cobrisse a cidade.}

\textls[10]{Da multiplicidade dos sons cotidianos desprendia-se de repente um zumbido que
não era nem o rangido de uma lata, nem o tinido distante de um molho de
chaves, nem o ronco de um motor: um som fácil de se reconhecer entre milhares
de outros e que pertencia à \textit{roda da sorte}.}

\textls[10]{Na escuridão da avenida se acendia, ao entardecer, um diadema de cintilações
coloridas, como uma primeira constelação, a da Terra. Outras logo eram acesas
em seguida e a avenida se transformava num corredor luminoso, ao longo do
qual eu passeava empedernido, como aquele garoto da minha idade que eu vira
numa edição ilustrada de Júlio Verne, contemplando pela janela de um
submarino, nas trevas suboceânicas, maravilhosas e misteriosas
fosforescências marinhas.}

\textls[15]{Poucos dias depois, a quermesse já estava instalada. O semicírculo das
barracas se organizava, se completava e de repente se tornava definitivo.}

Zonas bem estabelecidas a dividiam em regiões de sombras e de luzes --- as
mesmas todo ano. Primeiro, havia a série de restaurantes com dezenas de
guirlandas de lâmpadas coloridas e, em seguida, os cosmoramas com
monstruosidades, a fachada ébria de luz do circo e, por fim, as humildes e
obscuras barracas dos fotógrafos. As pessoas passeavam em círculo, passando,
sucessivamente, por luminosidades máximas e regiões entrevadas, como a lua no
desenho do meu livro de geografia, que percorria alternadamente zonas
tipográficas brancas e negras.

Eu costumava entrar nos cosmoramas pequenos e mal iluminados, com poucos
artistas, que nem telhado tinham, onde meu pai podia negociar com o diretor,
à entrada, um preço coletivo e reduzido para toda a nossa família, que era
numerosa.

\textls[10]{A representação inteira se revestia de um aspecto de improvisação e
insegurança. O vento noturno soprava frio pelas cabeças dos espectadores e,
acima de nós, todas as estrelas cintilavam no céu. Estávamos perdidos numa
barraca de quermesse, desnorteados pelo caos da noite num ponto espacial
ínfimo de um determinado planeta. Naquele ponto, naquele planeta, pessoas e
cachorros representavam em cima de um palco; as pessoas atiravam ao ar
diversos objetos para os agarrar em seguida, enquanto os cachorros saltavam
através de aros e andavam com duas patas. Onde acontecia tudo isso? O céu,
por cima de nós, parecia ainda mais imenso\ldots{}}\looseness=-1

\textls[-10]{Certa vez, numa daquelas barracas miseráveis, o artista anunciou ao público um
prêmio de cinco mil leus}\looseness=-1\footnote{Moeda romena. {[}\textsc{n.\,t.}{]}} \textls[-10]{para
quem imitasse o sensacional e facílimo número que ele viria a apresentar.
Éramos poucos sentados nos bancos. Um senhor gordo, já há muito tempo famoso
na praça por sua avareza, siderado pela inesperada possibilidade de ganhar
uma grande quantia num simples cosmorama de quermesse, mudou bruscamente de
lugar para ficar mais próximo do palco, decidido a acompanhar com a máxima
atenção cada gesto do artista a fim de reproduzi-lo e embolsar o prêmio.}\looseness=-1

Seguiram-se alguns instantes de um silêncio atroz.

O artista aproximou-se da ribalta: 

--- Senhores --- disse ele com uma voz rouca ---, trata-se de expulsar pela
    garganta a fumaça de um cigarro. 

\textls[15]{Ele acendeu um cigarro e, ao retirar a mão da gola, onde a mantinha todo o
 tempo, soltou um fiozinho de fumaça azulada através do orifício de uma
 laringe artificial, provavelmente consequência de uma cirurgia.}

\textls[10]{O senhor do banco da frente ficou aturdido e embaraçado; enrubesceu até às
 orelhas e, retornando ao seu lugar, murmurou bem alto entre os dentes:} 

--- Ah, claro, nem é de se admirar, com uma maquininha daquelas no pescoço!

Imperturbável, o artista respondeu de cima do palco:

--- Pois não, é só tentar --- disposto talvez a realmente conceder o prêmio a
    um colega de sofrimento\ldots{}

Em tais barracas, para ganhar o pão, velhos pálidos e macilentos engoliam
pedras e sabonetes diante do público, jovens moças contorciam seus corpos e
crianças anêmicas e esquálidas, após deixar de lado a espiga de milho que
roíam, subiam no palco e dançavam agitando guizos costurados nas calças.

\textls[7]{De dia, imediatamente após a hora do almoço, debaixo do calor ardente do sol,
a desolação da quermesse era infinda. A imobilidade dos cavalinhos de pau, de
olhos esbugalhados e crina bronzeada, embebia-se de não sei qual terrível
melancolia de vida petrificada. Das barracas vinha um cheiro forte de comida,
enquanto um único realejo, ao longe, insistia em deixar escorrer uma valsa
asmática, de cujo caos irrompia, volta e meia, uma nota metálica assobiada,
como um jato brusco, alto e delgado, vindo da massa de um tanque d'água.}

\textls[10]{Eu costumava ficar horas a fio diante das barracas dos fotógrafos,
contemplando pessoas desconhecidas, em grupos ou sozinhas, empedernidas e
sorridentes, diante das paisagens cinzentas com cascatas e montanhas
longínquas. Todas as personagens, devido ao cenário comum, pareciam membros
de uma mesma família que haviam excursionado ao mesmo lugar pitoresco, onde
se deixaram fotografar um depois do outro.}

Uma vez, num desses escaparates, deparei-me com minha própria fotografia. Esse
encontro brusco comigo mesmo, imobilizado num gesto fixo, ali, na margem da
quermesse, provocou em mim um efeito deprimente.

\textls[5]{Antes de chegar à minha cidade, ela com certeza viajara por outros lugares,
por mim desconhecidos. Num instante, tive a sensação de existir apenas
naquela fotografia. Essa inversão de posições mentais me acontecia com
frequência nas mais diversas circunstâncias. Ela surgia de supetão e
transformava-me repentinamente o corpo interior. Num acidente de rua, por
exemplo, durante alguns minutos eu observava o acontecimento como um
espectador qualquer. De repente, contudo, toda a perspectiva se
modificava --- como naquela brincadeira em que devemos reconhecer na pintura
da parede um animal estranho qualquer, que em outro dia não encontramos mais
por vermos no seu lugar, formado pelos mesmos elementos decorativos, uma
estátua, uma mulher ou uma paisagem --- e no acidente de rua, embora tudo
permanecesse intacto, eu via, de repente, as pessoas e todas as coisas ao meu
redor, do ponto de vista do ferido, como se eu estivesse estendido no chão e
visse tudo a partir da minha posição de acidentado, de baixo para cima, do
centro para a periferia, com a nítida sensação do sangue escorrendo de mim.
Da mesma maneira, sem qualquer esforço, como uma consequência lógica do
simples fato de ver, eu me imaginava, na sala de cinema, vivendo intimamente
as cenas projetadas sobre a tela, assim como me vi, diante da barraca do
fotógrafo, no lugar daquele que me fitava, fixo, no cartão.}\looseness=-1

Toda a minha própria vida, a daquele que estava em carne e sangue do lado de
lá do escaparate, de repente se me apresentou indiferente e sem significado,
da mesma maneira como, para as pessoas vivas deste lado do vidro, pareciam
absurdas as viagens por cidades desconhecidas do eu fotográfico.

\textls[-10]{Da mesma forma que a fotografia que me representava passeava de um lugar para
o outro, contemplando pelo vidro sujo e empoeirado perspectivas sempre novas,
eu mesmo, do lado de cá do vidro, sempre levava para passear meu personagem
semelhante a vários lugares, eternamente observando coisas novas e
eternamente nada entendendo delas. O fato de que eu me movia, que estava
vivo, só podia ser puro acaso, um acaso que não tinha qualquer sentido, pois,
assim como eu existia desta parte do escaparate, eu podia existir também para
além dele, com a mesma face pálida, com os mesmos olhos, com o mesmo cabelo
desbotado que formavam no espelho uma figura fugaz e bizarra, difícil de
definir.}\looseness=-1

Assim chegavam até mim, do exterior, diversas advertências para me mobilizar e
me tirar bruscamente da compreensão de todos os dias. Elas me estupefaziam,
me detinham e resumiam, num só instante, toda a inutilidade do mundo.

\textls[10]{Naquele segundo, tudo me parecia caótico, assim como eu tapava minhas orelhas
ao ouvir uma fanfarra e, ao descolar os dedos por um instante, toda a música
me parecia, naquele momento, barulho puro.}

Eu passeava o dia todo pela quermesse, sobretudo pelo campo dos arredores,
onde os artistas e os monstros das barracas, reunidos em torno do caldeirão
com polenta, descabelados e sujos, desciam de seus belos cenários e
existências noturnas de acróbatas, mulheres sem corpo e sereias, para a massa
comum e a irremediável miséria de sua humanidade. O que diante das barracas
parecia admirável, descontraído e por vezes até mesmo fastuoso, aqui, detrás
delas, em plena luz do dia, caía numa familiaridade mesquinha e
desinteressante, que era, aliás, a do mundo inteiro.

Certo dia, assisti ao enterro do filho de um dos lambe-lambes.

As portas do cosmorama estavam escancaradas e, do lado de dentro, diante do
cenário que servia de fundo para as fotografias, jazia sobre duas cadeiras o
caixão destampado.

A tela de trás representava um parque esplêndido, com terraços em estilo
italiano e colunas de mármore.

Nessa paisagem onírica, o pequeno cadáver de mãos cruzadas sobre o peito, com
roupa de festa e adornos prateados na lapela, parecia mergulhado numa
beatitude inefável.

\textls[-10]{Os pais da criança e diversas mulheres choravam desesperançosamente em torno
do féretro, enquanto do lado de fora a fanfarra do grande circo, emprestada
grátis pelo diretor, entoava grave uma serenata do Intermezzo, a peça mais
triste do programa.}\looseness=-1

Naqueles momentos, o morto com certeza estava indizivelmente feliz e
sossegado, na intimidade de sua paz profunda, no silêncio infinito do parque
de plátanos.

Logo, entretanto, ele foi arrancado da solenidade em que jazia e posto numa
carroça para ser levado ao cemitério, para a cova fria e úmida que lhe estava
reservada.

Após sua partida, o parque ficou deserto e desolado. Na quermesse, a morte
também se revestia de cenários factícios e cheios de nostalgia, como se a
quermesse formasse um mundo à parte, cuja missão fosse demonstrar a
melancolia sem limites dos ornamentos artificiais, desde o início da vida até
o fim, com o exemplo vivo de existências pálidas, consumidas à luz difusa
do \textit{panopticum} ou no repartimento da parede ilimitada, embebida de
belezas supraterrestres, dos cosmoramas de fotografias. 

\textls[10]{A quermesse então se tornava, para mim, uma ilha deserta sob o bafejo de
auréolas desoladas, perfeitamente semelhante ao mundo inexplicável mas
límpido para onde conduziam-me as \textit{crises} de infância.}


\section{xxi} 

\letra{O}{}\textls[-15]{andar de cima da casa dos Weber, aonde eu ia com frequência desde a
 morte da velha Etla Weber, parecia-se com um verdadeiro \textit
 {panopticum}. Nos quartos ensolarados a tarde toda, a poeira e o calor
 flutuavam ao longo das cristaleiras repletas de objetos antiquados, atirados
 ao acaso nas estantes. As camas haviam sido levadas para o térreo, deixando
 os dormitórios vazios. Agora o velho Samuel Weber (\emph
 {Vendas \& Representações}) e seus dois filhos, Paul e Ozy, moravam nos
 aposentos de baixo.}\looseness=-1

No primeiro aposento, que dá para a rua, foi instalado o escritório. Era um
cômodo cheirando a mofo, atulhado de registros e envelopes contendo amostras
de cereais, forrado por reclames velhos manchados de moscas.

\textls[10]{Alguns deles, depois de tantos anos na parede, haviam se integrado por
completo à vida familiar. Acima da caixa registradora, o reclame de uma água
mineral representava uma mulher alta e esbelta, em véus diáfanos, vertendo o
remédio salvador dos enfermos a seus pés. É claro que, nas horas secretas da
noite, Ozy Weber vinha beber da fonte milagrosa, com braços finos como
flautas e com o manúbrio do esterno escapando da camisa como o peito inchado
de um peru.}

Outro reclame familiar era o de uma firma de transportes que --- com seu barco
deslizando por sobre ondas encrespadas --- completava a personalidade de
Samuel Weber ao constituir, junto com o boné de capitão de navio e os óculos
de lentes grossas que costumava portar, um terceiro elemento de marinheiro.
Quando o velho Samuel fechava um dos livros de registro e o colocava debaixo
da prensa, girando a barra de ferro, dava realmente a impressão de que
manejava o timão de uma caravela por mares desconhecidos. Longos fiozinhos do
algodão cor-de-rosa com que tapava os ouvidos pendiam de suas orelhas, e isso
com certeza era uma sábia precaução contra as correntes marítimas.

No segundo cômodo, Ozy lia romances populares, afundado numa poltrona de
couro, erguendo bem o volume para que sobre ele batesse a luz fraca que vinha
da rua, atravessando o escritório. Num canto escuro, cintilava o anteparo de
uma escarradeira monumental de lata, em forma de gato, e, na parede, um
espelho refletia estranhamente um quadrado de luz cinzenta, como uma
lembrança fantasmática do dia que escoava lá fora.

\textls[-13]{Eu costumava visitar Ozy assim como os cachorros entram em quintais alheios
quando o portão está aberto e ninguém os afugenta. Atraía-me sobretudo uma
brincadeira bizarra, inventada não sei mais por qual de nós e em que
circunstâncias. A brincadeira constava em manter, com a maior seriedade,
diálogos imaginários. Tínhamos que permanecer sérios até o fim e não revelar
de maneira alguma a inexistência das coisas sobre que falávamos.}\looseness=-1

Um dia entrei e Ozy me disse, num tom terrivelmente seco, sem erguer os olhos
do livro:

--- A aminopirina que tomei ontem à noite para transpirar me provocou uma
    tosse horrenda. Fiquei me revolvendo nos lençóis até de manhãzinha. Até
    que há pouco veio a Matilda --- não existia Matilda alguma --- e me
    aplicou uma fricção.

\textls[-10]{O absurdo e a estupidez da coisas que Ozy narrava atingiam-me a cabeça como
poderosos martelos. Talvez eu devesse sair imediatamente do aposento, mas,
com uma pequena volúpia de me colocar de propósito no seu nível de
inferioridade, eu lhe respondia no mesmo tom. Acho que era esse o segredo da
brincadeira.}\looseness=-1

--- Olhe que eu também me resfriei --- disse-lhe, embora estivéssemos no mês
    de julho --- e o Doutor Caramfil --- que existia --- prescreveu-me uma
    receita. Pena que esse médico\ldots{} sabe, ele foi preso hoje de
    manhã\ldots{}

Ozy ergueu os olhos do livro.

\textls[-10]{--- Viu, faz tempo que eu lhe dizia que ele falsificava dinheiro\ldots{}}\looseness=-1

--- Mas é claro --- acrescentei ---, pois de onde ele teria para gastar com
    tantas atrizes de revista?

Nessas palavras havia sobretudo um prazer um pouco nojento de mergulhar na
mediocridade do diálogo e, ao mesmo tempo, uma vaga impressão de liberdade.
Podia, assim, caluniar à vontade o médico, que morava ali perto e que eu
sabia que se deitava toda noite precisamente às nove horas. 

\textls[-8]{Dessa maneira, éramos capazes de falar de tudo, misturando coisas verdadeiras
com coisas imaginárias, até a conversa inteira se impregnar de uma espécie de
independência aérea, desprendendo-se de nós e pondo-se a flutuar pelos
cômodos, como um pássaro bizarro --- de cuja existência exterior, aliás, caso
o pássaro realmente aparecesse diante de nós, só duvidaríamos se as nossas
palavras não tivessem nenhuma relação com nós mesmos.}\looseness=-1

Ao ganhar de novo a rua, eu tinha a sensação de ter dormido profundamente. Mas
o sonho parecia continuar, e eu fitava com surpresa as pessoas conversando
seriamente entre si. Será que elas não percebiam que se pode falar com
gravidade sobre qualquer coisa, sobre absolutamente qualquer coisa?

\textls[10]{Às vezes Ozy não estava a fim de conversa e me levava consigo para rebuscar o
andar de cima. Naqueles poucos anos de abandono e, graças ao costume do velho
Weber de mandar \textit{para cima} todos os objetos inúteis, acabaram se
acumulando ali as coisas mais diversas e extraordinárias.}

\textls[-15]{Nos quartos, um sol incandescente entrava pelas janelas empoeiradas e sem
cortinas. As cristaleiras vacilavam levemente aos nossos passos pelo assoalho
velho, como se batessem os dentes. Separando dois quartos, uma cortina de
miçangas servia de porta.}\looseness=-1

\textls[-20]{Eu subia já um pouco zonzo pelo calor do dia. A absoluta desolação dos
aposentos me inquietava. Era como se eu houvesse existido num mundo conhecido
há tempos e do qual não podia me lembrar muito bem. Meu corpo era tomado por
uma bizarra sensação de leveza e desprendimento. Essa sensação se aprofundava
ao ter de passar entre os dois cômodos separados pela cortina de miçangas.}\looseness=-1

Procurávamos nas gavetas sobretudo correspondências velhas para arrancar os
selos dos envelopes. Os pacotes amarelados soltavam uma nuvem de poeira e
insetos que corriam pelos papéis em busca de abrigo. Vez ou outra uma carta
caía no chão, abrindo-se e revelando uma caligrafia antiga e caprichada, com
tinta desbotada. Havia nela um quê de tristeza e resignação, uma espécie de
conclusão cansada do passar do tempo desde que fora escrita e um sono sereno
na eternidade, como o das coroas mortuárias. Encontrávamos também fotografias
fora de moda, madames vestidas em crinolina, senhores meditabundos com o dedo
na testa, com sorrisos anêmicos. Na parte debaixo da fotografia, dois anjos
seguravam um cesto de frutas e flores, debaixo do qual estava escrito \emph
{porte visite} ou \emph{souvenir}. Por entre fotografias e os objetos das
cristaleiras --- a fruteira de vidro rosa com espirais nas bordas, as bolsas
de veludo dentro das quais só havia a seda roída pelas traças, diversos
objetos com monogramas desconhecidos ---, por entre tudo isso reinava um ar
de perfeita compreensão como uma espécie de vida própria, idêntica à vida de
outrora, quando as fotografias, por exemplo, correspondiam a pessoas que se
moviam e que estavam vivas, quando as cartas eram escritas por mãos quentes e
verdadeiras --- uma vida, contudo, reduzida a uma escala menor, num espaço
mais restrito, no limite das cartas e das fotografias, como num cenário
contemplado por meio das lentes grossas de um binóculo, cenário este que
permaneceu intacto em cada um de seus componentes, embora minúsculo e
distante.\looseness=-1

Ao cair da tarde, quando descíamos a escada, costumávamos cruzar com Paul
Weber, que mantinha lá em cima um armário de roupas, no primeiro quarto, onde
ia se trocar.


\section{xxii} 

\letra{E}{le} era um menino ruivo, de mãos grandes e cabelo desgrenhado,
 lábios carnudos e nariz de palhaço. Mas pairava em seus olhos uma candura
 indizivelmente calma e tranquilizadora. Por causa desse olhar, tudo o que
 Paul fazia tinha um ar de desprendimento e indiferença.

\textls[-10]{Amava-o muito, mas em segredo; meu coração disparava quando o encontrava na
 escada. Gostava da simplicidade com que falava comigo, sorrindo sempre, como
 se a nossa conversa tivesse, para além do seu significado próprio, um outro,
 remoto e efêmero. Ele conservava aquele sorriso durante as conversas mais
 sérias e mesmo quando tratava dos mais diversos negócios com o velho Weber.
 Eu amava Paul também pela vida secreta que mantinha para além das ocupações
 diárias, e da qual chegavam até mim apenas ecos sussurrados com estupefação
 pelos adultos ao meu redor. Paul gastava com mulheres, no teatro de revista,
 todo o dinheiro que ganhava. Em sua devassidão havia uma fatalidade
 irremediável, contra a qual o velho Weber se chocava como se ela fosse uma
 parede. Certa vez, ressoou na cidade inteira o boato de que Paul desatrelara
 os cavalos das charretes da praça e os levara para a sala do teatro de
 variedades, onde improvisou um tipo de circo, com a participação dos mais
 eminentes bêbados da cidade. De outra feita, correra um boato de que ele
 tomara banho de champanhe com uma mulher. Mas o que é que não se dizia
 dele?} 

\textls[-22]{Era-me impossível definir minha simpatia por Paul. Enxergava bem as pessoas ao
meu redor, enxergava bem a inutilidade e o aborrecimento com que consumiam
suas vidas, jovens moças no jardim dando risadas estúpidas; comerciantes com
olhares ardilosos e cheios de importância; a necessidade teatral do meu pai
de desempenhar o papel de pai; o violento cansaço dos mendigos dormindo em
esquinas imundas; tudo isso se confundia num aspecto geral e banal, como se o
mundo, assim como era, esperasse havia tempo dentro de mim, construído em sua
forma definitiva, e eu, a cada dia, não fizesse outra coisa senão verificar
em mim o seu conteúdo envelhecido.}\looseness=-1

\textls[15]{Tudo era muito simples; só Paul estava fora disso tudo, numa densidade de vida
compacta e absolutamente inacessível à minha compreensão.}

Mantinha profundamente dentro de mim todos os seus gestos e os seus menores
movimentos, não como uma lembrança, mas como uma existência dupla. Muitas
vezes me esforçava em andar como ele, estudava um determinado gesto e o
repetia diante do espelho até convencer-me de que o repetira de maneira
exata.

\textls[15]{No andar de cima da casa dos Weber, Paul era a figura de cera mais enigmática
e mais elegante. Logo ele levou para lá a mulher pálida que faltava, com
gestos e andar produzidos por um mecanismo silencioso\ldots{}}

Assim, o andar completava sua galeria de \textit{panopticum}, começando pelo
capitão de navio Samuel Weber e acabando com o fenômeno delicado e disforme
do infantil Ozy.


\section{xxiii} 

\letra{E}{u} também encontrava velharias e objetos cheios de melancolia num
 outro andar abandonado, na casa do meu avô. Ali as paredes ainda estavam
 cobertas por quadros estranhos, emoldurados por frisos grossos de madeira
 dourada ou frisos mais estreitos de \textit{plush} vermelho. Havia ainda
 algumas molduras feitas de conchinhas, uma ao lado da outra, trabalhadas com
 uma minúcia que me fazia contemplá-las horas a fio. Quem teria aplicado
 aquelas conchas? Quais foram os vivos e miúdos gestos que as uniram? De tais
 obras defuntas renasciam de súbito existências inteiras, perdidas na neblina
 do tempo como imagens de dois espelhos paralelos, obstruídos em profundezas
 esverdeadas de sonho. Num canto jazia um fonógrafo com o bocal virado para
 baixo, belamente pintado em fatias amarelas e róseas, como uma enorme porção
 de sorvete de baunilha e rosa; em cima da mesa havia várias estampas, duas
 das quais representavam o rei~Carol~\textsc{i} e a rainha Elisabeta.

\textls[10]{Esses quadros me intrigaram por muito tempo. O artista parecia-me muito
talentoso, pois os traços eram muito finos e seguros, embora eu não
compreendesse por que ele os executara numa aquarela cinzenta, empalidecida,
como se o papel tivesse ficado muito tempo debaixo d'água. Um dia, fiz uma
descoberta assombrosa: aquilo que eu considerava uma cor apagada não era
outra coisa senão um amontoado de letras minúsculas, que só podiam ser
decifradas com uma lupa.}

\textls[10]{Em todo o desenho não havia um único sinal de lápis ou pincel; tudo era uma
junção de palavras que contava a história da vida do rei e da rainha.}

Minha estupefação fez de repente desabar a incompreensão com a qual eu fitava
os desenhos. No lugar da minha descrença na arte do desenhista, nasceu uma
infinita admiração.

Nela, senti a mágoa de não ter observado antes a qualidade essencial do
quadro, fazendo crescer em mim, ao mesmo tempo, minha grande insegurança em
tudo o que via: já que eu contemplara por tantos anos aqueles desenhos sem
descobrir a própria matéria de que eram constituídos, não seria possível que
me escapasse, devido a uma miopia semelhante, o significado de todas as
coisas ao meu redor, significado este nelas inscrito, talvez, tão claramente
quanto as letras que compunham aqueles quadros?

\textls[10]{Em torno de mim, as superfícies do mundo embeberam-se subitamente de brilhos
estranhos e opacidades inseguras como as das cortinas, opacidades que se
tornam transparentes e que nos apresentam de repente a profundidade de um
espaço assim que uma luz se acende atrás delas.}

Por trás dos objetos, contudo, jamais se acendeu luz alguma, de maneira que
eles permaneciam sempre presos aos volumes que os encerravam hermeticamente,
e que por vezes pareciam estreitar-se para deixar entrever seu verdadeiro
significado.


\section{xxiv} 

\letra{A}{quele} andar ainda tinha outras curiosidades que só a ele
 pertenciam. Por exemplo, o panorama da rua visto das janelas da frente.

Sendo as paredes da casa bastante espessas, as janelas eram profundas,
formando nichos onde se podia estar muito confortavelmente.

Instalava-me num deles como num quartinho de vidro e abria a janela que dava
para a rua. 

A intimidade do nicho, bem como a delícia de observar a rua de uma posição
prazerosa, deu-me a ideia de um veículo nas mesmas medidas, com almofadas
macias em que me deitasse, com janelinhas pelas quais eu olhasse para
diversas cidades e paisagens desconhecidas, enquanto o veículo percorreria o
mundo.

\textls[10]{Certa vez, enquanto meu pai me contava lembranças da infância, perguntei-lhe
qual fora seu desejo secreto mais ardente, no que ele me respondeu que, acima
de tudo, desejara possuir um veículo miraculoso em que pudesse ficar deitado
e que o levasse pelo mundo afora.}

\textls[-15]{Eu sabia que, na infância, ele dormia no quarto do andar de cima com as
janelas que dão para a rua, e então lhe perguntei se ele gostava de se deitar
nos nichos das janelas para olhar para baixo.}\looseness=-1

\textls[10]{Surpreso, respondeu-me que, de fato, a cada noite, quando subia para se
deitar, ele se metia num dos nichos onde permanecia horas a fio, várias vezes
adormecendo ali mesmo. Ele provavelmente sonhou com aquele veículo no mesmo
lugar e nas mesmas circunstâncias que eu.}

No mundo havia, portanto, para além de lugares malditos que secretavam
vertigens e desânimos, outros espaços mais benéficos, cujas paredes
destilavam imagens belas e prazerosas.

\textls[10]{As paredes do nicho destilavam o sonho de um veículo que percorre o mundo e
quem se deitasse naquele lugar acabava lentamente se impregnando dessa ideia
como do fumo entorpecedor do haxixe\ldots{}}


\section{xxv} 

\letra{P}{elo} andar de cima podia-se entrar também em dois sótãos, sendo que
 de um deles tinha-se acesso ao telhado por meio de uma janelinha. Era por
 ali que eu subia para o topo da casa. Cinzenta e amorfa, a cidade toda se
 estendia ao meu redor, até os campos longínquos, onde trens minúsculos
 passavam por cima de uma ponte, frágil como um brinquedo.

\textls[-15]{O que eu mais queria era não ficar tonto e poder atingir uma sensação de
equilíbrio igual à que eu tinha lá embaixo, no chão. Queria continuar minha
vida \textit{normal} em cima do telhado, movendo-me, sem medo e sem qualquer
impressão especial de vazio, no ar sutil e cortante das alturas. Achava que,
se conseguisse isso, eu sentiria no corpo um peso mais elástico e mais
vaporoso, que me transformaria totalmente e que faria de mim uma espécie de
homem-pássaro. Tinha certeza de que a atenção de não cair era a que mais
pesava em mim, enquanto o pensamento de estar numa grande altura me percorria
como uma dor que eu queria arrancar pela raiz.}\looseness=-1

\textls[10]{Para que nada me parecesse excepcional durante a estada lá em cima, cada vez
eu me obrigava a executar algo preciso e banal: ler, comer ou dormir.}

\textls[10]{Pegava as ginjas e o pão que meu avô me dava e subia para o telhado. Dividia
as ginjas em quatro pedaços e assim os comia, um a um, para que minha
atividade \textit{normal} durasse o máximo possível. A cada ginja terminada,
esforçava-me em arremessar o caroço lá embaixo na rua, num caldeirão enorme
em exposição na frente de uma loja.}

Após descer, apressava-me em ver quantos caroços eu acertara dentro do
caldeirão. Eram sempre três ou quatro. O que me decepcionava, porém, era que
em derredor eu só encontrava outros três ou quatro caroços. Havia comido,
portanto, pouquíssimas ginjas, embora tivesse a impressão de ter ficado horas
a fio em cima do telhado. No quarto do meu avô, o mostrador de faiança verde
do relógio igualmente me demonstrava que apenas alguns minutos haviam passado
desde que eu subira. O tempo provavelmente se tornava cada vez mais denso à
medida em que \textit{transcorria} mais em cima. Em vão eu tentava
prolongá-lo ao ficar o máximo possível no telhado. Sempre que descia, eu
tinha de reconhecer que passara muito menos tempo do que imaginara. Isso
reforçava minha estranha sensação, no solo, de indefinido, de inacabado\ldots
{} O tempo aqui embaixo era mais rarefeito que na realidade, ele continha
menos matéria do que nas alturas e, assim, participava da fragilidade de
todas as coisas, que, ao meu redor, pareciam tão densas embora fossem tão
instáveis, sempre na iminência de abandonar seu significado e seu contorno
provisórios para surgir sob a forma de sua exata existência\ldots{}


\section{xxvi} 

\letra{O}{}andar de cima se decompôs pedaço por pedaço, objeto por
 objeto depois da morte do meu avô. Ele morreu naquele quartinho mesquinho e
 úmido do quintal, que escolhera como refúgio de sua velhice e que não queria
 mais abandonar a não ser para o caminho derradeiro.

\textls[10]{Todo dia eu o visitava ali, às vésperas da morte, e assisti à oração dos
moribundos que ele recitou sozinho, com voz trêmula e desprovida de emoção,
após vestir uma camisa branca nova para que a reza fosse mais solene.}

\textls[-10]{Naquele quartinho eu o vi morto poucos dias depois, estirado sobre uma mesa de
lata para o seu último banho. Meu avô tinha um irmão que era alguns anos mais
novo do que ele, com o qual se parecia assustadoramente: ambos tinham a mesma
cabeça muito redonda como uma pequena esfera, coberta de cabelos brancos
brilhantes, os mesmos olhares vívidos e penetrantes e a mesma barba com fios
escassos como uma espuma cheia de buracos.}\looseness=-1

Esse tio pediu à família para ter a honra de lavar o morto e, embora fosse
velho e fraco, ele pôs mãos à obra com afinco.

Tremia da cabeça aos pés enquanto trazia da torneira do quintal baldes cheios
de água para esquentá-la na cozinha.

Assim que a água esquentou, ele a levou para o quartinho e começou a lavar o
cadáver com sabão e feixes de palha.

Enquanto esfregava, lágrimas se misturavam aos seus dentes e --- como se meu
avô pudesse ouvir o que dizia --- falava-lhe aos sussurros, suspirando com
amargor: ``Veja só aonde chegamos\ldots{} veja até onde me trouxeram meus
dias miseráveis\ldots{} agora você está morto e eu aqui o lavando\ldots{} ai
de mim\ldots{} tive de viver tantos anos\ldots{} para chegar a este momento
tão triste\ldots{}''.

\textls[-15]{Com a manga da roupa ele secava as faces e a barba úmida de lágrimas e
transpiração, continuando a lavar com ainda mais afinco.}\looseness=-1

\textls[-15]{Os dois velhos, espantosamente parecidos, um morto e o outro o lavando,
formavam uma cena um tanto alucinante. Os empregados do cemitério, que
costumavam realizar esse trabalho, recebendo por ele gorjetas de toda a
família, estavam num canto e observavam contrariados esse intruso que lhes
roubava a ocupação. Falavam entre si aos sussurros, fumando e cuspindo no
chão em todas as direções. Após uma hora de trabalho, o irmão do meu avô
terminou.}\looseness=-1

O cadáver estava deitado de bruços na mesa.

\textls[10]{--- Pronto? --- perguntou um dos funcionários, um homenzinho de cavanhaque
    ruivo que estalava nervosamente os dedos, cheio de malícia.}

--- Pronto --- respondeu o irmão do morto. --- Agora vamos vesti-lo\ldots{}

\textls[10]{--- Aha! Quer dizer que está pronto --- disse de novo o homenzinho, espumando
    ironia. --- Você realmente acha que está pronto? Você acha que é assim
    que se enterra um morto? Nessa sujeira desgraçada?}

\textls[-20]{No meio do aposento, com um feixe de palha na mão, o pobre velho ficou
surpreso, fitando-nos a todos e implorando com seu olhar emudecido que o
defendêssemos. Sabia muito bem que lavara o morto com extremo cuidado e não
achava que merecia tal insulto.}\looseness=-1

--- Agora eu vou lhe mostrar que não se deve meter o nariz onde não é
    chamado\ldots{} --- prosseguiu o homenzinho atrevido e, arrancando da mão
    do velho o feixe de palha, precipitou-se com ele para a mesa,
    introduziu-o com um movimento rápido no ânus do morto e extraiu dele um
    pedaço espesso de excremento\ldots{}

--- Está vendo como você não sabe lavar um morto? --- disse ele. --- Queria
    sepultá-lo com essa sujeira dentro dele?

O irmão do meu avô, após ser sacudido por um tremor violento, desabou em
prantos\ldots{}

O sepultamento ocorreu num dia canicular de verão: nada mais triste e mais
impressionante do que um enterro em pleno calor e à plena luz do sol, quando
as pessoas e as coisas parecem um pouco maiores, em meio aos vapores da alta
temperatura, como se vistas através de uma lupa.

Que outra coisa senão enterrar seus mortos podiam fazer as pessoas num dia
como esse?

No bafo e no langor do ar, seus gestos pareciam ter centenas de anos, os
mesmos de outrora, de hoje e de sempre. A cova úmida aspirou o morto num
frescor e numa escuridão que com certeza o trespassaram de uma felicidade
suprema. Os torrões de terra caíram pesados sobre a tampa do ataúde enquanto
as pessoas, cobertas de poeira, transpiradas e cansadas, continuavam a viver,
do lado de cima da terra, suas vidas imperiosas. 


\section{xxvii} 

\letra{P}{oucos} \textls[10]{dias após o enterro do meu avô, Paul Weber se casou. Paul estava meio cansado no casamento, porém mantivera o sorriso; um sorriso
triste e forçado, em que se vislumbrava o início de um devotamento.}

\textls[10]{No colarinho duro, aberto na frente, o pescoço pelado e vermelho movia-se de
maneira estranha; suas calças pareciam mais compridas e mais delgadas do que
de costume; as abas do fraque dependuravam-se grotescas como se fossem de um
palhaço. Paul concentrava sobre sua pessoa todo aquele ar grave e ridículo da
cerimônia. O ridículo mais secreto e mais íntimo da cerimônia era eu que
corporificava. Eu era o pequeno palhaço em quem ninguém prestava atenção.}

\textls[-20]{No fundo do salão escuro, a noiva aguardava sentada em sua poltrona sobre o
estrado. Véus brancos cobriam seu rosto e só depois de sair do baldaquim e os
erguer é que eu vi Edda pela primeira vez\ldots{}}\looseness=-1

\textls[-20]{As mesas dos convidados estendiam-se brancas pelo quintal numa só fileira; na
entrada, amontoaram-se todos os vagabundos da cidade; o céu tinha uma luz
indecisa de barro amarelo; donzelas pálidas em vestidos de seda azul e rosa
distribuíam pequenos bombons prateados. Era um casamento. Os músicos faziam
ranger uma valsa velha e triste; de vez em quando, o seu ritmo inchava,
crescia e parecia revigorar-se, para em seguida a melodia definhar de novo,
cada vez mais, até dela só restar o fio metálico da flauta solitária.}\looseness=-1

\textls[-10]{Dia terrivelmente longo; um dia inteiro é demais para um casamento. No fundo
do quintal não aparecia ninguém, lá ficava a estrebaria do hotel e uma
elevação de onde eu podia ter um panorama de tudo, enquanto ao meu redor
algumas galinhas bicavam pequenos grãos entre os fios de grama e, do quintal,
vinha o sopro da valsa triste misturando-se ao aroma fresco de feno úmido da
estrebaria. De lá eu vi Paul fazer algo extraordinário: ele conversava com
Ozy e com certeza contava-lhe algo divertido, talvez uma piada, pois o doente
começou a rir tanto que ficou roxo, quase se sufocando por debaixo do
plastrão arqueado da camisa engomada.}\looseness=-1

\textls[-10]{Finalmente anoiteceu.~As poucas árvores do quintal entranha-\linebreak
ram-se
 no escuro, nele escavando um parque misterioso e invisível.}\looseness=-1

\textls[-5]{Na sala mal iluminada, a noiva continuava no estrado ao lado de Paul,
inclinando a cabeça na sua direção sempre que ele lhe sussurrava alguma
coisa, abandonando languidamente o braço entre os seus dedos que o
acariciavam ao longo da brancura da luva.}\looseness=-1

\textls[-5]{Trouxeram alguns bolos até a mesa. Sobretudo um, monumental como um castelo
reforçado com ameias e milhares de contrafortes de um creme róseo. As pétalas
das flores de açúcar que o cobriam brilhavam opacas e oleosas. Cravaram no
meio dele uma faca que fez uma das rosas ranger fino sob o gume,
esmigalhando-se como vidro em dezenas de pedaços. As velhas senhoras
passeavam majestosas em seus vestidos de veludo, com inúmeras joias no peito
e nos dedos, avançando lenta e solenemente como pequenos altares de igreja
ambulantes, cobertas de ornamentos.}\looseness=-1

Pouco a pouco, o salão foi tomado pela névoa e tudo o que eu via tornou-se
ainda mais vago e absurdo\ldots{} Adormeci olhando para minhas mãos vermelhas
e ardentes.


\section{xxviii} 

\letra{O}{}quarto em que despertei cheirava a fumaça azeda. Num espelho à
 minha frente, a janela refletia a aurora, que surgia como um quadrado
 perfeito de seda azul. Eu estava deitado numa cama bagunçada, cheia de
 travesseiros. Um zumbido soava nos meus ouvidos como se dentro de uma
 concha; no aposento, ainda flutuavam camadas finas de fumaça.

\textls[10]{Ao tentar me levantar, minha mão entrou nas formas esculpidas da madeira da
cama; algumas preenchiam meus dedos, e outras, à luz embaciada do quarto,
pareciam agigantar-se longe da cama, formando milhares de ameias, buracos e
bolores rendados. Em poucos instantes, o quarto recheou-se imaterialmente com
todo o tipo de volutas pelas quais eu tinha de passar para chegar à porta,
afastando-as e fazendo espaço entre elas. Minha cabeça soava sem parar e
todas as cavernas do ar pareciam repetir esse zumbido. No corredor, a luz
branca lavou meu rosto com seu frescor, fazendo-me despertar por completo.
Deparei-me com um senhor envergando uma longa camisa de noite, que me fitou
com um ar bastante zangado, como se me censurasse por estar já vestido tão
cedo pela manhã.}

\textls[-10]{Fora dali não havia mais ninguém. Lá embaixo, no quintal, restaram as mesas
descobertas dos convidados, feitas de tábuas de pinho. A aurora estava fria e
enfadada. O vento espalhava papéis de bala coloridos pelo quintal deserto. Em
que posição a noiva manteve a cabeça? Como é que ela a havia deitado sobre o
ombro de Paul? Em certos \textit{panoptica}, a mulher de cera tinha um
mecanismo que a fazia deitar a cabeça para um dos lados e fechar os olhos.}\looseness=-1

As ruas da cidade haviam perdido todo sentido; o frio entrava por debaixo da
minha roupa; estava com sono e com frio. Ao fechar os olhos, o vento aplicava
sobre meu rosto um outro rosto, mais frio, e, deste lado das pálpebras, eu
sentia como se portasse uma máscara, a máscara da minha face em cujo interior
fazia escuro e frio como o verso de uma máscara metálica de verdade. Qual
casa deveria explodir no meu caminho? Que poste deveria se contorcionar como
um bastão de borracha, mostrando, assim, que me faz caretas? Em nenhum lugar
e em nenhuma circunstância, nunca acontece algo no mundo.


\section{xxix} 

\letra{A}{o} \textls[10]{chegar à praça, alguns homens descarregavam carne para as
 barracas dos açougueiros. Seguravam nos braços metades de reses vermelhas e
 roxas, úmidas de sangue, altas e imponentes como princesas mortas. No ar,
 pairava um cheiro quente de carne e urina; os açougueiros penduravam as
 reses com a cabeça para baixo, com seus olhares globulosos e negros
 direcionados para o chão. Estavam agora enfileiradas nas paredes brancas de
 porcelana como esculturas vermelhas, entalhadas no material mais frágil e
 diverso, com o reflexo aguado e irisado da seda e a limpidez turva da
 gelatina. Do ventre aberto, dependuravam-se a renda dos músculos e colares
 pesados de contas de gordura. Por ali os açougueiros enfiavam suas mãos
 vermelhas, retirando preciosas miudezas e as arranjando em seguida sobre a
 mesa: objetos redondos de carne e sangue, largos, elásticos e quentes.}\looseness=-1

A carne fresca brilhava aveludada como pétalas de rosas monstruosas,
hipertrofiadas. A aurora se tornara azul como aço; a manhã fria cantava com
um som profundo de órgão.

Os cavalos das carroças fitavam as pessoas com seus olhos eternamente
lacrimejantes; uma égua soltara na rua um jorro fervente de urina. Na poça
que se formou, espumosa aqui e transparente ali, o céu se refletia negro e
profundo.

\textls[15]{Tudo se tornara longínquo e desolador. Era de madrugada; as pessoas
 descarregavam carne; o vento entrava por debaixo da minha roupa; tremia de
 frio e de insônia; em que espécie de mundo estava vivendo?} 

\textls[10]{Pus-me a correr como um louco pelas ruas. O sol surgiu vermelho de novo na
margem dos telhados. As ruelas de casas altas ainda estavam governadas pela
escuridão, e só no cruzamento das ruas a luz brotava cintilante como por
entre portas abertas ao longo de corredores abandonados.}

\textls[15]{Passei pelos fundos da casa dos Weber; as venezianas pesadas do andar de cima
estavam cerradas; tudo era triste e deserto; o casamento se consumara.}


\section{xxx} 

\letra{O}{} \textls[15]{andar de cima da casa dos Weber iluminou-se tão logo chegou Edda
 trazendo sombras e frescor, assim como certas claridades em bosques espessos
 se desanuviam com a luz verde filtrada pelas folhas.}

\textls[15]{A primeira coisa que Edda fez foi cobrir as janelas com cortinas e espalhar
pelo chão tapetes macios que abafaram todos os ecos desolados do andar de
cima.}

\textls[15]{Toda manhã eu ficava na sacada, inventariando aquela quantidade imensa de
objetos contorcionados e artificiais que saíam das cristaleiras.}\looseness=-1

Junto com Ozy, eu os limpava conscienciosamente, para em seguida atirá-los
numa caixa que ia para o lixo.

\textls[15]{Edda entrava e saía da sacada, envergando um roupão azul e uns chinelos cujos
saltos estalavam a cada passo. Por vezes permanecia apoiada no parapeito,
cerrando um pouco as pálpebras e fitando o céu resplandecente.}

\textls[10]{O andar de cima embebeu-se de um aroma inefável, que lhe modificou o conteúdo
como uma essência forte misturada a uma bebida alcoólica.}\looseness=-1

Assim, todos os acontecimentos na minha vida eram fadados a surgir de forma
brusca e intermitente, sem que os pudesse compreender, encerrados em si
mesmos e isolados do passado. Edda tornou-se um objeto a mais, um simples
objeto cuja existência me torturava e irritava, como uma palavra repetida
inúmeras vezes que acaba se tornando ininteligível à medida em que a
necessidade de entendê-la nos parece mais imperiosa.

\textls[10]{A perfeição do mundo estava prestes a transparecer de qualquer lugar, como um
broto que ainda tem de transpor uma última casca para rebentar ao ar livre.}

\textls[10]{Nas manhãs de verão, na sacada do andar de cima, algo acontecia e todo o meu
corpo debalde se esforçava em compreender o quê, exatamente.}\looseness=-1

\textls[10]{Para me encontrar com Edda, eu me armava com todos os amargores, todas as
humilhações e todo o ridículo necessário a uma aventura.}\looseness=-1


\section{xxxi} 

\letra{M}{antiveram} a cortina de miçangas entre os quartos. As cristaleiras
 passaram a exibir peças de lingerie branca com laços grandes de fitas
 coloridas, e a casa dos Weber mudou completamente. Em torno de Edda
 iniciou-se uma pantomima de quatro personagens: Paul tornara-se sério e
 fiel; o velho Weber comprou um boné novo e um par de óculos com armação de
 ouro; Ozy ofegava de emoção esperando que Edda o chamasse para cima,
 enquanto eu permanecia na sacada, lançando olhares aguados que se perdiam no
 vazio.

Todo sábado à tarde nos reuníamos no aposento da frente, onde o gramofone
tocava árias orientais de Kismet enquanto Edda nos servia quitutes agridoces
feitos de mel e amêndoas. Havia uma fruteira repleta de avelãs, das quais se
servia sobretudo Samuel Weber, que engolia com vagar e intensidade, fazendo
seu pomo-de-adão dançar como uma boneca de borracha.

\textls[-15]{Mantinha as pernas uma sobre a outra, o que era uma posição de descanso
 totalmente imprópria para um homem de negócios ou comerciante de cereais,
 mais própria de um ator em cena e, quando falava, franzia os lábios para não
 deixar entrever seus dentes de ouro.}\looseness=-1

Ele temia encostar no mais mínimo objeto e, ao passar pela cortina de
miçangas, voltava-se e unia devagar as duas metades atrás de si, para não
produzir nenhum tilintar.

\textls[10]{No caso de Ozy, todas as deformidades se acutizaram e se curvaram numa posição
de atenção extrema. O manúbrio do seu esterno parecia saltar ainda mais, como
se se esforçasse em apreender as mínimas palavras de Edda e surpreendê-las
com um segundo de antecipação.}

\textls[15]{Sozinho, Paul caminhava por cima dos tapetes, calmo e seguro de si. Exibia
gestos plenos em que não havia nada a acrescentar ou tirar e, sempre que
cingia Edda nos seus braços, todos nós, os outros três, no final das contas
nos consolávamos por ele fazer isso melhor do que qualquer um.}

\textls[10]{No que diz respeito a mim, não sei muito bem o que é que havia comigo naqueles
dias.}

\textls[10]{Numa tarde, enfiado numa das poltronas, comecei a pressionar com força minha
cabeça no \textit{plush}. Pelos pontiagudos miúdos do veludo entraram no meu
rosto, o que me produziu uma dor intensa. Num segundo surgiu dentro de mim,
ridículo e magnífico, um desejo imperioso de heroísmo, assim como só numa
tarde de sábado, no fastio da música do gramofone, podem jorrar os mais
diversos e absurdos pensamentos.}

Enfiava a cabeça no \textit{plush} cada vez mais e, na medida em que a dor se
tornava mais violenta, aumentava a tenacidade da paciência de suportá-la.

\textls[15]{Talvez exista dentro de nós um outro tipo de fome e um outro tipo de sede que
não as orgânicas --- algo em mim precisava, naquele momento, ser aplacado com
uma dor simples e aguda. Com força cada vez maior, eu mergulhava o rosto e o
esfregava nos pelos ásperos, torturando-me com um sofrimento que já começava
a me dilacerar.}

De repente, Edda estacou com um disco de gramofone na mão, observando-me
estupefata. Ao meu redor criou-se um silêncio que me incomodou
desmesuradamente. 

--- Mas o que é que há com ele? --- perguntou Edda. 

\textls[15]{Vi meu reflexo no espelho. Era ridículo, perfeitamente ridículo: no rosto, uma
mancha roxa deixava entrever aqui e ali gotinhas de sangue.}

\textls[15]{Com os olhos bem abertos e o rosto sangrando, ao observar-me ao espelho, não
pude deixar de constatar uma semelhança alegórica com a capa de um romance
popular que estava na moda, representando o tzar russo, ensanguentado, com
uma das mãos no maxilar, logo após sofrer um atentado.}

\textls[-10]{Muito mais do que a dor no rosto, torturava-me agora o destino miserável do
 meu heroísmo, que acabara me fazendo interpretar, em carne e osso, um
 capítulo dos Mistérios da Corte de Petrogrado.}\looseness=-1

Edda molhou um lenço no álcool e limpou meu rosto. Tanto ardia que fechei os
olhos. Sentia a pele ferver, ardendo como uma chama.

Imerso em tontura, desci a escada para que as ruas ávidas me recebessem de
novo em sua poeira e monotonia.


\section{xxxii} 

\letra{O}{} verão inchara caoticamente o parque, as árvores e a atmosfera,
 como em um desenho feito por um doido. Todo o seu sopro tórrido e amplo fez crescer monstruosamente as folhagens, que
se tornaram gordas e exuberantes.

\textls[10]{O parque extravasara como lava; as pedras ferviam; minhas mãos estavam pesadas
e vermelhas.}

\textls[10]{Naquela desolação mole e quente, eu passeava com a imagem de Edda por vezes
multiplicada em dezenas de exemplares, dez, cem, mil Eddas, uma ao lado da
outra no calor do verão --- estatuárias, idênticas e obcecantes.}

\textls[-20]{Havia nisso tudo um desespero violento e lúcido, que contaminava tudo o que eu
via e sentia. Em paralelo com minha vida simples e elementar, decorriam em
mim outras intimidades --- quentes, apaixonadas e secretas, como uma
espantosa e fantástica lepra interior.}\looseness=-1

\textls[10]{Eu compunha os detalhes de cenas imaginárias com a mais detalhada exatidão.
Via-me em quartos de hotel, com Edda deitada ao meu lado, enquanto a luz do
pôr do sol entrava pela janela filtrada por espessas cortinas cuja sombra
fina se desenhava porosa sobre o seu rosto adormecido. Via o desenho do
tapete ao lado da cama, sobre o qual estavam os seus sapatos e sua bolsa
entreaberta em cima da mesa, da qual saía um pedacinho do seu lenço. O
armário com espelho em que se refletia metade da cama e a pintura das flores
nas paredes\ldots{}}

Tudo isso produzia em mim um gosto bastante amargo\ldots{}

Eu perseguia mulheres desconhecidas no parque, andando atrás delas passo a
passo, até voltarem para casa, onde costumava ficar diante de uma porta
fechada, destruído, desesperado.

Num entardecer, segui uma mulher até a entrada de sua casa.

\textls[10]{A casa tinha um jardinzinho na frente, fracamente iluminado por uma lâmpada
elétrica.}

Num impulso brusco e imprevisível, abri o portãozinho e fui atrás da mulher
pelo quintal, rápida e sorrateiramente. Nesse meio tempo, ela já havia
entrado em casa sem perceber minha presença, de maneira que acabei ficando
sozinho no meio da alameda. Uma ideia estranha passou pela minha cabeça\ldots
{}

No meio do jardim havia um canteiro de flores. Num instante coloquei-me em seu
centro e, ajoelhado com a mão no coração, a cabeça descoberta, assumi uma
posição de prece. Eis o que eu desejava: permanecer assim o máximo possível,
imóvel, petrificado no meio do canteiro. Há muito me torturava essa vontade
de cometer um ato absurdo num lugar absolutamente desconhecido e, agora, isso
me viera espontaneamente, sem esforço, quase como uma alegria. O entardecer
zumbia quente ao meu redor e, nos primeiros instantes, senti uma enorme
gratidão por mim mesmo pela coragem de ter tomado semelhante decisão.

\textls[-10]{Minha ideia era ficar completamente imobilizado mesmo que ninguém surgisse
para me colocar para fora e eu tivesse de permanecer assim até a manhã do dia
seguinte. Pouco a pouco, minhas pernas e meus braços se enrijeceram e minha
posição ganhara uma casca interior de uma calma e uma imobilidade infindas.}\looseness=-1

Quanto tempo permanecera assim? De repente ouvi vozes na casa e em seguida a
luz de fora se apagou.

\textls[-15]{No escuro, sentia melhor o bater da brisa noturna e o isolamento em que me
encontrava, no quintal de uma casa desconhecida.}\looseness=-1

\textls[15]{Alguns minutos depois, a luz se acendeu novamente e depois apagou-se. Alguém
de dentro da casa a ligava e desligava para ver que efeito isso teria sobre
mim.}

Continuei imóvel, decidido a enfrentar experiências mais graves que um simples
jogo de luz. Mantinha a mão sobre o peito e o joelho no chão.

\textls[10]{A porta se abriu e alguém saiu no quintal, enquanto uma voz grossa de dentro
da casa gritava: ``Deixe, deixe-o em paz, ele vai embora sozinho''. A mulher
que eu perseguira pôs-se ao meu lado. Agora ela estava de roupão, de cabelos
soltos e chinelos. Olhou nos meus olhos e, durante alguns segundos, nada
disse. Permanecemos ambos calados. Por fim, ela pôs a mão no meu ombro e
disse com brandura: ``Vamos\ldots{} agora acabou'', como se quisesse fazer-me
entender que compreendera o meu gesto e que permanecera por algum tempo
calada justamente para permitir que ele se concluísse a seu modo.}

\textls[10]{Essa compreensão tão espontânea me desarmou. Levantei-me e sacudi a poeira das
calças. ``Suas pernas não estão doendo?'', perguntou-me ela. ``\ldots{} seu
não poderia ficar tanto tempo imóvel\ldots{}'' Quis dizer algo mas não
consegui senão murmurar um ``boa noite'' e fui embora às pressas.}

Todos os meus desesperos berravam de novo dolorosamente dentro de mim.


\section{xxxiii} 

\letra{E}{u} era um garoto alto, magro, pálido, com um pescoço fino saindo da
 gola demasiado larga da túnica. Os braços longos dependuravam-se da roupa
 como dois bichos recém-esfolados. Os bolsos explodiam de papeluchos e
 objetos. Era difícil encontrar no fundo deles um lenço para limpar a poeira
 das minhas botas, quando eu ia andar pelas ruas do \textit{centro}.

\textls[15]{Ao meu redor produziam-se as coisas mais simples e elementares da vida. Um
porco se coçava na cerca e eu ficava parado minutos a fio, contemplando. Nada
superava a perfeição do rangido do pelo áspero em contato com a madeira;
encontrava nele uma imensa satisfação e uma garantia tranquilizadora de que o
mundo continuaria existindo\ldots{}}

\textls[15]{Numa rua suburbana, havia um ateliê de escultura popular, onde eu também
passava muito tempo parado.}

\textls[10]{No ateliê havia milhares de coisinhas brancas e lisas, em meio às aparas
onduladas que caíam da plaina e enchiam o espaço com sua espuma rígida,
cheirando a resina.}

\textls[15]{O pedaço de madeira debaixo da ferramenta tornava-se mais fino, mais pálido,
 suas nervuras surgindo límpidas e bem traçadas, como sob a epiderme de uma
 mulher.} 

Ao seu lado, em cima de uma mesa, alastravam-se inúmeras bolas de madeira,
calmas e pesadas, que preenchiam toda a superfície da pele da minha mão com
um peso macio e inefável.

Havia também peças de xadrez com perfume de verniz fresco e paredes
completamente cobertas por flores e anjos.

Assim, por vezes surgiam da matéria alguns eczemas sublimes com supurações
rendadas, pintadas ou esculpidas.

\textls[-25]{O orvalho congelado pululava no frio do inverno, nas formas torneadas da água
 aflita, e no verão brotavam flores em milhares de explosões miúdas, com
 flamas petaladas vermelhas, azuis, alaranjadas.}\looseness=-1

Ao longo de todo o ano, o mestre escultor, com óculos de uma lente só, extraía
da madeira anéis de fumaça e flechas indígenas, conchas e samambaias, penas
de pavão e orelhas humanas.

\textls[10]{Em vão eu acompanhava seu trabalho vagaroso para surpreender o momento em que
o pedaço úmido e esfarrapado de madeira expiraria numa rosa petrificada.}

\textls[10]{Em vão eu mesmo tentava cometer o milagre, passo a passo. Numa das mãos
segurava o pedaço rude de pinho, pedregoso e desgrenhado, e eis que de sob a
plaina de repente saía algo escorregadio como um desmaio.}

\textls[15]{Talvez, no momento em que eu começava a desbastar a madeira, um sono profundo
me envolvesse, enquanto forças extraordinárias surgiam do ar como tentáculos
que a penetravam, produzindo o cataclisma.}

\textls[15]{Talvez, naqueles momentos, o mundo inteiro parasse e ninguém soubesse do tempo
que passara. Dormindo profundamente, o mestre decerto esculpira todos os
lírios das paredes e todos os violinos com espirais.}

Ao despertar, a madeira me revelava as linhas da sua idade, como a palma de
uma mão aberta exibindo as linhas do destino.

\textls[15]{A diversidade dos objetos que eu segurava, um após o outro, me aturdia. Em vão
pegava uma das bolas, meus dedos escorregavam em torno dela, apertavam sua
superfície, giravam-na e a soltavam para que rolasse\ldots{} Em vão\ldots
{} em vão\ldots{} era impossível entender qualquer coisa.}

Ao meu redor, a matéria dura e imóvel me circundava por todas as partes ---
aqui, sob a forma de bolas e esculturas ---, na rua, sob a forma de árvores,
casas e pedras; imensa e inútil, ela me encerrava dentro de si da cabeça aos
pés. Não importa para onde eu dirigisse meus pensamentos, a matéria me
rodeava, a começar pela minha roupa e continuando até os mananciais dos
bosques, passando por muros, por árvores, por pedras, por cristais\ldots{}

\textls[-5]{A lava da matéria saíra de cada canto da terra, petrificando-se ao ar livre,
sob a forma de casas com janelas, de árvores com galhos que não paravam de se
alçar para espetar o vazio, de flores que preenchiam lânguida e coloridamente
pequenos volumes de espaço curvo, de igrejas desabrochadas com a cúpula cada
vez mais alto, até a cruz delgada da ponta, onde a matéria interrompera seu
escorrimento vertical, incapaz de atingir maiores altitudes.}\looseness=-1

\textls[15]{Por toda a parte ela infestara o ar, irrompendo nele, preenchendo-o com os
abcessos enquistados das pedras, com os ocos feridos das árvores\ldots{}}

Caminhava enlouquecido pelas coisas que via e das quais o destino não me
permitia escapar.

\textls[15]{Acontecia-me por vezes, entretanto, encontrar um lugar isolado onde podia
deixar minha cabeça descansar à vontade. Ali, ao menos por um instante, todas
as vertigens se calavam para que eu me sentisse melhor.}

\textls[15]{Certa vez, encontrei um desses refúgios no lugar mais estranho e imprevisível
da cidade.}

Era realmente tão bizarro que nem eu teria imaginado pudesse constituir uma
toca tão admirável e solitária.

\textls[15]{Acho que só aquela sede ardente de preencher o vazio dos dias, de qualquer
modo e em qualquer lugar, fizera com que eu enveredasse por uma nova
aventura.}


\section{xxxiv} 

\letra{U}{m} \textls[-25]{dia, passando em frente ao teatro de variedades da
 cidade, tomei coragem e entrei. Era uma tarde calma e luminosa. Atravessei um pátio sujo, com muitas portas
fechadas --- no fundo encontrei uma delas aberta, dando acesso a uma escada.}\looseness=-1

No vestíbulo, uma mulher lavava roupa. O corredor cheirava a lixívia. Comecei
a subir os degraus, no início a mulher nada me disse, mas, quando cheguei na
metade da escada, ela virou a cabeça na minha direção e murmurou mais para si
mesma: ``Aha!\ldots{} você veio!\ldots{}'', confundindo-me decerto com um
conhecido.

\textls[-20]{Muito tempo depois daquele acontecimento, quando relembrei esse detalhe, as
palavras da mulher não me pareceram mais tão simples: elas talvez contivessem
o anúncio de uma fatalidade que presidia meus tormentos e que, por intermédio
da voz da lavadeira, demonstrava que os lugares de minhas aventuras eram
previamente estabelecidos; eu estava fadado a chegar até eles como se caísse
em emboscadas muito bem armadas. ``Aha, você veio, dizia a voz do destino,
veio porque tinha de vir, porque não tinha escapatória\ldots{}''}\looseness=-1

Cheguei a um corredor comprido, fortemente aquecido pelo sol que entrava por
todas as janelas que davam para o pátio.

\textls[15]{As portas dos aposentos estavam fechadas; não se ouvia o mínimo ruído em parte
alguma. Num canto, uma torneira de água pingava incessante. O corredor estava
quente e deserto; o ralo aspirava vagarosamente cada gota d’água, como se
sorvesse uma bebida gelada.}\looseness=-1

\textls[10]{No fundo, uma porta dava para o sótão, onde encontrei roupas penduradas num
varal. Atravessei o sótão e cheguei a um pequeno vestíbulo com quartinhos
limpos, recentemente caiados. Em cada um deles havia um baú e um espelho;
eram com certeza os camarins dos artistas de variedades.}

Num dos lados havia uma escada descendente, por onde cheguei até o palco do
teatro.

Vi-me portanto, de repente, no palco vazio, diante da sala deserta. Meus
passos se recobriam de uma estranha ressonância. Todas as cadeiras e mesas
estavam corretamente dispostas para uma representação. Encontrava-me sozinho
diante delas, no palco, em meio a um cenário de floresta.

Quis abrir a boca, sentia que tinha de dizer alguma coisa em voz alta, mas o
silêncio me petrificara.

De repente, vislumbrei o alçapão do ponto. Inclinei-me e olhei lá dentro.

Num primeiro momento nada pude distinguir, mas, pouco a pouco, descobri um
subsolo cheio de cadeiras quebradas e antigos objetos cenográficos.

\textls[-15]{Com movimentos prudentíssimos, enfiei-me no alçapão e desci.}\looseness=-1

\textls[10]{Por toda a parte, a poeira se depusera em grossas camadas. Num canto jaziam
estrelas e coroas de cartão dourado, que haviam provavelmente servido para um
espetáculo feérico. Num outro canto, uma mobília em estilo rococó, uma mesa e
algumas cadeiras com pernas quebradas. No meio, uma poltrona solene, quase no
gênero de um trono real.}

Nela afundei, exausto. Finalmente encontrava-me num lugar neutro, onde ninguém
podia saber que eu estava. Apoiei as mãos nos braços dourados da poltrona e
me deixei embalar por completo pela mais poderosa sensação de solidão.

\textls[10]{A escuridão ao meu redor dissipou-se um pouco; a luz do dia chegava suja e
empoeirada através de janelas duplas. Encontrava-me longe do mundo, longe das
ruas quentes e exasperantes, numa célula fresca e secreta, no fundo da terra.
No ar, o silêncio flutuava velho e embolorado.}

Quem poderia desconfiar do meu paradeiro? Tratava-se do lugar mais insólito na
cidade inteira, onde, ao dar-me conta de estar ali, eu me deixava invadir por
uma calma alegria.

Em torno de mim jaziam poltronas tortas, vigas empoeiradas e objetos
abandonados: era justamente o lugar comum a todos os meus sonhos. 

\textls[20]{Durante algumas horas permaneci sossegado, numa beatitude perfeita.}

\textls[10]{Até que por fim deixei o esconderijo pelo mesmo caminho que percorrera ao
chegar. Curiosamente, nem mesmo dessa vez não encontrei ninguém.}

O corredor parecia incendiado pelas chamas do sol que se punha. O ralo
continuava aspirando a água com sorveduras breves e regulares.

Na rua, tive por um momento a impressão de que nada daquilo acontecera. Minhas
calças, porém, estavam cobertas de poeira; deixei-as assim, sem limpar, como
prova mais à mão da longínqua e admirável intimidade que acabara de vivenciar
debaixo do palco. No dia seguinte, na mesma hora da tarde, fui subitamente
tomado pela nostalgia do subsolo isolado.

\textls[-20]{Tinha quase certeza de que, dessa vez, me encontraria com alguém, fosse no
corredor ou na sala. Durante algum tempo tentei resistir à tentação de
retornar. Eu estava, entretanto, abatido demais por causa do calor do dia
para que a possibilidade de um risco me assustasse. Eu tinha de voltar para
baixo do palco, custe o que custasse.}\looseness=-1

Passei pela mesma porta do pátio e subi pela mesma escada. O corredor estava
deserto como antes e não havia ninguém no sótão, nem na sala lá embaixo.

Em poucos minutos eis que me encontrava de novo no meu lugar, na poltrona
teatral, em minha deliciosa solidão. Meu coração batia mais forte; estava
mais do que emocionado com o êxito de minha extraordinária escapada.

\textls[-20]{Pus-me a acariciar extaticamente os braços da poltrona. Queria que o estado em
que eu me encontrava me penetrasse o mais profundamente, pesasse dentro de
mim o máximo possível, percorresse cada fibra do meu corpo para que eu a
percebesse como verossímil.}\looseness=-1

Dessa vez também fiquei durante muito tempo, tendo ido embora mais uma vez sem
encontrar ninguém\ldots{}

\textls[15]{Comecei a fazer essas visitas embaixo do palco regularmente, toda tarde.}

Como se já fosse algo perfeitamente normal, os corredores estavam sempre
vazios. Despencava na poltrona, esmagado de felicidade. A mesma luz azulada e
fresca de porão atravessava as janelas sujas. Impunha-se a mesma atmosfera
secreta de uma perfeita solidão da qual eu não conseguia me fartar.


\section{xxxv} 

\letra{C}{erta} \textls[15]{tarde, essas excursões diárias ao subsolo do teatro
 terminaram, da mesma estranha maneira como haviam começado.}

\textls[10]{Ao sair do sótão ao entardecer, uma mulher no corredor pegava água da
torneira.}

Passei lentamente ao seu lado, com o risco de que me perguntasse o que estava
fazendo ali. Ela, porém, continuou seu serviço, com aquele ar defensivo de
indiferença que as mulheres adotam ao pressentirem que um desconhecido quer
lhes falar.

\textls[10]{Detive-me no alto da escada, desejoso de travar uma conversa com ela. Havia,
por um lado, minha hesitação e, por outro, sua certeza amuada de que eu iria
lhe dirigir a palavra. O sussurro da água da torneira dividia friamente o
silêncio em dois campos bem distintos.}

Voltei-me e me aproximei dela. Ocorreu-me perguntar-lhe se por acaso conhecia
alguém que pudesse posar para mim como modelo para uns desenhos. Pronunciei a
palavra \textit{alguém} num tom perfeitamente relaxado, para que não se
subentendesse um mero desejo trivial de ver uma mulher nua, mas apenas a
preocupação exclusivamente artística e abstrata de desenhar.

Alguns dias antes, um estudante, decerto contando vantagem, me dissera que ele
chamava em sua casa em Bucareste jovens moças sob o pretexto de desenhá-las,
para depois se deitar com elas. Tinha certeza de que nada disso era verdade e
sentia, não sei como, falta de destreza na história do estudante, como se
recontasse por conta própria um caso do qual ouvira falar. Apesar de tudo,
ela permanecera bem marcada em minha memória e, agora, surgia uma ocasião
maravilhosa para utilizá-la. De maneira que a história de um desconhecido
distante, após atravessar o terreno infértil de outra pessoa, tornara-se de
novo suficientemente madura para aplicar-se à realidade.

A mulher não entendia, ou fingia não entender, embora eu me esforçasse em
explicar as coisas da maneira mais límpida.

\textls[-30]{Enquanto falava, uma porta se entreabriu e apareceu outra mulher.}\looseness=-1

Ambas se aconselharam aos sussurros.

--- Muito bem, vamos levá-lo então até a Elvira, ela não tem mesmo o que
    fazer --- disse uma delas.

\textls[-24]{Conduziram-me até um quartinho escuro e acanhado, que eu jamais percebera, ao
lado do sótão. No seu interior, em vez de janela havia duas fendas na parede,
pelas quais entrava uma corrente de ar frio. Era a cabine cinematográfica de
onde, durante o verão, projetavam-se os filmes no jardim do teatro de
variedades. No chão, viam-se as marcas do pedestal de cimento sobre o qual se
apoiava o projetor.}\looseness=-1

Num canto, uma mulher doente jazia na cama, coberta até a boca, batendo os
dentes. As outras mulheres foram embora, deixando-me sozinho no meio do
aposento.

Aproximei-me da cama. A doente tirou uma das mãos de sob o cobertor e a
estendeu na minha direção. Era uma mão comprida, fina, gelada. Disse-lhe em
poucas palavras que tudo aquilo era uma confusão, que eu havia sido trazido
até ela por engano. Balbuciei umas desculpas, dizendo-lhe vagamente do que se
tratava: uns desenhos para um concurso artístico.

Retendo de tudo o que lhe dissera apenas a palavra \textit{concurso}, ela me
respondeu com uma voz débil:

--- \ldots{} Certo\ldots{} certo\ldots{} vou prestar-lhe o meu concurso\ldots
     {} quando recobrar a saúde\ldots{} agora eu não posso\ldots{}

Ela tinha entendido que eu precisava de auxílio financeiro. Desisti de dar
explicações e fiquei sem graça por alguns instantes, sem saber como preparar
minha partida.

\textls[20]{Enquanto isso, ela começou a se lamentar com um tom bastante natural, como se
ainda se desculpasse por não me poder oferecer nada.}

--- Está vendo, estou com gelo em cima da barriga\ldots{} estou com
    calor\ldots{} estou com calor\ldots{} me sinto muito mal\ldots{}

\textls[20]{Fui embora entristecido, sem nunca mais voltar ao teatro de variedades.}


\section{xxxvi} 

\letra{O}{} \textls[-10]{outono chegou com seu sol avermelhado e suas manhãs embaçadas. As
 casas da periferia, amontoadas sob a luz, cheiravam a cal fresca. Eram dias
 desbotados de céu nublado como uma roupa suja. A chuva tamborilava
 infinitamente no parque desolado. Pesadas cortinas de água se agitavam pelas
 alamedas como num imenso salão vazio. Eu rumorejava pela grama úmida
 enquanto a água jorrava pelos meus cabelos e braços.}\looseness=-1

Nas ruelas sujas do subúrbio, quando a chuva parava, as portas se abriam e as
casas sorviam ar. Eram interiores humildes com armários torneados, buquês de
flores artificiais arrumados em cima da cômoda, estatuetas de gesso
abronzeado e fotografias dos Estados Unidos. Vidas sobre as quais eu nada
sabia, perdidas nos cômodos meio embolorados sob tetos baixos, sublimes em
sua indiferença resignada. Queria morar naquelas casas, deixar-me penetrar
por sua intimidade, dissolvendo em sua atmosfera, ácido fortíssimo, todos os
meus devaneios e amargores.

O que eu não daria para poder estar nesse ou naquele quartinho, entrando com
familiaridade e me atirando exausto sobre o velho divã, em meio a almofadas
de cretone florido? Ali eu poderia alcançar outra intimidade interior,
respirar outro ar e tornar-me outro eu\ldots{} Estendido no divã, poderia
contemplar, do interior da casa, por detrás das cortinas, a rua por onde eu
andava (eu procurava imaginar o mais exatamente possível o aspecto da rua
vista a partir do divã, pela porta entreaberta), poderia encontrar de
imediato dentro de mim lembranças que jamais vivi, lembranças desconhecidas
da vida de sempre que eu levava, lembranças pertencentes à intimidade das
estatuetas abronzeadas e ao globo velho do abajur com borboletas azuis e
violetas.

Como eu me sentiria bem naquele cenário barato e indiferente, que nada sabia
de mim\ldots{}

\textls[-5]{Na minha frente, a ruela suja espalhava sem parar sua pasta lamacenta. As
casas se seguiam umas às outras como dobras de um leque, umas brancas como
cubos de açúcar, outras pequenas, com o telhado puxado por cima dos olhos,
cerrando os maxilares como boxeadores. Podia ver uma carroça de feno ou, de
repente, coisas extraordinárias: uma pessoa andando pela chuva, levando nas
costas um candelabro com ornamentos de cristal, toda uma vidraria que
tilintava como sininhos em cima de seus ombros, enquanto pesados pingos
d'água escorriam por todas as facetas brilhantes. Em que consistia, no final
das contas, a gravidade do mundo?}\looseness=-1

\textls[10]{No parque, a chuva lavava as flores e as plantas murchas. O outono acendia
nelas incêndios acobreados, vermelhos e roxos como chamas que brilham mais
forte antes de se apagarem. Na feira, água e barro escorriam em tromba dos
montes enormes de verdura. No corte das beterrabas surgia de repente o sangue
escuro e vermelho da terra. Mais para lá jaziam batatas bondosas e brandas,
ao lado de cabeças cortadas amontoadas de repolhos enfolhados. Num canto
erguia-se a pilha, exasperadamente bela, das abóboras inchadas e horrendas,
com sua casca estorricada de tanto sol sorvido o verão inteiro.}

\textls[-15]{No meio do céu, as nuvens se agrupavam e depois se dissipavam, deixando entre
elas espaços raros como corredores infinitamente perdidos ou, noutras vezes,
vazios imensos que eram mais bonitos do que o vazio dilacerante que pairava
sempre sobre a cidade.}\looseness=-1

\textls[-5]{Naquele momento, a chuva caía ao longe, de um céu que não tinha mais limites.
Gostava da cor modificada da madeira molhada e das grades enferrujadas cheias
d'água, diante dos quintais bem comportados das casas, que o vento, misturado
a torrentes de água, atravessava como uma gigantesca crina de cavalo.}\looseness=-1

\textls[10]{Por vezes eu queria ser um cachorro para poder observar esse mundo molhado a
partir da perspectiva oblíqua dos animais, de baixo para cima, levantando a
cabeça. Aproximar-me mais do chão, com os olhares nele fixados, estreitamente
unido à coloração violeta do barro.}

Esse desejo, que há muito me habitava, revolveu-se frenético naquele dia de
outono no arrabalde\ldots{}


\section{xxxvii} 

\letra{N}{aquele} \textls[10]{dia, meus passos levaram-me até a margem da cidade, no
 descampado onde se organizava a feira de animais.}

\textls[-20]{Diante de mim, o arrabalde se estendia encharcado pela chuva como uma imensa
poça de lama. O esterco exalava um cheiro ácido de urina. Por cima de nós, o
sol se punha num cenário esfarrapado de ouro e púrpura. Diante de mim se
estendia, perdendo-se de vista, o barro quente e mole. Que outra coisa, além
dessa massa limpa e sublime de sujeira, poderia preencher meu coração de
alegria?}\looseness=-1

Primeiro, hesitei. Dentro de mim ainda lutavam, com as forças de um gladiador
moribundo, os últimos vestígios da educação. Num abrir e fechar de olhos,
porém, eles naufragaram numa noite negra, opaca, e perdi o controle de mim
mesmo.

Entrei na lama primeiro com um pé, depois com o outro. Minhas botas deslizaram
suavemente na massa elástica e pegajosa. Eu passara a ser uma excrescência do
barro, estava unido a ele na mesma substância, como se eu houvesse brotado da
terra.

Agora eu tinha certeza de que as árvores também não passavam de barro
solidificado, irrompido da crosta terrestre. A sua cor era eloquente. Mas só
as árvores? E as casas, e as pessoas? Sobretudo as pessoas. Todas as pessoas.
Não se tratava, claro, de nenhuma lenda estúpida dizendo que ``da terra
saíste e à terra retornarás''. Isso era vago demais, abstrato demais,
inconsistente demais diante do descampado lamacento. As pessoas e as coisas
despontaram justamente desse mesmo esterco e dessa mesma urina em que eu
afundava minhas \textit{concretíssimas} botas.

Em vão as pessoas haviam se coberto com sua pele branca e sedosa e se vestido
com roupas feitas de tecido. Em vão, em vão\ldots{} Nelas havia a lama
implacável, imperiosa e elementar; a lama quente, gordurosa e fedorenta. O
tédio e a estupidez com que preenchiam sua vida demonstravam isso de sobra.

\textls[15]{Eu mesmo era uma criação especial do barro, um missionário por ele enviado a
este mundo. Naqueles momentos, sentia isso perfeitamente enquanto recordava
minhas noites convulsas e escuridões febris, quando minha lama essencial se
arrojava inutilmente e se esforçava em atingir a superfície. Eu fechava os
olhos, mas o barro continuava fervendo na obscuridade, em murmúrios
ininteligíveis\ldots{}}\looseness=-1

Em derredor se estendia o arrabalde cheio de lama\ldots{} Essa era a minha
carne autêntica, despojada de roupas, despojada de pele, despojada de
músculos, despojada até a lama.

\textls[-12]{Sua umidade elástica e seu cheiro cru me acolhiam até suas profundezas, pois
eu lhes pertencia por completo. Algumas aparências, puramente acidentais,
como por exemplo os poucos gestos que eu era capaz de realizar, o cabelo liso
e fino, os olhos vítreos e úmidos, me separavam de sua imobilidade e de sua
antiquíssima sujeira. Era pouco, pouco demais diante da imensa majestade do
lodo.}\looseness=-1

Caminhava em todas as direções. Afundei os pés até o tornozelo. Uma chuva
mansa caía e, ao longe, o sol ia dormir atrás da cortina de nuvens
ensanguentadas e purulentas.

De repente me inclinei e meti as mãos no esterco. Por que não? Por que não?
Tinha vontade de berrar.

A pasta era morna e branda; minhas mãos se moviam nela sem dificuldade alguma.
Ao fechar o punho, o barro escapava por entre os dedos sob a forma de belas
fatias negras e brilhantes.

\textls[-20]{O que haviam feito as minhas mãos até agora? Onde haviam perdido seu tempo? Eu
as movia para lá e para cá, conforme ditava o coração. O que haviam sido até
então senão pobres aves prisioneiras, amarradas aos ombros com uma terrível
corrente de couro e músculos? Pobre aves, fadadas a voar em gestos estúpidos
de bom comportamento, aprendidos e ensaiados como algo de grande
importância.}\looseness=-1

\textls[-15]{Pouco a pouco, elas voltaram a ser selvagens e passaram a gozar de sua velha
liberdade. Agora elas revoluteavam a cabeça no esterco, arrulhavam como
pombos, batiam as asas, felizes\ldots{} felizes\ldots{}}\looseness=-1

\textls[-15]{De alegria, comecei a agitá-las por cima da cabeça, fazendo-as voar. Gotas
grandes de lama caíam sobre meu rosto e minha roupa.}\looseness=-1

Por que as teria limpado? Por quê? Era apenas o começo; meu gesto não gerava
nenhuma consequência grave, nenhum tremor no céu, nenhum abalo da terra. Logo
passou pelo meu rosto uma mão cheia de sujeira. Fui tomado por uma imensa
alegria, há muito não me sentia tão bem disposto. Levei ambas as mãos ao
rosto e ao pescoço e, em seguida, esfreguei-as no cabelo. 

A chuva começou de repente a cair mais fina e mais cerrada. O sol ainda
iluminava o arrabalde como um imenso candelabro no fundo de um salão de
mármore cinzento. A chuva caía à luz do sol, uma chuva dourada, cheirando a
roupa lavada.

O arrabalde estava deserto. Aqui e ali, um montinho de espigas secas comidas
pelas vacas. Peguei uma delas e a abri atentamente. Tremendo de frio, minhas
mãos cheias de barro limpavam com dificuldade as folhas de milho. A
atividade, porém, me interessava. Havia muito o que descobrir numa espiga
seca. Ao longe, no fim do descampado, entrevia-se uma choupana coberta de
junco. Corri até ela e me protegi sob o beiral. O telhado era tão baixo, que
eu batia a cabeça nele. A terra junto à parede estava perfeitamente seca.
Deitei-me no chão. Apoiei a cabeça nuns sacos velhos e, de pernas cruzadas,
podia agora dedicar-me livremente a uma análise minuciosa da espiga.

Estava feliz em poder me debruçar sobre uma pesquisa tão apaixonante. Os
canais e as cavidades da espiga me preenchiam de entusiasmo. Abri-a com os
dentes e encontrei dentro dela uma penugem macia e adocicada. Esse forro era
maravilhoso para uma espiga; se também as pessoas tivessem artérias
revestidas por uma penugem tão suave, com certeza a sua escuridão seria mais
doce e mais fácil de suportar.

Ao fitar a espiga, o silêncio dava uma risada mansa dentro de mim, como se, do
lado de dentro, alguém estivesse fazendo bolhas de sabão.

\textls[-5]{Chovia, o sol brilhava e, ao longe, em meio à névoa, a cidade fumegava como
uma montanha de lixo. Alguns telhados e torres de igrejas cintilavam
estranhamente nesse crepúsculo úmido. Estava tão feliz que nem sabia que ação
absurda deveria cometer primeiro: analisar a espiga, deitar-me ou observar a
cidade ao longe.}\looseness=-1

\textls[-10]{Um pouco mais além dos meus pés, onde o barro começava, uma rã de repente
pôs-se a pular. Primeiro se aproximou de mim, para em seguida mudar de ideia
e se dirigir ao arrabalde. ``Adeus, rã!'', gritei enquanto ela se
distanciava, ``Adeus! Meu coração se despedaça por você me abandonar tão
cedo\ldots{} Adeus, minha bela!\ldots{}'' Comecei a improvisar uma longa fala
dirigida à rã e, ao seu término, atirei a espiga na sua direção, na tentativa
de acertá-la\ldots{}}\looseness=-1

Finalmente, observando fixamente as vigas acima de mim, fechei os olhos
cansados e adormeci.

Um sono profundo logo se apoderou de mim até a medula dos ossos.

\textls[10]{Sonhei que passeava pelas ruas de uma cidade cheia de poeira, muito ensolarada
e repleta de casas brancas; talvez uma cidade oriental. Caminhava ao lado de
uma mulher vestida de preto, com grandes véus de luto. Estranhamente, porém,
a mulher não tinha cabeça. Os véus estavam muito bem arranjados no lugar onde
a cabeça deveria estar, mas no lugar dela só havia um buraco escancarado, uma
esfera vazia até a nuca.}

\textls[-15]{Ambos estávamos muito apressados e, um ao lado do outro, acompanhávamos uma
carroça com o símbolo da Cruz Vermelha, em que se encontrava o cadáver do
marido da senhora em negro.}\looseness=-1

Dei-me conta de que estávamos em tempo de guerra. De fato, logo chegamos a uma
estação de trem e descemos as escadarias até um subsolo com uma iluminação
elétrica muito fraca. Havia acabado de chegar um comboio de feridos, e as
enfermeiras iam e vinham agitadas na plataforma, com cestinhos de cerejas e
biscoitos que eram repartidos entre os inválidos do trem.

\textls[15]{De repente, desembarcou de um compartimento da primeira classe um senhor
gordo, bem vestido, com a insígnia de uma condecoração na lapela.}

\textls[15]{Usava monóculo e polainas brancas. A careca se escondia debaixo de alguns fios
de cabelo prateado. Nos braços, um cachorro pequinês branco, com olhos como
se fossem duas bolinhas de ágata boiando em óleo.}

Por alguns instantes ele andou para cima e para baixo pela plataforma, como se
procurasse algo. Finalmente, encontrou: tratava-se da vendedora de flores.
Escolheu do cesto alguns buquês de cravos vermelhos e pagou, tirando o
dinheiro de uma carteira elegante e delgada, com um monograma de prata.

\textls[10]{Em seguida, subiu de volta ao trem e, pelo vidro, observei como instalara o
cachorrinho em cima da mesinha junto à janela e lhe dava de comer, um por
vez, os cravos vermelhos, que o bicho engolia com visível apetite\ldots{}}

Um sobressalto horrível me despertou.

\textls[10]{Agora chovia muito forte. As gotas golpeavam o chão bem ao meu lado e tive de
me estreitar junto à parede. O céu se enegreceu e, ao longe, não se via mais
a cidade.}

Embora estivesse com frio, minha face ardia. Sentia muito bem a sua
efervescência por debaixo da crosta de barro ressequido. Ao tentar me
levantar, levei um choque elétrico nas pernas. Ficaram completamente
amortecidas e tive de descruzá-las devagar, uma depois da outra. As meias
estavam frias e molhadas.

\textls[10]{Pensei em me abrigar dentro da choupana. Mas a porta estava trancada e, no
lugar de janela, a casinha só tinha uma abertura fechada com tábuas pregadas.
Como o vento dispersava a água da chuva em todas as direções, eu não tinha
como me proteger dela. Começara a anoitecer. Em pouco tempo, o arrabalde
afundou na escuridão. Bem na sua margem, da direção de onde eu viera, uma
taberna se iluminara.}

Fui imediatamente para lá; tencionava entrar, beber alguma coisa, estar num
lugar aquecido em meio às pessoas e ao fedor do álcool. Rebusquei meus bolsos
e não encontrei um centavo. Diante da taberna, a chuva caía abundante através
de uma cortina de fumaça e vapores que bafejava fedorenta do interior.

Tinha de me decidir, ir para casa, por exemplo. Mas como?

No lamentável estado de sujeira em que me encontrava, não era possível. Mas
abdicar da sujeira eu também não queria. Um amargor indescritível tomou conta
da minha alma, como aquele que alguém sente ao perceber que, diante de si,
não tem absolutamente mais nada por fazer, nada por realizar.

\textls[-5]{Comecei a correr pelas ruas na escuridão, saltando por sobre as poças e
entrando até os joelhos em algumas delas. Num determinado momento, o
desespero cresceu dentro de mim compelindo-me a gritar e bater com a cabeça
nas árvores. Entretanto, logo toda a tristeza se coagulou num pensamento
tranquilo e suave. Agora eu sabia o que deveria fazer: tendo em vista que
nada mais podia continuar, só me restava acabar com tudo. O que eu deixaria
para trás? Um mundo úmido, feio, em que chovia devagar\ldots{}}\looseness=-1


\section{xxxviii} 

\letra{E}{ntrei} em casa pela porta dos fundos. Passei sorrateiro pelos
 cômodos, evitando ver-me no espelho. Procurava algo rápido e eficaz que
 despejasse de uma vez nas trevas tudo o que eu via e sentia, assim como se
 esvazia uma carroça carregada de pedras ao se retirar uma tábua do fundo.

Comecei a remexer pelas gavetas, em busca de um veneno fulminante. Enquanto
rebuscava, nenhum pensamento vinha-me à cabeça; tinha que terminar, o mais
rápido possível. Era como se eu tivesse de cumprir uma tarefa como qualquer
outra. 

\textls[-7]{Encontrava toda sorte de objetos que não podiam servir para nada: botões,
cordões, barbantes coloridos, papeluchos, tudo com um cheiro forte de
naftalina. Tantas, tantas coisas, todas incapazes de provocar a morte de uma
pessoa. Eis o conteúdo do mundo nos momentos mais trágicos: botões, barbantes
e cordões\ldots{}}\looseness=-1

\textls[-10]{No fundo de uma gaveta, encontrei uma caixa cheia de comprimidos brancos.
Podia ser um veneno, como também podia ser um remédio inofensivo. Veio-me à
mente, porém, que se tomados em grande quantidade deveriam ser tóxicos de
qualquer modo.}\looseness=-1

\textls[10]{Pus um deles na língua, que soltou na boca um gosto meio salgado e insípido.
Ao triturá-lo entre os dentes, seu pó absorveu toda a saliva. Minha boca
ficou seca.}

Havia muitos comprimidos na caixa, mais de trinta. Fui até a torneira do
quintal e, aos poucos, pacientemente, pus-me a engoli-los. Para cada
comprimido eu tomava um gole de água; precisei de muito tempo até terminar a
caixa. Os últimos não escorregavam mais, era como se a garganta tivesse
inchado.

\textls[10]{Uma escuridão completa envolvia o quintal. Sentei-me num degrau e comecei a
esperar. No meu estômago se desencadeou uma ebulição terrível, mas, fora
isso, eu me sentia bem, o farfalhar da chuva parecia-me agora
indescritivelmente íntimo. Parecia compreender o meu estado, penetrando-me
profundamente no intuito de me fazer bem.}

\textls[15]{O quintal se transformou numa espécie de salão em que eu me sentia leve, cada
vez mais leve. Todas as coisas se esforçavam desesperadas em não afundar na
obscuridade. De repente, dei-me conta de que transpirava exageradamente. Meti
a mão por dentro da camisa e a retirei molhada. Ao meu redor, o vazio crescia
vertiginosamente. Em casa, quando me atirei à cama, o suor me escaldava da
cabeça aos pés.}


\section{xxxix} 

\letra{E}{ra} \textls[10]{uma cabeça bonita, extraordinariamente bonita. Aproximadamente três vezes maior que uma cabeça humana, girando lentamente sobre um eixo de latão que a atravessava do pescoço até a moleira.} 

\textls[10]{No início, eu só conseguia ver a nuca. De que material podia ser? Tinha um
brilho apagado de azulejo velho, com nuances de marfim. Pequenos desenhos
azuis estavam impressos por toda a sua superfície, como uma espécie de
filigrana que se repetia geometricamente, como o desenho de um linóleo. De
longe, parecia uma caligrafia miúda e fina aplicada sobre um papel
marfinizado; algo de uma beleza inimaginável.}

\textls[10]{Assim que a cabeça começou a se mover, rodando sobre o eixo, fui invadido por
uma profunda vertigem. Sabia que, dentro de alguns segundos, apareceria a
face do crânio --- um rosto terrível e assustador.}

\textls[-5]{Era um rosto aliás bem formado, com todos os relevos humanos habituais: olhos
fundos, queixo bem proeminente e um triângulo escavado debaixo de cada maçã,
como nas pessoas magras.}\looseness=-1

\textls[10]{Por outro lado, a pele era fantástica: formada por finas lâminas de carne
delgada, uma junto à outra, como as lamelas cor de café do lado de baixo dos
cogumelos.}

\textls[10]{As lâminas eram tantas e tão apertadas umas às outras, que, ao observar a
cabeça fechando um pouco as pálpebras, nada parecia anormal, e as minúsculas
linhas se assemelhavam a sombras hachuradas de uma gravura em cobre.}

Às vezes, durante o verão, ao contemplar as castanheiras de longe, carregadas
de folhas, percebia que elas se pareciam com enormes cabeças cravadas em
troncos, com a face profundamente escavada, assim como as lamelas da \textit
{minha} cabeça.

Quando o vento soprava pelas folhas, a face ondulava como as ondas de um
trigal.

\textls[-25]{Da mesma maneira, a cabeça tremulava às oscilações do pedestal.}\looseness=-1


\section{xl} 

\textls[10]{\letra{P}{ara} confirmar que a face era feita de lamelas, bastava afundar só
 um pouquinho o dedo na carne. O dedo entrava sem a menor resistência, como
 se mergulhasse numa pasta úmida e mole. Ao retirá-lo, as lamelas voltavam ao
 lugar sem deixar marcas.}\looseness=-1

Uma vez, na infância, assisti à exumação e ao sepultamento de um cadáver.

Era o de uma menina que morrera jovem e fora enterrada com vestido de noiva. 

\textls[13]{O espartilho de seda se desfizera em faixas compridas e sujas e, em alguns
pontos, restos de bordado se misturavam à terra. O rosto, porém, parecia
intacto, tendo mantido quase todos os traços. Sua cor era roxa, ao passo que
a cabeça parecia ter sido modelada em papel machê.}

\textls[10]{Quando o caixão foi retirado da cova, alguém passou a mão pelo rosto da morta.
Naquele momento, todos nós ali presentes tivemos uma surpresa macabra: aquilo
que acreditávamos fosse o rosto bem preservado não passava de uma camada de
mofo com dois dedos de espessura. O mofo substituíra a pele e a carne do
rosto em toda a sua profundidade, mantendo-lhe a forma intacta. Por baixo,
era puro esqueleto.}

Assim também era a \textit{minha} cabeça, a única diferença é que, ao invés de
mofo, ela era recoberta por lâminas de carne. Por entre elas, porém, eu era
capaz de tocar o osso com o dedo.

\textls[-20]{Embora horrenda, a cabeça era um refúgio garantido contra o ar.}\looseness=-1

Por que contra o ar? No quarto, o ar estava eternamente em movimento, viscoso,
pesado, escorrendo, sempre prestes a endurecer em estalactites negras e
horrorosas.

\textls[-20]{Foi nesse ar que a cabeça apareceu pela primeira vez, tendo-se criado ao seu
redor um vazio como uma auréola que crescia sem parar.}\looseness=-1

\textls[10]{Fiquei tão alegre e satisfeito com o seu surgimento, que tinha vontade de dar
risada. Mas como eu poderia rir na cama, no escuro da noite?}\looseness=-1

\textls[-3]{Comecei a nutrir um amor insaciável pela cabeça. Era a coisa mais preciosa e
mais íntima que eu possuía. Ela vinha do mundo das trevas, de onde penetrava
em mim apenas um zumbido miúdo, como uma ebulição contínua no crânio. Que
outras coisas se encontravam ali? Abria bem os olhos e, em vão, perscrutava a
escuridão. Além da cabeça marfínea, não havia mais nada.}\looseness=-1

Perguntei-me, com certo receio, se essa cabeça não viria a se converter no
centro de todas as preocupações da minha vida, substituindo-as, uma a uma,
até que, finalmente, eu ficasse no escuro a sós com ela. Naquele momento, a
vida parecia imbuir-se de um sentido preciso, verdadeiro. Por enquanto, ela
crescia no ar como um fruto pleno que atingira a maturidade.

A cabeça era o meu refresco e a minha felicidade pessoal. Talvez, se
pertencesse ao mundo inteiro, uma terrível catástrofe teria acontecido. Um
único momento de felicidade plena seria capaz de estupeficar o mundo para
todo o sempre.

Contra a \textit{cabeça} lutava incessante, cada vez mais impotente, o
escorrer gorduroso do ar.

\textls[10]{Por vezes, ao seu lado, aparecia o meu pai, vago e indistinto, como uma massa
de vapores esbranquiçados. Sabia que ia colocar a mão sobre a minha testa; a
mão estava fria. Tentava lhe explicar a luta entre a cabeça e o ar, enquanto
sentia meu pai abrindo minha camisa e deslizando o termômetro até minha
axila, como uma lagartixa delgada de vidro.}

Em torno da cabeça desencadeou-se um movimento enervante, como um drapejar de
bandeira.

Impossível interrompê-lo; a flâmula continuava a esvoaçar.

\textls[-15]{Lembrei-me do dia em que, durante a hora do chá, no andar de cima da casa dos
Weber, Paul deixara a mão pendurada para baixo ao lado da cadeira, enquanto
Edda, da cama, com a ponta do sapato, batia na palma dele de brincadeira. Com
o passar do tempo, esse gesto impregnara-se de uma virulência incomum. Ao
recordá-lo, o sapato começava a arranhar freneticamente a mão de Paul até
produzir uma pequena ferida e, em seguida, um buraco na carne. O mecanismo
enfadonho do sapato não cessava um só instante: escavava sem parar a palma
esburacada, o braço inteiro, o corpo todo\ldots{}}\looseness=-1

\textls[15]{Assim se desencadeara o movimento da bandeira no quarto. Agora havia o risco
de ele fazer um furo em tudo, de me devorar, talvez\ldots{}}\looseness=-1

Gritei desesperado, encharcado de suor.

--- Quanto? --- perguntou uma voz na sombra.

--- Trinta e nove --- respondeu o meu pai, indo embora e deixando-me à mercê
    das tormentas que aumentavam.


\section{xli} 

\letra{D}{e} manhã, a convalescença se anunciou como uma extrema fragilidade
 da luz. No quarto em que eu dormia, ela entrava por uma janela fixada no
 teto. O volume do quarto perdera estranhamente a densidade. A claridade
 tornava os objetos mais leves e, por mais fundo que eu respirasse,
 permanecia no peito um vazio amplo, como se uma importante quantidade de mim
 houvesse mesmo desaparecido.

Em meio aos lençóis quentes, migalhas escorregavam por baixo das coxas. O pé
procurava a barra de ferro da cama, e o ferro o traspassava com uma faca de
frieza.

\textls[-15]{Tentei sair da cama. Tudo corria conforme minhas suspeitas: o ar,
inconsistente demais, não era capaz de me sustentar. Caminhava difuso por
ele, como se atravessasse um riacho morno e vaporoso.}\looseness=-1

Sentei-me numa cadeira, debaixo da janela do teto. Ao meu redor, a luz
expulsava a exatidão dos objetos como se os lavasse com força até despi-los
de seu brilho.

A cama, no canto, jazia mergulhada na escuridão. Como conseguira eu, durante a
febre naquele escuro, distinguir nas paredes cada grão de cal?

\textls[10]{Comecei a me vestir lentamente; as roupas também estavam mais leves do que de
costume. Dependuravam-se do meu corpo como pedaços de mata-borrão, exalando o
cheiro de lixívia do ferro de passar.}\looseness=-1

\textls[10]{Flutuando por águas cada vez mais raras, saí de casa. O sol me entonteceu de
imediato. Manchas imensas de brilhos amarelos e esverdeados cobriam
parcialmente as casas e os transeuntes. A própria rua parecia fraca e fresca,
como se também ela tivesse saído da febre de uma grave enfermidade.}

Os cavalos das carruagens, cinzentos e derreados, moviam-se anormais. Ora
trotavam extremamente devagar, pesados e bamboleantes --- ora desembestavam,
respirando com energia pelas narinas para não caírem débeis no meio do
asfalto.

\textls[15]{O longo corredor de casas balançava levemente ao vento. De longe vinha o
cheiro forte do outono. ``Um belo dia de outono!'', disse para mim mesmo.
``Um esplêndido dia de outono!''\ldots{} Pus-me a passear muito lentamente ao
longo das casas cobertas de poeira. Na vitrine de uma livraria, encontrei um
brinquedo mecânico agitando-se.}\looseness=-1

\textls[-10]{Era um palhacinho vermelho e azul, que batia dois minúsculos pratos de latão.
Estava fechadinho nesse seu quarto, na vitrine, entre livros, bolas e
tinteiros, batendo os pratos alegre e indolente.}\looseness=-1

\textls[-15]{Enternecido, meus olhos encheram-se de lágrimas. Havia tanta pureza, tanto
frescor e tanta beleza naquele cantinho de vitrine!}\looseness=-1

Não podia haver lugar melhor nesse mundo para bater pratos tranquilamente,
envergando belas roupas coloridas.

\textls[10]{Eis finalmente algo simples e límpido depois de tanta febre. Na vitrine, a luz
do outono caía mais íntima, mais agradável. Que bom seria se eu pudesse
substituir aquele palhacinho alegre! Em meio a livros e bolas, rodeado por
objetos puros, dispostos com esmero em cima de uma folha de papel azul. Tim!
Tim! Tim! Que gostoso, que gostoso é estar na vitrine! Tim! Tim! Tim!
Vermelho, verde, azul; bolas, livros e tintas. Tim! Tim! Tim! Que belo dia de
outono!\ldots{}}

\textls[10]{Aos poucos, porém, despercebidamente, os movimentos do palhaço começaram a
diminuir. Primeiro, os pratos não encostavam mais um no outro, e depois, de
repente, o palhaço parou com os braços fixos no ar.}

\textls[10]{Dei-me conta, quase horripilado, de que o palhaço parara de se mexer. Algo
dentro de mim petrificou-se dolorosamente. Um instante de alegria e beleza
congelara no ar.}

Abandonei rapidamente a vitrine e me dirigi a um pequeno parque público no
centro da cidade.

As castanheiras livraram-se de suas folhas amareladas. O velho restaurante
feito de tábuas estava fechado e, diante dele, vários bancos quebrados
estavam jogados em desordem.

\textls[-15]{Não sei como, mergulhei numa profunda escuridão, de maneira que despertei
quase deitado de costas, olhando para o céu. O sol enviava por entre os
galhos uma luz despedaçada, cheia de cristais.}\looseness=-1

Por algum tempo permaneci assim, com olhares perdidos para o alto,
enfraquecido, indescritivelmente enfraquecido.

\textls[-20]{Sentou-se de repente ao meu lado um garoto robusto, com as mangas da camisa
arregaçadas, de pescoço vermelho e forte e manzorras sujas. Por alguns
instantes, coçou a cabeça com todos os dez dedos e, em seguida, tirou do
bolso da calça um livro que pôs-se a ler.}\looseness=-1

Segurava o livro apertado na mão para que o vento não virasse as folhas,
enquanto lia murmurando alto; de vez em quando passava uma mão no cabelo como
se quisesse entender melhor.

\textls[10]{Dei uma tossida significativa e o interpelei: ``O que você está lendo?'',
perguntei-lhe, atirado no banco, com os olhos fixos nos galhos das árvores.}

\textls[10]{O garoto pôs o livro na minha mão como se eu fosse um cego. Era uma longa
história versificada sobre bandoleiros, um livro gordurento, cheio de manchas
de óleo e de sujeira; via-se bem que passara por muitas mãos. Enquanto dava
uma olhada, ele se levantou e ficou em pé na minha frente, forte, seguro de
si, com as mangas arregaçadas e o pescoço descoberto.}

Algo tão prazeroso e calmo quanto bater pratos numa vitrine.

--- E\ldots{} sua cabeça não dói enquanto lê?\ldots{} --- perguntei,
    devolvendo-lhe o livro.

Ele parecia não compreender.

\textls[10]{--- Por que me doeria? Não dói nada --- disse ele, sentando-se de novo no
    banco para continuar a leitura.}

\textls[-10]{Existia, assim, uma categoria de coisas no mundo de cuja participação o
destino me vedava, palhaços indolentes e mecânicos, garotos robustos que
nunca têm dor de cabeça. Ao meu redor, por entre as árvores, sob a luz do
sol, soprava uma corrente vívida e ampla, repleta de vida e pureza. Estava
fadado a permanecer eternamente à sua margem, entupido de escuridão e surtos
de desmaio.}\looseness=-1

Estiquei as pernas no banco e, apoiando as costas numa árvore, encontrei uma
posição bastante cômoda. Definitivamente, o que me impedia de ser eu também
forte e indolente? Sentir em mim a circulação de uma seiva fresca e vigorosa,
assim como circula por milhares de galhos e folhas de árvores, estar na
vertical sem motivo sob a luz do sol, em pé, sóbrio, com uma vida segura e
bem definida, fechada em mim como uma armadilha\ldots{}

\textls[10]{Para isso, deveria talvez, antes de tudo, tentar respirar mais fundo e mais
devagar: eu não respirava certo, o meu peito estava sempre cheio ou vazio
demais. Mas comecei a respirar o ar com segurança. Senti-me melhor depois de
alguns minutos. Um fluido de perfeição --- que, embora fraco, sentia
inflar-se a cada instante --- começou a correr pelas minhas veias. O barulho
da rua relembrou-me a cidade ao longe, mas agora a cidade girava muito
lentamente ao meu redor como um disco de gramofone. Tornara-me uma espécie de
centro e eixo do mundo. O essencial era eu não perder o equilíbrio.}

\textls[5]{Uma vez, num circo, de manhã, enquanto os artistas ensaiavam, assisti a uma
cena que agora voltava-me à mente\ldots{} Um aficionado do circo, um simples
espectador sem qualquer preparo, subiu, sem hesitar, com muita coragem, na
pirâmide de cadeiras e mesas sobre a qual subira pouco antes o acróbata do
circo. Admirávamos todos a precisão com que ele escalava a perigosa
construção; o frenesi de conseguir ultrapassar os primeiros obstáculos
embriagava o aficionado com uma espécie de ciência do equilíbrio, plena de
inconsciência, que o fazia colocar a mão no lugar exato, esticar a perna com
precisão e encontrar nele mesmo o peso mínimo com o qual devia abordar o
nível seguinte. Aturdido e feliz com a segurança de seus próprios gestos, em
poucos segundos ele alcançou o topo. Uma vez lá em cima, porém, ocorreu algo
absolutamente incomum: ele de repente se dera conta da fragilidade do ponto
de apoio em que se encontrava, bem como de sua extraordinária audácia.
Batendo os dentes, pediu com uma voz esmaecida que lhe trouxessem uma escada,
ao mesmo tempo que recomendava inúmeras vezes que a segurassem bem, sem
balançá-la. O corajoso aficionado desceu infinitamente precavido, degrau por
degrau, transpirando da cabeça aos pés, pasmo e irritado com a própria ideia
de ter-se metido lá em cima.}\looseness=-1

A minha atual posição no jardim era a de quem está no topo da frágil pirâmide.
Sentia claramente como circulava em mim uma seiva nova e forte, mas tinha de
me esforçar por não cair da altura de minha admirável certeza.

Fui visitado pelo pensamento de que era assim que eu deveria ver Edda, calmo,
seguro de mim, pleno de luz; faz tempo que eu não passava por lá. Queria
apresentar-me diante de alguém, pelo menos uma vez, inteiro e inabalável.

Calado e imponente como uma árvore. Isso mesmo --- como uma árvore. Enchi o
peito de ar e, estirando-me bem de costas, dirigi uma calorosa saudação de
camaradagem aos galhos acima de mim. Havia algo de rude e simples na árvore,
que se aparentava maravilhosamente com minhas novas forças. Acariciava o
tronco como se batesse no ombro de um amigo. ``Camarada árvore!'' Quanto mais
contemplava com atenção a coroa infinitamente esparramada dos galhos, melhor
sentia como a carne se dividia em mim, gerando ocos por onde o ar vívido de
fora começava a circular. O sangue subia pelas veias majestoso e pleno de
seiva, espumando com a efervescência da vida simples.

\textls[10]{Levantei-me. Por um instante, os joelhos se dobraram inseguros como se
quisessem comparar, numa única hesitação, toda a minha força e fraqueza. Com
passos largos, pus-me na direção da casa de Edda.}

\textls[10]{A porta pesada de madeira que dava para o terraço estava fechada. Sua
imobilidade me desconcertou um pouco. Todos os meus pensamentos se
esfumaram.}

\textls[15]{Pousei a mão na maçaneta e a apertei. ``Coragem'', disse para mim mesmo;
detive-me, porém, para retificar. ``Coragem? Só os tímidos precisam de
coragem; os normais, os fortes não têm coragem nem covardia, eles
simplesmente abrem as portas, assim\ldots{}'' A escuridão fresca do primeiro
aposento me abrangeu com uma atmosfera plena de calma e alegria, como se
estivesse me esperando há muito tempo.}\looseness=-1

\textls[-15]{Dessa vez, a cortina de miçangas que se recompôs atrás de mim produziu um
tilintar estranho, que me deu a impressão de estar sozinho, numa casa
deserta, às margens do mundo. Seria essa a sensação de extremo equilíbrio no
topo da pirâmide de cadeiras?}\looseness=-1

Bati com violência à porta de Edda.

Alarmada, disse-me que entrasse.

Por que meus passos eram tão silenciosos?

\textls[15]{``Passos silenciosos?'' Parecia-me, contudo, que a presença de alguém como eu,
ou melhor, de uma árvore, deveria ser pressentida de longe.}

No aposento, entretanto, não se produziu a mínima surpresa, tensão ou emoção.

Por alguns segundos os pensamentos me precederam de maneira ideal, com uma
grande perfeição e sobriedade de gestos. Vi-me avançando com bastante
segurança e, num movimento desembaraçado, pus-me aos pés de Edda, na cama
onde estava deitada. Minha verdadeira pessoa, porém, fora deixada para trás
por esses belos projetos, como um reboque inútil e quebrado.

\textls[15]{Edda convidou-me a sentar, de maneira que me instalei numa cadeira a grande
distância dela.}

\textls[15]{O pêndulo batia entre nós um tique-taque enervante e sonoro. Curioso: o
tique-taque crescia como o fluxo e refluxo do mar, aproximando-se de Edda sob
a forma de ondas até quase não ouvi-lo mais, para em seguida retornar inchado
na minha direção, tão violento que estourava meus ouvidos.}

--- Edda --- comecei a falar, interrompendo o silêncio ---, permita-me lhe
    dizer uma coisa muito simples\ldots{}

Edda não respondeu.

--- Edda, você sabe o que eu sou?

--- O quê?

--- Uma árvore, Edda, uma árvore\ldots{}

Todo esse breve diálogo ocorreu, é claro, estritamente no meu interior;
nenhuma palavra foi realmente pronunciada.

Edda aninhou-se na cama, estreitando os joelhos na direção do seu corpo e
cobrindo-os com o penhoar. Em seguida, pôs as mãos sob a cabeça e começou a
me fitar atentamente. Daria qualquer coisa, com a maior alegria, para que ela
encontrasse outro ponto no aposento para onde olhar.

\textls[10]{Vi de repente numa prateleira um grande buquê de flores dentro de um vaso.
Minha salvação. Como eu não as vira antes? Todo o tempo eu concentrara meus
olhares naquele canto, desde que entrara. Para comprovar a sua aparição,
olhei por um instante para outro lado e depois retornei a elas. Lá no seu
lugar estavam elas, imóveis, grandes, vermelhas\ldots{} Como é que eu não as
observara? Comecei a duvidar da minha certeza de árvore. Eis que um objeto
surgira naquele quarto, num lugar onde antes não havia nada. Será que a minha
visão era sempre clara? Talvez houvessem permanecido no meu corpo vestígios
de fraqueza e escuridão que ainda circulavam pela minha nova luminosidade,
como nuvens num céu resplandecente, cobrindo-me a visão ao passarem pelo
humor vítreo, assim como as nuvens do céu por vezes tapam subitamente o sol,
mergulhando em sombras uma parte da paisagem.}\looseness=-1

--- Que flores bonitas aquelas --- disse a Edda.

--- Que flores?

--- Aquelas ali, na prateleira.

--- Que flores?

--- Aquelas dálias vermelhas são tão bonitas\ldots{}

--- Que dálias?

--- Como assim\ldots{} que dálias?

\textls[10]{Ergui-me e precipitei-me na direção da prateleira. Atirada sobre um monte de
livros, uma echarpe vermelha. No momento em que estiquei o braço e me
convenci de que de fato se tratava de uma echarpe, algo hesitou ao longe
dentro de mim, como a oscilação da coragem do equilibrista amador, no topo da
pirâmide, entre acrobacia e diletantismo. Certamente eu também tinha chegado
à minha altura máxima.}\looseness=-1

Agora, todo o problema se resumia a voltar e sentar-me na cadeira. E em
seguida, o que deveria fazer, o que deveria dizer?

\textls[10]{Esse problema me deixou tão aturdido por alguns instantes, que não fui capaz
 de executar o mínimo movimento. Como a grande velocidade das hélices de um
 motor que as fazem parecer imóveis, minha hesitação profundamente desesperada
 emprestava-me a rigidez de uma estátua. O tique-taque do pêndulo batia forte,
 cravando em mim pequenos pregos sonoros. A muito custo, me arranquei da
 imobilidade.} 

\textls[-10]{Edda estava na mesma posição na cama, fitando-me com a mesma calma estranheza;
poder-se-ia dizer que uma força maligna, pérfida ao extremo, conferia aos
objetos o seu mais comum aspecto a fim de me meter na maior confusão. Eis o
que lutava contra mim, eis o que era implacável: o aspecto comum dos
objetos.}\looseness=-1

Num mundo tão exato, qualquer iniciativa era supérflua, se não impossível. O
que me tirava do sério era o fato de Edda não poder ser diferente, o fato de
ela não passar de uma mulher de cabelo bem penteado, de olhos de um azul
violeta, com um sorriso no canto dos lábios. O que é que eu poderia fazer
contra uma exatidão tão áspera? Como poderia fazer com que ela entendesse,
por exemplo, que eu era uma árvore? Eu tinha de transmitir com palavras
imateriais e informes, por meio do ar, uma coroa de galhos e folhas, enorme e
imponente, assim como eu a sentia dentro de mim. Como poderia fazê-lo?

\textls[-20]{Aproximei-me da cama e me apoiei na barra de madeira. Sobre as minhas mãos
irradiou-se uma espécie de certeza, como se nelas houvesse de repente
penetrado todo o núcleo do meu desassossego.}\looseness=-1

E agora, então? Entre Edda e mim impunha-se, atordoante, o mesmo ar
esverdeado, impalpável e aparentemente inconsistente em que pairavam todas as
minhas forças que, todavia, de nada serviam. Hesitações de dezenas de quilos,
silêncios de horas inteiras, angústias e vertigens de carne e sangue, tudo
isso podia fazer parte daquele espaço miserável sem que ao menos a sua
aparência exibisse o colorido negro e a matéria embaciada que continha. No
mundo, as distâncias não eram simplesmente aquelas que vemos com os olhos,
ínfimas e permeáveis, mas outras, invisíveis, povoadas por monstros e
acanhamentos, por projetos fantásticos e gestos insondáveis, que --- caso se
coagulassem com a matéria da qual tendiam se compor --- transformariam o
aspecto do mundo num cataclismo terrível, num caos extraordinário, pleno de
violentas desgraças e felicidades extáticas.

\textls[5]{Naquele momento, fitando Edda, a materialização dos meus pensamentos talvez
pudesse realmente ter tido como resultado aquele gesto simples que assombrava
a minha mente: erguer o peso para papel da escrivaninha (com o rabo do olho
eu o observava, pressionando um maço de folhas como um nobre capacete
medieval) e atirá-lo sobre Edda; como consequência imediata, um formidável
jorro de sangue do seu peito, vigoroso como a torrente de uma torneira,
inundaria pouco a pouco o quarto, até eu sentir, primeiro, como os meus pés,
e logo também meus joelhos, chapinham no líquido morno e pegajoso e,
depois --- como nos filmes americanos que causam sensação com um personagem
condenado a ficar dentro de um espaço hermeticamente fechado, onde o nível da
água sobe sem parar --- como de repente sinto o sangue já na altura da minha
boca, afogando-me no seu gosto prazeroso e salgado\ldots{}}\looseness=-1

\textls[-25]{Comecei a mover os lábios involuntariamente e a engolir em seco.}\looseness=-1

-- Está com fome? --- perguntou Edda.

--- Ah, não, não\ldots{} não estou com fome, só estava pensando numa
    coisa\ldots{} absurda\ldots{} totalmente absurda.

\textls[-20]{--- Por favor me conte. Desde que chegou, você não disse uma palavra, nem eu
    lhe perguntei nada\ldots{} agora eu estou esperando, veja lá.}\looseness=-1

--- Olhe, Edda --- comecei a dizer ---, trata-se de algo que, no fundo, é muito
    simples\ldots{} demasiado simples até\ldots{} me perdoe por lhe dizer
    isso, mas eu\ldots{}

\textls[10]{Quis continuar dizendo ``eu sou uma árvore'', mas a frase agora não tinha mais
valor algum desde que fora invadido pela vontade de beber sangue. Ficou no
fundo da minha alma, desbotada e murcha, e até me surpreendi com o fato de
ela ter tido certa importância no passado. Comecei de novo.}

--- Veja só, Edda, do que se trata, eu não estava passando bem, me sentia
    fraco e combalido. A sua presença sempre me faz bem, basta vê-la\ldots
    {} você se aborrece por causa disso?

\textls[15]{--- De jeito algum\ldots{} --- respondeu-me ela, pondo-se a rir. Agora sim é
    que eu tinha ganas mesmo de cometer algo absurdo, sanguinário, violento.
    Recuperei com um gesto rápido meu chapéu. ``Agora eu vou embora.'' Num
    instante, eu já estava nos últimos degraus da escada.}

Uma coisa agora era certa: o mundo tinha um aspecto comum próprio, no meio do
qual eu despencara por equívoco; jamais poderei me transformar em árvore,
jamais poderei matar alguém, jamais o sangue jorrará em ondas. Todas as
coisas, todas as pessoas encontravam-se presas em sua triste e pequena
obrigação de serem exatas e nada mais que exatas. Era inútil acreditar que
havia dálias dentro de um vaso, quando o que havia era uma echarpe. O mundo
não tinha força de mudar o mínimo que fosse, encontrava-se tão mesquinhamente
preso em sua exatidão que era incapaz de se permitir tomar uma echarpe por
flores\ldots{}

Pela primeira vez, senti a mente apertada forte dentro do crânio. Terrível e
doloroso cativeiro\ldots{}


\section{xlii} 

\letra{N}{aquele} \textls[12]{outono, Edda adoeceu e morreu. Todos os dias anteriores,
 todos os meus passeios inúteis, todas as minhas fadigas e perguntas
 torturantes concentraram-se na dor e na angústia de uma única semana, como
 em certos líquidos em que a mistura de variadas substâncias condensa, de
 repente, a violência de um poderoso veneno.}

No andar de cima, o silêncio desceu uma oitava. Paul lograra encontrar, não
sei em que armário, um sobretudo velho e uma gravata tão puída que o nó ao
redor do pescoço parecia feito de barbante. Tinha uma cor violeta, como o véu
fino que as noites mal-dormidas deixam sobre o rosto.

\textls[-15]{--- Sofreu a noite toda --- disse-me ele. --- Ontem eu perguntei
 de novo ao médico o que ele acha, e ele me disse tudo, toda a verdade. É
 como se uma explosão houvesse se produzido nos rins, confessou-me o médico.
 É extremamente raro que tal doença surja tão brusca e virulenta. Em geral
 ela se insinua devagar, revelando sintomas que a anunciam muito antes de se
 tornar grave. Trata-se de uma verdadeira explosão nos rins; uma verdadeira
 explosão.}\looseness=-1

Paul falava rápido porém com longas pausas, como se desse tempo, entre uma e
outra palavra, para que a dor aguda dentro de si fervilhasse e se
consumisse.

O escritório do térreo estava escuro como uma caverna; o velho Weber, com a
cabeça enfiada num livro de registros, dava a ilusão de estar ocupado\ldots
{}

\textls[-15]{Cada manhã, o médico vinha com seu andar silencioso e, passando
 pelos quartos, conduzia os três Weber até o quarto de Edda.}\looseness=-1

\textls[10]{Eu os acompanhava, conversando com Ozy. Faz tempo que não brincávamos do nosso
jogo imaginário, de maneira que agora seria uma ocasião maravilhosa.}

Que bom seria falarmos da doença de Edda como se nada daquilo fosse verdade!

Ao subir as escadas, pensava na possibilidade extraordinária de que tudo não
passasse de uma brincadeira dirigida por Ozy, da qual participassem também o
médico, Paul Weber e o velho. Pela primeira vez o corcunda dirigiria, de
verdade, uma cena imaginária e inexistente. Ao chegarmos lá em cima, tinha
vontade de gritar: ``Já basta, chega, muito boa interpretação, Paul manteve
uma máscara realmente impressionante, vimos que o velho Weber de fato sofreu,
mas agora é o suficiente, já acabou, por favor, Ozy, diga-lhes que você
desistiu de continuar\ldots{}''.

Tudo, porém, estava muito bem montado para desistirem no alto da escada\ldots
{}

Enquanto o médico entrava no quarto de Edda, ficávamos no aposento
 contíguo o velho Weber, Ozy e eu. Devia ser a primeira vez na vida que o
 velho Weber tentava conter uma grande emoção. Com a cabeça apoiada na
 poltrona, ele olhava de maneira impessoal e vaga para a rua, como se nada
 soubesse ou nada esperasse. Num determinado momento, como grandes atores que
 tendem a rematar seu papel por meio de um detalhe inédito, ele se ergueu da
 poltrona e foi observar mais de perto um quadro pendurado na parede. Porém,
 como o grande ator que, engrossando demais a voz para uma tirada trágica
 acaba por transformá-la numa fala ridícula, digna dos risos da galeria, o
 velho Weber quis desempenhar seu papel com muita calma mas errou o efeito:
 enquanto estava de pé admirando o quadro, tamborilava nervoso os dedos no
 espaldar de uma cadeira atrás de si\ldots{}\looseness=-1

Paul pegou-me pela mão:

--- Edda quer vê-lo, venha comigo sem fazer barulho.

\textls[10]{Edda estava deitada na cama com lençóis brancos, a cabeça virada para a
janela. Seu cabelo estava estendido nos travesseiros, mais loiro e mais fino
do que nunca; as doenças têm tais sutilezas. Pairava no quarto uma espécie de
decomposição branca dos objetos pela superabundância de luz em que o rosto de
Edda desaparecia, inconsistente.}

De súbito, ela virou a cabeça.

\textls[-15]{Então era verdade\ldots{} Ou seja, naquele momento ocorreu-me algo tão
inexplicável, tão claro e tão surpreendente, que poderia ter constituído uma
verdade vinda do exterior\ldots{} A cabeça de Edda assemelhava-se
perfeitamente à cabeça marfínea das minhas noites de febre. A evidência era
tão atordoante que quase acreditei que eu mesmo inventara naquele momento a
forma exata da velha cabeça de faiança, com a mesma velocidade com que
construímos, nos sonhos, todo um episódio ao ouvirmos o som de um disparo.}\looseness=-1

\textls[-10]{Agora eu tinha certeza de que algo ruim e violento haveria de acometer Edda em
breve. Talvez eu tenha até imaginado isso mais tarde; em tudo o que diz
respeito a Edda, não consigo distinguir nada do que na verdade era parte de
mim ou parte dela.}\looseness=-1

\textls[10]{Ela tentou olhar-me nos olhos, mas acabou fechando as pálpebras, cansada. O
cabelo de lado revelava a testa amarela como um bloco de cera. Eu de novo me
fechava hermeticamente na presença de Edda, naquilo que ela representava
agora, como nas minhas noites de delírio. Em nenhum dos meus passeios, em
nenhum dos meus encontros eu pensava verdadeiramente em alguém além de mim
mesmo. Era-me impossível conceber uma outra dor interior, ou simplesmente uma
existência alheia. As pessoas que eu via ao meu redor eram tão decorativas,
tão efêmeras e materiais como qualquer outro objeto, como casas ou árvores.
Só diante de Edda, pela primeira vez, senti que minhas interrogações podiam
escapar e, após ressoar em outras profundidades e no interior de uma outra
existência, retornar sob a forma de ecos enigmáticos e angustiantes.}

Quem era Edda? O que era Edda? Pela primeira vez via-me do lado de fora, pois
na presença de Edda encontrava-se a interrogação do sentido da minha vida.
Foi no momento da sua morte que ela me abalou mais profunda e autenticamente;
a sua morte era a minha morte e, desde então, em tudo o que faço e em tudo o
que vivencio, a imobilidade da minha morte vindoura projeta-se fria e
obscura, assim como a vi no quarto de Edda.


\section{xliii} 

\letra{N}{a} aurora daquele dia levantei-me pesado e empedernido, incomodado
 com a presença de alguém ao lado da minha cama.

Era o meu pai, que esperava em silêncio que eu acordasse. Quando abri os
olhos, ele deu alguns passos no quarto, trouxe-me uma bacia branca e uma
caneca de água para eu lavar as mãos.

\textls[10]{Com uma convulsão dolorosa que apertou meu coração, entendi o que isso queria
dizer.}

--- Lave as mãos --- disse meu pai. --- Edda morreu.

Lá fora garoava, e por três dias continuou chovendo.

\textls[15]{No dia do enterro, o barro foi mais agressivo e sujo do que nunca. Rajadas de
vento atiravam a água sobre os telhados e os vidros das janelas. Toda a noite
uma janela permaneceu iluminada no andar de cima da casa dos Weber, no quarto
onde as velas estavam acesas.}

\textls[15]{O escritório do velho Weber foi todo revolvido para abrir espaço ao caixão que
tinha de passar; a lama entrou nos aposentos de maneira triunfal e
insinuante, como uma hidra de inúmeros prolongamentos protoplasmáticos --- eu
via muito bem como eles se estendiam pelas paredes, subindo pelas pessoas,
galgando os degraus e tentando escalar o caixão.}

\textls[15]{O assoalho de madeira ressurgiu, no escritório, por debaixo do linóleo que o
cobria e que foi retirado, revelando longas marcas de sujeira, assim como as
rugas negras que se aprofundaram no rosto de Samuel Weber.}

\textls[15]{Em torno de suas botas de borracha a lama subia lenta mas tenaz, certamente
penetrando pela pele até o coração --- suja, pesada, pegajosa. Era lama e
nada mais, era assoalho e nada mais, eram velas e nada mais. ``Meu enterro
será uma sucessão de objetos'', disse-me Edda certa vez.}

\textls[15]{Algo se debatia dentro de mim, ao longe, como se quisesse provar a existência
de uma verdade superior ao barro, algo que fosse diferente dele. Em vão\ldots
{} Minha identidade tornara-se há muito verossímil e, agora, de uma maneira
muito simples, apenas confirmava que no mundo não havia nada salvo o barro.
Aquilo que eu interpretava como dor dentro de mim não passava de um leve
fervilhar do barro, um prolongamento protoplasmático modelado em palavras e
raciocínios.}

As gotas caíam sobre Paul como num recipiente sem fundo; corriam-lhe sobre a
roupa, sobre as mãos que se dependuravam pesadas, fazendo com que curvasse as
costas. As lágrimas escorriam sujas por sua face em longos fios, como água
pelas vidraças.

Devagar, balançando-se sobre os ombros das pessoas, o caixão passou do lado do
navio de Samuel Weber, do lado dos velhos livros de registro e de dezenas de
frascos de tinta e medicamentos descobertos por ocasião da bagunça produzida
no escritório. O enterro era uma simples sucessão de objetos\ldots{}

\textls[-10]{Ocorreram ainda outros episódios, do lado de cá da vida: no cemitério, quando
retiraram do esquife o cadáver enrolado em panos brancos,  estes tinham a
marca de uma grande mancha de sangue.}\looseness=-1

Esse foi o último e mais insignificante episódio anterior ao subsolo do
cemitério quente, embolorado e repleto de corpos moles como gelatina,
amarelos\ldots{} purulentos\ldots{}


\section{xliv} 

\letra{Q}{uando} \textls[-5]{vez ou outra ponho-me a pensar nessas coisas, tentando
 inutilmente cristalizá-las em algo que poderia chamar de minha própria
 pessoa; quando as rememoro, o escritório do velho Weber logo se transforma
 no cômodo em que respiro bolor e sinto o cheiro de antigos livros de
 registro --- bem naquele momento ---, para imediatamente desaparecer e se
 converter no aposento que agora me apresenta o mesmo problema doloroso, o do
 modo como as pessoas passam a sua vida, utilizando-se, por exemplo, de
 quartos, ou sentindo-se como um corpo estranho, ramificado como uma
 samambaia e inconsistente como uma fumaça dentro de si, um aroma especial,
 como o aroma profundamente enigmático do bolor; quando acontecimentos e
 pessoas abrem-se e fecham-se dentro de mim como leques; quando minha mão
 tenta escrever esta estranha e incompreensível simplicidade, tenho então a
 impressão, por um instante, como um condenado que num segundo se dá conta,
 de maneira diferente de todas as outras pessoas ao seu redor, da morte que o
 espera (e ele gostaria de se debater de forma distinta como se debatem os
 demais, conseguindo salvar-se), tenho a impressão de que a partir de tudo
 isso surgirá, um fato novo e autêntico, ao mesmo tempo quente e íntimo, que
 me resuma tão claramente como um nome e que ressoe no meu interior com uma
 tonalidade única, jamais ouvida, mas que seja a do sentido da minha
 vida\ldots{}}\looseness=-1

\textls[-20]{Por que, se não por isso, persistirá em mim aquele fluido tão
 íntimo porém tão hostil, tão próximo porém tão rebelde na hora de o
 capturar, que se metamorfoseia sozinho, ora na imagem de Edda, ora nos
 ombros curvados de Paul Weber, ora no detalhe excessivamente preciso da
 torneira de água no corredor de um hotel?}\looseness=-1

Por que agora me retorna, nítida, a lembrança dos últimos dias de Edda? Por
que, num outro sentido (e as perguntas podem crescer caoticamente em milhares
e milhares de sentidos diferentes, como naquela brincadeira de criança de
dobrar um papel manchado de tinta, apertá-lo com força para que a tinta se
espalhe o máximo possível e nele ver reveladas, ao abri-lo, as mais
fantásticas e insondáveis contorções de um desenho bizarro), por que, num
outro sentido, me retorna essa lembrança e não outra?

Ademais, com cada lembrança incompreensível e exata, tenho de me dar conta de
que, como a dor violenta de um doente que relega a um segundo plano pequenas
moléstias momentâneas de desconforto, como uma posição errada das almofadas
ou o gosto ruim de um remédio, como uma dor que envolve e abarca todas as
minhas outras incompreensões e angústias, tenho de me dar conta de que, por
mais ininteligível e mesquinha que seja, cada lembrança é, apesar de tudo,
única, no sentido mais pobre do termo, tendo desempenhado um papel na minha
vida linear de uma só maneira, numa só exatidão, sem poder ser modificada e
sem o menor desvio de sua própria precisão.

\textls[15]{``A sua vida foi assim e não de outra maneira'', diz a lembrança, e essa frase
engloba a imensa nostalgia desse mundo fechado em suas próprias luzes e cores
herméticas, das quais nem mesmo uma vida tem permissão de extrair nada senão
o aspecto de uma banalidade exata.}

\textls[15]{Ela engloba a melancolia de ser único e limitado, num mundo único e
mesquinhamente árido.}

\textls[15]{Por vezes, à noite, desperto de um pesadelo terrível; é o meu sonho mais
simples e mais aterrador.}

\textls[-10]{Sonho que estou dormindo profundamente na mesma cama em que me
 deitei ao anoitecer. Desenrola-se no mesmo cenário e aproximadamente na
 mesma hora da noite; se, por exemplo, o pesadelo começa no meio da noite,
 ele me situa com exatidão naquele tipo de escuridão e de silêncio que reinam
 naquela hora. Vejo e sinto, no sonho, a posição em que me encontro, sei em
 que cama e em que quarto estou dormindo, meu sonho se molda como uma pele
 fina e delicada à minha verdadeira posição e ao meu sono daquele momento.
 Desse ponto de vista, poder-se-ia dizer que estou acordado: estou acordado,
 embora durma e sonhe com minha vigília. Sonho que estou dormindo naquele
 momento.}

E eis que de repente sinto como o sono se torna mais profundo, mais pesado e
como tenta, assim, me arrastar consigo.

\textls[10]{Quero acordar, mas o sono escorre pesado das minhas pálpebras e dos meus
braços. Sonho que me agito, que golpeio com as mãos, mas o sono é mais forte
do que eu e, depois de me debater um pouco, ele me prende com mais força e
mais tenacidade. Então começo a gritar, quero resistir ao sono, quero que
alguém me acorde, desfiro violentas bofetadas contra mim mesmo para
despertar, fico com medo de o sono me mergulhar em tamanhas profundezas que
não poderei nunca mais retornar, imploro aos berros que alguém me ajude, me
sacuda\ldots{}}

Por fim, meu último grito, o mais forte, me desperta. Vejo-me logo no meu
verdadeiro quarto, que é idêntico ao quarto do meu sonho, na mesma posição em
que eu sonhava, na mesma hora em que eu sabia que me debatia no pesadelo.

\textls[15]{O que eu vejo agora ao meu redor difere muito pouco do que eu estava vendo um
segundo atrás, mas tem não sei que ar de autenticidade que paira sobre os
objetos, sobre mim, como um brusco resfriamento da atmosfera no inverno, que
amplifica de repente todas as sonoridades\ldots{}}\looseness=-1

Em que consta o sentido da minha realidade?

\textls[-5]{Rodeia-me de novo a vida que vou viver até o próximo sonho. Lembranças e dores
presentes aferram-se fortemente a mim e eu quero resistir a elas, não quero
cair no seu sono, de onde talvez não venha a retornar nunca mais\ldots
{} Debato-me agora na realidade, grito, imploro para que me acordem, para que
me acordem numa outra vida, na minha vida verdadeira. Com certeza estou em
pleno dia, sei onde me encontro e que estou vivo, mas algo falta nisso tudo,
assim como no meu espantoso pesadelo.}\looseness=-1

\section{xlv} 

\letra{D}{ebato-me}, grito, atormento-me. Quem me despertará?
Ao meu redor, a realidade exata me arrasta cada vez mais para baixo, tentando
me puxar para o fundo.

Quem me despertará?

Sempre foi assim, sempre, sempre.