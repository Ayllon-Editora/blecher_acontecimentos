\hyphenation{Pierre}
\hyphenation{Herald}

\part{Mulheres}

\pagestyle{baruch}
\chapter{Renée, Marthe, Odette}

\section{i}

\letra{A}{inda} \textls[15]{não são oito horas. Ştefan Valeriu sabe disso pela posição do
facho de sol, que chegou só até a beirada de baixo da espreguiçadeira.
Sente como sobe pela barra de madeira, como envolve seus dedos, a mão, o
braço descoberto, quente como um xale. Passará um tempo --- cinco
minutos, uma hora, uma eternidade --- e, em torno das pálpebras
cerradas, uma cintilação azul vai se instalar, com vagas listras
prateadas. Então serão oito horas, e ele dirá para si mesmo, sem
convicção, que precisa se levantar. Assim como ontem, assim como
anteontem. Mas continuará sorrindo, pensando naquele relógio solar que
construíra, desde o primeiro dia, com uma espreguiçadeira e um canto de
terraço.}

Sente o cabelo ardendo ao sol, áspero como fios de cânhamo, e considera
que, no final das contas, não faz mal ter esquecido em Paris, no quarto
da rue Lhomond, o frasco de loção tônica Hahn, seu único requinte,
supremo, aliás. Gosta de passar os dedos pelo cabelo engruvinhado, do
qual o pente, pela manhã, só logra desatar três rodamoinhos, cabelo esse
que, agora, entre os dedos, ele sente pela aspereza quão loiro é.

Deve ser muito tarde. Puderam-se ouvir, há pouco, vozes pela alameda. Da
direção do lago, alguém gritou, uma voz de mulher, talvez a inglesa de
ontem, que nadava vigorosamente e o fitava, enquanto ele se admirava com
toda aquela luta contra a água, ela que só conhecia nado livre.

\textls[15]{Ştefan balançou a perna por cima da barra da cadeira e procurou na
grama, do jeito que estava, sem meia, vestígios de umidade. Ele sabe
que, mais para a esquerda, não muito longe, na direção da moita, tem um
lugar em que o orvalho dura mais, até a hora do almoço.}

Pois então. O corpo sonolento queimando ao sol, e essa sensação de
frescor vegetal.

Segunda à noite, ao descer no salão da pousada, depois de --- mal tendo
chegado da estação ferroviária, após uma longa viagem --- trocar de
camisa, a sérvia tagarela da mesa dos fundos disse em voz alta, para
todo o mundo ouvir:

--- \emph{Tiens, un nouveau jeune homme}.\footnote{Em tradução livre, ``Olhe, um novo jovem''. \textsc{{[}n.\,e.{]}}}

Ştefan se sentiu duplamente grato a ela. Tanto pelo \emph{nouveau}
como pelo \emph{jeune homme}.

\textls[-15]{Tinha sido velho uma semana antes, ao sair do último exame de residência
médica. Velho, mas não envelhecido. O cansaço das noites em claro,
manhãs no hospital, longas tardes na biblioteca, duas horas de prova,
numa sala escura diante de um professor surdo, com roupa grossa de
inverno e um colarinho que parecia sujo\ldots{}}\looseness=-1

Em seguida, o nome desse lago alpino, que encontrou ao acaso, numa
livraria, num mapa, a passagem de trem comprada na primeira agência de
viagem, as compras em lojas de departamento --- um pulôver branco, uma
calça cinza de flanela, uma camisa de verão ---, a partida como uma
fuga.

\emph{Un nouveau jeune homme.}

\asterisc

Não conhece ninguém. Lançam-lhe às vezes uma palavra ao acaso, mas ele
responde evasivo. Ştefan tem medo de sua pronúncia insegura e se
sentiria mal ao se trair já no primeiro dia: um estrangeiro. Após o
almoço, passa apressado por entre as mesas, ausente, quase crispado. Os
outros talvez o considerem rabugento. Mas ele é apenas preguiçoso.

Em cima, atrás do terraço, começa o bosque. Há ali uma pequena área de
grama alta, espessa e elástica. Ele a esmaga toda tarde sob o peso de
seu corpo adormecido, para reencontrar refeitos, no dia seguinte, cada
um de seus fios. Estatelado no chão, fica de braços abertos, pernas
esticadas, a cabeça mergulhada na grama, vencido por uma força contra a
qual gostaria de lutar.

\textls[15]{Um esquilo pula de uma aveleira para outra. Como será que se diz esquilo
em francês?}

Paira um silêncio imenso\ldots{} Não. Não paira um silêncio imenso. Isso
parece citação de livro. O que paira é uma imensa bagunça, uma imensa
balbúrdia zoológica, grilos que se arrastam, gafanhotos que se agitam,
escaravelhos que se chocam no ar, ferindo sonoramente os élitros e
depois caindo com um barulho denso, de chumbo. Em meio a tudo isso, sua
respiração, a de Ştefan Valeriu, é um mero detalhe, sinal irrisório de
vida, irrisório e capital como o do esquilo que pula, ou o do gafanhoto
que se detém na ponta da sua bota, achando ser uma pedra. É benfazejo
saber-se aqui, um animal, uma criatura viva, um quadrúpede sem
importância, que dorme e respira numa área de dois metros quadrados,
debaixo de um sol que pertence a todos.

\textls[15]{Se tivesse vontade de pensar, o que pensaria um grilo a respeito da
eternidade? E se, por acaso, a eternidade tiver o gosto dessa
tarde\ldots{}}

\textls[15]{Lá embaixo se veem, no terraço da pousada, cadeiras, xales, vestidos
brancos. Mais adiante, o lago azul, diáfano, idílico. Cartão postal.}

\section{ii}

\letra{F}{az} \textls[15]{frio ao anoitecer, um anoitecer azul, repleto de sons tênues que
chegam da cidade, por cima do lago, para além de onde se veem luzes
elétricas, distantes. É quinta-feira, tem concerto militar no parque
municipal.}

Quase toda a pousada tomou o barco das 8h27 para assistir. Ştefan
Valeriu ficou. Por todo o vale, que se desabre amplo diante do terraço,
reina um azul profundo de madeira de ginjeira.

--- Com licença, o senhor sabe jogar xadrez?

--- Sim.

Por que respondeu ``sim''? Teria sido tão simples responder ``não'' e,
agora, estaria desacorrentado do lado de fora, passeando pelo terraço.
Um ``sim'' apressado e ei-lo no saguão, diante do tabuleiro de xadrez,
condenado a prestar atenção.

\textls[15]{O parceiro é um homem alto, ossudo, escuro, maduro e feio. Joga devagar,
calculado.}

--- O senhor não foi ao concerto?

--- Não.

--- Nem eu. Minha esposa fez questão de ir e deixei. Mas eu\ldots{}

Ştefan perde uma torre, mas constrói, no canto esquerdo do tabuleiro, um
ataque implacável contra o rei.

--- O senhor é do Sul?

--- Não. Sou romeno.

\textls[10]{--- Não pode ser! O senhor fala como um francês. Ou talvez eu não esteja
acostumado com a pronúncia daqui. Pois eu também não sou da França. Sou
tunisiano.}

--- Tunisiano?

\textls[15]{--- Sim. Quero dizer, francês da Tunísia. Possuo plantações lá. Meu nome
é Marcel Rey.}

O ataque de Ştefan fracassa e, diante da devastação do campo, ele
abandona a partida. Nesse meio-tempo, retornam da cidade os que haviam
ido ao concerto. Ouve-se o apito do barco no cais.

Saem no quintal para aguardá-los. Vozes muitas, animadas, exclamações,
apertos de mão, cumprimentos ruidosos.

--- Ah, Marcel, se você soubesse como foi bonito.

\textls[15]{--- Renée, olhe aqui um senhor que pode se tornar nosso amigo. Minha
esposa.}

É uma mulher alta, magra. Na escuridão, só se veem os olhos. Ştefan
beija a mão dela. Uma mão miúda e fria, que nada diz.

\asterisc

\textls[15]{Foram a um pequeno passeio em Lovagny, visitar um castelo, os três.
Marcel Rey, a esposa e Ştefan. Tem também Nicolle, filha do casal Rey.
Caminharam bastante, deram risada, tiraram fotografias. O senhor Rey tem
uma pequena máquina de filmar, com a qual às vezes filma algumas cenas,
que depois manda revelar em Paris.}

--- Renée, fique ali com o senhor Valeriu. Mais para lá, na luz. Isso,
sorriam, conversem, mexam-se.

--- Se é para fazermos uma cena de filme --- sussurra Valeriu ---, eu
preferiria, minha senhora, uma de amor.

Disse isso ao acaso, leviano, para poder transformar facilmente numa
piada, caso necessário.

Renée dá um sorriso casual e nada diz. Ştefan brinca com os cachos do
cabelo de Nicolle. Senhor Rey filma.

\asterisc

Fica sabendo da história toda deles. Nasceram ambos na Tunísia, de velhas
famílias de colonos, numa pequena cidade. Ele viera uma vez à França, em
1917, para receber uma bala no ombro duas horas após ter entrado na
trincheira, e ser reenviado para casa uma semana depois. Até então ela
não havia ido além de Túnis. Casaram-se em 1920, tiveram uma filha ---
Nicolle --- em 1921, compraram uma vinha em 1922, uma plantação um ano
depois e, em seguida, duas a cada ano. Djedaida, a pequena localidade
deles, fica a cinquenta quilômetros de Túnis. Vilarejo de europeus,
rodeado por obscuras tribos locais que se concentram na periferia quando
chega a seca, e passeiam pelas ruelas com olhares ameaçadores. Nesses
períodos, o casal Rey dorme com uma espingarda do lado da cama. Todo
sábado no fim da tarde, momento de pagar os trabalhadores das
plantações, Renée fica junto ao telefone para pedir socorro de Túnis a
tempo.

\textls[15]{Ela conta tudo isso sossegada, sem empolgação, um pouco cansada, e
Ştefan Valeriu precisa fazer três perguntas para conseguir obter uma
resposta.}

\textls[5]{--- Pode me passar aquele xale? Estou com frio.}

Ele o joga por cima da espreguiçadeira e, ao tentar alcançá-lo, sua mão
se demora casualmente sobre o joelho dela. Renée se sobressalta,
assustada, e grita, sem sentido: ``Nicolle, Nicolle!''

Ao anoitecer, Ştefan responde a uma carta recebida de Paris: ``Nenhum
conhecido. Só uma família de tunisianos; ele, bom enxadrista, ela,
esposa virtuosa. O casamento parece funcionar''.

\asterisc

\textls[15]{Partiu com um bote do cais da pousada, remou até longe, de onde se vê,
simétrico, o desfiladeiro das montanhas, lançou a âncora e se deitou no
fundo do barco, deixando os remos ao movimento das ondas.}

\textls[15]{Está com preguiça, uma preguiça simples, sem arrependimentos, tranquila
como uma vasta ausência. Fecha os olhos. O sol o abrange por inteiro.}

\textls[15]{Um pouco antes, no saguão, reviu o jovem casal que chegara há pouco
tempo à pousada e que ocupou o quartinho isolado no quintal, longe de
todos. Lua de mel, provavelmente. Ela é admirável. Descera tímida, com
vestígios de negligência no vestir-se, e Ştefan suspeitou, pelos olhos
dela, que noite ardente devia ter passado. Poder-se-ia dizer que espalha
por toda a casa um aroma de alcova, de almofadas quentes, sensuais, com
um corpo de amante adormecida, com uma luz matutina difusa que
surpreende o ato de amor.}

--- Insuportável! Contagioso! Vale uma reclamação!

\textls[10]{Ştefan fala alto, sozinho. Responde-lhe um jato d'água que atinge o
barco, o grito remoto de uma nadadora, o relógio da Saint François de
Sales, na cidade, soa dez horas.}

\section{iii}

\letra{N}{o} \textls[10]{terraço, retraído num canto, um casal o intriga. Observou-os só hoje,
mas talvez já estivessem ali há mais tempo. A mulher, ainda não velha,
tem algo de glorioso em sua beleza. Trinta e cinco anos, talvez. Ou
mais. Alta, calma, de traços bem definidos, com um sorriso que não é
sorriso, mas uma grande descontração do rosto. O homem ao seu lado é um
menino. Não deve passar dos vinte anos. Está sentado no chão, na grama,
ao lado da poltrona da mulher, e fala rápido, vívido, com gestos miúdos.
Ela faz que o escuta, e passa a mão pelo cabelo dele, acariciando-o.}

Amante? Marido? Gigolô? Um pouco de cada, acha Ştefan Valeriu, que de
repente descobre dentro de si um sentimento de inveja, ou talvez de
humilhação, ele, animal jovem de vinte e quatro anos, vigoroso,
aprazível e sozinho, que aguarda, sem saber de onde, uma paixão que não
chega.

--- Quer cantar, Nicolle?

--- Quero.

\asterisc

\textls[15]{É muito tarde. Todos foram se deitar. Ştefan Valeriu está sozinho no
terraço.}

--- O sono é melhor que as estrelas, meu jovem --- disse-lhe, antes de
ir se deitar, o senhor Vincent, marselhês gordo e jovial.

\textls[20]{Não respondeu. Após o jantar, a mulher e seu jovem pajem desceram até o
lago. Ainda não voltaram. Ştefan acompanhou com o olhar, por algum
tempo, o passeio ao longo do cais: via-se bem o xale solto dela,
esvoaçante, e o pulôver branco dele. Agora não se pode distinguir mais
nada, mas eles vão voltar e a noite é uma criança.}

\asterisc

\textls[15]{Um baque forte. Um barco se choca contra o seu. Ştefan Valeriu,
surpreso, se ergue.}

--- Quem é?

--- Eu.

É o jovem pajem. Dá uma risada confusa, aparentemente alegre com o
encontro, mas se desculpando pelo acidente.

--- Estava remando para trás e não percebi que seu barco estava no meio
do caminho. Aliás, é meu primeiro passeio no lago. A água está boa?

--- Está.

--- Sabe nadar?

--- Sei.

--- Posso ancorar aqui também?

--- Pode.

\textls[10]{Ştefan se deita de novo no fundo do barco, rabugento assumido. O outro
monta no barco dele e começa a brincar com os pés na água, espalhando
gotas brancas ao sol.}

Ştefan Valeriu assobia.

--- Bolero?

--- Sim.

Silencia de novo. O outro continua assobiando a canção iniciada por
Ştefan.

Ao longe, lá em cima, no terraço da pousada, um vestido branco esvoaça
ao sabor da brisa como uma flâmula.

--- \emph{Hallo, hallo}\ldots{}

O jovem pajem gesticula, entusiasmado. Um braço lhe responde lá de cima,
solene.

\textls[15]{--- Você dizia que sabe nadar --- interrompe Ştefan o idílio que o
enerva.}

--- Sim, sei.

\textls[10]{--- Então vamos competir. Até o último cais, ida e volta três vezes, sem
parar. Quem ganhar manda o perdedor ir comprar cigarros na cidade.
Esqueci os meus no quarto.}

Ele precisa dessa vitória. Sente que é estúpido, infantil e mesquinho
tudo aquilo, mas ao mesmo tempo sente a necessidade de humilhar o outro,
de roubar um pouco de sua graça inconsciente. O jovem pajem concorda.
Num instante, estão ambos alinhados sob o sol como duas espadas\ldots{}
Saltam.

\textls[5]{Poderia deixá-lo passar à frente por algum tempo, poderia lhe dar a
ilusão da vitória e, em seguida, com duas braçadas bem dadas, alcançá-lo
e ultrapassá-lo. Mas não. Tem de ser uma derrota absoluta, clara,
esmagadora, do início ao fim. Ştefan está muito mais à frente. O outro
se desdobra. Ouve como arfa, como diminui o ritmo, como se vira de
costas para respirar e descansar. Cinco metros à frente. Dez. Um milhão.
Ganha. O adversário ficou completamente para trás. Ouve-se, da direção
da pousada, o sino chamando para o almoço.}

O jovem pajem enfim chega. Ştefan o ajuda a subir no barco. Está
crispado.

--- Chamaram para o almoço. Mamãe deve estar brava.

--- Mamãe?

--- Sim. Mamãe. Prometi a ela que voltaria a tempo.

--- A senhora de branco?

--- Sim.

--- Por que não me disse que é sua mãe?

--- Você não perguntou.

Não lhe dirige nem mais uma palavra. Não tem o que lhe dizer. Daria a ele
um abraço se tivesse tempo. Rema rápido, ora com uma mão, ora com a
outra, para ao mesmo tempo tirar a camiseta de nado, enxugar-se,
vestir-se. Na orla, amarra desajeitadamente o barco no cais e ajuda o
jovem pajem a amarrar o dele.

--- Vamos!

Toma-o pela mão e sai correndo, sem olhar para trás. A ladeira até a
pousada é íngreme, o sol da tarde bate forte. No entanto, terá de chegar
o mais rápido possível lá em cima.

--- Vamos parar um pouco.

--- Não. Qual é o seu nome?

--- Marc. Marc Bonneau.

\textls[15]{No portão da pousada, o vestido branco de pouco antes está à espera.
Ştefan o avista e se detém a dois passos dele, surpreso com a acolhida.
Só agora percebe que deve estar descabelado, suado, com a gola mal
abotoada, e se envergonha diante daquela mulher tão calma.}

--- Senhora, gostaria de lhe pedir perdão em nome do Marc. Ele se
atrasou por minha causa.

--- Que péssimo. Duas horas de castigo para os dois. Proibidos de
beberem água durante a refeição. Mas veja como estão suados.

Pega o lenço do bolsinho da camisa e passa na testa dele.

--- Está vendo?

\asterisc

\textls[15]{As noites são longas, monótonas. Senhor Rey, senhor Vincent e Marc
Bonneau jogam buraco. Marthe Bonneau e Renée Rey conversam. Ştefan
Valeriu, refugiado atrás da capa de um livro aberto, fuma.}

--- Senhor Valeriu, Nicolle foi se deitar faz tempo. Já está tarde para
crianças. Seria bom que você seguisse o exemplo dela.

\textls[15]{--- Só mais um pouquinho, dona Bonneau, termino esse capítulo e já vou.}

--- Oh, essas crianças de hoje em dia\ldots{}

Todo o mundo dá risada, exceto Ştefan, que parece absorvido pelo livro e
arqueia a sobrancelha em sinal de atenção e ausência.

--- Esse Seu Valeriu --- diz o gordo senhor Vincent --- não está bom da
cabeça. Outro dia mesmo o flagrei no terraço conversando com as
estrelas. Hoje de manhã não esteve no lago e, agora, vejam só, fica
calado e nem se aborrece. Claríssimos sinais, meus senhores, claríssimos
sinais\ldots{}

--- Mas deixe-o em paz, por favor, e chega de provocação. Não é que eu
sou a sua protetora, Seu Valeriu?

--- Sem dúvida, dona Bonneau.

\textls[-20]{Fita-a nos olhos, com um sorriso submisso, com uma inocência no olhar
que dá asas à imaginação e espaço à liberdade interior.}\looseness=-1

\textls[5]{Guardaria algum segredo aquela mulher tão bonita? Seus olhos grandes,
bem desenhados, piscam pouco e enxergam bem. Nenhum instante de
ausência, nenhuma sombra de melancolia. Vez ou outra, quando passa ao
lado de Ştefan, pousa a mão sobre o ombro dele, gesto que repetirá um
minuto mais tarde, com Marc. É tranquila, talvez por se sentir
protegida: protegida por sua própria maturidade, pela presença do filho,
por sua beleza solene e controlada.}\looseness=-1

\textls[-25]{--- O senhor ainda vai segurar o Marc por muito tempo no jogo?}

--- Até terminarmos.

\textls[15]{--- Muito bem, então com quem é que eu vou dar meu passeio noturno?}

--- Com o jovem Valeriu.

--- É mesmo? Deseja me acompanhar, Seu Valeriu?

\textls[-35]{--- Se me permitirem ficar acordado até tarde, dona Bonneau\ldots{}}

--- Tem razão. Vamos lhe dar uma dispensa especial esta noite. A senhora
vem conosco, senhora Rey?

--- Não. Tenho medo de que a essa hora tenha muita friagem perto do
lago.

\textls[-15]{Os dois descem a alameda que liga o portão da pousada à beira do lago.
Ficam bruscamente em silêncio tão logo atravessam a soleira do saguão
para o quintal, surpreendidos pela vastidão da noite que, do lado de
dentro, não imaginavam. Mal se veem, mas sabem que estão um ao lado do
outro, pelo barulho das sandálias no cascalho. À medida que se aproximam
do lago, a noite se torna menos compacta, como se iluminada por luzes
interiores da água, luzes que vão do verde ao azul. Abaixo, na orla,
ouve-se um rumorejo não se sabe bem de onde, do farfalhar do bosque, do
marulhar das ondas na margem, calmo como um pulsar, do sono das plantas
em derredor, do balanço de um barco desprendido do cais.}

\textls[15]{Senhora Bonneau se apoia em Ştefan Valeriu. Não é a mão que cede e se
perde: pelo contrário, é uma mão firme, segura de si, desprovida de
sensualidade.}

\textls[-25]{--- Dona Bonneau, queria lhe dizer que a senhora é muito bonita.}

--- E muito velha.

--- Talvez. Mas sobretudo muito bonita.

Longo silêncio.

--- E o que mais?

--- Nada mais. É tudo.

De vez em quando passa um automóvel, lançando um cone violento de luz
que atinge seu rosto, e desaparece em seguida na primeira esquina da
avenida, deixando-a um pouco embaraçada, como se um desconhecido
entrasse na sala no meio de uma conversa. Ştefan Valeriu percebe esse
primeiro instante de constrangimento e o inscreve entre as suas
vitórias.

\asterisc

\textls[20]{Marthe Bonneau está hospedada no térreo, na ponta do quintal. De seu
quarto, no andar de cima, numa ala oblíqua do edifício, Ştefan consegue
controlar a janela dela com facilidade, sem ser notado.}

\textls[-20]{Viu-a há pouco, depois do almoço, despedindo-se do grupo de amigos no
terraço e entrando na casa. Deteve-se na soleira e fez um gesto aos que
ficaram do lado de fora, um gesto de cansaço, de sono. Em seguida, abriu
a janela e fechou a persiana. Por um instante, seus braços cintilaram na
janela, com sua luminosidade opaca.}

Ştefan toma a decisão, antes de pensar melhor no que faz e no risco.
Desce rápido a escada, em poucos passos atravessa o quintal, bate
brevemente à porta e não espera resposta. Dirá o que lhe passar pela
cabeça, não importa o quê.

--- Marc não está aí?

\textls[15]{--- Você sabe muito bem que não tem como estar aqui. Está em Grenoble.
Você mesmo não o acompanhou ontem à noite até a estação ferroviária?}

--- Então\ldots{}

--- Então fique, já que você quis vir e veio.

Está deitada num sofá perto da janela e segura um livro por abrir. Fala
com ele como se a sua chegada não a surpreendesse.

--- Aproxime-se e sente-se.

\textls[15]{Ştefan leva a mão maquinalmente até a gola, como se quisesse endireitar
o nó de uma gravata imaginária. É curioso como, tão logo se encontra
diante dela, ele se lembra de um detalhe da roupa, inadequado ou
negligente, que lhe parece humilhante em comparação com o aspecto
sóbrio, com a simplicidade vigilante dela. Em especial uma mecha de
cabelo que o irrita caindo sempre sobre a testa, e que ele não para de
arranjar devidamente, com a sensação de que isso dota a sua presença com
um ar de negligência que, contrastando terrivelmente com a beleza
ordenada dela, deve incomodá-la.}

\textls[15]{--- Te vi nadando antes do almoço. Estava com a Renée Rey às margens do
lago e nos alegramos com o espetáculo. Você nada muito bem.}

\textls[15]{--- Senhora, vim para conversar sobre algo completamente diferente.}

\textls[15]{Queria segurar a mão dela, num gesto súbito, para simplificar as coisas,
mas a dúvida de não saber se deveria ou não agir o faz ficar sem
palavras, hesitante. Ela o fita com o mesmo sorriso de proteção e, com a
maior naturalidade, segura ela a mão dele, como se dissesse: ``Está
vendo, é tão simples, não precisa se torturar por uma ninharia dessas''.}

--- Olhe, olhe, Renée Rey está passando. Senhora Rey, não vem nos fazer
companhia? --- E, em seguida, dirigindo-se a Ştefan: --- Gosto dessa
mulher. Assim como gosto do Marc, assim como gosto de você. Vocês três
são jovens e isso é bonito de se ver quando se passa, como eu, para a
outra metade da vida.

\asterisc

\textls[5]{Pediu-lhe que a acompanhasse, domingo de manhã, na igreja de um vilarejo
vizinho, na qual haviam visto, noutra ocasião, de passagem, curiosos
vitrais do século 18. Está de vestido preto, comprido, sem decote, e de
chapéu de aba larga, sob a qual se revela a tranquilidade de sua
expressão. Apoia-se numa coluna central, com Ştefan à sua direita e, à
sua esquerda, Marc, ambos com roupas brancas de férias.}

\textls[10]{Ştefan Valeriu tem de repente --- imaginando o grupo visto de longe ---
a sensação de servir como um simples detalhe decorativo a um cenário
previamente montado. A igreja escolhida de propósito, as duas velhas
ajoelhadas que os rodeiam, o vestido austero, as camisas de gola
desabotoada, a sombra fria por debaixo da cúpula\ldots{}}

--- \emph{Maman, que tu es belle}\footnote{Em tradução livre, ``Mamãe, como você é linda''. \textsc{{[}n.\,e.{]}}} --- sussurra Marc.

\textls[-10]{Pela primeira vez, Ştefan a fita com hostilidade, sem erguer o olhar
diretamente para ela, por medo de perturbar a sua aparência, mas
mantendo-a oblíqua em seu campo de visão, desde o seu retiro simulado.
Como essa mulher deve ter calculado o lugar exato onde ficar, a coluna
na qual se apoiaria ao acaso, a mão que ficaria coberta pela metade,
pois o gesto de desabotoar a luva haveria de ser flagrado pelo som do
órgão e esquecido no meio! Como deve ter sido premeditado esse movimento
tênue da cabeça para trás, esse cansaço do lábio inferior, que não tem
mais força para aguentar um sorriso, essa leve inquietação das
narinas\ldots{}}

--- \emph{Maman, que tu es belle}.

\textls[-25]{Senhora Bonneau responde a Marc colocando a mão sobre o seu ombro. A
outra, sobre o ombro de Ştefan. Pelo bem da simetria.}

Por um instante, a ideia de se desprender do grupo o atiça. Sente um
gosto aguçado de ultraje.

Desloca-se leve e imperceptivelmente para a direita, e a mão dela,
desconcertada por um segundo, cai.

\asterisc

É o terceiro dia de recuo estratégico de Ştefan Valeriu. Desde o
incidente na igreja, só se encontrou com Marthe Bonneau junto com outras
pessoas. A piada de cada noite não surtia mais efeito.

--- Seu Valeriu, Nicolle já foi se deitar. Está tarde para crianças.

--- Tem razão, senhora. Está tarde.

\textls[10]{Ergueu-se, apagou o cigarro, fechou o livro e proferiu a todos um ``boa
noite'' geral e cordial.}

\textls[10]{Vez ou outra, os olhos dela procuravam os dele. Ele a fitava sem querer
e virava rápido a cabeça com aquele sobressalto de desculpa que nos
invade quando, involuntariamente, pousamos o olhar na carta da pessoa ao
lado.}

Vez ou outra, ela o convidava para a acompanhar até a cidade ou num
breve passeio pelas redondezas. Ştefan recusava respeitosamente,
alegando razões perfeitamente plausíveis.

\textls[-15]{--- Lamento, senhora, lamento. Prometi aos meus amigos de Aix
(lembra-se? Meus amigos romenos com quem me encontrei sábado passado no
lago?), prometi que iria vê-los. Se eu pudesse lhes telefonar, seria
mais fácil. Mas telefone neste fim de mundo\ldots{}}

\textls[-15]{Faz tempo que Ştefan Valeriu conhece esse tipo de impertinência cordial.
Está convencido de que, no final das contas, a pose firme da senhora
Bonneau não vai resistir. Pequenos sinais de irritação parecem surgir:
um vago sorriso ofendido, uma maneira brusca de colocar e tirar as
luvas, uma indiferença forçada no falar.}

\textls[-10]{Há pouco, logo depois do almoço, após se erguerem da mesa e irem para o
quintal distribuídos em grupos ao acaso, a senhora Bonneau, com quem
Renée Rey começara uma conversa que prometia ser longa, procurara-o com
o olhar e tentou lhe fazer um gesto para que esperasse, pois tinha algo
a lhe dizer. Mas ele, por estar justamente ocupado acendendo o cachimbo,
considerou poder se permitir a não levar em consideração aquele gesto
demasiado discreto. Distanciou-se e, com passos muito preguiçosos,
pôs-se a subir rumo ao bosque, para seu lugar de costume. Senhora
Bonneau o fitava em pânico, sem saber como explicitar melhor seu pedido
para que ficasse, enquanto se via impossibilitada de falar por causa do
discurso ardoroso da senhora Rey. Ştefan optou por ignorar tudo aquilo e
ostentar sua mais inocente expressão.}

\textls[15]{Ele agora evoca, desde o seu esconderijo, aquela breve cena e degusta,
com maldade, cada nuance. Dá uma risada sonora, sem modéstia: tinha
vencido.}

\textls[15]{Finalmente, Marthe Bonneau fica sozinha e o procura com o olhar do outro
lado da grade do terraço, parece avistá-lo e se dirige rumo ao bosque.
Ştefan ouve seu vestido farfalhando entre as árvores. Deitado na grama,
ele enfia bem os dedos na terra para ganhar autocontrole.}

\textls[20]{--- Dona Bonneau, aposto que a senhora chegou até aqui por acaso.}

--- Se apostar, vai perder. Vim vê-lo.

\textls[10]{Sua resposta é clara, precisa e inábil. A ironia de Ştefan Valeriu fica
em suspenso, sem objeto, como a tensão de alguém que chega com chaves de
sabedoria para abrir uma porta, mas a encontra aberta. A resposta dela
--- uma única resposta --- subverteu de repente uma vitória de três
dias, como um único movimento numa partida de xadrez.}

--- Posso me sentar ao seu lado?

\textls[10]{Ele, todo estirado na grama, ela só pela metade, com a cabeça apoiada
numa aveleira, dominando-o, portanto, só pelo fato de poder olhá-lo de
cima para baixo --- Ştefan sente também como essa diferença, talvez
casual, talvez premeditada, revela, por um lado, sua posição vigilante
e, por outro lado, a liberdade e a indiferença dele. E dá risada, sem
saber se a capacidade dela de sempre encontrar a posição mais digna e
segura é estratégia ou instinto. Estratégia ou instinto, não importa,
uma vez que dessa força de autocontrole emana a sua clara beleza, ainda
mais clara naquela tarde de sol.}

--- Não há dúvida, dona Bonneau, a senhora é muito bonita.

--- Não, meu caro amigo. Apenas muito calma. É verdade que, às vezes, é
a mesma coisa.

--- Por exemplo agora.

\textls[20]{--- Não, não agora. Porque não estou nada calma: vou embora amanhã.}

Ştefan pensa em falar, mas tem medo, pensa em se levantar, mas não se
sente determinado. Fecha os olhos e aguarda.

\textls[10]{--- Vou embora amanhã e me pergunto se não estaria indo tarde demais. Um
instante tarde demais.}

--- Isso significa?

\textls[-10]{Nenhuma resposta chega por um bom tempo e nenhuma sombra desce sobre o
rosto dela, o qual Ştefan, perscrutando-o, imagina devastado por dores
reprimidas. A mesma expressão definida, os mesmos traços simétricos
iluminados por um sorriso vigilante.}

--- Isso significa?

--- Isso significa que sua passagem pelo terraço de manhã, de camisa
branca, com o pescoço descoberto, com seu nome estrangeiro que ninguém
na pousada consegue pronunciar direito, com essa sua juventude decidida
e confusa, com sua vida desconhecida, com os jornais estrangeiros que
você recebe de lugares remotos, com as cartas que lhe chegam em
envelopes com selos estranhos, com suas crispações rabugentas, com suas
alegrias explosivas, com sua paixão pela leitura de livros e por rolar
na grama, é uma imagem agradável.

Ştefan pega na sua mão para beijá-la, mas a encontra tão tranquila, tão
admirada com seu aperto emocionado, tão segura de si, que, sem poder
mais soltá-la, com medo de que o gesto seja demasiado brutal e, também,
sem poder mantê-la presa na sua, ele sugere que parta.

\textls[15]{--- Está tarde, senhora. Nicolle ainda não foi se deitar, mas está
tarde.}

\section{iv}

\letra{A}{} \textls[15]{cena da ferroviária foi banal, cena de despedida numa estação serrana
ao término das férias, com repetidos apertos de mão, exclamações
impacientes, promessas de cartas e de reencontros. Toda a pousada fora
acompanhar Marthe Bonneau e todos faziam estardalhaço em torno dela, ela
sozinha, calma, vagamente intimidada pela efusão alheia, como se um
pouco embaraçada por não conseguir ser mais comunicativa como de
costume. Fazia carinho em Nicolle e respondia com precisão a perguntas
imprecisas.}

\textls[-10]{Num canto da plataforma, Marc conversava entusiasmado com Renée Rey;
Ştefan Valeriu, observando de passagem esse detalhe, se pergunta pela
primeira vez se realmente não acontecera alguma coisa entre os dois, sem
que ele, obcecado pelas próprias expectativas, percebesse. No entanto, o
pensamento só o preocupa durante um segundo, ao acaso, e logo ele é de
novo absorvido pela festa da plataforma, meio falsa, meio sincera. À
partida do trem, esticando o braço para fora da janela, dona Bonneau
grita em sua direção:}

\textls[30]{--- Estamos te esperando em Paris. Espero que venha visitar o Marc.}

O que poderia conter um significado secreto, dito especialmente para ele
e compreendido só por ele, mas que poderia também ser apenas: ``Espero
que venha visitar o Marc''.

\textls[15]{``Ridículo!'', conclui Ştefan um instante depois, já de volta, quando,
no quintal da pousada, se sente muito solitário e suspenso diante de
quatro semanas ainda por vir, que lhe parecem inúteis de antemão. E, com
esse ``ridículo'', ele decide concluir um incidente amoroso que, agora,
na ausência da mulher, lhe parece cansativo e distante.}

Consulta o calendário, nota que só está na metade de agosto, procura no
guia qual castelo nas redondezas ainda não foi visitado, acende o pito e
sai para espairecer.

\asterisc

À noite, após o jantar, Ştefan é acometido por um momento de angústia:
trata-se, em boa medida, da raiva de se ver desocupado num momento em
que, até o dia anterior, ele costumava passar na companhia de todos no
saguão. Quem substituirá o Marc no buraco, quem substituirá a dona
Bonneau nas conversas? As janelas do saguão estão abertas, ouvem-se
vozes familiares do lado de dentro, vê-se a fumaça azul do tabaco contra
o brilho das lâmpadas. O terraço parece maior do que antes, a noite,
mais profunda, enquanto os reflexos do lago, de longe, têm algo de fixo,
regulador. É bom, é muito bom ouvir o som dos próprios passos sobre a
terra molhada, roçar as árvores em derredor, apoiar-se no parapeito do
terraço, inclinado por sobre todo o vale, não esperar ninguém e não
querer nada.

Alguém sai do mato e se aproxima dele em silêncio. É Renée Rey. Ouve ao lado a sua respiração quente, próxima do rosto.

--- Por que está triste?

\textls[10]{Por um instante, Ştefan pretende responder com sinceridade: ``Não estou
absolutamente triste''. Mas, antes de falar, antes de pronunciar a
primeira palavra, a resposta sofre uma reviravolta, quase que sem a sua
permissão.}

--- Por que essa pergunta? Sabe muito bem por quê.

Os olhos dela cintilam intensamente.

--- É verdade?!

E então ela cai em seus braços, à procura de sua boca, beijando ao acaso
o que alcança, desajeitada, inabilidosa e sem experiência. Mas é
invadida por um momento de suspeita.

--- E a senhora Bonneau?

\textls[10]{--- Senhora Bonneau? Você não entendeu? Era uma brincadeira, precisava
esconder, precisava confundir sua atenção, precisava reprimir minhas
possíveis imprudências. Mas agora, já que você descobriu, vou
embora\ldots{}}

--- Não, não, não. Fique aqui, por mim, comigo. Ah, se você soubesse, se
você soubesse\ldots{}

E o beija de novo, tempestuosa e inabilmente, enquanto do saguão a voz
do senhor Rey a chamava para ir dormir, pois já passava da meia-noite e
a partida de buraco tinha terminado.

\asterisc

\textls[10]{De manhã, em geral um barulho de sino o desperta: são algumas vacas que
sobem a montanha, uma trilha atrás da casa, debaixo de sua janela
aberta. Por alguns instantes, permanece propositalmente de olhos
fechados a fim de se demorar no calor do sono encerrado e adivinhar por
entre os cílios o sol que inunda o quarto, desorientado pelo aroma
das plantas e pelos raros chamados que se ouvem de fora.}

\textls[10]{Foi um sono impetuoso, triunfante e completo, sob o qual parecia ter a
sensação de uma felicidade latente, a mesma sensação que um torrão de
terra negra deve ter ao ser profundamente permeado por uma nascente.}

\textls[10]{Essa ária íntima da vitória o incomoda. Ştefan Valeriu não se reconhece
autossuficiente. Ademais, a pequena cena de teatro da noite passada, que
deveria no máximo diverti-lo, o encanta. Isso é ruim.}

\textls[15]{``Sou um imbecil!'' Veste-se rápido, enfia as sandálias nos pés, passa
duas vezes os dedos pelo cabelo e, com a camiseta de nado jogada sobre
os ombros, desce a escada. No quintal, a manhã é mais brilhante do que
imaginava: ao longe, o lago emite reflexos promissores.}

--- Senhor Valeriu!

O chamado vem de algum lugar de cima, e Ştefan tem que circundar a casa
duas vezes com o olhar até descobrir, numa das janelas, Renée, que se
escondera mal atrás da cortina.

--- Bom dia, senhora Rey.

--- Pode vir um instante até aqui em cima, buscar seu livro?

--- Que livro?

--- O livro que você me deu\ldots{}

\textls[15]{Ştefan sobe de novo a escada, dessa vez rumo ao quarto do casal Rey. A
mulher o aguarda atrás da porta, trêmula, pálida e de camisola. Acabara
de se levantar, é visível: a cama está desarrumada. Ştefan leva-a nos
braços até a cama e a atira entre os travesseiros.}

--- E o senhor Rey?

--- Está fora.

--- Nicolle?

--- Está fora.

--- Como assim?

--- Te amo. Se você soubesse\ldots{} se você soubesse\ldots{}

Na verdade, Renée não sabe amar. Seu primeiro enlace é de uma visível
incompetência: nenhuma reticência nem demora no fato de ceder, mas
inúmeras hesitações, que não têm a ver com pudor, mas decerto com sua
inabilidade. No entanto, a brusquidão do acontecimento, as vozes que se
ouvem do quintal lá embaixo, a cama bagunçada, a janela aberta, a hora
inverossímil, tudo isso faz daquele momento amoroso algo curioso e
ilógico.

\asterisc

\textls[-15]{Renée Rey tem um corpo feio, mãos muito delicadas, delgadas e frágeis na
articulação, pernas temerosas, face morena, lábios queimados por uma
febre permanente e olhos ensombrecidos. Vestida, ela tem, apesar das
roupas que lhe caem bem, um ar encabulado, de modo que elas parecem
inadequadas e não lhe pertencer. Só ao anoitecer, quando esfria e joga
nos ombros um xale bordado de seda, que a cobre por inteiro, ela
recupera a graça vegetal que Ştefan nela percebera com indiferença,
aliás, desde o primeiro momento. Nua, torna-se muito mais jovem do que
é, os quadris se delineiam crus, impudicos graças às coxas compridas de
adolescente.}

--- Renée, você é a mulher mais nua do mundo.

--- Que besteira você está falando. Como pode uma mulher nua ser mais
nua do que outra mulher nua.

--- Pode ser. Mas talvez você não entenda. Porque estar nu não significa
estar sem roupa. Há mulheres nuas e há mulheres sem roupa. Você é uma
mulher nua.

\textls[15]{Renée se crispa, cansada por causa dessa distinção que não compreende, e
a crispação sublinha ainda mais os traços afilados de seu rosto.}

--- Tem algum tunisiano na sua família?

--- Um de verdade?

--- Sim.

--- Nenhum. Mas por quê?

--- Sei lá. Há algo não europeu em você. Não sei bem o quê: o cabelo
áspero demais, o corpo delgado demais, a pele opaca demais e os lábios,
esses lábios que ardem.

\textls[15]{--- Não. Não tem nenhum. Lá, todas nós somos assim. Talvez por causa do
sol\ldots{}}

\textls[-20]{Ştefan gosta de grudar seu rosto à pele dela e passá-lo às vezes por
todo o seu corpo ora ardente, nos momentos de paixão, ora frio,
escorregadio e impermeável, como as folhas da palmeira de interior.}\looseness=-1

Nos instantes de tranquilidade, quando a solta de seus braços, cansada,
de olhos fechados, Renée fica ao lado dele, ausente, disseminando ao seu
redor uma espécie de grande sombra vegetal.

Mais tarde, em seguida, reagindo ao seu chamado, ela tem uma crise
inexplicável de pudor, que a faz cobrir o rosto com as mãos, encolher
desesperadamente as coxas morenas, fechar-se dentro de si e recusar-se
de maneira teimosa, absurda e violenta, até o momento em que, por
cansaço ou por capricho, se submete com uma alegria desavergonhada e
pueril.

Após seu primeiro momento de amor, lá em cima, no quarto conjugal, numa
manhã imprevista, Renée Rey se refugiara sem explicações numa pose
perfeita de esposa virtuosa.

--- Senhora Rey --- sussurrou-lhe Ştefan à mesa, ao acaso ---, encontrei
um quarto no bairro antigo da cidade. Amanhã, às três horas,
planejaremos um breve passeio pelo lago e nos perderemos casualmente do
grupo. Vamos ver o quarto. Certo?

--- Não.

Não teve tempo de lhe perguntar por quê, pois o marido justamente se
aproximara. Mais tarde, ao lhe pedir explicações, ela falou, de maneira
estúpida, sobre seus remorsos, sobre seus deveres\ldots{} Coisa que não
a impedira, na tarde do dia seguinte, enquanto todos tomavam café no
quintal, de ir até o quarto dele, jogar-se aos seus braços, arrancar o
vestido com gestos de pânico e o beijar atropeladamente, murmurando de
vez em quando ``que não apareça o Marcel'', com voz apaixonada, como se
fossem palavras de amor, e não de medo, perdendo-se nos seus braços com
gritinhos arrepiados, ao mesmo tempo que se ouviam, através da porta
entreaberta, passos no corredor.

Eles estão agora no quarto do bairro antigo, aonde Renée não quis ir mas
foi, um cômodo de paredes brancas, mobília metálica, janelas abertas,
decoração desprovida de mistério, em meio à qual a imagem do ``marido
ultrajado'' seria tão ridícula que parece improvável. Só às vezes,
quando evocam o marido, Renée cobre o rosto e diz, com uma entonação que
não é dela:

--- Oh, não mereço um homem como ele.

O que é, aliás, uma réplica adotada recentemente, ouvida decerto em
algum espetáculo da Comédia Francesa, entre os vários que viu em Paris,
na sua breve passagem por lá.

O fato foi percebido ou não? Sim e não. É possível, pois, se Renée foi
imprudente e patética, Ştefan foi metódico e preguiçoso.

\textls[15]{Mas é possível que uma jovem mulher se deite com um jovem homem, numa
casa cheia de gente desocupada, sem que ninguém fique sabendo? Por
enquanto, nenhum indício. Fala-se ainda às vezes sobre Marthe Bonneau, o
que afasta outras suposições. Senhor Rey continua jogando bem xadrez e
seus apertos de mão não parecem de nada suspeitar.}

\textls[-10]{Só a Nicolle, do nada, certa vez irrompeu em prantos quando Ştefan lhe
perguntou uma coisa, algo totalmente sem importância.}

--- Por que, Nicolle? Por quê?

\textls[10]{Senhor Rey a castigou na hora, pois ``ninguém deve fazer nada sem motivo
na vida, nem mesmo chorar''.}

\textls[10]{Que homem estranho, pensa Ştefan consigo mesmo, fitando-o enquanto
prepara com imenso vagar os deslocamentos das peças de xadrez. De
qualquer modo, muito mais estranho que a Renée, tão cansativa, desigual
e apaixonada. Que mãos de lavrador. Que olhar de silvicultor. Que
silêncio teimoso, monótono, sem entrelinhas, sem preocupação.}

Houve um espetáculo de opereta certa noite na cidade, e todos decidiram
ir juntos. Combinaram de manhã uma noitada: vestidos longos e trajes
pretos. Ao se encontrarem no cais para esperar o barco, a aparição de
Marcel Rey, entre vestidos de seda e \textit{smokings}, foi constrangedora,
envergando um fraque que desfigurava a sua silhueta de jovem, e um
chapéu de plush grande demais, como se houvesse sido emprestado. Renée
teve uma pequena crise de histeria dificilmente reprimida, e Ştefan
sentiu vergonha de sua própria elegância, tão barata e tão triunfante.

\textls[-25]{O ombro direito do senhor Rey está mais inclinado que o outro.}

--- É impossível consertar esse mau hábito --- queixa-se Renée.

\textls[20]{--- Por que consertar? Acostumei-me assim: nesse ombro eu carrego a
espingarda.}

--- Espingarda! --- admira-se, assustado, o senhor Vincent.

\textls[20]{--- Sim, ao alvorecer e à noite, quando patrulho a plantação em
Djedaida.}

Djedaida! Quantas vezes Ştefan Valeriu tentou imaginar a vida dura de
lá, naquela família de antigos colonos, com avós que passaram pelas
guerras da primeira colonização, com primas jovens, que muito tempo
atrás fizeram uma viagem a Paris e desde então ficaram melancólicas, com
noites de festa, quando se reúnem todos na casa dos velhos Rey para
escutar discos no gramofone, com madrugadas de insônia no verão, à
espera do vento escaldante que sopra do deserto e branqueia a copa das
palmeiras com uma cinza fina e prateada ao luar\ldots{}

--- Oh, por que Marcel não quer que nos mudemos para Paris? Pense como
seria ótimo. Eu poderia te visitar, poderíamos sair juntos, tomar chá no
Berry, na Champs-Elysées\ldots{}

\textls[15]{--- Você tinha razão, Renée, não tem nenhum tunisiano na sua família.}

--- Por que diz isso?

--- Nada.

\section{v}

\letra{O}{dette} \textls[20]{Mignon tem dezoito anos, usa um barrete azul cobrindo
obliquamente a nuca, um vestido esportivo apertado por um cinto de couro
e, nos pés, sandálias brancas, sem meias.}

\textls[15]{Ştefan a conheceu uma noite, no terraço da pousada, quando, enquanto
seus amigos brincavam em roda com um anel e um barbante, ela olhava
noutra direção, observando como anoitecia por sobre o lago.}

--- Não quer brincar conosco?

--- Mas é claro.

\textls[15]{Entrou na roda e brincou com entusiasmo, cantando junto com todos quando
era hora de cantar.}

\begin{verse}
\emph{Il court, il court le furet}\\
\emph{Le furet des bois jolis\ldots{}}\footnote{Em tradução livre, ``Ele corre, ele corre, o furão; O furão dos bosques bonitos''. \textsc{{[}n.\,e.{]}}}
\end{verse}

\textls[10]{O anel passava de mão em mão, escondido, e a pessoa do meio tinha que
adivinhar quem é que estava com ele, o que obrigava os outros a passá-lo
rápido de um para o outro, ou fingir passar. Renée Rey, posicionada ao
lado de Ştefan, tinha a oportunidade justa de apertar a mão dele com
força, o que o fez, algumas vezes, deixar-se apanhar para poder sair da
roda e mais tarde se colocar ao lado de Odette, que brincava
concentrada, com boa-fé e espírito esportivo, sem apertos de mão que não
fossem estritamente necessários ao jogo.}\looseness=-1

\asterisc

Chovera o dia todo e só ao anoitecer, durante o jantar, lá pelas sete
horas, o céu se iluminou um pouco do lado direito do lago. Pelas janelas
do salão se viam, ao longe, as montanhas cobertas por uma luz violeta,
com labaredas crepusculares.

--- Arco-íris! --- gritou alguém e todo o mundo pulou da mesa, senhor
Vincent com o guardanapo no pescoço, Renée frenética, as crianças
espantadas, todos correndo até o terraço, de onde se podia avistar
melhor a maravilha: um arco-íris imenso, coroando todo o vale e tingindo
todo o lago de um azul angelical.

\textls[15]{Só não deixaram seus lugares Ştefan Valeriu --- que continuou comendo
sossegado --- e, no outro canto da sala, Odette Mignon, igualmente
insensível.}

--- Não tem curiosidade de ver o arco-íris?

--- Não.

--- Mas deve ser muito bonito.

--- Muito bonito; e um pouco trivial.

``Eu não teria encontrado essas palavras'', pensa Ştefan, dirigindo à
moça um gesto de aprovação, com a admiração desinteressada com que um
atacante do futebol cumprimentaria seu companheiro de time por ter
marcado um belo gol.

\asterisc

\textls[10]{Se lograsse fazê-lo sem ostentação, Ştefan não teria se sentado ao lado
de Renée Rey. Mas foi inevitável.}

É um ônibus de excursão, com bancos paralelos de três assentos cada. À
direita de Ştefan, Renée; à esquerda, Odette. Senhor Rey está bem na
frente, ao lado do motorista, com um guia na mão, dando informações
geográfico-históricas em voz alta.

--- Atenção, \emph{le col de la Caussade}!\footnote{Em tradução livre, ``a passagem de Caussade''. \textsc{{[}n.\,e.{]}}} Atenção, \emph{pont du
Query}! Altura de 1\,816 metros. Não, desculpe, 1\,716\ldots{}

\textls[-10]{Por vezes manda parar, para tirar uma fotografia ou gravar alguns metros
de filme. Como o sol ainda não nasceu e faz muito frio, estão todos
cobertos com mantas, ocasião para Renée pegar na mão de Ştefan e
apertá-la com emoção, enquanto, à sua esquerda, Odette agita as suas,
soltas, vívidas naquele ar gelado das cinco da manhã, ora apontando para
um álamo ao longe, ora para o pico de uma montanha, ou para uma rede de
pescadores no lago. Ştefan se deprime profundamente, desmesuradamente,
com essa mão cativa, e lhe parece que, se tivesse coragem de liberá-la
do aperto, ficaria subitamente feliz. Sente o braço dela pesado, mole,
ainda sonolento, com uma sensação de alcova que lhe parece obscena
naquele início de manhã que vibra de luz e som.}

--- Você não me ama mais.

\textls[15]{--- Oh, sim, sim. --- E se não soubesse que seria inútil, explicaria a
ela que agora se trata de outra coisa e que ela está confundindo de
maneira estúpida coisas diferentes, muito alheias umas das outras.}

Pela hora do almoço, param num mosteiro nas montanhas perto de Grenoble
--- uma cartuxa --- e fazem uma visita obrigatória às celas, à
biblioteca e à capela, conduzidos por um guia, que informa
objetivamente: aqui jazeu São Bruno por três anos seguidos, ali temos um
vitral do século \textsc{xiii}, aqui pernoitou o papa quando atravessou as
montanhas rumo a Avignon\ldots{}

Renée parece muito interessada nas explicações e fica sempre para trás
do grupo, pendurada no braço de Ştefan, a quem pede esclarecimentos
suplementares para, em seguida, escondida atrás de uma porta ou na curva
de um corredor, roubar-lhe um beijo.

Numa das celas, Ştefan a flagra tateando o estrado de madeira da cama e
então imagina, maldoso, que, naquele momento, ela deveria estar pensando
no quão incômodo seria fazer amor ali.

Mal tendo chegado a Grenoble, ele consegue se separar do grupo, feliz
com aquela liberdade inesperada de passear sozinho pelas ruas de uma
cidade desconhecida, onde ninguém o conhece e onde não conhece ninguém.
Ao passar pelas vitrines das lojas, vira a cabeça para observar seu
reflexo na vidraça, e aquela silhueta esguia lhe parece a de um amigo
resgatado.

Fica numa livraria folheando revistas e livros novos, todos publicados
naqueles dois meses em que desaparecera do mundo, pedindo avidamente
inúmeras informações ao livreiro, espantado com aquele freguês, que nada
compra e tudo quer saber.

Quase não nota Odette Mignon, que aparece também por lá após fazer umas
compras, surpresa por encontrá-lo.

--- Se quiser me fazer feliz, deixe-me escolher um livro para você. E
deixe-me oferecê-lo a você. Olhe, esse aqui, por exemplo.

Ao lhe estender o livro, ele tem a sensação de que aquele gesto de
oferecer algo consegue apagar, de repente, a lembrança daquela manhã
constrangedora, resgatando-a.

\asterisc

Senhor Rey pendurou na parede da sala de jantar um cartaz manuscrito em
letras grandes: ``Hoje à noite, um único e grandioso espetáculo de cinematógrafo, no
terraço da pousada. Na programação, curtas-metragens absolutamente
inéditos''.

\textls[15]{De fato, chegaram-lhe de Paris filmes que ele havia mandado revelar ---
todas as excursões, todos os longos passeios no lago, algumas tardes no
terraço\ldots{} Uma infinidade de cenas que haviam esquecido e que
consideravam definitivamente parte do passado, mas que continuam
existindo naquele baú de madeira entregue pelo carteiro.}

Estão todos ansiosos, como antes de uma estreia e, até o anoitecer,
passam o tempo com conversas irritadas, impacientes.

\textls[15]{--- Você vai ver como será bonito! Que ao menos seja nítido. Você se
lembra do chá no passadiço, quando passeamos pelo lago todo? Estou louco
para ver como saiu. Você vai ver que bonito, você vai ver.}

\textls[15]{Os outros visitantes da pousada, que não fazem parte do seu grupo,
também se deixam contaminar pela impaciência geral, pois estão todos
convidados para o espetáculo, uma espécie de estreia de gala da região.}

\textls[15]{Antes de escurecer, senhor Rey fixa a tela e prepara o aparelho de
projeção, enquanto Renée, a anfitriã, indica os lugares a todos,
mantendo Ştefan do seu lado, depois de colocar Odette casualmente
sentada no canto oposto, junto com o senhor Vincent e Nicolle.}

\textls[15]{Os primeiros metros de filme são recebidos com aclamação, cada um se
reconhecendo na tela e se cumprimentando --- as madames, com gritinhos
de entusiasmo (\textit{Ah! Nossa! Olha! Não! Isso não!}), os homens, com um
sorriso de vaidade leviana ou --- o senhor Vincent, por exemplo --- com
uma gargalhada bombástica, como se dissesse, a cada nova aparição na
tela: ``Ah, mas essa é muito boa''.}

Ştefan mal se reconhece na imagem projetada na tela, irreal, quase
impossível, pois lhe parece anormal que, enquanto ele está ali, no
jardim, imóvel na cadeira branca de vime, alguém que é também ele
caminha e dá risada, livre, mais livre do que ele mesmo, para sempre
fora de seu próprio controle.

Eis a Marthe Bonneau, num barco, eis o Marc correndo por uma alameda
sabe-se lá atrás de quem, eis de novo a Marthe grandiosa,
cinematográfica, eternamente bela\ldots{}

A cena muda de novo: o passeio até Lovagy (``Lembra-se, Ştefan, fazia
dois dias que havíamos nos conhecido'').

\textls[-10]{Mas por que na tela Renée o segura sempre pelo braço? Por que agora se
apoia no ombro dele? O que é esse ar de ternura, do qual não se lembra?
Não. É impossível. Não foi assim. Não podia ter sido assim. Eram
desconhecidos. Ele falava com ela respeitosamente. Ela lhe respondia com
frieza. Tudo se torna irreconhecível na tela; tudo é de outra maneira,
mais animado, mais caloroso, mais íntimo.}

\textls[-15]{Quanto mais avançam os filmes --- outras cenas, outros passeios ---,
mais atrevidos se tornam os gestos da mulher na tela, o ar deles, dos
dois, mais cúmplice, e todas aquelas imagens velozes, sem terem nada de
flagrante, mantêm um tom exagerado de intimidade. Há um quê de adultério
em todas as imagens e nem mesmo se pode dizer o que exatamente. Talvez o
olhar desvirtuado de Renée, talvez a ausência contínua do marido, que
não aparece em nenhum momento do filme, sempre ocupado com a filmagem.}

Ştefan tem a impressão de que se ri menos em torno dele, ou talvez mais
alto, de todo modo é um riso significativo, embaraçado, como se todo o
mundo houvesse entendido.

\textls[15]{--- Marcel, vamos terminar. Podemos continuar amanhã à noite. Já é
tarde.}

\textls[15]{--- Mas por que, Renée querida? Só são onze horas e todos estão se
divertindo. Não é mesmo, senhoras e senhores? Ademais, não projetei nem
metade ainda. Olhe essa cena, por exemplo. Você se lembra, no bairro
antigo, quando fiquei para trás para comprar selos\ldots{}}

--- Marcel, por favor\ldots{}

\textls[15]{--- \ldots{} e você continuou com o senhor Valeriu. Olhe ali, fotografei
vocês até virarem a esquina.}

Ele sabe? Caso sim, por que está tão calmo? Caso não, por que insiste em
explicar cada passagem e dar explicações constrangedoras que ninguém
pediu?

\textls[10]{Ştefan Valeriu não entende mais nada. Tem medo de erguer o olhar e, às
vezes, ao sentir um par de olhos dirigido a ele, sobressalta-se,
desconcertado. Só o olhar de Odette Mignon chega até ele, como sempre,
cristalino e sem rodeios. Ela pelo menos de nada suspeitará.}

\asterisc

Como as coisas se simplificam quando, pela manhã, no lago, Ştefan se
estica no fundo do barco à deriva, com remos soltos! Como se distanciam
e despencam intranquilos lá de cima, da pousada, dramas insignificantes
e heroínas cansativas!

\textls[15]{Só Odette Mignon, companheira de nado e remo, nua, bronzeada e boa
camarada como um novo Marc Bonneau, interrompe, na proa, o círculo azul
que Ştefan admira entre os montes à esquerda e à direita.}

--- Faz tempo que a senhora Rey é sua amante?

--- Amante?

--- Sim. Quero dizer, faz tempo que vocês se deitam juntos?

A precisão da pergunta não permite mais nenhuma resposta. Odette, aliás,
nem parece aguardar.

\textls[15]{--- Ah, a esse respeito, os filmes documentários do senhor Rey foram
verdadeiros documentários. Adorei. Se você não tivesse fechado tanto a
cara ontem à noite, poderíamos ter tido uma excelente conversa sobre
cada detalhe.}

--- Creia-me, foram minutos realmente desconfortáveis.

\textls[10]{--- Eu sei. Quero dizer, suponho, pois pessoalmente eu não podia fazer
nada além de me divertir. Acho que você exagerou e ainda está
exagerando. Ninguém percebeu.}

--- Você acha?

\textls[15]{--- Tenho certeza. Na tela só apareceram nuances: nenhum fato. Nenhum
beijo, por exemplo, ou qualquer outra coisa do gênero, incontestável. E
nenhum dos nossos amigos lá de cima é capaz de inferir algo a partir de
uma nuance. São gente comum, acostumada a olhar para arco-íris\ldots{}}

--- Mas e o senhor Rey?

--- Mistério. Ele me intriga tanto que, se eu soubesse que poderia me
dar uma resposta, iria agora mesmo, assim como você está me vendo, de
maiô de banho, molhada e descabelada, iria lhe perguntar. É uma pessoa
cruel ou um cretino.

--- Quantos anos você tem, Odette?

--- Você já me perguntou. Dezoito.

--- Você é muito inteligente e sabe de muitas coisas.

\textls[-10]{--- Sou virgem. Isso me ajuda a ser inteligente. Além disso, vivi muito
tempo sozinha. Meus pais se divorciaram quando eu tinha doze anos. Mamãe
continua jovem. Papai é rico e ambicioso. Tanto ela quanto ele
continuaram amando e tentando, cada um do seu lado. Ambos me fizeram
confidências como a uma amiga. Assim como se podiam fazer a uma menina
que cresceu rápido e que não os incomodava em nada. Aprendi com eles
tudo o que sei: acho que, às vezes, consegui lhes dar, em troca, bons
conselhos. Para o papai, quando precisava de uma gravata nova, no início
de uma relação amorosa; para a mamãe, quando a vida lhe parecia
terminada, por causa de um cretino que a abandonou.}

--- Você é um garoto, Odette.

--- Como quiser\ldots{}

--- Um garoto de jaqueta azul, saia branca, tamancos, cabelo loiro,
punhos estreitos e olheiras. Me dê sua mão e me deixe sacudi-la com um
aperto entre homens. Se você fosse um pouco mais feia, eu te daria um
par de galochas, te ensinaria a fumar cachimbo e iríamos juntos para as
montanhas, dormiríamos à noite nos refúgios alpinos, cada um numa cama
dura de madeira, longe do amor, dos desmaios e das complicações
psicológicas.

\asterisc

Renée Rey está doente. As persianas das janelas dela permaneceram
fechadas o dia todo. Faltou ao almoço, e o senhor Rey, soturno, desceu
só com a Nicolle, e almoçou com apetite.

--- É um carrasco --- disse alguém, uma madame da pousada.

``É uma pessoa simples'', pensou consigo Odette.

Ao entardecer, chegou o médico, que Renée não quis receber, mas que o
marido introduziu de maneira autoritária.

--- Preciso saber o que ela tem.

\textls[15]{``Não tem nada'', foi a resposta do médico ao partir, e isso quase
enfureceu o senhor Rey, que caminhava pelo quintal, sombrio, a passos
largos.}

--- Mas se ela não tem nada, você não precisa se aborrecer --- ponderou
Odette, com inocência.

\textls[-10]{--- Não tem nada, mas está pálida, não tem nada, mas não come, não tem
nada, mas desmaia. Para uma esposa de fazendeiro, é uma doença demasiado
sutil, senhorita. Lá, nós estamos sãos ou prostrados, estamos de pé ou
caídos. Quando estamos bem, estamos muito bem; quando estamos doentes,
estamos muito doentes.}\looseness=-1

\textls[15]{Logo após a partida do médico, enquanto Odette ouvia, no terraço, as
explicações do senhor Rey, uma empregada foi atrás de Ştefan Valeriu.}

--- A senhora Rey pede que o senhor suba sem falta.

\textls[10]{Encontrou-a nua, jogada de atravessado sobre a cama matrimonial, muito
pálida, mas com os olhos cintilantes de febre. A luz do crepúsculo,
filtrada pelas cortinas pesadas que tapavam por completo as janelas,
aumentava seu palor e atirava grandes manchas de sombra sobre os
travesseiros.}

--- Você está doente?

--- Não, eu te amo.

\textls[10]{--- Querida Renée, aprecio, mas isso é hora de me dizer? Agora, quando o
seu marido, alarmado, pode chegar aqui em cima a qualquer momento?
Quando a pousada toda não desprega os olhos das tuas janelas de
paciente?}

--- Você nunca vai entender nada. Põe a mão em mim e veja como estou
pegando fogo. Me dê um beijo e veja o quanto te esperei. Meu
marido\ldots{} toda a gente\ldots{} fique aqui.

\textls[10]{Seu corpo adotara uma inclinação clamorosa, parecendo revelar sua mais
obscura intimidade, sua mais surda raiz de vida. Tinha a impressão de
que, caso se aproximasse dela, sentiria seus lábios queimados, não as
coxas luzidias de árabe, mas a aorta pulsando de demasiado sangue, as
veias abertas para o receber, o coração como uma ferida.}

\textls[15]{Hesita por uma fração de segundo e, no tumulto que o arrasta na direção
da mulher, ergue-se dentro dele, como um grito, a ideia de que, naquele
momento, algo definitivo se decide, o que o faz se virar repentinamente
e abrir a porta, batendo-a atrás de si e correndo pela escada, liberto.}

\asterisc

Na hora do jantar, Odette encontrou na sala, em cima da mesa dela, um
telegrama. Abriu-o e o leu sem pressa.

--- É do papai. Vai passar aqui depois de amanhã, cedinho, com um
automóvel, para me pegar. Vamos para Antibes.

--- Você o esperava?

\textls[-25]{--- Não. Mas ele sempre faz essas surpresas e sabe que eu gosto.}

\asterisc

\textls[20]{Odette está lá em cima, no quarto dela, para onde subiu logo depois do
jantar, a fim de escrever algumas cartas e arrumar as coisas.}

\textls[-20]{--- Quero ficar livre amanhã o dia todo, e é melhor que eu prepare tudo
hoje para a partida. Papai não gosta de esperar. Boa noite.}

\textls[20]{--- Boa noite. Ainda fico por aqui. Mas, se você quiser, mais tarde,
quando for me deitar, bato na sua porta para mais um dedo de prosa.}

--- Encantada.

\textls[15]{Faz tempo que Ştefan não fica sozinho, à noite, no terraço. Talvez desde
a partida de Marthe Bonneau, quando nada dessa história de amor
acontecera.}

``Como deve ter sido bom naquela altura'', tenta lembrar enquanto olha,
no escuro, para a brasa do cachimbo, acendendo e apagando, como um
coração que bate.

\textls[15]{Ficaria feliz em poder esquecer tudo o que acontecera desde então, e
limitar sua memória àquela noite aconchegante e ao ponto de queima do
cachimbo.}

Que pena não estar chovendo. Ficaria assim, com a gola desabotoada,
cabeça descoberta, mangas arregaçadas, ficaria na chuva, apoiado num
tronco de árvore, e deixaria os filetes d'água passarem pelo cabelo,
pela testa, pelas faces, até se sentir, junto com a vegetação ao seu
redor, parte da noite da terra, da sua completa insensibilidade, para
sempre livre de escrúpulos e remorsos, livre da obsessão das janelas
surdamente iluminadas, lá em cima, onde aquela mulher patética arde num
amor excessivo.

No entanto, não chove, não há de chover e a noite é insuportavelmente
bela, com esse lago teatral, a lua cheia, as estrelas refletidas na
água, as montanhas prateadas.

\textls[15]{``Nunca chove na hora certa'', observa Ştefan, contrariado, e vai
dormir, contente, de qualquer forma, com a perspectiva de se demorar uns
quinze minutos no quarto de Odette, para baterem um papo trivial.}

À primeira batida na porta, no entanto, ninguém responde.

--- Odette!

\textls[10]{Embora a luz esteja acesa e possa se ouvir claramente a sua respiração,
abafada, do outro lado da porta.}

--- Que brincadeira é essa, Odette? Por que não responde?

--- Ah, é você? Boa noite.

--- Boa noite. Vim para batermos um papo.

--- Oh, já é tarde. Desculpe, estou com sono.

\textls[20]{--- Sem problema nenhum, mas abra a porta para eu apertar sua mão.}\looseness=-1

\textls[15]{Ştefan aguarda por alguns segundos, sem saber se aquilo tudo o diverte
ou o irrita.}

--- Ouça, Odette. Estou falando muito sério agora: se houver algo que te
impede de me ver, me diga que eu vou embora. Mas se não tem nada, abra
um momento. Quero apenas te desejar boa-noite e ir para a cama, eu
também estou cansado.

\textls[20]{--- Não tem nada, absolutamente nada, mas não posso abrir a porta.}

--- Por quê?

--- Porque sim.

--- Já vou avisando que não vou sair daqui enquanto você não me der uma
explicação plausível ou abrir a porta.

\textls[15]{Embora a moça não responda mais, Ştefan parece vê-la do outro lado da
porta, chateada, com punhos cerrados e lábio inferior caído, sorriso
amuado que ela assume sempre que se sente desarmada numa discussão.}

--- Fique sabendo que estou esperando. Olhe, acendi o cachimbo,
escorei-me na parede, pus as mãos no bolso e estou esperando. Até a uma,
até as duas, até de manhãzinha.

\textls[15]{Odette apagou a luz. Provavelmente foi se deitar e, da cama, mantém os
ouvidos atentos para descobrir se ele foi embora ou não. Às vezes, com
voz baixa, suplicante, batalhando contra o sono, murmura:}

\textls[15]{--- Ştefan, vá dormir. Ştefan, está tarde. Ştefan, você vai ficar
cansado amanhã\ldots{}}

\textls[15]{É a vez dele de não responder, crispado e rabugento, decidido a não sair
dali, embora saiba muito bem que, aquela noite, a porta não haveria de
se abrir.}

\asterisc

\textls[15]{Ştefan Valeriu saiu ao raiar do sol e só voltou à noite, depois do
jantar. Foi passear por algumas aldeias vizinhas, fumou muito e
conversou circunspecto, com os camponeses que encontrou, sobre a
colheita e o tempo.}

\textls[10]{``Talvez não seja bonito me esquivar'', pensou várias vezes no caminho,
mas as janelas misteriosamente fechadas de Renée Rey incentivavam, de
longe, a sua fuga.}

--- É melhor, é muito melhor assim.

\textls[15]{No que diz respeito a essa Odette, talvez não fosse má ideia puxá-la
pelas orelhas e lhe dizer que a piada da noite passada tinha sido uma
travessura de criança encrenqueira, da qual ele não gostara nada.}

De volta à pousada, sentiu-se satisfeito ao ver a sala de refeições
vazia e que todos haviam ido dormir, pois já era tarde. Luz baixa na
janela do casal Rey, escuridão na de Odette Mignon.

--- Melhor assim.

\textls[15]{No entanto, deveria lhe desejar boa viagem. Amanhã, antes do alvorecer,
ela haveria de partir e com certeza não a veria nunca mais.}

--- Era simpática às vezes\ldots{}

\textls[15]{Ao subir para o quarto, dá risada, percebendo que, com aquele ``era'',
ele encerrava o episódio com a moça bonita, um pouco tantã e de
olheiras.}

Sente-se cansado como um estivador e, a cada degrau, parece se aproximar
de uma felicidade sem igual: descalçar as botas, esticar os braços
desnudos, desabar sobre o lençol frio da cama.

\textls[15]{Seis passos ainda, dois, nenhum. Apanha a maçaneta da porta com todo o
peso do corpo, aperta, abre, entra --- tudo com a volúpia lenta de um
errante que chega ao destino --- e gira o botão da luz.}

--- Boa noite, Odette.

Por que não se sobressaltou? O lógico, o natural, o necessário seria
sobressaltar-se. Pelo menos isso: um sobressalto. Encontra-a, àquela
hora, no quarto dele, na cama dele, nua, calma, familiar e, naquilo que
deveria ser uma estupefação imensa, naquilo que deveria ser uma explosão
ruidosa, ele só consegue lhe dizer isso:

--- Boa noite, Odette.

--- Boa noite, Ştefan.

Aproxima-se dela, beija-a em ambas as faces, acaricia seus joelhos
roliços e, em seguida, tira das costas a mochila de excursão.

--- Sabe, estou bastante cansado. Caminhei muitíssimo hoje. Faz tempo
que está me esperando?

--- Sim. Umas duas horas.

--- E não se entediou?

\textls[15]{--- Não. Apaguei a luz, tirei a roupa e deitei na cama. É bonita a
paisagem daqui, com o bosque.}

Ele continua tirando a roupa, sem pressa, imperturbável.

--- Novidades na pousada?

\textls[15]{--- Nada. Nem hoje a senhora Rey desceu, e o senhor Rey perguntou por
você. Ao anoitecer, despedi-me de todos e fui fechar as malas. Só deixei
o vestido de viagem para fora e veja só, você acabou de sentar em cima
dele.}

--- Perdão. Apago a luz?

Deteve-se nu, em frente à cama, tranquilo, sem emoção, sem pudor,
cúmplice do corpo dela, como se o conhecesse há tempos.

--- Sim, apague.

Eles se enlaçam em silêncio, ela se perdendo toda nos braços dele, ele a
cobrindo dos tornozelos à cabeça, feliz por aquele corpo robusto e
delicado não estremecer e não se apressar. Sente os seus seios calmos,
escuta as batidas ritmadas do seu coração, ouve sua respiração
sossegada. As coxas da moça se abrem como asas, dóceis, mas com um quê
de decisão no movimento.

\textls[15]{É um corpo obediente e atento, que acompanha o seu com confiança,
respondendo preciso às suas intenções, como ao toque das teclas de um
piano. Na escuridão, eles não se procuram, não se perdem, não se falam:
tudo é harmonioso como dois caules que crescem juntos.}

\textls[-20]{E o grito de Odette, um único grito, de dor, de triunfo, de liberdade,
não assusta nem a si mesma, nem a ele, puro que é, alto, aguçado, saindo
pela janela aberta rumo ao bosque, perdendo-se por entre as árvores para
despertar um esquilo ou se encontrar flutuante com o grito remoto de um
gato selvagem, igualmente livre.}

--- Está chorando, Odette?

Não, não está. Ela só está mais quente, e seu corpo ferido se inclina
mais sobre o dele, firme, decidida, mas com os ombros um pouco mais
pesados e os braços derrotados sobre o travesseiro.

--- Está com sono, Odette?

Não, não está. Nunca esteve mais desperta, nunca esteve menos atordoada,
nunca percebeu melhor o que acontece ao seu redor. Olhe, essa é a sua
mão, esse é o meu joelho, essa é a sua boca áspera, essa aqui é a minha
orelha, que você beija sem me dar arrepios, essa é a sua omoplata larga
demais, esse aqui é o meu pulso e logo ali, veja, é o alvorecer que se
insinua\ldots{}

\textls[10]{Em breve, da direção de Serrier, vai-se ouvir o rumor de um automóvel
que se aproxima. Deverá partir, respondendo à buzina lá de baixo, da
rua, que a chamará.}

\textls[15]{Por que não chora, por que não lhe pede que a segure ali, por que não se
agarra mais febril, por que permanece indiferente ao lado dele e por que o ama como se fosse por uma eternidade, e não só por um instante?}

\asterisc

\textls[15]{Odette Mignon está na soleira, com o mesmo vestido branco, a mesma
jaqueta azul do dia em que chegou, pronta para partir, de mala na mão.}

--- Cuide-se, Ştefan.

Detém-se na soleira.

--- Odette.

--- Sim.

\textls[-20]{--- Agora me diga: por que você não abriu a porta ontem à noite?}

Ela reflete um pouco.

--- Não sei, Ştefan, juro que não sei.

\pagebreak

\section{vi}

\letra{C}{hegou} bonito o mês de setembro, com luzes cansadas. No lago, rarearam
os botes, guardaram as velas, os barcos brancos passam com menos
frequência. Um cartaz avisa no cais que o barco das 8h27 da noite foi
suprimido.

\textls[15]{A cada dia, a persiana cobre novas janelas da pousada: as pessoas estão
indo embora.}

\textls[15]{Amanhã de manhã, essa janela, que hoje sorri ao sol, com cortinas
brancas esvoaçantes, ainda se abrirá? E a do seu lado? E a de cima dela?}

Fecham-se uma após outra, como luzes que se apagam.

Já faz alguns dias que Renée Rey passou a descer do quarto e caminhar
após a refeição, por volta das duas horas, ao sol, sozinha ou com o
senhor Rey, de braços dados, calados. Por vezes para e faz carinho no
cachorro da casa, um pastor, imenso e peludo. Está muito mais pálida,
parece mais alta e, quando olha para alguém, dá um sorriso de
convalescente.

Trocou algumas palavras banais com Ştefan Valeriu, não com mais tristeza
do que com outras pessoas.

--- É tão agradável aqui fora e era tão angustiante lá em cima. Estou
com saudade do sol do nosso país.

\textls[15]{Dizem que vão embora em um ou dois dias. Escreveram para Marselha para
perguntar sobre o tempo, Renée tendo necessidade de uma travessia
sossegada.}

\textls[15]{Ao entardecer, ela fica no terraço, espichada na espreguiçadeira,
enquanto o senhor Rey e Ştefan jogam xadrez. Como nos primeiros dias.}

Quando a noite se instala, veem-se ao longe, bem para além do lago, as
luzes da estação ferroviária e, lá pela meia-noite, o trem que vai para
Paris, como uma serpente articulada e fosforescente.

\textls[20]{Detêm-se no meio da partida e o acompanham com o olhar, à distância.}\looseness=-1

\textls[15]{--- Levamos uma vida dura --- diz o senhor Rey, do nada. --- Não me
arrependo, e não a trocaria por outra. Mas é dura. Tenho certeza de que
os olhos de Renée se enchem de lágrimas ao fitar, como nós, a passagem
desse trem, no qual ela não haverá mais de embarcar, quem sabe, pelos
próximos anos. Nunca mais, talvez. Eu não me assusto com isso, mas, está
vendo, há algo dentro de mim, uma espécie de aflição que me deixa
pensativo. Vai passar, eu sei. Vai passar no caso dela também. O
trabalho recobre tudo isso. O sol, a lavoura, o deserto, o vento
noturno, os árabes\ldots{} Mas deve-se compreender como as coisas são
diferentes aqui, como elas nos seduzem e como uma mulher, em especial,
não consegue resistir a elas\ldots{}}

Esquece a partida e fala com calma, com a marca entre seus olhos mais
profunda. Em seguida, ergue-se de repente.

--- Vou subir para fazer as malas. Vamos embora amanhã. Fique com a
Renée até eu voltar.

Ştefan se dirige até o terraço, onde se vê, opaco na escuridão, o xale
da senhora Rey.

\textls[20]{--- O senhor Rey subiu e me pediu para lhe fazer companhia. Posso?}\looseness=-1

--- Claro.

--- Parece que vocês vão embora amanhã.

--- Não sabia, mas é melhor assim.

Ele se senta na grama e permanece calado por um bom tempo, ouvindo a
respiração da mulher ao lado. Avista um vagalume, apanha-o, mantém-no na
palma para observar como o pobre verme apaga a lâmpada da cabeça, mas
ela pede que lhe dê o vagalume e o coloca no cabelo. No escuro, aquele
ponto de fogo parece um pente mágico, que lança em derredor uma luz
tênue, suficiente para coroar sua cabeça com uma linha branca.

\textls[15]{Tudo parece plenamente apaziguado, no momento em que Renée irrompe em
lágrimas, um choro bom e amistoso, que ele incentiva acariciando as mãos
dela e recebendo-o sem hostilidade, como a uma chuva.}

--- Você ainda vai ficar muito tempo, Ştefan?

\textls[15]{--- Não sei ao certo. Estou esperando notícias de casa. Talvez ainda uma
semana. Talvez mais.}

--- Meu choro não te incomoda?

--- Por que, Renée? É noite. Ninguém está te vendo. E, finalmente,
alguém precisava chorar por todos nós.

\asterisc

Os dias passam infrutíferos, sem novos acontecimentos, deixando para
trás uma impressão de casa desabitada, sem mobília, com aposentos que
ecoam passos de transeunte solitário.

A luz da manhã é crua como clara de ovo, a luz do anoitecer é quente
como o globo de porcelana dos lampiões a gás.

Chegou de Marselha uma fotografia da família Rey, enviada na véspera do
embarque, com saudações cordiais. Ştefan colocou-a na moldura do espelho
e pensou em deixá-la ali ao partir.

Veio também uma carta para Odette Mignon, e a proprietária a deu a ele,
por não saber para qual endereço encaminhar. Nem ele sabe. É estranho
que Odette não lhe tenha dito nada a esse respeito, e mais estranho
ainda é que ele não tenha perguntado.

No quarto dela, encontraram umas coisas deixadas para trás, um bordado,
um livro, uma echarpe, três ou quatro fotografias amadoras. Nelas,
pode-se ver uma Odette atrapalhada, com a saia agitada ao vento, o
barrete azul torto na cabeça, os braços atirados ao ar, tentando
alcançar sabe-se lá que bola imaginária.

\textls[10]{``Passou como uma moça com quem se encontra por acaso no bonde'', diz
Ştefan para si mesmo, fitando, pela janela, o lago deserto, atravessado
por um barco a vela que singra apressado, como um pássaro alarmado.
Junto ao cais veem-se algumas gaivotas, voando baixo, atingindo com o
peito o espelho d'água e em seguida se alçando, desorientadas.}

Ouvem-se na escada os passos arrastados de Aneta, a empregada coxa que,
antes do anoitecer, verifica todos os quartos.

--- Madame Bernard mandou perguntar se o senhor quer que acenda o fogo.
Esfriou, e ela disse que à noite vai chover.

\chapter{Emilie}

\section{i}

\letra{P}{or} \textls[15]{que Emilie Vignon manteve a virgindade até a noite em que conheceu
Irimia \textsc{c}. Irimia, não sei dizer. Preguiça ou falta de imaginação.}

\textls[10]{Tudo tinha como impedir essa castidade tardia. O exemplo das amigas, os
costumes liberais do bairro, sua vida miserável e sem alegrias. Quando a
conheci, era uma moça de vinte anos, encorpada, robusta, de olhar
apagado e face nodosa. Perguntava-me às vezes como ela teria sido antes,
na infância, e, a despeito de todo o esforço, não conseguia imaginar.}

Era, na verdade, um animal manso e, mesmo feia como era, tinha por vezes
um ar de resignação que me atraía.\label{resignacao} Ainda hoje, depois de tanto tempo, só
consigo pensar nela com a sensação de uma amizade triste, e a ideia de
escrever a história dela me consola um pouco de tê-la perdido. É uma
sensação que ela não poderia compreender. Seus olhinhos piscariam e,
sentindo porém se tratar de algo que lhe diz respeito, daria um sorriso
--- aquele seu sorriso de sempre, desfigurado e ausente.

\textls[10]{Lembro-me da noite em que a conheci. Era janeiro, num período em que eu
tentava cicatrizar alguns segredos melancólicos que trouxera, uns três
meses antes, de umas férias passadas na montanha, às margens de um lago
nos Alpes, para onde tinha ido no verão para descansar dos extenuantes
exames de residência médica, e de onde tornara com a lembrança, nem
mesmo hoje plenamente curada, de uma moça loira que me amou sem motivos
e que desapareceu sem explicação. Tentava me recuperar e retomar minhas relativas conquistas de jovem rapaz, esperando, naquele dia de janeiro,
Mado, moça graciosa com quem conversara fazia pouco tempo numa estação
de metrô, e que ainda hesitava em receber o meu amor. (Mais tarde,
descobri que, no bairro dela, toda relação séria pressupunha três
encontros prévios. É uma lei de bom comportamento que devi respeitar).
Chovia naquele domingo, e procuramos em vão um lugar onde ficar. Nenhuma
mesa livre no \emph{bal musette},}\looseness=-1\footnote{\textit{Bal musette} é um tipo de baile popular francês, que se tornou comum depois do final do século \textsc{xix}. São geralmente associados a composições musicais que incluem o acordeão. \textsc{{[}n.\,e.{]}}} \textls[15]{nenhum ingresso sobrando no
cinematógrafo. Caminhávamos frustrados na chuva, detendo-nos vez ou
outra debaixo de raras marquises para nos refugiar, eu, enfastiado com
aquela aventura demasiado virtuosa, ela, Mado, tremendo de frio no meu
braço. Finalmente, desesperada naquela busca inútil, e ante a chuva
que aumentava, a moça se decidiu:}\looseness=-1

--- Vamos na Emilie.

Logo entendi que sua virtude cedera e não pedi explicações.

\textls[-5]{Foi então que entrei pela primeira vez no quarto de Emilie Vignon.
Ficava na mansarda de uma casa suja e torta, nas redondezas da Porte de
Saint-Ouen, curvada por sobre os trilhos da ferrovia. Lá de cima,
podia-se ouvir, a intervalos regulares, o trem suburbano e, se a janela
estivesse aberta, vinha da rua todo aquele barulho surdo de periferia.
Não olhei bem para a dona da casa. Sei apenas que, ao entrarmos, num
canto, no meio da escuridão, erguera-se uma sombra de mulher em cujos
ombros Mado bateu amistosamente e que depois se esgueirou pela porta,
calada.}

Aqui não é o lugar adequado e nem seria interessante falar sobre a Mado.
Basta dizer que era uma amiga dedicada e carinhosa. Naquele dia, ela
precisava se vingar das duas horas de caminhada pela chuva, e se vingou,
nua e ardente.

Só depois de saciar seus primeiros arroubos amorosos é que observei,
horrorizado, que a sombra da mulher, que achava que tinha saído do
quarto, continuava ali, num canto, numa cadeirinha na escuridão. Como
não suporto esse tipo de safadeza, estive prestes a discutir duramente
com a moça ao meu lado, que esticava seu corpo nu, de potro domesticado,
mas ela, compreendendo a minha crispação, me disse, despreocupada:

--- Ah, não é nada. É a Emilie.

\textls[10]{Pronunciou essas palavras com uma indiferença sem fim, como se me
houvesse dito que era um gato, uma cadeira ou uma mesa. Por outro lado,
a sombra da mulher no canto não dava nenhum sinal de vida, de modo que
eu, embora a ideia de me deitar com uma mulher na frente de outra seja
intolerável, não prestei mais atenção nela e respondi devidamente aos
elãs de minha jovem namorada.}

Desde então, encontrei várias vezes essa Emilie Vignon. Mado a mandava
vir até mim com os mais diversos favores, e a pobre moça tinha que
percorrer quilômetros inteiros para me trazer um bilhetinho de amor.
Vejo-a ainda no salão do hospital Trousseau, onde realizava meu estágio
de residente, com aquele seu chapéu de veludo embranquecido, com aquele
seu casaco comprido desbotado, revolvendo entre os dedos o envelope, sem
saber como o estender até mim. Cada movimento que deveria fazer era um
suplício, e acho que jamais vou esquecer o momento embaraçoso que passei
com ela na sala de plantão, para onde a chamara para comermos juntos,
imaginando que lhe faria prazer. Não sabia o que fazer com as mãos, como
escondê-las, e sofria profundamente em sua inquietação aparente.

\textls[10]{Creio que sua vida fora envenenada por aquelas duas mãos, que ela
carregava com a sensação instintiva de sua inutilidade. Pareciam alheias
ao corpo, feitas de madeira, demasiado pesadas. Do ponto de vista do
observador, tinha a impressão de que aquelas mãos estavam sempre tensas
e arqueadas. Uma corveia permanente, e me pergunto se, no fim, Emilie
não teria morrido de cansaço, caso --- conforme veremos --- um acidente
não a houvesse matado um pouco antes.}

Sempre que Emilie se enrolava, ou ficava triste, ou furiosa, estendia as
mãos ao longo do vestido, como se as tentasse esconder ou apoiar em
algo. Costumava pensar que, se as roupas de Emilie tivessem bolsos, sua
vida poderia ter sido muito mais simples.

\textls[5]{Havia na sua rigidez algo doloroso, que não me deixava dar risada. A
única inflexão de seu corpo se devia a um passo disforme. Emilie não era
propriamente coxa: quero dizer, não era doente. No entanto, costumava
apoiar o corpo mais na perna esquerda do que na outra, hábito do
trabalho. Ela trabalhava no subsolo de uma grande loja, no departamento
de embrulho, e toda a sua atividade consistia em acionar, por meio de um
mecanismo de pedal, o barbante da embalagem. Fazia alguns anos que
realizava esse trabalho diariamente, oito horas por dia, e a perna
esquerda acabou absorvendo o ritmo regular do pedal. Não tinha mais como
desaprender.}

\textls[5]{Mas não vou mais lhes dar outros detalhes do aspecto de Emilie. Já lhes
disse que era feia, e isso basta. Pois, de qualquer modo, vocês não têm
como compreender quanta delicadeza fazia parte de sua feiura. Sentia
apreço por ela, e suas amigas, que a torturavam com toda espécie de
favores bárbaros, jamais haverão de esquecer o seu olhar obediente.}

Tinha a discrição de uma toupeira. Esgueirava-se e sumia calada sempre
que se sentia a mais, não falava, não perguntava. Quando a levávamos
conosco para os bailes do bairro, ela ficava sozinha, vigiando nossos
casacos. E quando uma de suas amigas precisava de uma acompanhante para
uma expedição amorosa, Emilie sempre ia e assistia às mais exatas cenas
de volúpia. Não sei dizer se as amigas faziam de propósito essa
safadeza, para a exasperar. Não sei dizer se ela, Emilie, sofria ou não
diante do espetáculo. Só sei que permanecia ali, indiferente, observando
com um olhar imóvel o que acontecia, calma, impassível.

\textls[15]{Como é que Emilie Vignon, levando uma tal vida, ainda mantinha sua
inútil virgindade, é difícil de entender. Pudor, Emilie não tinha.
Convenções sociais não a impediam, pois, em seu universo, ser virgem com
mais de quinze anos de idade era sinal de decadência.}

\textls[10]{Creio que o amor fosse para ela uma dificuldade mais física do que
moral. Se eu não temesse uma expressão equívoca, diria que o amor se
tornava, no seu caso, uma questão de equilíbrio. Aquilo que deve ter lhe
parecido impossível no amor deve ter sido a mudança do eixo de rotação.
Ser um animal vertical e de repente passar para uma posição horizontal
--- eis o que deve ter torturado seus sonhos sensuais, se é que os teve.
Creio que o mistério do amor, para ela, se concentrava por completo
nessa queda, que a vida, em sua inteireza, se alicerçava nesse fato, o
que ultrapassava as suas forças.}

\textls[15]{Pediria perdão ao leitor por esses detalhes desavergonhados, mas, para
ser sincero, pouco me importa o leitor, e muito me importa Emilie
Vignon. Conto a vida dela em primeiro lugar porque eu quero chegar a
compreender alguma coisa da alma dessa moça, a quem no passado eu talvez
não tenha dado a devida atenção.\label{atenção}}\looseness=-1

\textls[15]{Digo, portanto, que só essa rigidez corpórea impedia que Emilie fosse
uma boa amante. Quem sabe que amor simples poderia ter aquecido seus
olhos cinzentos. Mas como amar alguém com um corpo daqueles, que parecia
talhado num bloco de pedra, inarticulado, inflexível? Penso nas coxas
compridas de Mado, penso nos revoluteios de seu corpo miúdo, à noite,
quando estremecia nos meus braços, e tento imaginar a mesma cena com
Emilie. Não, não. A imagem me parece grotesca. Se cada um de nós
nascesse de acordo com a nossa vocação e natureza, Emilie Vignon deveria
ser uma perna de mesa mal esculpida. Seria a única coisa que ela poderia
fazer bem e facilmente na vida.}\looseness=-1

Vai saber. Talvez tivesse ela uma determinada anatomia e, a seu modo,
uma graça que eu não consegui enxergar. Havia conexões secretas no seu
corpo. Quando erguia um ombro, tinha de inclinar um joelho. Como se cada
movimento rompesse um equilíbrio, que precisava ser restabelecido por
meio de um movimento complementar. Emilie não conseguia mover um dedo
independentemente dos outros quatro: articulando-se com o pulso, a palma
toda se erguia. Um colega residente, que a vira algumas vezes no pátio
do hospital, quando Emilie vinha me procurar, a mando de Mado, uma vez
me disse, em tom de brincadeira:

\textls[15]{--- Que curioso! Essa moça parece se mexer graças a uma série de
luxações.}

A observação foi exata. Após cada gesto executado por Emilie, eu sempre
esperava ouvir o estalo de um osso quebrado.

Espero que estes detalhes todos, contados desta maneira, não criem dela
uma imagem repugnante. Seria uma pena. Havia, no seu ser, algo
bem-comportado, comedido, um ar de objeto doméstico que, embora não
sirva para mais nada, não jogamos fora por termos nos acostumado com ele
e porque, de certo modo, o apreciamos. Gostava de Emilie assim como era
e, mesmo se jamais lhe disse isso, tenho a impressão de que ela entendeu
e que vai guardar certa gratidão por mim. Fora talvez uma das poucas
alegrias daquela existência que, aos vinte anos de idade, não esperava
mais nada, de direção alguma. Parecia fadada a passar inalterada, até a
morte, e com certeza assim teria sido se o acaso não a houvesse feito
encontrar Irimia \textsc{c}. Irimia.

\section{ii}

\letra{N}{ão} \textls[20]{posso dizer que fiquei feliz naquele 14 de julho, quando ele me
interpelou na ponte Saint-Michel. Estava folheando uma pilha de revistas
antigas num sebo às margens do rio, e esse é um dos poucos prazeres que
não divido com ninguém.}

--- Ei, Valerie!\ldots{} --- Embora nos conhecêssemos fazia tanto tempo,
nunca foi capaz de pronunciar corretamente o meu nome, Ştefan Valeriu
sendo para ele um nome estranho, ao passo que ``Valerie'' lhe soava mais
familiar e campestre.

\textls[15]{Irimia se implantou na minha frente e, pelo silêncio, compreendi que não
sairia sozinho dali.}

--- Como é que vai, Irimia?

--- Pois é, dando uma olhada.

\textls[-20]{De fato, estava dando uma olhada. Olhava para a água fluindo por debaixo
da ponte, sem mesmo piscar aqueles seus olhos grandes.}

Levei-o comigo para passearmos à beira-rio. Contou-me, com aquela sua
voz áspera, que concluíra em junho a licença em Direito em Bucareste, e
que pensava em cursar o doutorado em Paris, para o que obtivera uma
bolsa e chegara havia uma semana. Até o outono, com o início das aulas,
queria aprender francês.

--- Porque agora, sabe, não dá. Não dá, mesmo.

Falava com dificuldade, com fragmentos de frases inacabadas, cada
pensamento expresso até o fim era uma vitória. Lembrei-me do sofrimento
dele na escola, quando era obrigado a repetir a lição diante do
professor de história: parecia arrancar, com um martelo, cada palavra de
uma coluna de pedra que, aliás, na cabeça dele, deveria ser clara e
íntegra.

\textls[-10]{Pobre Irimia! Que ironia do acaso o levara, camponês de Ialomiţa, àquela
classe de garotos requintados do liceu Lazăr?}\looseness=-1\footnote{Nome do mais
  prestigioso liceu de Bucareste. {[}\textsc{n.\,t.}{]}} \textls[-10]{Que cálculo
equivocado o afastara de seu destino de lavrador nato e o levara para
amargar entre coisas que não compreendia? Fomos colegas desde o primeiro
ano de liceu, de modo que tive tempo de o conhecer: alto e corpulento
como era, de ombros imensos, pernas enormes, mal cabia na carteira. Vez
ou outra, após repetir a seu modo a lição aprendida com tanto esforço no
dia anterior, o professor, enfastiado, o mandava de volta para o seu
lugar: ``Irimia \textsc{c}. Irimia vá para o seu lugar''. Eu então imaginava que,
um dia, Irimia talvez se dirigisse quietinho para o cabide, pegasse com
calma o seu casaco do pino e declarasse, com aquela sua voz branda de
sempre, que aquele não era o seu lugar.}\looseness=-1

Mas não. Ele não fazia parte da raça dos revoltados. Voltava quieto para
a carteira, pousava as mãos no peito e permanecia em silêncio, olhando e
ouvindo. Sua estatura de gigante rígido era descabida naquele lugar
estreito. Tinha a impressão de que, na submissão de Irimia, havia uma
melancolia de animal domado, que atura e esquece, mas que mantém em
algum lugar, nas dobras mais profundas da alma, o prazer de outra vida,
a vocação de outro horizonte. Talvez me enganasse. Mas não podia
compreender de outra maneira o sorriso cândido daquele rapagão, sorriso
embaraçado, como se pedisse perdão por um erro permanente.\label{erro}

\textls[-20]{Guardo duas recordações precisas de Irimia, que não se relacionam entre
si, mas cujos detalhes acabaram ficando na minha mente.}

\textls[10]{Estava no pátio do liceu. No portão, um velho camponês de sobretudo,
carregando uma trouxa, olhava pela grade, sem se atrever a entrar.
Perguntei-lhe quem procurava.}

--- Bem, o filho da minha irmã.

--- Como é que ele se chama?

--- Bem, se chama Irimia.

Fui atrás do Irimia \textsc{c}. Irimia, embora devesse haver outros garotos no
liceu com o mesmo nome. Algo, porém, me dizia que o velho do portão
estava procurando o meu colega. Talvez por causa dos mesmos olhos azuis,
um pouco assustados. De fato, era ele. Aproximou-se do portão, sem
pressa, sem surpresa, tirou o boné da cabeça (gesto atávico com que seus
ancestrais devem ter tirado o gorro séculos a fio) e se inclinou para
beijar a mão do velho, uma mão escura, ossuda e descarnada. Não dei
risada. Na mesura do gigante, arqueado por sobre o velhote à sua frente,
havia algo de estremecedor. Eu, que vivi num mundo de tradições falsas e
leis falsas, tive a sensação de uma espécie de eternidade que o meu
colega Irimia \textsc{c}. Irimia materializava ali, na rua, na minha frente,
beijando a mão do parente velho.\label{velho}

\textls[10]{A segunda recordação que guardo do Irimia é completamente
insignificante, e até me pergunto se preciso contá-la. Foi também no
liceu, durante uma aula de literatura francesa. O professor lhe ordenou
que lesse em voz alta uma passagem de Racine. É curioso como --- embora
se trate de uma ninharia --- até hoje não esqueci o fragmento exato a
ser lido. Era a quarta cena do primeiro ato de \emph{Andrômaca}.}

\begin{verse}
\emph{Songez-y bien; il faut désormais que mon coeur}\\
\emph{S'il n'aime avec transport, haïsse avec fureur.}\footnote{\textls[15]{Conforme tradução de Jenny Klabin Segall (São Paulo: Martins, 1963): ''Para o meu coração a alternativa é uma; Se não arder de amor, que de ódio se consuma.'' {[}\textsc{n.\,t.}{]}}}
\end{verse}

\textls[15]{Difícil dizer em que se transformavam aqueles versos nos lábios de
Irimia. Uma espécie de búlgaro esmagado entre os dentes, um dialeto sem
vogais, triturado, moído, socado entre dois pedaços de sílex. A classe
toda se divertiu, eu mesmo também caçoei dele. Sossegado, ostentando a
testa estreita, as faces ásperas, um pouco proeminentes, os maxilares
firmes de carnívoro, com as mãos imensas aferradas à lombada do livro,
Irimia \textsc{c}. Irimia continuou lendo Jean Racine.}

Um colega de carteira, com quem pessoalmente antipatizo, embora nesse
meio-tempo tenha se tornado famoso escrevendo folhetins semanais num
grande jornal reacionário, garoto capaz e intelectual (reconheço tudo
isso para que não achem que eu sinta por ele qualquer tipo de inveja, eu
que não atingi a mesma fama e nem sou literato), um colega de carteira,
portanto, naquela altura, sussurrou-me ao ouvido, olhando para o Irimia:

--- É um primitivo.

Não. Irimia era apenas um camponês do Bărăgan.\footnote{Extensa
  planície, muito fértil, entre Bucareste e o Mar Negro.
  {[}\textsc{n.\,t.}{]}} Ali, junto a nós, recitando versos franceses,
parecia ridículo. Mas imaginei-o num entardecer de julho, às sete horas,
após uma longa jornada de trabalho, retornando ao vilarejo, descalço, ao
lado do trigal, à luz do crepúsculo, e pensei comigo mesmo que nenhum de
nós, absolutamente nenhum, nunca teve, por um único instante de nossa
vida de garotos inteligentes, pelo menos uma migalha daquela grandeza
simples do Irimia.

\textls[15]{Sou insensível ao que se denomina ``clamor da terra''}\looseness=-1\footnote{Provável
  menção à corrente ideológica e literária romena do Sămănătorism, em
  voga na primeira década do século 20, que se manifestou por um
  especial interesse pela realidade camponesa. {[}\textsc{n.\,t.}{]}} \textls[15]{e
zombo desse tipo de literatura. Mas gosto de ver um animal robusto que
viceja no seu próprio ambiente. E às vezes me acontece de sofrer quando
vejo, no circo, um cachorro enfeitado com laços e guizos, erguido em
duas patas, sabendo que o seu destino seria o de enfrentar lobos no alto
de uma montanha, diante de estrelas brancas e de Deus.}\looseness=-1

Por isso, acho, comportei-me simpaticamente com Irimia e, se algumas
vezes acabei dando risada dele, foi por preguiça ou covardia: era
difícil não fazer como todos os outros. De modo que nutri por ele uma
amizade franca e sincera.

\section{iii}

\letra{N}{o} \textls[-20]{entanto, naquele 14 de julho, ele me entediava. Anoitecia belamente,
os barcos brancos que desciam da direção de Vincennes agitavam suas
bandeirinhas ao longo de todo o Sena, Notre-Dame se tornava azul à minha
esquerda, esfumaçando-se no crepúsculo. Ele concluíra a história, eu não
tinha mais perguntas para fazer, ambos estávamos calados e ele
continuava caminhando, detendo-se quando eu me detinha, retomando o
passo quando eu retomava. Queria me perder sozinho naquele universo de
festa, correr ao meu bel-prazer, parar quando quisesse. Impossível. As
galochas do Irimia ressoavam do meu lado.}

\textls[-10]{Comecei a ficar preocupado. Às dez da noite, eu tinha encontro marcado
com Mado no Quartier Latin, e não via modo de me desvencilhar do Irimia.
Digo-o não para me justificar, mas para esclarecer com exatidão o papel
involuntário que desempenhei na desgraça que se seguiria: fiz todo o
possível para o afastar. Gosto de piadas, mas mentem os meus amigos que
alegaram ser uma farsa por mim montada e premeditada a união entre
Irimia \textsc{c}. Irimia e Emilie Vignon. Talvez eu tenha culpa em outras
coisas, que vou admitir no momento adequado. No entanto, com relação ao
resto, tenho a consciência tranquila: não fui eu que levei o Irimia ao
café d'Harcourt, ele é que não me largava e acabou me acompanhando até
ali, contra a minha vontade. Aliás, nem sabia que Emilie devia vir
também aquela noite. Encontrei-a num canto da mesa e mal falei com ela,
pois Mado logo se atirou ao meu pescoço, beijando-me com todo o ardor de
uma amante. Fazia alguns dias que não nos víamos e o seu amor suportava
férias muito mal. Ademais, era 14 de julho}\looseness=-1\footnote{Festa nacional da
  França. {[}\textsc{n.\,t.}{]}} \textls[-10]{e Mado era republicana convicta.
Pendurou-se, portanto, no meu braço e me arrastou para a rua, onde as
pessoas dançavam, para festejarmos juntos a Queda da Bastilha.
Festejamos. Éramos vários colegas, cada um com a respectiva namorada, e
fizemos uma farra maluca, dançando no meio da rua, beijando-nos e
atirando serpentinas. Claro que, vez ou outra, retornava até a nossa
mesa, na calçada, para beber ou fumar, mas, envolvido que estava na
festa, não percebi nada.}\looseness=-1

Só mais tarde alguém me chamou a atenção.

--- \emph{Oh! Regarde les amoureux!}\footnote{Em tradução livre, ``Oh! Veja os apaixonados!''. \textsc{{[}n.\,e.{]}}}

Embora não tenha ganas de fazer piada, ainda mais agora que sei que
triste fim teve toda aquela história, escapa-me o riso sempre que recordo
a imagem do casal na calçada, entre luzes e serpentinas. Emilie e
Irimia! Estavam sentados um ao lado do outro, rígidos, sérios, um pouco
confusos, um pouco ausentes, às vezes se fitando nos olhos, longa e
fixamente. Permaneciam calados. Não nutriam decerto nenhum sentimento de
ternura, mas o fato de, no meio de toda aquela gente, só os dois ---
imóveis, calados --- permanecerem à mesa os unia, ao menos a nosso ver.

\textls[10]{Confesso que, instintivamente, tenho um certo gosto pela crueldade.
Talvez não chegue à tortura, mas, quando participo de uma festa, gosto
de encontrar um brinquedo, um alvo acessório para certas maldades. É
vulgar, eu sei. Mas é assim. Naquela noite, nem precisava inventar o
alvo: ele mesmo se oferecia. Aliás, não fiz nada além de assistir,
divertindo-me, a uma brincadeira que os outros --- em especial Mado ---
teriam feito também na minha ausência: consideravam Emilie uma amante e
Irimia, um namorado. Passavam a mão nos dois, lançavam alusões
desavergonhadas, elogiavam a beleza de Emilie, admiravam a força de
Irimia. Mas eis que, embora um pouco perplexos, eles se mantinham
sérios, o que aumentava o ridículo da situação, pois realmente assumiam
um certo ar de noivos. Ademais, ele não compreendia nada do que falavam,
girando a cabeça com um olhar suplicante, constrangido, cuja lembrança
me faz mal até hoje. As coisas teriam parado por aí e eu teria esquecido
tudo rápido no turbilhão dos festejos se, de repente, depois da
meia-noite, não houvesse caído a chuva. Foi uma verdadeira tempestade de
verão. Rápida e brusca. Num só instante, a pracinha da Sorbonne ficou
deserta, com mesas caídas e copos partidos na pressa. Fragmentos de
serpentina esvoaçavam molhados, como folhas de outono. Fugimos cada um
para um lado e por pouco não tive tempo de pegar Mado e levá-la comigo
correndo para casa, que nem ficava longe. Esqueci-me de Irimia e Emilie.
Quem recordar a chuva daquele 14 de julho, não pode me culpar por tal
esquecimento.}\looseness=-1

No entanto, Irimia me contou, mais tarde, em detalhe, tudo o que
aconteceu após a minha partida e, dado que eu o conhecia tão bem quanto a ela, posso imaginar com exatidão o ocorrido. Eles ficaram
sozinhos à mesa, debaixo da chuva, e não sabiam como se separar. Não se
conheciam, jamais haviam se falado, não conseguiam se falar e, apesar de
tudo, sentiam dificuldade em ir cada um cuidar de sua própria vida. Tal
decisão superava a força e imaginação de ambos. Estavam juntos?
Tinham então de permanecer juntos.

Puseram-se então em movimento, debaixo da chuva, sem trocar uma única
palavra. O vestido dela estava todo encharcado, filetes de água
escorriam da aba do seu chapéu. Ele tirou o casaco e cobriu a moça,
apertando-a ao seu lado, como a um tronco. Era fácil conduzi-la: ela
lhe chegava até a cintura. Assim caminharam horas a fio. Até Saint-Ouen,
onde Emilie morava, deveriam ser uns dez quilômetros. Percorreram tudo
aquilo a pé, de madrugada, arrastando os pés pelas ruas úmidas e
desertas. Chegaram tarde ao portão da casa de Emilie, ao raiar do dia.

Como se deu o resto, não sei. Como Irimia subiu até a mansarda dela,
como se arremessaram um aos braços do outro, como se esborracharam
vestidos no assoalho --- não sei.

Talvez fosse o sono atrasado que os invadia, da mesma maneira como se
deixam entorpecer os animais de tração exaustos. Talvez a vertigem
daquela noitada musical, com tochas e fogos de artifício --- 14 de julho
republicano, que pulsava em seus corações com um esplendor tardio ---,
talvez a chuva que acariciara suas faces, a Marselhesa que ainda
ressoava em seus ouvidos, modulada como uma canção de amor. Ou talvez
fosse, sobretudo, a necessidade de expressar alguma coisa. Sentiam
vagamente a simpatia elementar que um pangaré tem que sentir por outro
quando puxam juntos a mesma carga e, como não falavam a mesma língua,
encontraram, num momento de intuição, a maneira mais simples de se
expressarem.

\textls[-10]{Mas por que me perder em suposições? A verdade é que, no dia seguinte,
ao entardecer, Irimia veio bater à minha porta. Ostentava uma expressão
séria e protocolar. Fazia rodeios, querendo me dizer alguma coisa, mas
não sabia como começar. Girou o chapéu com as mãos algumas vezes. Deu
umas tossidas. No final, falou de uma vez, sem introdução, assim como os
tímidos que, depois de passearem por uma hora com a xícara de chá na
mão, sem saber onde pousá-la, deixam-na cair no chão, resolvendo tudo.}\looseness=-1

--- Olha, ontem à noite eu me deitei com a Emilie.

Com certeza arregalei os olhos, pois logo em seguida ele baixou o olhar,
encabulado. Sabia que Irimia nunca mentia, mas o fato me pareceu tão
abominável que hesitei em acreditar. Quis dar risada, e até tentei.

\textls[20]{--- Malandro! --- disse-lhe, ameaçando-o com o dedo, admirativo.}\looseness=-1

\textls[15]{Ele mal conseguiu sorrir e, em seguida, retomou o ar protocolar de
antes. Suspirou. Depois, do fundo do coração, com um toque de
arrependimento, do qual eu não seria capaz nem se matasse alguém, ele
confessou:}

--- E ela era virgem.

\textls[-15]{Justamente aí começa a minha culpa, minha pequena culpa. Pois eu sabia
que Emilie Vignon era virgem. Sabia também que, em seu universo, tal
fato não tinha significado algum, e que o amor era feito sem
responsabilidade. No entanto, distraído pela feição desgraçada de
Irimia, divertindo-me com a sua imensa ingenuidade, continuei a
brincadeira. Quem sabe? Talvez, se naquela altura eu tivesse abordado as
coisas com leveza e dito a ele que o ocorrido não era grave, a história
teria tido outro desdobramento. Mas eu, pelo contrário, assumi uma
expressão circunspecta, caminhei pelo aposento e fitei Irimia de cima
para baixo, como a um réu.}\looseness=-1

\textls[15]{Ele nem se atreveu a enfrentar a minha reprovação. Disse-me, com
simplicidade:}

--- Não faz mal. Eu vou levá-la.

--- Como assim, levá-la?

--- Sim, vou levá-la para o altar.

\textls[15]{Embora soubesse que ele não estava brincando, achei que, no final das
contas, as coisas não chegariam tão longe. Deixei-o partir, em seguida
dei muita risada e esqueci.}

\textls[10]{Só uma semana depois, um colega de faculdade me contou, ao acaso, que
Irimia ia se casar. Fiquei perplexo. Saí correndo atrás dele, esperando
chegar a tempo para salvar o rapaz daquela enrascada. Encontrei-o
sossegado, com aquele sossego de quem travou as pazes com a própria
consciência. Tentei tirá-lo daquela inconsciência, fazê-lo raciocinar,
convencê-lo.}

\textls[15]{--- Está certo, seu desgraçado, preste atenção no que faz. Você é pobre,
vai ter que trabalhar, vai ter que voltar para casa, seus pais estão te
esperando.}

Ele deu de ombros.

--- Mas se ela era virgem\ldots{}

Era esse o seu raciocínio; vi muito bem que não conseguiria demovê-lo
daquela sua honra bitolada de lavrador. Então fui tentar falar com
Emilie. Não que o destino de Irimia me interessasse tanto: mas aquele
casamento me parecia monstruoso. Do ponto de vista humano, o
acasalamento do gigante com a aleijada, feita de madeira, era horrendo,
bestial. Que vida poderiam levar aqueles dois, lá em cima, naquela
mansarda de Saint-Ouen, comunicando-se por sinais, pois nem a mesma
língua falavam, sem saber nada um do outro, grunhindo quando quisessem
dizer alguma coisa e se enroscando à noite, calados, como cachorros?

Não logrei convencer ninguém. Emilie, pobre alma que se deixava levar
por qualquer um, recebera sem surpresa a proposta de Irimia. Não
entendia muito bem por que ele fazia questão de se casar, mas nem tinha
por que o rejeitar. As amigas dela, ademais, vendo naquele casamento uma
oportunidade perfeita de piada, se apressaram a emaranhar as coisas de
tal modo que ninguém mais pudesse desemaranhar. Encontrei-me, portanto,
diante do fato consumado, e acabei desistindo de qualquer contestação.

Exatos quinze dias depois, participei do casamento civil entre Emilie
Vignon e Irimia \textsc{c}. Irimia. Havia alguns colegas do noivo no salão da
subprefeitura do décimo quarto distrito, e toda uma legião de vendedoras
e costureiras: colegas do subsolo da loja em que Emilie trabalhava.
Cheguei ali como se para uma feliz coincidência. Pelo contrário, foi um
espetáculo melancólico. O caráter ridículo da cena era comovente, e não
me lembro de ter visto alguém sorrindo. As moças lacrimejavam.

\textls[15]{Só Emilie, de braço dado com o genro, lançava um brilho modesto de
rainha, e sua feiura de sempre irradiava solene como uma auréola de
castidade.}

\section{iv}

\letra{P}{or} \textls[10]{muito tempo não soube mais nada da vida deles. Parti de Paris em
agosto, para uma cidadezinha no Sul, onde tive que substituir um médico.
Voltei de lá bem mais tarde, em novembro, com dez mil francos no bolso e
com a esposa do médico substituído, uma mulher feia e afetada. (Mas essa
é uma outra história\ldots) Claro, havia me separado de Mado, cuja
relação de qualquer modo durara demais. Por conseguinte, não tinha mais
de quem receber notícias do casal Irimia.}

Avistei-os, no entanto, num dia de domingo, no Jardin des Plantes,
olhando os animais. Estavam de mãos dadas, assim como saem para passear
os soldados com as empregadas no Cişmigiu,\footnote{Nome do primeiro
  parque público de Bucareste. {[}\textsc{n.\,t.}{]}} o que emprestava
àquele parque parisiense uma triste lembrança dos nossos subúrbios. Os
dois se postaram em meio a um grupo de crianças, em frente ao pavilhão
dos elefantes. Irimia tirou do bolso do casaco um pão enrolado em papel,
desembrulhou-o e se pôs a distribuí-lo entre os paquidermes.

\textls[15]{Chamava-os com carinho, em romeno:}

--- Vem pro tiozinho, vem, vem pro tiozinho!

\textls[20]{Sempre que a tromba de um elefante ultrapassava a grade, dependurando-se
no ar e descendo na direção da mão de Irimia, Emilie pulava assustada e
o puxava para trás, mas ele permanecia tranquilo. Entendia-se bem com os
paquidermes, e eles o reconheciam. Não me aproximei deles e os evitei,
para não perturbar o idílio.}

Revi-os, porém, três meses depois, em circunstâncias trágicas. Estou
acostumado com a morte e mais de uma vez fechei os olhos dos mortos nos
salões brancos de hospital, nos quais passei a juventude pensando em
outras coisas para além dos rostos que se decompunham do meu lado. O que
é que vocês querem? É uma insensibilidade profissional, diante da qual o
calafrio do fim deixa de existir. A morte de Emilie, no entanto, me
estremeceu. Foi horrenda.

\textls[15]{Num dia de março, Irimia me procurou no hospital e me pediu para
encontrar um leito para Emilie na minha seção, pois estava grávida e
haveria de dar logo à luz.}

--- Muito bem, Irimia, olhe só o que você foi arranjar! Rebentos? Por
que não fez mais cedo o que era para ser feito?

\textls[20]{Irimia pareceu não compreender. Olhou-me confuso e, ao entender
finalmente que eu lhe falava de aborto, fez por instinto o sinal da
cruz.}

Eu não fazia clínica obstétrica no Trousseau. Mas disse ao Irimia que
interviria junto à direção e tentaria encontrar o leito necessário. Em
dois dias, estava tudo arranjado. O residente da seção de maternidade
era meu amigo e prometeu cuidar com especial atenção de Emilie. Eu
mesmo, quando não estava de plantão ou quando meus doentes me davam
trégua, cruzava o corredor e entrava no salão \textsc{xviii}, para
auxiliá-lo.

\textls[10]{Quando levei Emilie, desde o primeiro momento tive certeza de que ela
não haveria de sair mais dali. Jamais vira uma tal gravidez. Não era
propriamente anormal: os sintomas eram comuns e a paciente, firme. Mas
todo o seu corpo estava deformado, o ventre enorme, os membros pesados e
afastados do corpo. Respirava com dificuldade e, por vezes, virava os
olhos, como os gansos de engorda. Imaginar que aquele corpo nodoso, que
rangia como um guincho por lubrificar, aquele corpo torto, troncudo, com
as articulações fora do lugar, com reflexos selvagens, que aquele corpo
pudesse levar dentro de si uma criança! Era uma aberração, uma
impossibilidade física. Emilie, para quem pegar um copo de um lugar e
pôr em outro era um problema de acrobacia, agora tinha de dar à luz uma
criança! Seu corpo rígido precisaria se tornar elástico, se moldar
conforme os movimentos internos do feto, acompanhando suas inflexões de
verme cego!}\looseness=-1

\textls[10]{Foi um horror. A mulher parecia um tronco de árvore tombado. Se pudesse
se debater, talvez sofresse menos, talvez até escapasse com vida. Mas
não: mantinha-se aferrada ao lençol e de lá nos fitava com um par de
olhos aflitos, suplicantes, de cachorro que se afoga. Por vezes gritava,
e seu berro ecoava longe, pelos imensos salões do hospital, assim como
se deve ouvir, no abatedouro, o mugido das vacas sacrificadas. Pensamos
em preparar o fórceps, mas o patrão, que conduzi até a cabeceira da
paciente, não permitiu. Dizia que, de qualquer modo, era um caso
perdido.}\looseness=-1

\textls[-15]{Tudo aquilo durou três dias e três noites. Irimia, apesar das minhas
tentativas de retirá-lo dali, ficou ao lado do leito da esposa o tempo
todo, defendendo-se com teimosia. Não conhecia essa sua força de
vontade. Permaneceu imóvel, sem dizer uma palavra, sem soltar um
suspiro, olhando ora para mim, ora para Emilie, à espera.}

\textls[15]{Na terceira noite, lá pelas três da madrugada, a mulher deu à luz uma
menina. Irimia a recebeu das minhas mãos. Levou-a para perto de uma
lâmpada, fitou-a longamente, em seguida a devolveu à enfermeira e foi
dormir.}

No dia seguinte, ao retornar ao hospital, encontrei-o do lado da cama de
Emilie. A mulher agonizava. No início da manhã, sofrera uma forte
hemorragia e, agora, fora diagnosticada com septicemia geral. Arquejava.
Ao me avistar, Irimia levou um dedo aos lábios, fazendo sinal para que
eu pisasse mais leve.

\textls[20]{--- A partir de agora --- disse-me --- ela vai melhorar. Conseguiu
escapar.}\looseness=-1

Não tive coragem de replicar, e ele, sem compreender meu semblante
sombrio, acrescentou:

--- Está vendo como ela mexe a boca? Isso não é nada: são os nervos que
se restabelecem. --- Em seguida, autoconfiante e exalando um profundo
orgulho paternal: --- Mas e a criança?\ldots{} A menina?\ldots{} Você
viu que bonita? Vem ver.

E me arrastou atrás dele, até um salão adjacente.

\textls[20]{Emilie morreu no dia seguinte, ao entardecer. Não sei quem fechou seus
olhos.}

\textls[10]{Nós a enterramos numa manhã de fim de março, límpida e ensolarada. Fazia
calor e saímos sem casaco, sorridentes naquela luz clara de primavera.
No meio do caminho, as vendedoras de lírio-do-vale expunham pequenos
buquês de um franco cada. Nós os compramos todos, em memória de Emilie.
Irimia envergava seu sobretudo preto para ocasiões formais, o mesmo que
usara, nove meses antes, no casamento.}

Num canto do cemitério, Mado, com o rosto contraído e uma expressão
carregada, de um jeito que jamais a vi, chorava e soluçava como uma
criança e, de longe, ao me avistar, sorriu entre as lágrimas. Era uma
moça boa e sentimental.

\chapter{Maria}

\section{i}

\letra{S}{ua} \textls[10]{revelação repentina de ontem à noite me surpreendeu e também me
aborreceu um pouco. Creia-me, eu não esperava. Tinha certeza de que,
entre nós dois, as coisas estavam bem claras e inequívocas e, às vezes,
quando por acaso eu me apoiava no seu ombro (embora esse gesto sempre
enfureça o Andrei), eu o fazia por um prazer amical, quase sem perceber,
mais um entre tantos outros gestos familiares.}

\textls[15]{Por que você terminou do mesmo jeito que todos os outros? Deixe-me
repreendê-lo. Você merece e, no final das contas, está vendo, isso me
faz bem.}

\textls[10]{Não pense você, em primeiro lugar, que no meu silêncio e na minha
partida apressada do baile tenha havido algo da indignação de uma mulher
ultrajada. Já sou velha, mas você não quer acreditar, já ouvi tantas
vezes, em outras circunstâncias mais ou menos semelhantes, as mesmas
palavras que você me disse ontem à noite, que o ocorrido não me parece
mais digno de nota, de modo que consigo passar por cima dele com certa
frivolidade. Sim, com certa frivolidade.}

\textls[-10]{Portanto, não entenda mal a minha crispação de ontem. Não o censuro em
nada. Considero apenas que a nossa amizade poderia ter se abstido de tal
acidente e que e o fato de você me amar, ou de querer me amar, ou de achar que me ama, complica em excesso uma relação que eu apreciava e que me
pareceu possível durante muito tempo. Veja, você estragou de tal maneira
a disposição das coisas, que ora me pergunto se é prudente lhe dizer que
eu gostava de você e que eu esperava sempre com alegria as suas visitas,
como uma festa íntima. Você é insuportável --- é isso o que você é.}

Ontem, assim que você terminou de falar, senti de repente como algo
desaba e se embaraça, e isso me enervou tanto que não fui capaz de
permanecer ali, naquele salão de baile, de modo que pedi ao Andrei que
me levasse para casa, embora soubesse que o contrariava ao pedir aquilo, pois ele estava justamente travando uma conversa animada com
Suzy Ioaniu e, com certeza, tencionava passar a noite toda dançando com
ela.

(Só depois me dei conta de que o Andrei poderia ter se utilizado daquela
minha partida súbita como uma pequena cena de ciúmes e que isso poderia
indiretamente lisonjear a Suzy, o que com certeza me aborreceu, mas era
tarde demais para consertar e, no fundo, aquilo não tinha grande
importância. Para ser sincera, talvez eu tenha até gostado de separá-los
à força).

\textls[15]{Mas agora vamos conversar de maneira normal, como duas pessoas
bem-comportadas. O concerto do Brailowsky}\looseness=-1\footnote{Alexander Brailowsky
  (1896--1976), célebre pianista francês. {[}\textsc{n.\,t.}{]}} \textls[15]{no
  Ateneu}\looseness=-1\footnote{Ateneul Român (Ateneu Romeno), principal sala de
  concertos de Bucareste, inaugurada em 1888. {[}\textsc{n.\,t.}{]}} \textls[15]{será
  segunda-feira à noite e, como nenhum de nós dois deixará de ir (pelo
  menos espero que você não cometa essa besteira), vamos nos encontrar lá.
  Pois bem, não quero um aperto de mão medroso, não quero que fiquemos nos
  espiando com olhares desconfiados, que conversemos constrangidos sobre a
  chuva que cai lá fora, sabendo que um segredo persiste entre nós. Pelo
  contrário. Quero, ao término do concerto, poder pedir a você que me
  acompanhe até em casa, como de costume. Acho que estarei sozinha, pois
  você sabe que o Andrei não gosta de música e, aliás, segunda-feira à
  noite ele deverá estar na casa da Suzy, onde vão ensaiar um espetáculo
  para o príncipe Mircea, no qual o Andrei tem um número em que interpreta
  o rei do tango.}\looseness=-1

Talvez seja difícil esclarecer as coisas entre nós e me pergunto se
terei coragem de escrever esta carta até o fim, mas eu não posso, juro
que não posso acrescentar à minha vida, já complicada o bastante, mais
um segredo, mais uma situação obscura. No meu pequeno universo íntimo,
você é a única pessoa com quem posso falar abertamente, e não vou perder
essa ocasião. Às vezes sinto, eu, que me acostumei às mentirinhas dos
outros e às minhas próprias, às vezes sinto uma tal saudade de
sinceridade que me vêm lágrimas aos olhos. Sempre que os arranjos miúdos
com que construí minha vida me pressionam, sinto que desejo me vingar
--- me vingar por me vingar, de maneira inútil, boba --- declarando, de
uma vez por todas, a verdade verdadeira, até os mínimos detalhes, sem
reserva, sem me importar com as consequências. Quantas vezes não pensei
em lhe dizer, quando você vinha tomar chá em casa, ao entardecer. Mas
você é um homem lógico, e me responde sempre com argumentos
desencorajadores. Sabe que, quando eu lhe estendia ao acaso a cigarreira
ou o prato de biscoitos, embora você já tivesse um cigarro na mão ou já
houvesse se servido de doces, não raro eu fazia esse gesto inútil só
para mudar de assunto e deter a tempo uma confissão tentadora.

\textls[15]{Mas hoje vou lhe dizer sem rodeios.}

Você sabe que eu amo o Andrei? Não diga que escolhi uma hora ruim: você
não precisa ser poupado, nem eu sou capaz de acrobacias. A única coisa
que interessa é sermos claros.

Descobri que as pessoas falam da minha relação com ele como um acaso
qualquer: uma ``colagem'' que já dura cinco anos e que deve terminar de
um dia para o outro. Talvez por isso as mulheres se permitam flertar com
Andrei na minha frente, e os homens, me abordar em particular com certa
liberdade. Talvez por isso você costume utilizar palavras duras ao se
referir a ele, ou lhe dirigir sorrisinhos sardônicos na minha frente,
que ele bem pode merecer, mas que me fazem mal.

\textls[15]{Pois, no fundo, você é ingênuo. Você acredita na inteligência, no bom
gosto, na discrição, na sutileza, e não entende que, por cima de tudo
isso, alguém é capaz de amar o Andrei, sim, o Andrei, amigo seu e amante
meu.}

\textls[10]{Já surpreendi mais de uma vez o seu sorriso de incredulidade quando me
aproximava dele para admirá-lo ou lhe contar um segredo. Você talvez
imagine que aquele homem bonito só sabe se divertir tocando \textit{jazz}, além
de perder duas horas pela manhã arrumando o penteado, e que pregou na
parede o retrato do Rodolfo Valentino, recortado de uma revista de
cinema. Embora eu não tenha vaidade pessoal, naqueles momentos tinha
ganas de me aproximar e lhe dizer que você é insípido e medíocre, e que
seu cotovelo dói por ele ser muito mais bonito que você. Sua
superioridade me irrita. Sempre que você me diz algo inteligente, tenho
a impressão de que é uma reprimenda. Quero lhe dizer que eu sei. Que eu
conheço o Andrei, assim como você também o conhece, mas que isso não
muda nada.}\looseness=-1

Uma vez eu lhe contei que, num concurso eliminatório de tênis, Andrei
tinha ganhado o segundo lugar. Você apagou o cigarro, me olhou e me
disse ao acaso, indolente:

\textls[15]{--- ``Nem ao menos o primeiro''.}

Que ódio senti de você aquele dia.

Pois, está vendo, você me compelia a julgar um sentimento que eu recebia
com simplicidade, sem cálculo. Você era como um biólogo que faria
questão de demonstrar que, conforme todas as leis da ciência, eu não
deveria gostar da flor que adoro.

\textls[15]{Você foi a primeira pessoa que me obrigou a me perguntar por que amo o
Andrei; pergunta, aliás, ridícula, que não tem como levar, nunca, a
lugar algum.}

\textls[15]{Por que o amo? Meu Deus! Porque sim.}

\textls[-15]{Conheci-o um ano depois do divórcio. Não tinha disposição alguma para
uma grande paixão. Gostava de me vestir bem, inventando sozinha meus
próprios vestidos de passeio para usar em dias de sol. Estava me
preparando para sair e, visto que hesitava entre o mar e a montanha,
adiava minha partida de uma semana para a outra, embora Bucareste já
começasse a me entediar.}

\textls[20]{Então apareceu o Andrei. Ele me cortejou meio malandro, com uma certa
impertinência de galã e achei aquilo engraçado, assim como o príncipe de
Gales deve achar engraçado os transeuntes que, sem saber quem é,
perguntam-lhe na rua que horas são ou lhe pedem um favor. Dei risada e,
para continuar a brincadeira, incentivei-o. Pois, de qualquer modo, eu
não tinha como dizer àquele homem simpático, arrojado e seguro de si que
ele se enganava.}

\textls[15]{``O senhor sabe com quem está falando?'' seria uma resposta
profissional. Quanto a mim, gosto de passar por quem as pessoas acham
que sou.}

\textls[20]{Se eu soubesse, na altura daqueles primeiros dias de verão, aonde essa
brincadeira daria, talvez tivesse parado na hora. Cerro os olhos e, sem
me arrepender de nada, imagino de que outra maneira teria sido minha
vida. Que sossego bom, que cansaço gostoso!}

\textls[15]{Andrei entrou na minha existência num momento de desatenção, quando
deixei portas abertas e persianas erguidas. Você o conhece: toda
precaução com ele é pouca. Não é difícil, não é mau, não é bom, não é
simples, não é complicado. Anoitece, escurece no quarto, giro cansada o
interruptor e, naquela luz repentina, surge Andrei, num canto.}

\textit{--- Você estava aí?}

\textls[-15]{Sim. Estava. E já que estava, e já que eu estava cansada, e porque já
era tarde, peço-lhe um favor, chame a empregada, passe-me o livro da
estante, diga-me que gosta do vestido verde que usei ontem, puxe a
mesinha de chá para mais perto, cante aquela canção de amor que ouvi
ontem à noite no Modern e que esqueci.}

\textls[10]{O verão corria imperceptível, bem assim, com prazeres miúdos. Certo dia,
olhei o calendário e vi que já passava de 15 de agosto. Tarde demais
para sair de Bucareste. As noites eram quentes, abafadas, com não sei
que traço de aflição que me fatigava. Andrei me encontrou junto à
janela, com a testa grudada à vidraça, como no passado, na adolescência,
quando ficava em casa sozinha, não esperava ninguém e espiava da sacada,
com o olhar para além do bairro, para além da cidade, algo que deveria
chegar, não sei bem o quê.}\looseness=-1

\textls[15]{Disse-lhe que não iria mais embora. Tornou-se loquaz, entusiasmado,
beijou-me as mãos, ajoelhou-se, ergueu-se numa pirueta --- tudo isso com
um sotaque desculpável, pois ainda estávamos brincando.}

--- \emph{Madame, ma voiture vous attend}.\footnote{Em tradução livre, ``Senhora, meu carro a está esperando''. \textsc{{[}n.\,e.{]}}}

\textls[-15]{Apontou para a porta, sorridente, juvenil, cavalheiro --- e seu convite
em francês me fez lembrar algo que eu ouvira várias vezes relacionado a
ele, dito com indulgência e faceirice: ``\emph{Ce cher André}''.}\looseness=-1\footnote{Em tradução livre, ``Este querido André''. \textsc{{[}n.\,e.{]}}}

\textls[15]{Dei risada, dei-lhe o braço --- num gesto de camaradagem --- e descemos.
Naquela época, ele tinha um pequeno Chenard Torpedo, o qual, se bem me
recordo, ficou destruído dois anos atrás na Itália, no acidente que
sofremos nos Alpes. Ele se sentou ao volante e me colocou ao seu lado
com um sorrisinho de superioridade que, naquela época, ficava muito bem
nele e de certo modo me intimidava.}

Estou escrevendo rápido, sem revisar. Contando-lhe tudo isso, tenho a
dimensão exata de toda a obscenidade. Sim, com a cumplicidade de um
automóvel, de uma noite de agosto e de um sorriso, tornei-me amante do
Andrei. Eu, que guardo do amor um gosto amargo de tempo perdido, talvez
não tivesse o direito de me arrepender desse ``acesso à paixão'' por um
portãozinho lateral. Imagino, às vezes, porém, um amor solene, que
começasse puro, imaculado, a partir de um casamento místico entre mim e
um homem --- o que talvez não passe de um vestígio da minha educação de
boa burguesa, coisa que talvez o faça rir, mas talvez possa ser algo
mais difícil de explicar.

\textls[15]{Não estava tonta naquela noite e creio ter mantido o tempo todo uma
certa reticência, pelo que aliás senti ódio de mim, pois queria viver,
livre de qualquer bom gosto, aqueles momentos triviais e prazerosos;
mas, ao mesmo tempo, deixando a cabeça se aninhar no ombro de Andrei, ao
sopro do vento, ao mesmo tempo, eu repetia para mim mesma que tudo
aquilo não passava de um simpático \textit{vaudeville}.}

\textls[5]{Depois de Otopeni, as luzes rarearam. Andrei tinha, do lado esquerdo do
para-brisa, um pequeno farol portátil. Ele me passou o farol para que eu
iluminasse o caminho nas encruzilhadas. Era algo absorvente,
apaixonante. Tínhamos diante de nós apenas a faixa branca dos faróis da
frente, em seguida uma listra preta, opaca, e mais longe, bem longe, a
mancha incandescente do pequeno farol que eu manipulava. Com a leve
pressão de dois dedos, eu deslocava aquela luz instável e acompanhava o
seu alcance sobre um marco miliário, sobre um galho recurvado, sobre um
pilar de ponte. Olhava para frente, sem virar a cabeça para nenhum lado,
perscrutando a escuridão e os desvios inesperados da estrada. Era uma
vigilância intensa. Tinha apenas a sensação imediata e aguda da minha
tensão: para além de mim, havia apenas o estrépito nervoso do motor, a
mão dominadora do homem ao meu lado, segurando firme o volante, as
árvores que passavam negras, farfalhantes, à direita e à esquerda, os
mostradores prateados ao lado do freio e talvez ainda --- pressentida ao
longe, lendária --- a noite que nos circundava.}

E, agora, sentada à minha mesinha, curvada sobre esta folha de papel,
pensando naquela viagem, ainda sinto por debaixo das têmporas as lufadas
de vento.

\textls[15]{Passado algum tempo, não sei mais quanto, não sei onde, Andrei freou.
Viam-se próximas, entre as árvores, algumas luzes de casas esparsas.}

--- \emph{Madame, la nuit vous attend}.\footnote{Em tradução livre, ``Senhora, a noite a espera''. \textsc{{[}n.\,e.{]}}}

\textls[15]{Que frase bonita. Não dê risada. É uma frase bonita, parece o refrão de
uma canção de amor nova, de versos horríveis, mas que não paramos de
cantar e dos quais gostamos pela sua beleza passageira. Você não tem
como entender isso, você é inteligente demais para entender.}

Acompanhei-o.

\section{ii}

\letra{F}{icamos} \textls[15]{duas semanas. Era um vilarejo depois de Câmpina, não muito longe
de Sinaia. Até hoje não sei o nome dele. Quis voltar a Bucareste no dia
seguinte. Comuniquei a Andrei o meu desejo, assim como dizemos a alguém,
durante uma festa, à meia-noite, que já é tarde e queremos ir. Ele deu
risada e ergueu os ombros.}

--- Impossível.

--- Seja razoável, Andrei. Foi muito bonito, mas preciso ir.

--- Ir como?

--- No seu automóvel.

--- Quebrou.

--- Até ontem ele não tinha nada.

--- Mas hoje tem. Um automóvel é uma criatura caprichosa.

--- Deixe de brincadeira.

--- Juro. Hoje de manhã, você ainda estava dormindo, desci até o quintal
e quebrei o carburador. Sou um homem prudente. Imaginei que você
quisesse fugir. Então decidi que essa nossa aventura devia ser protegida
contra todos, contra você em especial. Creia-me: um automóvel em
perfeito funcionamento é um perigo. Um automóvel enguiçado, a quarenta
quilômetros de distância da cidade, é uma garantia de constância.
Enguicei o meu.

--- Você é louco.

--- Sou.

\textls[-15]{Você reconhecerá, nessa conversa, o estilo do Andrei, aquele estilo que
você dizia ser cordial e ingênuo, querendo certamente dizer indiscreto e
arrogante, mas do qual eu gostava e ainda gosto. Pois, veja, hoje, com o
cansaço desses cinco anos e com a crueldade que pacientemente aprendi
com ele, não posso deixar de sorrir ao me lembrar daquele Andrei que me
aprisionava, atrevido, feliz e inconsciente, seguro de si, orgulhoso de
sua proeza. Estava vestido com uma calça branca de verão e uma camisa de
mangas arregaçadas, gola desabotoada, o que o rejuvenescia e o dotava,
naquele quintal do interior, de um certo ar rústico, simplicidade,
alegria.}

\textls[-10]{Já lhe disse que sou velha. Naquela altura eu também era. Menos, mas
era. Há dentro de mim um cansaço antigo, que vem não sei de onde, que me
torna vulnerável a tudo o que diz respeito a coragem, a um gesto brusco,
a uma frase atrevida, a um jovem rosto inconsciente. Vai saber? Deve ser
algo semelhante à melancolia dos finais de verão, quando o sol ainda
está inteiro e a luz ainda é branca, mas a copa das árvores se arrepia
ao anoitecer com o pressentimento do declínio que está por vir, e que
elas guardam dentro de si como uma brasa íntima, embrulhada em miolo de
pão.}

\textls[15]{Naquela hora da manhã, Andrei era um conquistador, ``conquistador'' no
sentido clássico da palavra, significado que só então eu descobri, pois
até então eu só encontrara conquistadores no cinema e no teatro,
provavelmente para que a revelação se tornasse completa mais tarde.}

\textls[15]{Apertei a mão dele, consentindo em tudo. Abraçou-me entusiasmado e
barulhento, infantil, muito infantil, mas ao mesmo tempo ostensivamente
dominador --- o que não me descontentava, pois nunca me descontenta
deixar nas pessoas por quem tenho afeto a impressão de que um capricho
delas seja uma ordem. Não me lembro de quanta ironia havia na minha
sujeição. Deve ter sido pouca. Mas muito pouca: o mínimo necessário para
me desculpar diante daquilo que você chama de meu ``tato''. Joguei fora
esse meu tato, joguei fora assim como jogamos fora um chapéu que nos
incomoda e ficamos com a cabeça descoberta e o cabelo desgrenhado ao
vento.}

--- E, apesar de tudo isso, não podemos ficar.

--- Por quê?

\textls[15]{--- Muito simples. Só tenho esse vestido que estou usando, e você, só
esse terno.}

\textls[20]{Deu risada, beijou-me, atirou num gesto ágil o paletó sobre os ombros e
saiu correndo, gritando apenas que voltaria antes do almoço.}\looseness=-1

Correu vários quilômetros até chegar à estrada, parou o primeiro carro
que passou, enfiou-se nele para o estupor do motorista, em uma hora
chegou a Braşov, passou pelo mercado e, uma hora depois, estava de
volta, cheio de pacotes, caixas, sacolas, que abrimos juntos aos risos,
descobrindo cada aquisição, pois nem ele mais sabia bem o que trouxera,
e cada pacote desembrulhado era uma surpresa para ambos: pijamas, uma
vitrola com um único disco, um vestido de lã, roupa de ginástica, doces,
livros, raquetes de pingue-pongue, lenços, pó de arroz, água de colônia,
óculos de sol, todo um bazar fantasioso constituído por bagatelas
encantadoras e um mau gosto estridente.

É provavelmente deslocado dizê-lo a você, além de ser uma besteira, mas
você saberá perdoar este meu momento de candura: foram os mais belos
dias que guardo na memória.

Tenho uma pilha de fotografias daquele período e não raro as contemplo
--- hoje ainda --- sem arrependimento, sem censura, por tudo o que
aconteceu desde então, feliz por descobrir um novo detalhe, nessas
fotografias, que conheço de cor e em que tudo é emocionante, tudo, os
sapatos brancos dele, meu cordão arrebentado (estava correndo e ele
enganchou numa calha, num dia em que estávamos voltando do rio Prahova,
onde tomamos banho, num lugar bem escondido, completamente pelados, pois
é verdade que não tínhamos maiô de banho, mas também porque essa loucura
nos rendia uma imensa alegria), e tudo está intacto, próximo, familiar ---
como lhe dizer? ---, não como uma evocação desesperada de coisas
perdidas, mas como o reconhecimento brando e sossegado de uma paisagem
em que vivemos com a sensação de que ela seja nossa para sempre.

Desde então, Andrei foi, inúmeras vezes, brutal, obsceno e maldoso, mas
tudo isso e também muito mais --- oh, infinitamente mais do que eu seria
capaz de contar, não por ter medo de você, ou porque me sentiria
humilhada (pois faz tempo que não tenho mais orgulho), mas porque eu
passaria mal enumerando tudo, mas tudo isso, como eu dizia, tomba e se
apaga na lembrança daquelas duas semanas iniciais.

\textls[15]{Acho que já contei que, entre as coisas trazidas de Braşov, havia uma
vitrola e um disco. Um único disco. Não sei por que veio só um e
justamente aquele. Num lado, uma dança húngara de Brahms e, no outro,
uma dança espanhola de Granados, ambas interpretadas ao violino por
Jascha Heifetz. Era um disco vermelho da \emph{His Master's Voice}, do
qual me lembro perfeitamente ainda hoje, embora, desde então, não sei
por quê, não tive mais curiosidade de ouvir. Durante duas semanas, de
manhã, na hora do almoço, à noite, especialmente à noite, após o jantar,
esperando na varanda passar da meia-noite, ouvíamos aquelas duas
melodias, que se tornaram tão familiares, que deixaram de ser duas peças
musicais, transformando-se numa espécie de ritual doméstico, parte
integrante daquela nossa vida, assim como todos os sons familiares de
uma casa --- passos reconhecíveis no vestíbulo, o tique-taque do relógio
na parede, o barulho do abrir e fechar das portas\ldots{}}

\textls[15]{Está vendo, passaram-se tantos anos, esqueci tantas coisas, esquecerei
ainda outras tantas, mas acho que sempre vou me recordar daquelas
canções, não por serem bonitas, nem sei se são bonitas, mas por aquelas
férias embutidas dentro delas, tão profundamente, que nem o Andrei, ele
que só conhece tangos, se esqueceu delas, e às vezes, provavelmente de
maneira involuntária e sem intenção, calha de ele as cantarolar enquanto
come ou se barbeia.}

Muito me alegrará se ele não souber o que são, nem onde as ouviu, e que
as leve consigo dessa maneira, indiferente, assim como levamos, no bolso
interior do casaco, uma carta antiga do ano passado, que ali esquecemos,
considerando-a extraviada.

De qualquer modo, nunca lhe perguntei nada sobre isso.

\section{iii}

\letra{R}{etornei} \textls[15]{tarde a Bucareste, em fins de setembro, quando encontrei as
pessoas já há muito de volta das férias e as árvores, nas avenidas,
amareladas.}

\textls[-15]{Foi nessa mesma época que te conheci, você tinha acabado de voltar de
Paris, depois de não sei que estágios médicos, sobre os quais Andrei tinha me falado com ceticismo (``Esse Ştefan Valeriu nunca vai ser alguém''), e
talvez se recorde como fiquei embaraçada aquele dia, em que ele te
chamou no meio da rua para me apresentar, cintilante de orgulho e cheio
de entrelinhas. Sentia que ele esperava ser parabenizado por mim, sua ``conquista''. Sentia que estava encantado pelo meu vestido, pelos
meus olhos, pela sua surpresa.}\looseness=-1

Quanto lutei, naquela hora, para temperar o entusiasmo e a indiscrição
dele. Não fui prudente e nem agora creio ser. Mas me assustavam os
boatos, as suposições, os comentários, as noites de estreia em que, ao
passar pelo saguão, sentia às minhas costas um caudal de sussurros, as
idas ao restaurante, quando me recebiam, já na porta, trinta olhares que
sabiam alguma coisa e queriam saber tudo, as perguntas delicadas, as
alusões\ldots{} Se eu pudesse ter comunicado a todos, por meio de uma
circular, que eu estava amasiada com Andrei, e se eu tivesse certeza de que
isso poria fim à curiosidade pública, eu o teria feito. Cansava-me a
``atmosfera de sensacionalismo'' com que me circundavam.

Tentei explicar tudo isso ao Andrei. Disse-lhe que ninguém me
incomodava, mas que deveríamos dar ao nosso amor o tempo necessário para
encontrar a ``fórmula social'' que melhor lhe conviesse (não me lembro
de ter utilizado esse termo horroroso, mas, no fundo, foi o que eu quis
dizer). Pedi-lhe que deixasse que as coisas encontrassem sozinhas o
próprio caminho.

--- Você é uma burguesa --- disse-me.

\textls[15]{Não fiquei brava. Tinha razão, de certo modo. Naquele período, ele
estava febril, entusiasmado, alucinado por projetos para o futuro,
apaixonado: eu estava calma, um pouco cética, de todo modo lúcida e,
embora feliz com o nosso namoro, exausta com suas explosões juvenis.}

\textls[15]{Eu fazia questão de uma coisa em especial: que morássemos separados.
Quanto a isso, não estava disposta a fazer nenhuma concessão. Ele se
mostrou atrevido, autoritário e ameaçador, mas eu fui tenaz e isso foi o
bastante.}

\textls[15]{Mas, pelo amor de Deus, por quê? Você me recebe para dormirmos juntos.
Para comermos juntos. Saímos juntos pela cidade, por que não moraríamos
também juntos? Por que você não se muda para a minha casa? Por que não
compramos uma casa maior, para nós dois?}

Teria sido difícil lhe responder e teria sido difícil que ele
compreendesse. Nem tentei. Mantive, porém, minha decisão.

\textls[-15]{Tinha necessidade de uma casa em que eu pudesse comandar e ficar
sozinha; um quarto em que ninguém pudesse entrar sem bater à porta, um
cofre em que eu pudesse trancar o que quisesse, quatro paredes entre as
quais eu pudesse ficar longe de tudo e de todos.}

\textls[-20]{``Temperamento de castelhana'', disse-me você uma vez, e não soube o que
responder. Não acho que seja isso. Só sei que gosto do meu interior, não
tenho alegria mais garantida do que retornar para o meu interior à
noite, mantive muito clara a noção de ``refúgio'' da casa (a volta do
filho pródigo é a única passagem da Bíblia que realmente me comove) e,
se não me deixei dilacerar pela vida até hoje, é em boa parte graças a
este quarto de onde ora lhe escrevo. Quantas possíveis loucuras não
refreei aqui, quantos gestos estridentes, quantas providências
falsas\ldots{} E quantas vezes não entrei aqui ferida, com o rosto
desfigurado, os braços tombados ao longo do corpo, incapaz de decifrar
sabe-se lá que catástrofe que me surpreendera, dizendo para mim mesma
que tudo acabara e provavelmente tomando esse ``tudo'' pela própria
vida.}\looseness=-1

Um ou dois dias depois, quando você me encontrou na rua, eu ria comigo
mesma, imaginando quantos danos íntimos sustentavam a calma pela qual
você me parabenizava e da qual eu me orgulhava; no entanto, por outros
motivos, acredite.

\textls[15]{A casa foi a única coisa que mantive, passando todo o resto, dia após
dia, à posse de Andrei, que sabia pedir e obter com um verdadeiro
instinto de criança terrível.}

Gostava da sua aparência agitada, do modo de soar nervoso e hostil, do
gesto com que atirava o chapéu na entrada, das perguntas desordenadas,
das respostas curtas, dos passos rápidos com que passava em revista os
aposentos, trocando o lugar das coisas, surpreendendo-se com qualquer
coisa, querendo saber de tudo, impaciente, intolerante e tirânico, como
lhe caía bem ser.

\textls[15]{--- Você é impertinente, Andrei.}

\textls[15]{Isso o lisonjeava. Sentia-se de repente forte, prepotente, e me lançava
um sorriso sem modéstia --- sorriso que eu adorava enormemente, pois eu
me sabia reticente e vulnerável, ao passo que ele se mostrava ousado e
presunçoso.}

\textls[15]{Desde o início percebi a vaidade dele, mas cultivei-a com alegria, pois
era seu ponto mais sensível e porque ela, embora o tornasse às vezes
impossível, o dotava de não sei que graça áspera de adolescente, que
desconhece obstáculos. Deduzia, assim, no meu íntimo, que, dentre nós
dois, eu era a mais forte, e pensei diversas vezes que todas as suas
vitórias sobre mim, ao menos no início, foram, na verdade, pequenas
concessões da minha parte, para evitar que ele se zangasse.}

Não sei dizer quando exatamente o Andrei se tornou indispensável.
Provavelmente não foi num momento preciso. Meu amor por ele cresceu de
maneira sedimentar, lenta, a partir de pequenas coisas, hábitos, até
que, um dia, me vi prisioneira dele. Por muito tempo me considerei
livre, olhando-o com distanciamento e, pelo fato de o julgar com frieza,
consciente de todos os seus defeitos divertidos, eu nutria a ingenuidade
de me acreditar independente dele e pronta para enfrentar, de coração
aberto, uma eventual separação. De nenhum modo eu permitiria que aquele
homem, dotado de fúrias e fantasias dominadoras, pudesse me fazer
sofrer, eu que o via com tanta indulgência e ironia.

\textls[15]{Tendo a impressão de que ele brincava de tirano comigo, conformei-me em
dar réplicas de escrava, assim como gente grande aceita se assustar com
uma criança que se cobre com um pano branco e faz \textit{uuu-uuu} como um
fantasma\ldots{}}

Não sabia que essa brincadeira era tão perigosa.

\section{iv}

\letra{N}{ossa} primeira briga mais séria foi por causa do meu nome; Andrei não
gostava nada dele. Tinha dificuldade em pronunciá-lo devidamente; mas, além de feio, com
certeza, comum e desprovido de um som especial, foi o nome que me deram
desde pequena: Maria.

Entre nós, na intimidade, ele tentava todo tipo de substituições e
diminutivos afetuosos, que eu recusava com firmeza.

--- Estou avisando, Andrei, que, sempre que você me chamar por um nome
que não é o meu, não vou responder.

Seu maior sofrimento era quando tinha de me apresentar a um desconhecido
e, então, baseando-se no meu constrangimento, ele retificava o meu nome,
pronunciando-o à francesa, à inglesa, ou abreviando, conforme sua
preferência, a primeira sílaba.

\textls[15]{Assim, fui chamada de Marie, Mary, Ria. E, sempre que ele armava essa
brincadeira, eu a desarmava com tranquilidade, para a sua fúria
infinita.}

--- Meu nome é Maria. O Andrei não gosta de me chamar assim; é uma velha
brincadeira dele.

Não dê risada: era algo de extrema importância e, no final das contas,
consegui resistir, pois desconfiava que aquilo significava o grande nó
do que nos distinguia, o símbolo mais evidente de que ele e eu
pertencíamos a dois universos distintos.

\textls[15]{Por que não gostava de ``Maria''? Não sei bem. Perguntei algumas vezes e
ele não soube explicar.}

\textls[15]{--- Você está entendendo --- dizia ele, contrariado ---, todas as
mulheres em torno de nós têm um nome normal. Por exemplo, a Suzy Ioaniu.
Ela se chama Suzana, mas reconheça que seria horrível chamá-la assim.
Outro exemplo, Bébé Stoian. Anny. E assim por diante: Lulu, Lily, Ritta,
Gaby\ldots{} Só você\ldots{} só você\ldots{} Maria, que nome de caipira!}

\textls[15]{Eu dava risada. Sinceramente me divertia aquela sua paixão por nomes
amarfanhados, estranhos, amassados, nomes escritos com ípsilon ou com um
``e'' mudo, com consoante dupla, estética leviana que o Andrei devia ter
aprendido nos espetáculos dos bares noturnos e que queria impor ao meu
nome de três vogais, monótono, insípido, caipira, mas do qual eu gostava
e com o qual me habituara por ser tão simples e comum.}

\textls[20]{Como Andrei se sentiu triunfante no dia em que, ao vir me ver, descobriu
que a minha empregada, contratada fazia pouco tempo, se chamava como eu,
Maria. Nem se deu ao luxo de rir: simplesmente esticou a palma, em tom
demonstrativo, diante da evidência, esperando o reconhecimento da
verdade. Que briga, meu Deus, que briga tivemos ao lhe dizer que aquilo
absolutamente em nada me incomodava, e que eu deixaria meu nome em paz,
assim como o da empregada Maria, porque esse nome é capaz de atender a
ambas na mesma casa, junto com todas as Marias do mundo.}

Concluiu a discussão de maneira severa.

\textls[15]{--- Você é irremediavelmente burguesa. Só que você devia usar também
vestidos de 1915, não só o nome.}

Muito mais tarde, só mais de dois anos depois, em Paris, numa
circunstância que o faria dar risada, mas que me comoveu sinceramente,
Andrei consentiu com o fato de que, definitivamente, um nome não é
melhor nem pior que outro, chegou mesmo a reconhecer que o meu nome
combina e se parece comigo. Havíamos ido à rue La Boétie visitar algumas
galerias de pintura e, por acaso, na Bernheim-Jeune, descobrimos uma
curiosa exposição de uma atriz italiana, Maria Lani, que, por conhecer
pessoalmente um grande número de pintores famosos, pedira que cada um
pintasse o retrato dela, reunindo, por conseguinte, uma coleção de
imagens excepcionais do mesmo modelo.

\textls[15]{Acho que Andrei se surpreendera com o fato de que uma mulher tão
distinta como aquela atriz italiana pudesse se chamar como eu, Maria.
Creio, ademais, que ficou orgulhoso com a semelhança, me lembro de ele
ter me observado ali, na exposição, algumas vezes, com uma centelha de
admiração que há muito tempo não demonstrava por mim e que, desde então,
creio que nunca mais nutriu.}

\textls[15]{Ao anoitecer, fomos jantar em Montmartre num bar acolhedor e não muito
badalado, onde Andrei se comportou como nos bons tempos, infantil,
vibrante, direto, e me falou um monte de besteiras encantadoras.}

--- Maria, fui um burro e você não deve me perdoar. Você tem justamente
o nome que merece, nome de uma graça invisível, nome que não suporta
diminutivos, belo como você, simples como você, talvez um pouco sério
demais, pois eu gostaria às vezes de te ver diferente, mais animada,
mais inquieta, não como eu, eu sou completamente maluco, mas um pouco
diferente\ldots{}

\textls[20]{--- Mais jovem, Andrei, mais jovem --- ajudei-o a calibrar os
pensamentos.}

\textls[15]{Protestou com veemência, recordou-me que ele é mais velho que eu,
contou-me dos seus fios de cabelo branco, mas, apesar dos protestos, eu
sabia ter dito a verdade e, naquela noite, Andrei, orador arrogante e
enfatuado, de quem você sempre dá risada, me disse as coisas mais justas
e delicadas que podiam ser ditas sobre o nosso amor conturbado.}

\textls[15]{Percebi, então, que, só depois de não sei quantos meses de resignação,
minha relação com Andrei não estava definitivamente comprometida, uma
vez que aquele momento de compreensão fora possível, e --- quem sabe?
--- com um pouco de paciência, certa perseverança caso necessário e uma
pitada de sorte, eu conseguiria finalmente descobrir, em nossa dinâmica
íntima, o pedal exato que deveria pisar para que a sintonia daquela
noite entre nós pudesse durar.}

\textls[15]{Durar! Está vendo, esse deve ter sido, não só na minha relação com
Andrei, como também com as outras pessoas e com a própria vida, o meu
maior erro. Durar! Apavora-me a ideia de que algo possa ser
completamente aniquilado --- uma coisa, uma pessoa, um sentimento ou um
objeto apenas ---, de que possa desaparecer da noite para o dia e, na
passagem das coisas, obseda-me apenas a sua possível eternidade, a marca
que pode se demorar e permanecer.}

\textls[15]{O que me dilacerava na presença de Andrei era aquele seu ar de uma provisoriedade contínua, de pessoa que entrou por acaso numa casa, com o
chapéu enterrado até as orelhas, sem saber se vai embora, se volta ou se
fica. Por vezes eu era tomada por uma tentação infantil de colocar a mão
em cima do seu ombro e perguntar, com toda a seriedade:}

--- Você está aqui?

Não ache que esse meu anseio por permanência fosse opressor ou coercivo.
Sabia muito bem que Andrei tinha que ficar solto para se mover, trair e
--- como lhe explicar? --- creio que justamente isso, aquele seu ar
indefinido de vagabundo me comovia e me fazia amá-lo, pois ele era o
tipo de pessoa dos ventos aleatórios e do contato social, ao passo que
eu, o da espera e eternidade. Uma eternidade barata, claro, tanto a
minha como a dele, mas que precisava ser protegida com afinco, contra
tantos obstáculos.

\textls[-15]{E, se eu apreciava a sua presença e precisava dela, tão fantasiosa,
barulhenta e desequilibrada, presença de pequeno aventureiro, que
aterroriza e impõe, é porque isso oferecia uma espécie de brisa matinal
à minha vida ``séria'', como dizia ele, ele também talvez precisasse da
minha tranquilidade, ao menos para descansar, pois, às vezes, as suas
fugas, escapulidas e caprichos destrambelhados acabavam por cansá-lo e,
então, ele talvez gostasse de encontrar, ao meu lado, um dia ou uma
semana de paz, provavelmente um pouco monótona demais e demasiado
burguesa, mas estável.}

Creio que Andrei finalmente compreendera isso naquela noite em Paris e,
ao retornarmos tarde para casa, a pé, por ruas meândricas ao longo do
Sena, levemente embriagados pelo vinho e levemente impacientes pela
noite de amor que prometíamos um ao outro de maneira tácita, eu estava
feliz de braço dado a ele.

\section{v}

\letra{E}{} o \textls[20]{que mais? Pergunto-me se ainda tenho o que contar. Agora que me pus
a falar de tantas coisas do passado, das quais não me esqueci por um só
instante, e que guardei até hoje em algum lugar dentro de mim, como numa
gaveta em que não precisamos remexer, agora que evoquei tantos
acontecimentos, arrependimentos e equívocos, confesso que me sinto
apegada a eles e, ao relatá-los, sinto também uma espécie de prazer
doloroso, como se liberasse das ataduras um braço aprisionado tempo
demais.}

\textls[15]{No entanto, não vou lhe contar as histórias restantes --- brigas,
traições, explicações --- porque sinto que é tarde para fazer o caminho
de volta. Aceitei-as faz tempo, assim como foram e como ainda serão, e
tento transformar a companhia delas numa espécie de volúpia familiar ---
provavelmente a única que me resta. De qualquer modo, você já conhece
todas elas, assim como eu e todo o mundo, pois o Andrei não poupou nada
e teve o cuidado de que cada nova aventura sua fosse um evento público,
iniciado e terminado à vista de todos.}

\textls[10]{Você se lembra de quando ele fugiu, em 1924, para Paris, com a Didi,
aquela vedete loira que foi embora com ele no meio da temporada,
abandonando o teatro de revista do Cărăbuş}\looseness=-1\footnote{Inaugurado em 1919,
  é o mais célebre teatro de revista romeno, localizado no centro de
  Bucareste. Atualmente porta o nome de seu fundador, o ator Constantin
  Tănase (1880--1945). {[}\textsc{n.\,t.}{]}} \textls[10]{e deixando toda a Bucareste
  perplexa? Contaram-me que se fala disso até hoje e que, nos teatros, sua
  proeza de então se tornou lendária. Enquanto isso acontecia, eu saía
  na rua, recebia visitas, fazia visitas --- tudo com uma tranquilidade
  absurda, com um sorriso lavado, com um desprendimento que deve ter
  paralisado as mais brutais alusões.}\looseness=-1

\textls[-15]{Como fiz isso? Não sei. Fui corajosa? Depreciativa? Insensível?}

Não sei, juro que não sei, mas creio ter tido a consciência de que
aquela miséria não pertencia à circunferência da minha vida, não tinha a
ver comigo, não tinha como ter a ver comigo, havia um muro sólido entre
o que acontecia e o que eu era.

\textls[5]{Ninguém entendeu nada e Andrei entendeu muito menos que qualquer um. Ao
retornar, ele evitou me encontrar por duas semanas. Enviou-me
cumprimentos vagos e emissários assustados, em missão de reconhecimento.
No fim, veio me ver em casa certa manhã, de surpresa, desorientado, sem
saber o que fazer, se deveria se explicar, ou me culpar; de todo modo,
estava decidido a ``ter razão''. Não lhe dei a oportunidade. Recebi-o
como se nada houvesse acontecido, como se tivéssemos nos visto na noite
anterior, dirigi-me a ele amistosamente, demos risada juntos, pedi que
ficasse para o almoço, abordei numerosas ninharias. Após a refeição, pus
para tocar na vitrola uns discos que eu havia comprado para ele na sua
ausência, tangos argentinos, e pedi que me mostrasse um passo de dança
que começou a entrar na moda naquela época e que eu não tinha conseguido
aprender. Ele adorou a ideia, pois se sentiu lisonjeado por eu recorrer
à sua competência, e senti como esse fato insignificante o fez recobrar
a segurança, deixando de lado o ar de culpado e voltando a ser, assim
como eu o conhecia, o dono da situação.}

Ao entardecer, enquanto tomávamos chá, provavelmente porque gostou dos
docinhos (sabe, aqueles folhados porosos), ele ergueu os olhos da xícara
e me disse, condescendente:

\textls[-15]{--- Você não é o meu tipo de mulher, mas é uma moça simpática.}

Fitei-o, pela primeira vez, crispada, pois, sem que me ofendesse, aquele
adjetivo me pareceu leve demais para os meus ombros, que já haviam
aprendido a suportar tanto.

--- Você também, Andrei, você também\ldots{}

Olhava-o enquanto comia na minha frente, à mesa da qual se ausentara por
tanto tempo, ávido, alegre e comunicativo, com um candor que lhe caía
perfeitamente e com uma inconsciência que poderia desculpar não uma
traição, mas um crime. Sempre gostei de olhar para ele enquanto comia, e
creio que a avidez seja a única coisa profundamente boa nele (talvez
seja uma besteira o que eu diga agora, mas acredito nisso e vou dizer do
mesmo jeito), pois um homem ávido tem um quê de criança, alguma coisa
que diminui a sua aspereza, a sua importância, o seu terror de macho. Se
mulheres simples e burras lograram viver a vida toda ao lado de homens
grandiosos, reis, generais, cientistas, talvez seja justamente porque
comiam junto com eles à mesa, tendo assim acesso àquela imagem de
crianças bicudas e esfomeadas, a única coisa que as protegia de suas
majestades.\label{majestades}

Ah, com certeza não era o caso do Andrei, embora ele fosse também um
pequeno tirano a seu modo, e o meu único momento de superioridade em
relação a ele, a única ocasião em que o sentia depender de mim, como se
esperasse a minha proteção e a minha decisão, era à mesa, quando, ao
desdobrar o guardanapo, me perguntava com os olhos o que é que lhe daria
de comer.

\textls[-20]{É estúpido e pueril o que estou lhe contando agora, sinto que seja
estúpido e você não tem como entender nada disso, mas não me leve a mal,
sou a única culpada se estou dizendo a você, que é homem, coisas que só
uma mulher é capaz de compreender e sentir.}

Foi uma das poucas alegrias que devo ao Andrei, a alegria de o servir,
de cuidar dele, de o ver mimado por debaixo daquela máscara de homem
enérgico, de bajular seus gostos e os educar. Ele se submetia, com a
satisfação de quem recebe qualquer coisa por se sentir no direito, e
havia algo de majestoso naquele seu modo de permitir que eu tomasse
conta dele.

\textls[15]{Talvez fosse a única coisa que fizesse o Andrei se sentir realmente
conectado a mim, e eu sabia que, por mais longe que fugisse, com quem ou
para onde quer que fosse, um belo dia ele voltaria, de cansaço, sabendo
que, em algum lugar, um corpo dócil de mulher, uma cama conhecida, uma
boa refeição e uma vitrola com tangos novos o esperavam.}

Está vendo, não tenho vergonha em dizer que, no meio disso tudo, eu
contava com a preguiça, o cansaço e a avidez dele, talvez também, às
vezes, com a vaidade dele, pois eu era uma mulher elegante --- não? ---
que ficava bem de braço dado com um homem, nas noites de estreia, no
teatro ou no restaurante.

\textls[10]{Por isso, jamais me desesperei diante de nenhuma das escapulidas do
Andrei, e nem fui atrás dele, convicta de que, mais cedo ou mais tarde,
voltaria sozinho. Não digo que tenha sido fácil, especialmente no
início. Quantas vezes não o esperei em vão à mesa depois de ele prometer
atender ao meu convite, quantas vezes não fiquei em casa arrumada para
sair, à noite, já tendo passado da hora do início do espetáculo, atenta
à campainha e vendo o tempo passar, cinco minutos, mais cinco minutos,
até compreender que não viria mais e então eu tirava a roupa, não
indignada, mas triste por perder uma noite. E como era constrangedor
dizer à empregada que eu almoçaria sozinha depois de avisá-la para
colocar talheres para duas pessoas, ou que não sairia mais, depois que
me vira, pouco antes, me arrumando para o teatro ou para passear. Você
talvez não tenha conhecido essas pequenas misérias e não saiba que
rastro de fadiga elas deixam, mas eu, que as vivi a cada dia, teria
preferido a elas um grande golpe, uma grande dor, que ao menos nos
atinge de frente, nos derruba ou revigora.}

\textls[10]{Nunca pedi explicações ao Andrei e o impedi sempre que as tentou me dar,
pois eu sabia muito bem que nada poderia apagar a minha íntima sensação
de dúvida, e o resto --- fatos, argumentos, justificativas --- pouco me
importava. Ademais, o prazer em revê-lo era toda vez tão novo e vivo que
tudo desaparecia, pelo menos durante a sua presença.}

\textls[10]{Você se lembra de quando nos encontramos, na primavera passada, na época
das eleições gerais, certa manhã, na Calea Victoriei,}\looseness=-1\footnote{Rua da
  Vitória, umas das principais artérias do centro de Bucareste.
  {[}\textsc{n.\,t.}{]}} \textls[10]{e você me disse que havia visto o Andrei pouco
  antes, preparando-se para partir no fim da tarde para Turda,}\looseness=-1\footnote{Cidade
  na Transilvânia. {[}\textsc{n.\,t.}{]}} \textls[10]{onde era candidato?
Respondi-lhe apática, como se soubesse de tudo aquilo, como se a
presença de Andrei em Bucareste e sua partida para Turda fossem óbvias.
Na verdade, não sabia de nada e fazia um mês que não via o Andrei, desde
quando me contara pela primeira vez da candidatura, que não aprovei,
parecendo-me aquilo uma piada séria demais para ele.}\looseness=-1

\textls[-20]{Portanto, estava em Bucareste! Passeava pela rua, conversava com as
pessoas, divertia-se\ldots{} Invadiu-me uma saudade dele, uma saudade
desencorajada, em que não havia censura nem revolta, mas apenas uma
ideia tímida de poder vê-lo, ouvi-lo, apertar-lhe a mão.}

\textls[-10]{À tarde, fui até a estação de trem e cheguei uma hora mais cedo, o que
me deixou aflita, pois imaginei que, naquele meio-tempo, Andrei talvez
passasse pela minha casa e não me encontrasse. Aguardei-o na plataforma,
antes do primeiro vagão, para poder observar o trem inteiro, temendo não
identificar o Andrei em meio à multidão apressada. Ele chegou tarde,
poucos minutos antes da partida do trem e, quando me avistou de longe,
estacou bruscamente, de espanto ou de medo, pois provavelmente previa um
escândalo. Fiz-lhe um gesto breve com a mão, o que o incentivou a se
aproximar, com uma alegria exagerada e obrigando-se a ser loquaz, o que
era desnecessário, pois não queria dele outra coisa senão vê-lo. Foi
cordial, afetuoso e --- dependurado na escada do vagão --- apertava
efusivo a minha mão, sem largá-la, embora esperasse, de um momento para
o outro, o apito da partida. Por um instante imaginei que ele talvez
saltasse daquela escada, por milagre, e, com a mala numa mão e o meu
braço na outra, se dirigisse comigo até a saída e me dissesse que
ficaria em Bucareste aquela noite. Por um instante fui fulminada pela
ideia de lhe pedir que ficasse; mas, por sorte, mordi os lábios,
calei-me e sorri.}

\textls[10]{O trem partiu e Andrei, na escada, fazia movimentos espalhafatosos com o
braço, até nos distanciarmos bastante, ele cintilante de satisfação e
orgulho, eu hesitante diante de um grande vazio, sabendo apenas que não
deveria chorar.}

Aquele momento, talvez, resumiu tudo o que havia entre nós.

\section{vi}

\letra{É}{} tarde \textls[10]{demais --- não é? --- para arrependimentos e remorsos. Se Andrei
não houvesse existido\ldots{} Pois então, se Andrei não houvesse
existido, confesso que não sei o que teria sido. Ele entrou tão
completamente na minha existência, mobiliou-a com tantas coisas, como a
uma casa abandonada, barricou com sua silhueta de boxeador tantas portas
que poderiam dar acesso a outra coisa, a outras pessoas, a outros
acontecimentos, de modo que, agora, ao imaginar que jamais houvesse
existido, sou incapaz de ver ao meu redor algo além de um imenso vazio.
Sinto que não teria me tornado tão exausta e que não traria comigo essa
íntima sensação de desprendimento que me serve de escudo; no entanto,
para além da ausência de Andrei, não consigo ver mais nada.}

Tentei às vezes, noite passada inclusive, após a sua surpreendente
declaração de ontem, tentei colocar ordem dentro de mim e julgar com
frieza o meu amor por esse homem, que conheço o suficiente para não me
iludir. Tenho um senso aguçado quanto ao que é ou não é pertinente, uma
espécie de instinto básico de justiça entre as pessoas, e eu sempre
achei que havia algo de inadequado na minha relação com o Andrei, que há
em mim certas coisas que teriam desabrochado nas mãos de outra pessoa,
tendências, não preciosas nem brilhantes, mas que talvez pudessem
iluminar a vida do homem amado. Achei também que tudo isso seria pesado
demais ou leve demais para o Andrei, que ele não teria o que fazer com
isso e que, permanecendo ao lado dele, eu desperdiçaria de maneira
estúpida um punhado de vida que talvez fosse necessária em outro lugar,
para outra pessoa, e que isso desorganizaria o universo inteiro, pois,
no frigir dos ovos, identificariam a falta dessa reserva extraviada de
amor.

\textls[15]{Infantil, não é mesmo? Nem tão infantil assim, contudo, se você entender
que o assombro ante o fato de eu não ter desertado da minha verdadeira
vocação realmente me atormenta e que, às vezes, com Andrei do meu lado,
deixo-me de repente invadir pela ideia arrepiante de ser uma prisioneira
e que, em algum lugar, não sei onde, de todo modo distante, muito
distante, ainda está à minha espera uma outra existência, com outro
homem, existência essa que interrompi sem saber, cinco anos atrás,
naquela noite de agosto em que me tornei amante de Andrei por preguiça e
leviandade.}

Serei completamente sincera, e digo que tenho pensado em você. Não quero
dizer besteiras, sobretudo agora, depois do que aconteceu ontem à noite,
mas por que não te conheci seis meses antes de te conhecer? Tudo talvez
tivesse sido diferente.

\textls[-15]{Você foi, para mim, talvez sem desconfiar disso, o único apoio com que
eu contava. Gostava de sabê-lo fora da minha relação com o Andrei, fora
de todas as nossas complicações comuns, porque assim eu sabia que, para
além daquelas angústias e comédias, restava-me algo, um terreno neutro
de vida, uma ilha pacífica, em cuja fronteira os temores, as suspeitas e
as inseguranças se dissolviam. Eu me parabenizei por resistir à tentação
de lhe contar tudo isso e fui grata por você jamais perguntar nada ---
justamente por isso impedi qualquer confusão entre o meu amor por Andrei
e a minha amizade por você, que permanecem duas coisas alheias uma à
outra, mas entre as quais, acredite, jamais pensei ter de escolher.}\looseness=-1

\textls[-15]{Por que você estragou o nosso acordo? Como as coisas entre nós agora se
enrolam e se desajustam! Ontem à noite, tive a sensação de uma pequena
catástrofe, e me perguntei, agitada, se havia alguma saída. Ainda hoje,
ao começar a lhe escrever, senti-me desconcertada, sem saber muito bem o
que lhe dizer, temendo me expressar de maneira equivocada e que você me
entenda mal, temendo sobretudo que a ligação entre nós seja interrompida
para sempre por esse acidente brutal, tão difícil de esclarecer, em meio
a tantos sentimentos contraditórios, que me dividem e confundem.}\looseness=-1

Agora, no entanto, após ter lhe contado tanto --- e, à medida que
contava, comecei a enxergar com maior clareza ---, creio poder lhe
dizer, com toda a sinceridade, que nada está perdido: retomemos as
coisas de onde você as deixou ontem à noite e esqueçamos o que se
seguiu. Isso é possível. Escute o que digo, acredite em mim: isso é
possível.

Deixe-me continuar com o Andrei, com quem tenho acertos, acordos e
desavenças que não podem ser interrompidos em um dia, nem em dois, nem
em um ano. Cheguei longe demais para voltar, estou cansada demais para
sucumbir. Uma relação amorosa me parece algo tão complicado, uma
engrenagem tão opressiva e minuciosa, que me é impossível
desvencilhar-me das roldanas, interromper as ligações miúdas que me
fixam, passar por cima do assédio de todos os detalhes a partir dos
quais ela se desenvolveu e entre as quais eu me tranquei.

\textls[10]{Há convenções físicas entre mim e Andrei, há preconceitos comuns, há
hábitos, cujo gosto de paixão talvez eu tenha perdido ou esquecido, mas
do qual não consigo me separar, pois sei que se tornaria um novo
suplício caso o fizesse, exatamente como uma mão enferma que nos deixa
em paz enquanto a mantemos imobilizada, mas que nos faz estremecer de
dor no momento em que, desatentos, a tiramos do lugar. Tenho medo de me
aproximar dessa ferida antiga que se chama amor, e sinto que é
desmesuradamente melhor deixar os remorsos e as revoltas em paz, em suas
profundas camadas anímicas, hoje calmas, porque, pelo menos aqui onde me
encontro, sou uma velha conhecida delas e me sinto de certo modo em
casa. No dia em que souber que o Andrei nunca mais vai bater à minha
porta --- com aquele seu ruído breve e dominador, que reconheço entre
milhares ---, vou me desorientar para sempre. Entenda e perdoe essa
minha vulnerabilidade. Entenda, sobretudo, que aquilo que resta para
além disso, entre nós dois, nossas longas conversas enquanto tomamos chá
em casa, nossos passeios pelos salões de exposição, nossas discussões a
respeito de uma pintura ou de um livro, as noites de concerto, os
pequenos sinais de cumplicidade que trocamos em sociedade, quando, com
uma troca de olhares, confirmamos estarmos de acordo quanto a um gesto
ou a um fato, tudo isso realmente não constitui a coisa mais
insignificante da minha existência e, talvez, nem da sua. É, na verdade,
a única coisa que me reabilita diante de mim mesma e a única
circunstância que me revela que o Andrei, na vida, não passa de um
homem que errou de endereço e que bateu a uma porta desconhecida,
equívoco que durou cinco anos e que vai durar mais cinco, sem deixar,
porém, de ser um equívoco, algo pelo que não posso me responsabilizar e
pelo que não tenho por que me arrepender.}

\textls[15]{Você vem segunda-feira, não vem?}

\chapter{Arabela}

\section{i}

\letra{R}{ecusei} \textls[20]{hoje mais uma oferta. Espero que seja a última. \textsc{j.\,k.\,l.}\,Wood, correspondente e redator do \emph{New York
Herald}, me fitou com estupefação sincera, rasgou o cheque que havia
agitado por meia hora em cima da mesa e me disse, ríspido: ``Com todo o
respeito, o senhor não sabe fazer negócio''.}

\textls[20]{Não sei. Mas a ideia de que poderia contar, para uma revista ilustrada,
uma história que me diz respeito me parece de um ridículo sem igual. Não
me perdoam por ter sequestrado do mundo do espetáculo um número
sensacionalista, \emph{Arabela and partner}, acabara de me dizer
\textsc{j.\,k.\,l.}\,Wood, número retribuído toda noite
com 620 dólares, além das despesas de transporte. O público quer saber o
verdadeiro motivo de eu ter recusado a retribuição e a cobertura
daqueles custos. Ele quer saber onde se encontra Arabela e --- se
possível --- por que Arabela me amou ou por que a amei eu.}

\textls[15]{Na gaveta da direita, guardo uma fotografia dela, do primeiro verão que
passamos juntos. Deve ter sido no fim de agosto, em Talloires.
(Precisaria, um dia, organizar meus papéis e, na medida do possível,
datar as fotografias. Que memória detestável.) Estava usando um vestido
azul, bem claro, com uma gola branca de liceu, só de sandálias nos pés,
sem meias, de cabeça descoberta, sem o mínimo de pó de arroz, um pouco
pálida, mas branca e relaxada sob o sol. Na fotografia, capturei-a com
um braço erguido bruscamente na minha direção --- gesto assustado que
não era dela, pois queria, acho, me dizer para que eu esperasse um pouco
antes de bater. Imagino essa pequena fotografia publicada num jornal e
estremeço. Não por decência, preconceito nem sentimentalismo: mas guardo
para mim as coisas que são só minhas, e me parece que o gesto de
surpresa tem, ainda hoje, o calor imediato de um grito.}

É de surpreender como os dias em que algo definitivo acontece na vida
não têm nada de diferente daqueles em que nada acontece. Nenhum sinal.
Nenhum pressentimento. Naquela noite de novembro, encontramo-nos na
place Pigalle com o adido de imprensa de uma legação aliada, com vistas
a redigir juntos uma nota. Ele acabou não vindo. Irritou-me a ideia de
voltar cedo para casa. Subi com ela até o Medrano.\footnote{Originalmente
  Circo Fernando, inaugurado em 1875. O Circo Medrano é, até hoje, um
  dos circos mais célebres de Paris. {[}\textsc{n.\,t.}{]}} Gosto do
cheiro das arenas de circo, o vermelho violento das cortinas, atrás das
quais se ouve o relincho dos cavalos que esperam a vez, o mau gosto das
atendentes, o bigode clássico do cavaleiro-diretor e a multidão em
derredor que dá risadas ensandecidas, fáceis. Prazer de um esteta
desabusado. Prazer de todo modo.

\textls[20]{Começara fazia tempo. Ao entrarmos, reinava o silêncio. O silêncio que,
no circo, precede os números de salto mortal. Dei uma olhada na
programação.}

\begin{verse}
%\begin{center}
\textsc{trio darties}\\
\emph{Dikki et Miss Arabela}
%\end{center}
\end{verse}

De um lado e de outro da arena, bem alto, havia duas barras em que, como
numa moldura, se balançavam dois atletas de vermelho. No centro, uma
barra de madeira, amarrada com duas cordas ao teto do circo, oscilava
solta, à espera. Na arena, uma espécie de palhaço careca indicava, com
gestos exagerados, o que haveria de acontecer. Os dois atletas de
vermelho deveriam saltar ao mesmo tempo, um da direita, o outro da
esquerda, para a barra móvel do meio, apanhá-la cada um deles com uma
única mão e, em seguida, com o mesmo movimento, saltar dali para as
outras duas barras laterais, um para o lugar do outro. À altura de uns
trinta metros, um salto duplo de grande extensão, sem rede.

Apagaram-se as luzes. Quatro refletores de cores diferentes atiravam
fortes fachos para cima. Na direção dos atletas da direita e da
esquerda, dois. Na direção da barra livre do meio, um refletor branco,
que dava às oscilações por cima do circo um ar de fatalidade. O quarto
refletor projetava mais para cima uma luz azul fraca, revelando ali uma
presença que eu não notara: numa tela de seda, uma mulher. Morena,
decorativa, de maiô prateado, com um bracelete de pedras grandes no
pulso direito, de pernas cruzadas, dali ela fitava tudo com absoluta
indiferença, apenas revelando, no canto esquerdo dos lábios, um sorriso
casual.

\textls[15]{Os tambores se precipitaram a bater. Ao meu redor, não se ouvia a
respiração de ninguém. Em seguida, um golpe surdo de tímpano, um salto,
quatro mãos agarrando a barra do centro, uma única oscilação e\ldots{}
pronto\ldots{} A barra balançava de novo vazia por cima da arena
profunda, os dois atletas sorriam, de um lado e de outro, lugares
trocados, a mulher continuava olhando com o mesmo ar inseguro de
ausência.}

\textls[-15]{Um número medíocre. Foi o que achei na época, em que eu só julgava as
coisas pelo meu gosto de diletante. É o que ainda acho, depois de tantas
turnês exitosas. Um número medíocre. Com certeza interessante e
perigoso, mas mal apresentado, com detalhes inúteis, uma cenografia de
quermesse e não sei que expressão declamatória, que desprezo. Mais
tarde, em minhas peregrinações pelas casas de espetáculo europeias,
aprendi que virtuosismo e simplicidade são duas coisas idênticas. Foi o
que me disse uma vez Rastelli,}\looseness=-1\footnote{Enrico Rastelli (1896--1931)
  célebre malabarista italiano. {[}\textsc{n.\,t.}{]}}, \textls[-15]{em Hamburgo, onde
por acaso estávamos nos apresentando, ele com suas bolas, tochas e
balões, e eu com Arabela.}\looseness=-1

\textls[10]{Mas voltemos. Na verdade, toda essa passagem deve ser suprimida. Hábito
estúpido de cabotino, incapaz de esquecer o próprio ofício, e que fala
dele sem parar. Voltemos então àquela noite de novembro, quando eu não
passava de um especialista técnico do Ministério da Saúde da Romênia
junto à Comissão Internacional de Cooperação Médica.}

\textls[10]{O número terminara. Os dois cavaleiros de vermelho agradeceram o público
no meio da arena. Entre eles, surgiu um terceiro, também de vermelho,
porém mais jovem e menos robusto, que até então ficara escondido não sei
onde. Em torno deles, o palhaço careca descrevia amplos círculos, uma
espécie de dança grotesca, sem nenhuma graça. Lá no alto, por cima de
todos nós, a mulher da tela de seda fitava ao longe a fumaça de um
cigarro imaginário. Movia levemente os pés, gesto imperceptível de
bailarina em posição de descanso. Desceu mais tarde, quando os aplausos
já haviam terminado por completo, por uma corda, trocando devagar,
preguiçosa, as mãos, uma após outra, avançando para baixo, até o centro
da arena, que atingiu com a ponta dos pés. Em seguida saiu, sem pressa e
sem se curvar, dentre as duas fileiras de criados de \textit{libré}.}

Durante o intervalo, como de costume, fui aos bastidores ver os cavalos.
Estavam sendo preparados para o espetáculo, em baias abertas, numa
cocheira nos fundos. Cheiro de areia, estrume, sangue e perfume ---
cheiro desordenado, complicado, que só ali conheci e que mantenho na
memória com acentuada precisão. Mulheres elegantes passavam de baia em
baia para acariciar uma crina preta, para limpar os olhos sujos de um
potro, para oferecer ao cavalo predileto o açúcar comprado na entrada.

Num canto, trepada na barra de madeira de uma baia, a mulher de roupa
prateada conversava com um cavalo preto, com a mão fraternalmente por
cima do pescoço dele. (Fazia-me lembrar a imagem de um antigo
cartão-postal com felicitações, daqueles que se enviavam no passado, na
época da minha infância, por ocasião do Ano-Novo, foto em que uma
amazona e um focinho de cavalo se olhavam com ternura por debaixo de uma
ferradura de flores). Mais tarde vim a saber que Miss Arabela não se
encontrava ali, naquela noite, de bom grado, mas que era obrigada
contratualmente a passear, durante o intervalo, pelos corredores, em
traje de espetáculo, junto com os palhaços e ilusionistas, para conceder
aos bastidores um ar de atividade flagrante. Isso costuma agradar ao
espectador.

\textls[15]{Aproximei-me dela ao acaso. Creio que ao acaso também, quero dizer, sem
qualquer intenção precisa, ofereci-lhe um cigarro. Aceitou e, em
seguida, ao se lembrar de que era proibido fumar ali, e como ambos
permanecemos com o cigarro apagado na boca, ela me perguntou se eu não
queria ir até seu camarim para fumarmos.}

--- Olha, é aqui perto. Primeira porta à esquerda.

\textls[20]{Acompanhei-a, surpreso com a simplicidade da proposta. Falara comigo
como a um velho conhecido, com um misto de indiferença e amizade que, na
verdade, era o seu jeito natural de ser, mas que naquela altura --- pois
não a conhecia --- me pareceu dirigido a mim.}

\textls[15]{Teve início a segunda parte da programação, mas não era nada
interessante, pelo menos os primeiros números, e me atraía a ideia de
trocar algumas palavras com aquela mulher de roupa prateada, em seu
ambiente de cabotina. Havia algo de cinematográfico naquele encontro.}

\textls[10]{No entanto, na soleira, hesitei. Quando a porta se abriu, notei,
surpreso e desapontado, que não estávamos sozinhos, e que os quatro
parceiros da mulher haviam ocupado antes o camarim, cada um num canto,
trombudos, preocupados com a roupagem complicada e perfeitamente
desinteressados por nós. Entrei embaraçado, sem saber se devia
cumprimentar ou não, sobretudo desconcertado com aquele silêncio hostil.
Trocavam-se carrancudos, mudos, sem pressa, tirando sem pudor as calças
ou a camisa e atirando um ao outro, de vez em quando, por cima das
nossas cabeças, uma toalha, um pente ou uma calçadeira. Só o mais jovem,
o garoto franzino que eu só vira na arena no fim do número, ergueu a
cabeça por cima da bacia em que se lavava nu até a cintura e me fitou
por um instante.}

--- Beb --- chamou-o a mulher.

Em seguida, ela me fez sinal para ir até a sua penteadeira. Acendeu meu
cigarro e depois acendeu o dela com o meu.

\textls[20]{Pôs-se a desabotoar o sutiã, mas desistiu, irritada por não poder fumar
sossegada, ficando com o dorso nu e um seio descoberto pela metade
debaixo da roupa colada, que não permitia que ele deslizasse. Fumava
canhestra, o que me surpreendeu e me fez comentar.}

--- Por que você segura o cigarro entre o polegar e o indicador?

--- Não sei. Parece que me acostumei assim.

Permanecíamos os dois calados. Não sabia o que lhe dizer, e me sentia
contrariado por tê-la acompanhado, pois eu desempenhava um papel
ridículo entre aqueles quatro homens que continuavam ocupados ao meu
redor, como se eu nem existisse.

--- Seu número é muito interessante --- disse finalmente, detendo-me
nessa superficialidade; contente, no entanto, por encontrar algo a
dizer, pondo fim àquele silêncio constrangedor.

\textls[20]{--- Interessante? Não sei. Cansativo. Não é mesmo, Beb? Cansativo.}

Levou as mãos à nuca e se pôs a balançar a cabeça entre os braços
roliços.

\textls[20]{--- E eu não faço nada. Fico olhando lá de cima como eles trabalham. Mas
as luzes, toda aquela aparelhagem mal parafusada, o público que dá
risada\ldots{} Se você soubesse como tudo isso incomoda\ldots{}}

\textls[15]{Havia um tom de enorme exaustão em sua voz, e aquelas poucas palavras
enunciadas, entrecortadas por longos silêncios, com o olhar acompanhando
o fiapo de fumaça do cigarro, me comoveram pela simplicidade.}

\textls[15]{Enquanto isso, os homens haviam terminado de se vestir e, então,
aconteceu algo tão inesperado que, se eu não me sentisse tão inibido,
acho que teria irrompido em gargalhadas. Os quatro se enfileiraram
diante de Arabela e, um por vez, se aproximaram, detendo-se com certa
timidez, à distância de um passo dela. Ela analisou cada um à parte,
atenta, verificando autoritária os trajes e emitindo observações
detalhadas.}

\textls[15]{--- Você, troque de gola amanhã. Já disse que não se usa a mesma gola
dois dias seguidos.}

\textls[15]{--- E você, por que não engraxou as botas? Por que está com o chapéu na
nuca?}

\textls[10]{Eles ouviam os comentários como alunos intimidados, com sorrisos
embaraçados de criança que reconhece o erro e lança um olhar de quem
promete se corrigir. Com os dois primeiros ela terminou rápido (eram os
dois atletas de vermelho, que haviam executado o salto mortal). Ordenou
a um deles que enfiasse o lenço no bolso, pois estava muito para fora da
roupa e, para o outro, apontou uma leve mancha na lapela.}

--- Amanhã eu não quero ver isso.

Abriu uma gaveta e tirou um maço de dinheiro, que distribuiu entre
todos, cada um recebendo em silêncio o que lhe cabia, sem verificar,
saindo em seguida com um cumprimento canhestro.

Mais complicado foi com Dikki, o palhaço careca, pobre homem de rosto
chupado por causa da maquiagem, com cuja roupa Arabela perdeu mais tempo
e que, ao receber o dinheiro, reclamou um suplemento, recusado
incisivamente pela mulher.

\textls[15]{--- Melhor você ir embora. Chegue mais cedo em casa e não beba; está
entendendo? E você, Beb, espere aí. Olha, está faltando um botão no seu
casaco.}

O rapaz ficou ao lado da porta. Arabela coseu o botão enquanto ele a
fitava com um enternecimento que me pareceu cômico pela circunstância e
por serem todos marmanjos.

Tudo isso se deu de maneira tão inesperada que, ao ficar finalmente a
sós com Arabela, não soube mais o que dizer. Embora a cena de família à
que assistira devesse me divertir, senti que me comovera, pela auréola
de ditador daquela jovem mulher, pela sujeição infantil de seus colegas,
pelo respeito atemorizado deles, pela sua condescendência irônica e
imperativa em relação a todos. Mais jovem que eles --- no máximo da
mesma idade ---, ela se comportava como uma irmã mais velha, e seu ar
maternal me pareceu zombeteiro naquele rosto de adolescente exausta.

--- Você não quer voltar para o circo? --- perguntou-me finalmente. ---
Vale a pena ver o número de equitação. É bom.

\textls[15]{Respondi num diapasão completamente diferente, lançando-lhe uma pergunta
repentina.}\looseness=-1

--- Você gosta dessa vida?

--- Que vida?

--- Essa que você leva.

--- Que perguntas você faz\ldots{}

\textls[10]{Desamarrou com cuidado o cadarço dos sapatos brancos de trabalho e se
pôs a preparar meticulosamente as meias, o vestido, o pó de arroz, com
tanta seriedade que aquela operação parecia ser o seu rito mais
importante. Não me pediu para olhar para o outro lado, apagar a luz ou
sair. Trocou-se na minha frente, com uma completa falta de pudor, mas
havia na indiferença de seus movimentos algo inexplicavelmente casto,
algo que paralisava em mim qualquer segunda intenção.}\looseness=-1

--- Se você quiser --- disse-lhe ---, vamos em algum lugar aqui perto
beber alguma coisa juntos e bater papo.

--- Com prazer, mas é uma pena que você perca o espetáculo.

Saímos. Caía uma chuva miúda, chuva noturna, em que tudo parece molhado,
a luz dos lampiões, a luz das vitrines, as paredes das casas subitamente
incendiadas pelos faróis de um automóvel que passa. De longe, o letreiro
luminoso de uma cafeteria piscava convidativo, e me alegrei de antemão
com a lufada de calor e burburinho surdo que haveria de nos receber à
entrada.

\textls[5]{Assim é a noitada nas cafeterias parisienses, em novembro, quando a
meia-noite nos flagra diante de uma garrafa vazia, afogados em vagalhões
de fumaça permeada pelo vozerio das pessoas, pelo ruído dos dados, pelo
som da moeda no zinco, e tudo isso chega distante, impenetrável e
consolador até nós, num rumor que nos assegura de não estarmos sozinhos
naquele fim de outono. Como é boa a companhia da gente desconhecida,
todos são amigos e confidentes na noite que nos une, entre espelhos
embaçados, entre mesas verdes de sinuca, entre grandes vidraças
enfeitadas por gotas da chuva do lado de fora que, deslizantes, desenham
mapas e continentes efêmeros. Tudo é anônimo e familiar nessa cafeteria
de bairro, como num trem, como num salão de navio, e a sensação de nunca
mais podermos reconhecer esses rostos amistosos que nos circundam, não
sei por quê, nos enche de uma vontade de confidenciar os mais ocultos
segredos ao primeiro companheiro de mesa.}

\textls[15]{Naquela noite, escutei tudo o que Arabela tinha para contar, sem me
surpreender e sem fazer perguntas, deixando-a livre para falar sobre o
que quisesse.}

\textls[15]{--- Foi um suíço, diretor de cabaré perto de Montreux, que me fez subir
a primeira vez naquela tela branca que você viu no circo. Disse que não
empregaria os garotos se eu também não me apresentasse. ``Mas ela não
sabe fazer nada'', disse Dikki. ``Não faz mal'', retrucou. ``Basta ficar
ali junto com vocês, visível para o público --- sem mulher não dá
certo.''}

\textls[15]{No final, o número ficou assim e não mudamos mais. Ando com eles, cuido
deles, pois os quatro são uns cabeças de vento, e nos apresentamos
juntos à noite. Nos apresentamos\ldots{} Você viu como. Dikki é um
beberrão, Beb fuma, Jef anda atrás de mulher e Sam não anda atrás de
nada. (São ridículos com esses nomes, mas me acostumei a chamá-los assim
e eles também só me chamam de Arabela). Se não fosse eu para os
controlar\ldots{} Um é meu irmão, um é meu amigo, e o outro nem sei mais
o que é. Habituei-me a eles e de todo modo é melhor assim do que sem
eles. Ou dá na mesma. Só às vezes um tédio me toma e não consigo mais
entender o que é que estou fazendo lá em cima, naquelas cordas, onde não
faço nada além de esperar que o nosso número termine.}

\textls[15]{Em geral não fumo, mas não costumo recusar nada --- pois nem costumo
receber nada ---, de modo que aceitei o seu cigarro. Espero que não se
incomode.}

\textls[10]{Deixei-a falar por muito tempo. Hoje não me lembro mais de tudo o que me
contou. Banalidades, acontecimentos triviais, reflexões, perguntas,
recordações --- tudo narrado de maneira indiferente, com o mesmo tom de
voz, comedida e desprovida de brilho nos olhos, o que demonstrava quão
pouca importância tinham todas aquelas coisas, e eu ouvia tudo aquilo,
por ela, e ela falava, provavelmente por cansaço.}

\textls[20]{Saímos tarde dali, eram quase duas da manhã. As estações de metrô já
haviam fechado fazia muito tempo e não havia nenhum táxi à vista.}

Propus que fôssemos para minha casa.

--- Impossível.

\textls[15]{--- Não, é possível, sim. Só para dormir, não para outra coisa. É mais
simples, e fica mais perto.}

\textls[15]{Refletiu um pouco; via-se com clareza que não era uma questão de pudor,
mas de comodidade. No final das contas, deve ter concluído que de fato
seria mais simples.}

--- Tá bom.

\textls[20]{Subimos até o terceiro andar de um hotel das proximidades, onde eu
estava morando, e, visto que só tinha uma cama, disse-lhe que eu iria
dormir no cômodo ao lado, numa poltrona. E o disse de boa-fé.}

\textls[15]{--- Não --- respondeu ela. --- Você vai dormir na cama também. É grande
o bastante para nós dois e não é impossível que nos entendamos bem.}

\textls[10]{Aceitei porque, para mim, dava na mesma, e não seria a primeira vez que
eu dormiria com uma mulher apenas como amigo, por camaradagem, pois não
raro, depois de uma festa, calhava de me acompanharem até em cima vários
amigos e amigas, e acabávamos dormindo como dava.}

\textls[10]{Quando apaguei a luz, e depois de reconhecer o calor do travesseiro, a
respiração da mulher ao meu lado, fraca e contendo não sei que íntima
tristeza, me pareceu tão antiga e familiar e, ao rumor surdo da chuva,
que ainda se ouvia da rua, a pulsação de seu corpo era tão próxima, que
peguei as suas mãos e as enrolei no meu pescoço, feliz por tê-la ao meu
lado.}

\textls[15]{Ela se entregou fácil, sem censurar a minha arremetida, mas também sem
se entusiasmar, submissa e absurdamente tranquila. Tinha um gosto de
miolo de pão, e essa sensação, que ainda trago dentro de mim, é a única
certeza que me ficou de Arabela, hoje, depois de tantos anos de vida em
comum e outros tantos desde a nossa separação.}

\textls[15]{Ao se soltar dos meus braços, ela se virou para o lado da janela e
adormeceu na hora, profundamente.}

Disse-me apenas que estava cansada.

\section{ii}

\letra{J}{amais} \textls[20]{tentei explicar a ninguém como é que Arabela ficou comigo e como
é que fui aceitar --- eu, tão obstinado no meu anseio por liberdade ---
uma tal complicação repleta de tantas consequências.}

\textls[20]{Tudo foi tão simples e espontâneo que tenho certeza de que qualquer
explicação seria falsa.}

Arabela ficou por preguiça. Assim como veio.

\textls[15]{--- Que tal --- disse-lhe no dia seguinte --- se você ficar aqui e
abandonar a turnê?}

--- Sei lá. Vamos tentar.

À noite, encontrei-a no meu quarto, instalada com simplicidade. Trouxera
uma malinha e alguns objetos de toalete: sua bagagem toda. Perguntei
como havia se resolvido com os colegas.

--- Foram embora.

--- Foi difícil?

\textls[15]{--- Não. Dei-lhes as contas e foram embora. De qualquer modo, eu não
fazia nada.}

\textls[10]{Minha situação devia ser embaraçosa diante daquela mulher, que conhecera
fazia apenas um dia e que, por causa de umas palavras ditas ao acaso,
abandonou uma existência e, de certo modo, uma carreira, para se
instalar na vida de um homem desconhecido, com quem nada tinha em comum.
Mas como explicar a sensação de tranquilidade que tive desde aquele
primeiro momento, o ar familiar que Arabela trazia para o meu quarto, o
timbre de recordações comuns que eu pressentia nos seus passos pela
casa?}\looseness=-1

\textls[15]{Tinha o talento de abrir exatamente a devida gaveta, de encontrar as
coisas em seus devidos lugares, de acender a luz sem me perguntar onde
está o interruptor, de recolocar um livro na prateleira certa.
Encontrava tudo, sozinha, por instinto. Por vocação, provavelmente.}

Saímos para comer juntos, em seguida fomos ao cinematógrafo do bairro e
voltamos tarde da noite, sem pressa, pelo menos da minha parte, pois,
embora gostasse do calor de seu braço e imaginasse com prazer que,
dentro em pouco, a teria toda nua do meu lado, na cama, aquela sensação
me parecia familiar, tudo permeado por um gosto de amor antigo e paixão
tranquila, como se entre mim e ela houvesse longos anos de entendimento
físico.

\textls[15]{Fui inquieto e rabugento a vida toda, rebelde sempre que uma mulher
tentasse me prender, exclusivamente preocupado com minha liberdade,
solteiro predestinado, e até então não compreendera como seria possível
viver a dois, a simples ideia de reencontrar toda noite o mesmo corpo,
com os mesmos frêmitos, parecendo absurda para mim, que desejei o tempo
todo surpresas e acordes passageiros.}

\textls[15]{Que milagre fizera com que Arabela derrotasse, desde o primeiro
instante, minha vocação de vagabundo no amor e conseguisse me segurar?
Poderia arriscar uma explicação caso me esforçasse. Mas por que convocar
a psicologia para esclarecer algo tão natural e que eu aceitara
voluntariamente? Não, não. Arabela daria risadas se lesse.}

Tudo o que lembro é que cheirava bem. Tinha um aroma desbotado de
perfume que, aquecido pelo sangue, dava-lhe um gosto todo pessoal, uma
nuance de cheiro de animal, evocativo da noite. Era incrível como a água
de colônia, química e inexpressiva, se transformava nas suas rendas num
aroma tão envolvente e rarefeito, como se fizesse parte de sua
respiração.

--- \emph{Que tu sens bon}\footnote{Em tradução livre, ``Que você cheira bem''. \textsc{{[}n.\,e.{]}}} --- dizia-lhe com sinceridade, quando queria
lhe dizer o quanto a amava e teria dificuldade em traduzir isso para o
romeno. Creio que soaria ridículo.\label{traduzir}

À noite, quando voltava do trabalho, de longe eu já estremecia à
lembrança daquele aroma que me preenchia a boca e as narinas, como um
perfume quente de castanhas cozidas, e subia as escadas apressado como
um adolescente até chegar em casa e poder apertá-la nos meus braços e
grudar nossas faces uma à outra.

Não tive muitas mulheres na minha vida. Suficientes, de qualquer modo.
Tantas quantas um homem comumente feio pode ter quando é gentil e quando
sabe, às vezes, insistir. Não me gabo, pois sei que um amigo meu, mais
alto do que eu, mais moreno e com um rosto mais aprazível, teve dez
vezes mais ``aventuras''. De todo modo, jamais encontrei uma mulher ---
e algumas delas até mesmo amei ---, jamais, que me desse aquela sensação
de volúpia calma que eu encontrava nos braços de Arabela, sorvendo o seu
cheiro de carne jovem, distendida na preguiça e indiferença.\label{indiferença}

\textls[10]{Pois Arabela não era uma mulher passional. Quando teve início o que
chamaram de ``meu declínio'', ou seja, quando o ministério na Romênia me
comunicou que eu havia sido demitido, sei que meus amigos mais próximos
lamentaram minha sorte, preocupados, anunciando por toda parte que eu
havia sido vítima de uma mulher fatal. Eu dava risada enquanto olhava
para aquela minha mulher fatal, com seus vestidos bem-comportados sempre
de uma cor áspera e quente, andando para lá e para cá pelo aposento como
uma genuína dona de casa, arrumando as coisas ou me trazendo um livro
que eu extraviara. Havia nela algo de tão conjugal e maternal (aquele
seu ar sério da noite em que a conheci, o cenho franzido, que ela ora
ostentava para cuidar de mim, do mesmo modo como cuidou de Beb ou
Dikki), de modo que a ideia de que alguém pudesse tomar Arabela por uma
heroína sombria de romance me fazia gargalhar como uma criança.}\looseness=-1

--- O que foi, Ştefan? Por que está rindo?

--- Nada, moça bonita. Gosto de te olhar.

\textls[15]{--- Você não é sério. Nada sério.}

\textls[-10]{Não. Não era sério. No dia em que terminassem os trabalhos preliminares
da comissão internacional, à qual fora enviado como especialista médico
do Ministério da Saúde Pública, eu deveria retornar à Romênia e
apresentar meu relatório, o correto seria eu retornar. Inclusive foi o
que Arabela me aconselhou a fazer.}

\textls[-5]{Não fui capaz. Não que eu tivesse me sentido um desgraçado, não que sem
ela eu não pudesse mais viver, não que a despedida fosse imensamente
difícil. Nada disso. No entanto, não fui embora. Largá-la, largar a sua
proximidade, seu corpo roliço, um pouco roliço demais, para dizer a
verdade, mas tão quente e tão familiar, esquecer aqueles dois braços
tranquilos, que eu desenhava com os lábios noite após noite, dos ombros
até o pulso --- isso realmente me parecia um esforço demasiado complexo.
Por outro lado, levá-la comigo eu não podia, por diversas razões (entre
as quais estava também certa covardia, pois teria dificuldade em
aparecer com Arabela em Bucareste, onde me aguardavam inúmeros amigos
honrados e em especial uma mulher --- Maria --- que eu amara inutilmente
no passado, mas que eu apreciava e ainda aprecio, justamente por nunca
ter logrado ultrapassar, com ela, uma afeição cuidadosamente controlada
por ambas as partes).}

Ir embora dali --- que chateação! Simplesmente fiquei, sem nada de
heroico na decisão, assim como no passado, na época do liceu, me
aconteceu algumas vezes, em certas manhãs de inverno, despertar bem
cedo, olhar sobressaltado para o relógio e, em seguida, após uma breve
hesitação, enfiar a cabeça debaixo da coberta, decidindo feliz: hoje não
vou para a escola.

\textls[-10]{Portanto, decidi, na manhã daquele dia de janeiro, não ir de novo para a
escola: deixei a valise diplomática para Bucareste esperando e, enquanto
fitava pelas janelas do nosso quarto trinta centímetros de céu
parisiense nublado, disse a Arabela que ficaria.}

As sanções chegaram rápido. Em primeiro lugar, uma advertência por
negligência grave no trabalho. Em seguida, a demissão. Se não fui
processado nem punido mais seriamente, foi porque, naquele meio-tempo,
Andrei Giorgian, amigo meu, que fora no passado deputado por Turda, mais
por diletantismo, se tornara um importante subsecretário de Estado no
Ministério dos Negócios Estrangeiros, e parece ter intercedido a meu
favor. Ademais, Andrei me mandou uma carta pessoal (lembro-me de ter
achado muito engraçado o fato de ter sido datilografada, e ainda por
cima escrita em papel timbrado do ministério), chamando seriamente minha
atenção ao meu gesto de leviandade.

No \textit{postscriptum}, anunciou-me estar se casando oficialmente com Maria
(com quem, aliás, vivera muito tempo junto, a mesma Maria a quem, alguns
anos atrás, durante um baile, fiz confidências inábeis, das quais até
hoje me arrependo). Não entendi, porém, aquele \textit{postscriptum}, pois o
casamento havia sido anunciado nos jornais e eu o parabenizara por
escrito.

\textls[-10]{Recebi com indiferença todas as censuras e pedidos insistentes de
retorno à ordem. Não por querer provocar ou afrontar alguém. Não. Mas
porque não tinha o que dizer e, sobretudo, não tinha como falar de
Arabela, da felicidade comedida que eu encontrara em nosso amor, da
volúpia lenta, ordenada e límpida de nossas noites na rue Tholozé. Era
engraçado notar que, aos olhos dos outros, eu era vítima daquela moça
morena que, nos seus instantes de paixão mais acentuada, sorria do mesmo
jeito que sorria no passado, lá de cima, na tela de seda, exausta e
ausente.}

\section{iii}

\letra{A}{s} \textls[15]{coisas correram muito bem enquanto eu tinha dinheiro. Por uns quatro
meses. Ainda contava, numa conta bancária dos bons tempos, com quase
trinta mil francos. O bastante para não me preocupar no início. Por
enquanto tinha com que viver --- e, além disso, mesmo se fosse um
verdadeiro boêmio, nada mais me interessava.}

Dos primeiros meses do meu amor por Arabela praticamente não guardo
recordações. Vejo uma Arabela bem-comportada, monótona e caseira,
deixando-me ir durante o dia aonde eu quisesse, esperando-me em casa,
dócil, aninhando-se ao meu braço quando saíamos juntos, estendendo-se de
noite na cama como um gato branco, ronronando quando lá fora fazia frio
e do lado de dentro fazia calor, e eu me aproximava dela para a beijar.

\textls[15]{Esquecera-me de que tinha uma profissão, esquecera-me de que deveria
trabalhar e voltei a ser, de certo modo, o estudante destrambelhado de
outrora, que abandonava as aulas de anatomia do primeiro ano para ir aos
concertos da orquestra Colonne. Dessa vez encontrei outra ocupação: de
manhã ia passear no Louvre e, à tarde, na Biblioteca Nacional, sempre na
mesma poltrona --- 118 --- e debaixo da mesma lâmpada com globo verde,
lia \emph{Vidas dos artistas}, de Vasari.}

\textls[15]{Era grato a Arabela por me ter tirado, involuntariamente, do caminho de
um destino razoável e de ter feito daquele senhor sério que conheceu
numa noite de novembro um indivíduo que esquece ser médico especialista
diplomata, para voltar a ser o que sempre quisera: um jovem.}

No verão, viajamos para Talloires, onde nos hospedamos numa pousada
muito barata, mas de ambiente refinado (um deleite para o gosto burguês
de Arabela), e lá desempenhamos, despretensiosamente, o papel de
``jovens recém-casados felizes'', na companhia de gente decente e
fofoqueira. Arabela cintilava de orgulho em meio às amigas da pousada,
todas elas esposas criteriosas, e como lhe caía bem dessa vez a aura de
``mulher casada'', ela que durante tantos anos perambulara por um
universo duvidoso e agitado. Sentia-me realmente contente por ter
concedido àquela mulher a única volúpia para a qual provavelmente fora
predestinada: a ilusão do amor legítimo. E me alegrava ao ver como
Arabela aos poucos perdia a sombra de dúvida --- ou talvez de pânico ---
que algumas vezes cobriu, no passado, o seu sorriso.

\textls[20]{Aqueles meses idílicos terminaram subitamente com nosso retorno a Paris,
quando, ao calcular o dinheiro que ainda restava, descobri que só
restavam seis mil francos.}

\textls[20]{--- Preciso fazer alguma coisa --- disse a ela, dando de ombros,
embaraçado.}

--- Precisamos fazer alguma coisa --- corrigiu-me.

\textls[15]{À noite, ao chegar em casa, encontrei-a bem-disposta, não muito, é
verdade (pois as coisas estavam realmente difíceis), mas, de qualquer
modo, corajosa e sobretudo decidida quanto ao que deveríamos fazer.}

\textls[-10]{--- Ouça, Ştefan, seis mil francos é muito dinheiro. Você não tem como
saber disso. Serão o bastante por pelo menos cinco meses. Acabei de
dizer ao porteiro que, a partir do dia 15, o apartamento estará
disponível. Vamos nos mudar para outro bairro, de preferência na margem
esquerda, na direção de Porte d'Orléans ou Porte de Versailles. Naquela
zona há quartos baratos e, com algo entre cento e cinquenta e duzentos
francos, com certeza encontraremos um bom, mesmo se for num andar mais
alto. Jantaremos em casa, cuido eu disso. No almoço, podemos comer bem
num pequeno restaurante, você vai ver. Passaremos mais tempo em casa e,
quando sairmos\ldots{} pois então, quando sairmos, parece que existe um
bilhete de segunda classe no metrô, não é?}

--- \textit{Huh}, por quarenta e cinco centavos.

\textls[15]{--- Não fale assim: você não sabe o que significam quarenta e cinco
centavos. Ah\ldots{} quase esquecia: tente fumar Gauloises. São mais
baratos e têm gosto melhor.}

--- Ou seja, pobreza\ldots{} --- concluí, com um suspiro propositadamente
exagerado para esconder minha real preocupação.

--- Pobreza, não. Certeza. Por cinco meses, apenas, mas certeza de todo
modo. Depois\ldots{} então, depois, veremos.

\textls[20]{Arabela voltara a ser aquela mulher atenta e controladora que eu
conhecera na primeira noite, no camarim do Medrano, passando em revista
os colegas e distribuindo ordens, que acatavam submissos.}

\textls[15]{Em dez dias, havíamos nos mudado. Arabela cuidara de tudo ---
negociações, pagamentos, discussões, chateações --- enquanto eu, depois
de alguns dias em pânico (``o que fazer?\ldots{} o que fazer?''),
retomei meus passeios metódicos pelos bairros parisienses, com raras
visitas a galerias de pintura ou livrarias, retornando ao entardecer
para casa, exausto, porém com uma calma indescritível por saber que
alguém tratava por mim das ``dificuldades da vida'' --- assim como dizia
Arabela, quando conversávamos a sério entre nós.}

\textls[-10]{Nossa nova residência era um quartinho no sexto andar --- último --- num
conjunto de casas escuras, com paredes revestidas pessimamente, num
pátio imenso, com plantas malcuidadas e inúmeras crianças. Próximo à
Porte de Versailles. Não me lembro de um único dia em que eu não tivesse
visto roupas estendidas nas janelas, para secar debaixo de um sol
presumido, uma vez que jamais era visível, devido aos beirais unidos dos
telhados. Todo dia, na mesma hora, ouvia-se um fonógrafo tocando discos
antigos --- uma voz rouca, teimosa, que se ocultava não sei em que
andar, pois nunca consegui localizar. Para chegar em casa, tínhamos
que subir uma escadaria complicada, e eu parava umas cinco vezes, diante
de portas diferentes, sobre as quais podia ler os mesmos cartões de
visita boêmios: Alexandre Merenski, pintor artístico; Theodor van Haas,
tenor; Marcel Charde, pintor paisagista. Só pintores, poetas e cantores
--- gente duvidosa, que atraía pobreza, e em meio à qual eu me sentia
estrangeiro, eu, que jamais tivera nada em comum com a arte e que, além
de um apreço muito relativo por livros, não reconhecia em mim nenhuma
vocação.}

\textls[-10]{No entanto\ldots{} Esquecia o bairro e os vizinhos, as paredes
multicoloridas da casa, o pátio adornado com ceroulas úmidas esvoaçando
ao vento nas janelas, a escadaria coberta de bolor e cartões de visita
artísticos, esquecia tudo isso e tudo mais ao abrir a porta e entrar no
nosso quarto do sexto andar, quarto que o gênio doméstico de Arabela
transformara num ambiente de ordem extrema e leve graça, desde a cor da
madeira passada na plaina da mobília até as flores frias de inverno que
nunca faltaram na nossa mesa, nem mesmo nos dias em que não havia
dinheiro para o pão! Deveria dar risada à memória da beleza demasiado
honesta daquele quarto, presidido pelo avental branco de dona de casa
que Arabela usava com vago orgulho --- mas, se não dou risada, é porque
sempre fui uma criatura desprovida de bom gosto e, em segundo lugar,
porque ali, naquele quarto, ficou algo de que não se ri.}

Rapidamente fizemos conhecidos no bairro. Arabela se concentrou em
especial nos nossos fornecedores, prevendo tempos mais difíceis.
Cultivou-os com assiduidade e gentileza. Dava-se muito bem com a
vendedora da leiteria Maggi da esquina, cumprimentava cordialmente o
açougueiro, perguntava das crianças do padeiro, que estavam sempre
doentes e a quem Arabela propunha diversos tratamentos. Graças a tais
relações tão úteis, obtivemos certo prestígio entre os vizinhos e
lembro-me de que Arabela se sentiu quase orgulhosa, certo dia, ao ficar
sabendo pelo padeiro, quem lhe segredara uma semana antes, que o preço
do pão diminuiria de 2,40 para 2,35. Não lembro se eu também não me
envaideci com o fato, que, de qualquer modo, fazia de nós, em certa
medida, uma espécie de sumidade no bairro.

Ao anoitecer, quando passávamos vindo da praça Vaugirard rumo à nossa
casa, em geral de braços dados, pois nos amávamos e fazia frio, nossos
amigos nos lançavam sorrisos desde os seus negócios e não raro ouvíamos
em derredor sussurros cúmplices: \emph{Voilà le jeune ménage du
sixième qui passe}.\footnote{Em tradução livre, ``Aqui está o jovem casal do sexto que está passando''. \textsc{{[}n.\,e.{]}}} O jovem casal do sexto andar éramos nós, e esse
apelido, se a mim não incomodava, era uma verdadeira felicidade para
Arabela, como se confirmasse tudo o que havia de respeitável na nossa
situação.

Parávamos no meio do caminho para negociar o jantar e sempre me admirava
com a capacidade deles de criar, a cada vez, com pouco dinheiro e
ingredientes não muito variados, uma refeição nova e uma pequena
surpresa. Tentei várias vezes organizar um almoço semelhante e jamais
consegui, o que --- por mais que soe infantil --- me deixa desconsolado.
Por vezes me ocorre pensar em Paris, e sonho em retornar, rever-me de
volta àquelas ruas que amo tanto que a lembrança delas me emociona como
a lembrança de gente, mas preciso confessar a seguinte besteira: a
primeira coisa que eu adoraria fazer seria entrar numa salsicharia e
pedir um pedaço de \emph{céleri rémoulade} a 1 franco e 25 centavos.
Aqui, na Romênia, expliquei a todos os meus senhorios essa receita, mas,
apesar dos experimentos empreendidos, o salsão com molho de mostarda que
me era servido está muito longe daquele \emph{céleri rémoulade}
picante, perfumado e revigorante que fazia a alegria das nossas
refeições naquele quartinho de sexto andar na rue D'Alésia, sob o olhar
de Arabela, que compreendia todas as glutonarias, mais ainda quando eram
tão inocentes e baratas.

Ela, que até onde sei não nutria nenhum tipo de vaidade, reagindo aos
elogios que recebia no circo com um erguer de ombros cansado e, mais
tarde, ao atingir a glória, respondendo com não sei que sorriso perplexo
às críticas entusiastas, pois então, ela, tão indiferente e simples,
corava de orgulho, como uma criança, quando eu lhe dizia que gostava da
comida ou que o molho era bem-feito. Naquelas noites, seu amor se
redobrava e, ao nos deitarmos, sentia-a mais emocionada e mais grata que
de costume, se comparado a seu temperamento de moça jovem, saudável e
comedida. Eram suas volúpias íntimas de dona de casa, pobre casa que mal
se mantinha de pé com o resto de um dinheiro prestes a acabar, apesar do
mais severo controle. Deixara todas as contas nas mãos de Arabela e só
tinha no bolso uns tostões que ela me dava para comprar cigarro, pois,
mesmo quando calhava de eu arranjar algum dinheiro --- um empréstimo, um
livro vendido, um relógio penhorado ---, eu passava tudo para ela, que,
com sangue frio, lutava contra a pobreza, as dívidas e os fornecedores,
saindo vitoriosa com dificuldade cada vez maior.

\textls[-5]{Nada pesava mais sobre nossos ombros do que o aluguel não pago e, se
hoje me lembro com certa nostalgia dos dias de fome, das caminhadas
forçadas a pé por não podermos comprar passagens de ônibus, e das roupas
puídas, quando penso naqueles cento e setenta e cinco francos mensais
que não tínhamos para pagar ao senhorio, estremeço de aflição e lamento
não conseguir afastar definitivamente da memória esse capítulo. As
manhãs de inverno, quando tínhamos de descer as escadas na ponta dos pés
para que ninguém nos flagrasse! As altas horas da noite, quando
rodeávamos o edifício várias vezes, até passar da meia-noite, esperando
que a luz da cabine do porteiro se apagasse para podermos subir,
segurando a respiração e colados à parede, seis andares --- um, mais um,
mais um ---, estremecendo diante da ideia de que, a qualquer momento,
alguém pudesse nos chamar e desviar do nosso caminho rumo à salvação. A
salvação era a porta do topo da escadaria, porta que batíamos atrás de
nós com um suspiro de alívio, trancando-a com duas voltas da chave e
apoiando-nos nela como no portão de uma cidadela arduamente conquistada.}

Tinha início então uma longa noite, cheia de paz e olvido, em que
mergulhávamos com a esperança de que não mais acabaria, num lento enlace
que passava gradualmente do beijo ao orgasmo, até o sono chegar e cobrir
o nosso cansaço de pobreza e amor.

A cabeça de Arabela pendia pesada sobre o meu ombro. Gostava de olhar,
no escuro, os reflexos de carvão do seu cabelo.

\section{iv}

\letra{E}{stávamos} em pleno inverno quando fomos obrigados a pensar seriamente
numa solução --- eu, aflito e ineficaz, Arabela, calma e prática.

--- É possível que eu chegue mais tarde hoje à noite --- disse-me certo
dia. --- Espere por mim na praça, do outro lado do correio, entre sete e
oito. Vamos ver\ldots{}

\textls[15]{Retornou à noite com algo mísero e incerto: um trabalho de acrobata,
numa espécie de teatro-cabaré, na periferia. Vinte e dois francos por
noite.}

--- Muito bem, Arabela, você quer retomar isso?

\textls[15]{--- Não quero retomar nada. E muito menos ``isso''. Quero que paguemos o
aluguel.}

\textls[15]{Voltava a ``isso'' sem desgosto ou revolta, com a simples consciência de
que devia trabalhar e ganhar dinheiro. Para ela, não havia problemas ou
hesitações.}

--- Preciso, você quer o quê?

\textls[15]{Com base no mesmo raciocínio simples, um mês depois ela haveria de me
pedir que a acompanhasse no teatro para fazermos um dueto.}

\textls[15]{Coisas perfeitamente absurdas, ditas num tom simples como o de Arabela,
se tornam tão naturais que, apesar da incoerência, aceitamos sem
discussão.}

\textls[15]{--- Sabe, a partir da semana que vem vai haver alteração no programa.
Vou acabar com a acrobacia e começo com a dança. Uma dança apache.
Precisamos de um acordeonista e parece que você sabia tocar piano.}

\textls[15]{Concordei. Se eu fosse explicar para alguém que me conhece, só
conseguiria dizer besteiras e a pessoa com certeza não entenderia nada.
Concordei.}

Oh, noites de Montrouge, naquele cabaré --- meio teatro, meio salão de
baile --- em que eu tocava acordeão em tons difusos, acompanhando a
dança de Arabela, cômica pela falta de desenvoltura (pois jamais fora
dançarina), mas elegante pelos movimentos instintivos do corpo, que
sabia obedecer bem a uma melodia! Do lado de fora ficava a outra vida,
da qual partira como um alucinado, mas à qual podia retornar a qualquer
momento, ao preço de um pequeno esforço --- mas a sensação de fuga
voluntária me fazia amar dez vezes mais o destino que encontrara entre
aquelas mesas de fregueses entusiasmados e copos compridos com hortelã
verde. Definitivamente, gostava da minha nova carreira e, por vezes,
quando fazia sucesso e me pediam para repetir uma música, acompanhado
pelas vozes de todos os bêbados presentes, atravessava-me um leve
calafrio de orgulho, que em mim provavelmente anunciava o surgimento do
cabotino.

\textls[-10]{Dali fomos nos apresentar em outros lugares, nos cinematógrafos de
bairro ou em bailes \emph{musette}, e modificávamos algumas vezes o
``gênero'' conforme as exigências do programa: hoje como locutor de
Arabela num número de acrobacia leve, amanhã a acompanhando ao piano
numa dança ou, caso necessário, realizando alguns passos junto com ela,
quando a dança exigia parceiro.}

Coincidíramos com uma temporada admirável aquele ano, em que
introduziam, nas avenidas, as primeiras instalações de filme sonoro,
enquanto os cinematógrafos ``mudos'' de periferia, assustados com a
concorrência da nova invenção, tentavam reter o público com ``atrações
excepcionais'' (assim como alardeavam os cartazes). Isso provocou uma
enorme afluência de ``homens-cobra'', ``mulheres-sereia'', cantores de
música popular, acrobatas e imitadores, entre os quais conseguimos
garantir o nosso lugar, pois a demanda era grande e os programas se
modificavam com frequência, o que nos obrigava a realizar verdadeiras
turnês pelas extremidades de Paris, de bairro em bairro, sem jamais
ultrapassar um círculo interior que, ao sul, era delimitado por
Denfert-Rocherau e, ao norte, por Batignolles. Nunca imaginei que um dia
atravessaríamos a fronteira da periferia para chegar do outro lado, ao
centro de Paris, onde os luminosos e os cartazes coloridos cintilavam
remotos, inacessíveis.

No entanto, não demorou muito até eu conhecer um jovem poeta e
pederasta, bem relacionado no mundo da noite. Ao ver Arabela dançar,
esse rapaz ardeu de entusiasmo pela arte dela, prometendo lançá-la. Sem
poder levá-lo a sério, eu, que em toda aquela história mantive bastante
senso crítico, bem sabendo que a dança de minha amada não valia grande
coisa, recebi aquelas promessas de glória com muita moderação. Não posso
deixar de mencionar, no entanto, que, graças a ele, passamos a fazer
parte da programação do Bobino, cabaré de fama popular em Montparnasse,
no que dancei com Arabela, por duas semanas, em plena rue de la Gaîté,
sem que nenhum cronista nos notasse, mas ganhando fabulosas centenas de
francos.

\textls[15]{Acabamos nos mudando da mansarda do sexto andar, sem, porém, deixarmos o
bairro, que de certo modo se tornara indispensável: para Arabela, pelas
relações honrosas que ali fizera, e para mim, pelas cores e pela
harmonia surda dos ruídos de periferia, com que acabara travando uma
amizade tão íntima que não podia ler ou pensar melhor do que em meio
àquela avalanche de sons --- uma loja que se abre ao rangido dentado da
porta de ferro, um chiado de fábrica, um canto perdido de sanfona, um
insulto arrepiante se erguendo da rua, em que dois motoristas se
confrontam.}

\textls[10]{Place de la Convention! Às vezes, de noite, quando tenho dificuldade
para pegar no sono, imagino-me passeando a esmo, com as mãos no bolso e
o casaco desabotoado, desde o centro daquela praça até a Porte de
Versailles, e avanço lentamente, primeiro pela calçada da direita,
depois pela da esquerda, detendo-me atento diante de cada loja para
rever o letreiro, a vitrine repleta de mercadoria barata e suntuosa, as
vidraças embaçadas\ldots{} Pairava um cheiro de batata cozida, marisco,
carne recém-cortada, banana\ldots{} Havia ao meu redor um desvario de
mercadorias atiradas ao acaso, chita, peixes, laranjas, botões e
suspensórios, manteiga, ovos, picles. E as vozes que se chocavam no ar,
e as luzes que se cumprimentavam de uma calçada a outra\ldots{} Revejo
os rostos conhecidos das pessoas de lá, a velha sorridente da soleira do
hotel Messidor, a jovem vendedora de alcachofra da esquina da rua
Blomet, o comerciante de frascos do outro lado da rua, que se parecia
com Napoleão \textsc{iii} e que adquiriu, devido ao fato de o ônibus \textsc{x}
parar bem em frente à sua loja, maneiras de chefe de estação\ldots{}}\looseness=-1

\textls[15]{E vejo Arabela descendo, vindo da direção da rue de la Croix Nivert,
onde morávamos, envergando seu impermeável curto, esgueirando-se
apressada entre os transeuntes, detendo-se, no entanto, longamente à
porta de uma loja para comprar uma quinquilharia, fitando com
perplexidade infantil as vitrines maiores e calculando quanto dinheiro
lhe restava no bolso, negociando em seguida uma garrafa de vinho da
Primistère, onde distribuíam cupons de prêmio que ela colecionava
circunspecta, na esperança de um dia ganhar o serviço de jantar ``para
doze pessoas'' fotografado na prateleira.}

Isso não só durante o tempo de vacas magras, quando lutávamos com o
pagamento do aluguel e as últimas passagens de ônibus, mas também mais
tarde, depois de termos ganhado algum dinheiro e feito de nosso novo
apartamento na Croix Nivert uma residência agradável. Arabela recusava
com teimosia se conformar à nossa situação de artistas que outrora se
apresentaram no Bobino e, embora por causa disso tenhamos conhecido uma
série de ``gente boa'' que nos frequentava, ela queria continuar sendo
dona de casa, exagerando por vezes com ostentação as suas preocupações
domésticas.

\textls[5]{Muito difícil definir o tipo de nossos novos amigos. Vagos pintores,
vagos poetas, vagos críticos --- todos jovens e desabusados, fisgados
nos cafés de Montparnasse a altas horas da madrugada, uns homossexuais,
outros apenas esnobes (de um esnobismo da devassidão, que exibiam
violentamente), outros, enfim, os menos numerosos, rapazes confiáveis,
mas preguiçosos e por enquanto desocupados. Não conseguia compreender
muito bem a situação deles, embora todos pintassem, escrevessem ou
fizessem teatro, alguns deles tendo me contado em detalhe sobre suas
relações literárias (mostrando-me, se necessário, uma carta assinada por
Cocteau) ou sucessos de um passado remoto --- juntando como prova um
cartaz velho ou algumas linhas elogiosas publicadas anos antes no
\emph{Les nouvelles littéraires}. Interessavam-me pelo seu caráter
pitoresco, pela agitação que conferiam aos nossos dois cômodos, onde,
aliás, só reinava a disposição de Arabela, constante e comedida como uma
chama por debaixo da brasa. Se eram talentosos ou chamados para realizar
algo artístico --- não sei. Talvez sim. Não entendo disso, mas não me
surpreenderia caso alguns deles tivessem chegado longe nesse meio-tempo
e, se eu lesse jornais estrangeiros aqui onde me encontro, talvez
soubesse de boas notícias suas.}

\textls[5]{Um gosto ingênuo pela fronda os havia trazido até nós, pois certamente
era revolucionário desprezar, como eles, os espetáculos da Grande Ópera
para admirar, por outro lado, uma dançarina de baile popular da
periferia parisiense. Um deles, inclusive, escreveu num manifesto
vanguardista --- um daqueles manifestos incendiários, que são lidos por
dezessete pessoas dispostas a virar o mundo de ponta-cabeça --- que
Wagner deveria ser queimado vivo e Bruno Walter (que, naquela época,
dirigia \emph{Os mestres cantores de Nuremberg} na Ópera) expulso, para
serem substituídos por minha amada Arabela. A mesma Arabela que, aliás,
não suportava nenhum deles, pois só apreciava gente sóbria e ponderada,
detestando radicalmente tudo o que fosse aventureiro, boêmio e
``artístico''. Essa moça, que vinha do universo circense, depois de
passar uma infância duvidosa e perambular por diversas localidades, essa
moça, mais fácil que virtuosa, que se deixou seduzir, pelo menos no meu
caso, já na primeira noite, sem que eu insistisse, tinha uma aversão
burguesa por tudo o que não fosse legítimo e imaculado. Sofria na
companhia dos pederastas e das lésbicas, que abundavam nos bastidores
dos bares em que nos apresentávamos, e se horrorizava com os costumes de
nossos amigos mais recentes, comportando-se com uma crispação casta e
rancorosa.}

Certa noite, depois de um longo jantar de mariscos regado a vinho
branco, no ateliê de um jovem pintor, onde casualmente nos encontramos
com um grupo de moças estabanadas e rapazes embriagados, fomos levados à
força, no automóvel de um amigo, ao Bois de Boulogne para assistir a uma
suruba. Arabela e eu só conhecíamos por alto e vagamente o significado
do termo, sabendo tratar-se de certos ritos sexuais que se realizavam em
grupo, no Bois, nas alamedas mais escuras.

\textls[5]{A experiência teve início misteriosamente, com trocas de sinais
luminosos, de um automóvel para outro, acendendo e apagando rapidamente
os faróis. Explicaram-nos que aquele era o sinal de reconhecimento entre
os iniciados. Uma longa fila de limusines avançou, com os faróis
apagados, para a parte central do bosque e, às vezes, da fila se
separavam dois ou três automóveis que, comunicando-se pelos mesmos
sinais de rigor, viravam para a direita ou para a esquerda, aos pares.
Parece que a primeira condição para uma suruba exitosa era a de que os
parceiros não se conhecessem: os homens trocavam de lugar, passando de
um carro para outro, trocando, assim, também de mulher. No escuro, com
luzes apagadas, cortinas fechadas, sem qualquer tipo de apresentação,
sem se verem, quase sem se falarem, eles se amavam, atiçados por aquele
mistério barato da escuridão e do desconhecido.}

Até então, para mim, toda aquela história não passara de uma lenda.
Agora, porém, desfilavam diante dos meus olhos aqueles veículos
misteriosos, acompanhava o piscar de seus faróis, via como se
esgueiravam na noite as sombras dos que chegavam num acordo. O espetáculo
era de uma força tal que despertava toda a minha curiosidade, anulando,
confesso, qualquer reserva moral. Não, não era repugnante. Era apenas
apaixonante. Seria necessária uma falta de imaginação, como a de
Arabela, uma honestidade primária, para alguém se revoltar ali, em nome
da decência. \textls[5]{Ela se dependurou no meu braço e gritava a plenos pulmões
que se recusava a assistir àquela sacanagem (sim, ela disse \textit{sacanagem},
para a minha vergonha e para o embaraço das pessoas requintadas que nos
acompanhavam). No entanto, naquele instante, nossos amigos notaram uma
limusine azul e passaram a persegui-la em grande velocidade para se
afastarem daquela alameda frequentada demais, em busca de um canto mais
propício para as primeiras tratativas. Era uma corrida da qual
participava com emoção sincera, surpreso com a simplicidade da aventura
e aguardando quase sem fôlego o seu desenlace --- com exceção de
Arabela, que não parava de se debater nos meus braços, gritando que
queria voltar para casa e ameaçando quebrar os vidros se não parássemos.}

\textls[15]{--- Vocês são uns safados, estão entendendo? Uns safados. E você também,
você também junto com eles! Por que não param? Quero que parem! Seus
bandidos. Vou dar queixa na polícia. Me dá um lápis para eu anotar a
placa deles. Me dá um lápis, está me ouvindo?}

Abriu a bolsa, tirou de dentro dela um envelope e, por não ter lápis,
pegou um batom e com ele escreveu no envelope, com dedos trêmulos, a
placa do carro que corria à nossa frente, cinco algarismos grandes, numa
letra infantil, de um vermelho vivo. Aquele seu gesto de pânico
aumentava a tensão da aventura e, naquele momento, me senti em pleno
romance policial.

Então, um assobio longo de alarme veio da nossa frente. Um assobio e
também um grito, parece. Paramos bruscamente. Seguiu-se um momento de
silêncio apreensivo em que ninguém se moveu, todos de ouvidos aguçados
na escuridão. Era como um trem parado inesperadamente, de noite, no meio
da planície, em cima de uma ponte. Ninguém sabe o que houve, ninguém se
atreve a adivinhar. Uma batida? Uma catástrofe? Uma ameaça? Só se ouve,
ao longe, o arquejo mecânico da locomotiva\ldots{}

\textls[-15]{Em seguida, surgiram ao nosso redor sombras e vozes. Alguns homens
correram para ver o que acontecera. Ouviam-se, por trás das portas
fechadas dos automóveis, sussurros de mulheres alarmadas. Descemos. Caía
uma garoa tranquila. Ao longe, luzes agitadas se amontoavam. Arabela me
acompanhou, caminhando do meu lado, calada. Passamos por entre os carros
parados em derredor e ouvimos, várias vezes, de dentro deles, longos
suspiros que provavelmente marcavam o culminar de enlaces que o alarme
não interrompera. Sentia, no ar gelado daquela noite de março, um cheiro
insuportável de alcova, perfume ou sangue, não sei ao certo.}

A uns cem passos de distância, formou-se um círculo largo, do qual nos
aproximamos com receio. Ocorrera um acidente. Adivinhamos pelos
sussurros das pessoas reunidas, pela distância respeitosa em relação a
algo que se encontrava no centro e que, só depois de nos aproximarmos
bem, vimos tratar-se de um ferido. Todos olhavam para ele, à distância,
intimidados, sem se atreverem a se aproximar. Era um garoto de dezesseis
anos, que passava por ali de bicicleta e fora atropelado por um
automóvel em alta velocidade, de faróis apagados.

\textls[-25]{Arabela atravessou o grupo e avançou sozinha. Um cacho do seu cabelo
negro lhe cobriu a testa, e nem teve a preocupação de o remover, tão
séria e atenta que estava. Ajoelhou-se junto ao ferido e o fitou
longamente, sem se emocionar. Em seguida, arregaçou as mangas, procurou
um lenço limpo e pediu água. Alguém se apressou por trazer, não sei de
onde, talvez do carburador de um carro.}

Ela ergueu a cabeça do ferido e a apoiou no joelho dela. Viu-se, sob a
luz dos faróis, uma cabeça fraturada e uma mecha de cabelo
ensanguentada. Uma mulher ao meu lado se pôs a gritar, histérica. Estava
usando um vestido elegante e amarrotado e, a julgar pelas faces coradas
e pela respiração entrecortada, o acidente a surpreendera no meio de um
orgasmo. Arabela a fitou longamente com um olhar severo --- um olhar que
eu não teria suportado se tivesse sido dirigido a mim.

\textls[-10]{Não havia nada a fazer. Dei-me conta, desde o início, de que o menino
estirado na terra molhada estava morto. Das minhas velhas lembranças
médicas, ao menos isso eu sabia. Nesse meio-tempo, a polícia chegou.
Peguei Arabela pelo braço e demos no pé. Caminhamos por muito tempo,
calados, eu grato àquela chuva murmurante que caía, que nos ajudava a
não pensar em nada e sobretudo em nós. Acabamos encontrando um táxi
perdido, que nos levou até em casa em meia hora, e durante todo aquele
tempo Arabela permaneceu imóvel, absolutamente ausente.}

\textls[20]{Quando nos vimos de volta ao nosso quarto, ela se atirou à cama, vestida
do jeito que estava, e se pôs a chorar, aos soluços, pueril, um choro
agitado que deveria redimir tudo, pelo perdão daquela noite.}

\section{v}

\letra{D}{esde} então não conversamos mais sobre o assunto. Por alguns dias,
evitamo-nos um ao outro e falamos muito pouco, sobre outras coisas,
procurando afazeres, fugindo do olhar do outro. Em seguida, esquecemos.

Hoje, ao evocar essas lembranças, a violência de Arabela naquela noite,
o seu pavor, o seu desgosto pela perversão e pela morte, que ela manteve
por alguns dias num sorriso extenuado, tudo isso me parece, de longe,
menos pueril e provido de uma certa aura que na época não compreendi,
pois a proximidade nos impede de compreender muitas coisas.

\textls[-10]{No entanto, naquelas suas características provavelmente repousava aquilo
que, mais tarde, os críticos chamaram de ``gênio'' de Arabela.
Escreveram-se tantos artigos sobre ela, sobre sua arte, sobre a emoção
de sua presença no palco --- e todos esses comentários não lograram
atingir o ponto central. Eu mesmo, acompanhando-a ao piano, estremecia
ao chamado de sua voz, embora me concentrasse o máximo possível; fiquei
intrigado com o mistério daquela emoção, como o mistério de um brinquedo
que inventamos, mas que foge ao nosso controle e nos supera.}

Quem diria, naquele dia em que recebemos a proposta de realizar um
número de canções no cabaré-teatro do interior, quem diria aonde nos
levaria aquela aventura insignificante? Para mim, a proposta era
absurda.

--- Cento e sessenta francos por noite --- disse Arabela ---, não é um
negócio que podemos encontrar a toda hora.

--- Não, claro. Mas sendo acrobata ou no máximo dançarina, não vejo,
moça bonita, como é que você vai cantar, mesmo que seja por cem vezes
cento e sessenta francos.

Ela deu risada, admirada com a minha lógica. A dela era mais simples.
Havíamos sido convidados para cantar num cabaré do interior. Cantar, não
dançar. Então vamos cantar.

\textls[-15]{--- Meu querido, a vida inteira fiz o que os outros me pediram. O que eu
quero, o que eu sei, o que eu posso: nunca ninguém me perguntou. E
também nem dei explicações. Você está dizendo que essa oferta que
recebemos é confusa. Pode ser. Mas, em primeiro lugar, é um contrato, e
um contrato não se recusa: se assina.}

Não era a primeira vez que eu me rendia diante de Arabela e da aventura.
Em poucos dias, nosso programa estava pronto. Realizamos ensaios febris
--- eu ao piano, Arabela zanzando pela casa com afazeres --- com algumas
canções que estavam na moda, duas canções de amor clássicas e, para
agradar ao público, uma antiga balada auvernesa, da região em que
haveríamos de nos apresentar, balada retirada das confidências musicais
do Seu Pierre, dono da loja de frascos, excelente barítono e
clarinetista da fanfarra distrital. A voz de Arabela me parecia
aprazível, e nada mais, de modo algum esperava o sucesso desmedido,
esmagador, que haveria de nos reter um mês inteiro no mesmo teatro e,
depois, alguns meses pelas grandes cidades do Sul, portos opulentos ou
balneários anacrônicos de temporada.

Que noites festivas, com um entusiasmo talvez demasiado cordial por
parte do público, que nos encorajava aos gritos, como numa partida, mas
também com momentos de um silêncio comovido, quando toda a sala
permanecia com a respiração suspensa até o último acorde da balada da
Auvérnia, que Arabela cantava devagar, como se desfizesse à noite, entre
os dedos, um novelo de seda, atenta ao seu fio cintilante.

Pude então sentir o tom pessoal de tristeza que hesitava na canção de
Arabela e que se esforçava por se livrar de sua inabilidade, de toda a
sua pose e comportamento de ``artista''. Sem saber muito bem que
delicado filão de emoção eu dirigia, decidi tentar simplificar o nosso
número, amontoado heterogêneo de melodias, e ordená-lo melhor desde o
início.

Quem chegou a ouvir Arabela mais tarde em Paris, naquela noite do
Empire, que ficou famosa nas crônicas do mundo do espetáculo, jamais
saberá por quantas configurações passara o nosso número, até chegar
àquela imagem perfeita que acabou por a popularizar nas revistas
ilustradas. Tudo, o vestido de Arabela, seu penteado reto e liso, as
cortinas pretas que nos emolduravam, minha posição ao piano, a ordem das
melodias no programa, o bracelete de prata que Arabela usava no pulso
esquerdo, seus braços compridos, pendendo preguiçosos ao longo do corpo,
com as mãos entrelaçadas como dois pássaros no joelho, tudo foi
conquistado lentamente, dia após dia, depois de inúmeras correções e, se
me esforçasse por lembrar, talvez conseguisse dizer, hoje, com precisão,
de quando e de onde vem cada detalhe do nosso espetáculo, em Brest ou em
Nîmes, pois, passando de cidade em cidade, simplificávamos a cada nova
experiência o nosso movimento no palco e a cenografia.

\textls[15]{O mais difícil foi encontrar para Arabela um lugar no palco, pois,
embora calma e absolutamente desinibida, era completamente incapaz de
encontrar uma posição natural enquanto cantava. Não sabia o que fazer
com as mãos, não sabia dar dois passos, não sabia em que se apoiar.
Troquei-a várias vezes de lugar, ora em pé do meu lado, como se
acompanhasse a partitura por cima dos meus ombros, ora à direita, na
frente do palco, apoiada numa prateleira baixa, ora à esquerda,
levemente inclinada sobre a cauda do piano.}

\textls[15]{Nenhuma posição funcionava. Foi um autêntico triunfo o dia em que,
durante o ensaio, gritei-lhe, exasperado, que subisse no piano.
Sentou-se comodamente, como numa poltrona e, de repente, reencontrou a
autoconfiança, pois, bem-posicionada sobre o piano, de pernas cruzadas e
com as mãos calmamente unidas sobre o joelho, lembrava sua antiga
posição na tela de seda, suspensa no circo, de modo que recobrou o
sorriso de melancolia e indiferença que a iluminara calorosamente desde
a noite em que nos conhecêramos.}

Não, não sei de onde exatamente vinha aquela vibração íntima do seu
canto, a beleza imaculada e, no entanto, embaçada das melodias. Ainda
hoje escuto, às vezes, na vitrola, alguns discos que gravamos
(felizmente poucos e difíceis de encontrar) e me pergunto que chave me
escapa para poder desvendar aquele mistério, do qual participei e que se
desenrolou debaixo do meu nariz. Tinha uma voz de pequena amplitude, sem
modulação, quase sempre monótona, com leves nuances falsas, mas, desde
as primeiras notas, parecia afastar não sei que cortinas pesadas,
revelando uma imensa janela para o sonho.

Cantava com empenho e ao mesmo tempo com inaptidão, como um iniciante
que tateia uma melodia ao piano, surpreendendo-se ao ouvi-la, e ela
manteve esse ar de indecisão e hesitação o tempo todo, mesmo quando o
sucesso lhe dava o direito de se sentir segura de si. Mas era totalmente
desprovida de elã, incapaz de gestos ou sorrisos ingênuos, cantando
qualquer coisa --- balada ou copla --- com a mesma voz invariável e
esforçada de criança triste.

Creio jamais ter flagrado nela um momento de cabotinagem ou ao menos de
orgulho, pois tinha a consciência de estar fazendo um trabalho nem
melhor nem pior que outro e se, em vez de cantar, devesse coser ou
bordar, tenho certeza de que investiria nisso a mesma boa-fé e
simplicidade. Tratava-se de uma atividade honesta e pura, que nada tinha
a ver com ``arte'', e se a canção de Arabela era de fato comovente, isso
não derivava do fato de ela cantar, nem de sua voz modesta, mas de outro
lugar, do seu cansaço íntimo, de sua melancolia remota, que atirava
sobre a canção e também sobre sua vida, sobre suas coisas e suas
recordações, um véu baço de luz.

\textls[10]{Certo dia, antes do nosso retorno a Paris e da estreia no Empire, ao
recordar umas férias passadas muitos anos atrás, na minha época de
estudante, num vilarejo alpino, onde conhecera e amara por uma única
noite uma moça, que desde então desaparecera por completo, ao recordar
aqueles tempos, pedi a Arabela que cantasse os versos da canção de uma
brincadeira de roda que então ouvira.}

\begin{verse}
\emph{Il court, il court le furet,}\\
\emph{Le furet des bois jolis}\\
\emph{Il a passé par ici}\\
\emph{Il a passé par là bas}\\
\emph{Il repassera par là\ldots{}}\footnote{Em tradução livre, ``Ele corre, ele corre o furão; O belo furão dos bosques; Ele passou por aqui; Ele passou por ali; Ele passará por aqui novamente.''. \textsc{{[}n.\,e.{]}}}
\end{verse}

\textls[10]{Pedi-lhe, sem saber que descobriria, por acaso, a mais bela peça do
nosso programa, a única, na verdade, à qual passaríamos a dever dentro
em pouco a nossa celebridade, melodia que haveria de circular um ano
inteiro pelas metrópoles europeias, ouvida na vitrola, no rádio, em
ritmo de \textit{jazz}, em orquestras vienenses ou assobiada à noite nas ruas
desertas por um transeunte atrasado, canção que inicialmente
surpreendera por sua simplicidade (pois era no mínimo ousado apresentar
num grande teatro, num palco famoso, uma melodia que crianças berravam
na escola, na hora do recreio), mas justamente essa ingenuidade deveria
agradar e criar, baseada no próprio exemplo, toda uma moda. Vivia-se um
retorno repentino da valsa, das canções de amor de 1900, dos quadros de
Toulouse-Lautrec ou Vuillard, dos vestidos longos e dos tricórnios, tudo
filtrado de acordo com o gosto da época, com certo carinho por aqueles
anos anteriores à guerra, felizes e sentimentais.}

Em meio a toda a boa sorte e glória da nossa carreira, só reivindicaria
o fato de ter sentido possível a poesia de um programa de canções fora
de moda e que, enquanto nossos amigos nos incentivavam a montar um
recital de música moderna, e Arabela tendia para as canções novas,
insisti em nos atermos exclusivamente a algumas melodias anacrônicas,
dentre 1900 e 1920, sabendo que uma canção de amor de dez anos atrás,
ressuscitada, seria superior a uma recente, no que toca à nostalgia e ao
leve ridículo que recobre todas as nossas emoções miúdas do passado.

\textls[-15]{Desenterrei então, do esquecimento das gerações mais sofisticadas, todo
um repertório de baladas de amor, que no passado teriam inspirado
prantos ou danças, e as reatualizei, sem ironia pelo seu mau gosto, mas
interpretando-as com a sinceridade inicial de seus dias de glória. Colhi
canções de amor fora de moda de vários lugares, Inglaterra, Alemanha,
França, anteriores à guerra ou imediatamente posteriores, e em seguida
fiz uma seleção, e nisso o gosto de Arabela foi decisivo, pois ela ao
menos não julgava por critérios estéticos, mas com base em seu
entendimento de boa moça que, ao ouvir uma ``canção bonita'', anota-a
num pedaço de papel e, em seguida, à noite, se está triste e tem vontade
de chorar, a canta.}

\textls[-15]{Talvez ainda hoje se cante \emph{Adelaide's Dream} ou \emph{When the Red
Bill}, que descobrimos num amontoado de partituras no sótão de um sebo,
uma de 1890, a outra de 1910, que em seguida lançamos num bar parisiense
de clientela anglo-saxã, onde Arabela cantava, com sincera compaixão, o
sonho daquela Adelaide de fim de século. Mais tarde, quando estivemos
pela primeira vez em Londres, precedidos pelos ecos de um sucesso que
começava a invadir toda a Europa, mais graças à estranha fórmula de
nossos concertos do que à sua qualidade, então, mais tarde, tivemos que
montar um programa inglês (oh, cômicas e ternas noites de cabaré
londrino, em que o público da sala balançava à sua interpretação de
\emph{Venetian Moon}, canção de amor pueril, esquecida já em
1920\ldots), o que nos obrigou, em seguida, já em turnê, a preparar,
para cada capital, um programa específico, realizando verdadeiras
incursões de folclore na área do tango e das valsas do passado.
Tornei-me profundamente sentimental na época em que Arabela ensaiou, às
vésperas de nossa estreia vienense, \emph{Wien, du Stadt meiner Träume},
que ela cantava com um terrível sotaque estrangeiro e infindável
nostalgia, embora só compreendesse metade das palavras\ldots{}}

Encontrávamo-nos realmente nos cumes da glória e as revistas ilustradas
de teatro começaram a publicar, na coluna de curiosidades, a cifra de
nossos ganhos e cartas de amor anônimas, endereçadas a Arabela a cada
manhã.

\emph{Arabela and partner}! Cartazes azuis, brancos, vermelhos e
verdes se alastravam pelo continente e pelo mapa da Europa como
bandeirinhas multicoloridas indicando o itinerário da vitória --- e
agitando nossos nomes em paragens remotas, nos painéis dos teatros, nas
janelas dos bondes, nos maços de cigarro de luxo.

\textls[15]{Gostava muito de ver a nossa imagem na primeira página dos programas,
Arabela em primeiro plano, desenhada em azul vivo --- cor que lhe caía
bem ---, e eu, em perspectiva, escondido ao piano, caracterizado por
traços negros que sombreavam meu rosto e me mantinham anônimo.}

\emph{Arabela and partner} me parecia um título de espetáculo de
cabaré que continha a dose exata de um necessário mistério e, ademais,
sem me esconder de ninguém, pois nem pensava em voltar a ser o que tinha
sido, contentava-me em me resguardar, dessa maneira, da curiosidade de
antigos conhecidos.

Nos cartazes eu não passava de um simples ``parceiro'', o que afastava
de mim as luzes e as gazetas. No meu íntimo, sem ousar assumi-lo, estava
contente sobretudo com o fato de que ninguém em Bucareste ficaria
sabendo de nada.

\textls[15]{Várias vezes quase recebemos convites para a Romênia e, enquanto nos
apresentávamos em Viena ou Budapeste, telegramas de Bucareste contendo
propostas nos assediavam em todos os hotéis, o que tornava a recusa mais
difícil, pois eu não podia dizer a Arabela que não podíamos aparecer lá
por causa de uma mulher que eu não queria que me visse. Era uma
criancice --- e eu sabia muito bem disso.}

Ao realizarmos nosso filme na Paramount, foi um verdadeiro calvário a
minha luta com o diretor, que fazia questão de colocar Arabela e eu
debaixo da mesma luz branca. Tive que lançar mão de toda minha
habilidade e insistência para o convencer de que, para a pureza do
canto, para a simplicidade da imagem, eu tinha que permanecer na
penumbra, mera silhueta escura, cujo movimento das mãos sobre o teclado devesse ser vez ou outra sublinhado por um facho de luz, perdendo-se em seguida
entre as cortinas. Na tela permaneceria apenas um círculo branco
contendo Arabela, demasiado indiferente ao que acontecia em derredor
para se incomodar com o exército de refletores nela focalizados.

\textls[10]{A verdade é que me assustava o fato de que aquele filme pudesse
inevitavelmente passar em Bucareste, e o espetáculo me parecia indecente
diante de uma sala repleta de gente conhecida, apavorando-me sobretudo
ao imaginar Maria, que --- grande cinéfila --- me fitaria desde a sua
poltrona, perplexa, certamente ouvindo o comentário arrogante de Andrei,
inclinado sobre ela para sussurrar com negligência: ``Eu te disse que
esse Ştefan Valeriu nunca seria alguém''.}

\textls[-10]{As obras-primas devem ser o que são graças a tais truques, pois a
estreia do nosso filme foi recebida por uma avalanche de elogios e
comentários, todos os críticos explicando com competência e termos
técnicos, a mim desconhecidos, o valor da luz e da sombra em nosso
curta-metragem. Talvez tivessem razão, embora não conseguisse levar a
sério todas aquelas verdades estéticas, sabendo que elas ocultavam
apenas uma história de amor medíocre em que, Deus é testemunha, nada
havia sido premeditado.}

\textls[15]{O que não me impedia de ir, às vezes, nas noites livres, a um
cinematógrafo de bairro que apresentava o nosso filme, para assistir a
ele e ouvir Arabela, simples e comovente na tela, assim como era no
palco ou em casa, com aquele seu sorriso embaçado como um beijo.}

\section{vi}

\letra{N}{ão} sei dizer quando exatamente ocorreu o pequeno incidente que segue.
Na época, não lhe dei importância, e nem hoje estou completamente
convencido de que tenha alguma ligação com a nossa separação ulterior.

\textls[15]{Certo dia, eu falava sobre seus antigos colegas de circo. Falava com
bastante indiferença, para não deixar espaço para arrependimentos.
Então, Arabela me disse, de repente, como se lembrasse aquilo só naquele
momento:}

--- Você sabia que o Dikki foi meu marido?

\textls[15]{Claro que não sabia e nem desconfiava. Dikki, aquele indivíduo sem
idade, careca e alcoólatra? A revelação era mais cômica do que
preocupante.}

--- Por que só agora é que você me diz isso?

--- Sei lá\ldots{} calhou agora.

\textls[15]{--- Você é um fenômeno, Arabela. Um fenômeno. Vivemos juntos há tanto
tempo, você me conta as coisas mais inimagináveis, passamos horas
inteiras matraqueando, e só agora você resolve me dizer uma coisa que de
todo modo é mais importante que muitas outras.}

--- Sou assim, Ştefan. Esquecida.

Silenciei por alguns instantes, desarmado pela simplicidade da resposta
--- e em seguida irrompi, com certa violência.

\textls[-15]{--- E por que justamente o Dikki? Dos quatro, por que justo ele?}

--- Porque era mais simples com ele. Entende? É muito difícil viver,
como eu vivia, com quatro homens, como se fosse uma família. Casada com
um, os outros três tinham que permanecer sossegados, afastados. Numa
história dessas, o mais importante é não ter que se haver com amores e
complicações. E com Dikki, pelo menos, não tinha como se tratar de amor,
não é?

\textls[15]{Lembrei-me então daquele seu olhar dirigido a Beb, que flagrei aquela
noite no camarim do Medrano, e logo me perguntei se a partida de Arabela
não teria sido mais séria do que eu imaginara, e se não teria deixado
para trás lembranças mais profundas do que as de um pacto entre
cabotinos. Aquele rapaz, Beb, talvez tivesse algo a dizer.}

Depois, não muito depois, num teatro-cabaré no norte da Alemanha,
durante uma de nossas turnês, reencontrei vestígios de seus antigos
colegas, e devo reconhecer que quem se emocionou mais fui eu, e não ela.
Eles haviam nos precedido em duas semanas no programa, e agora já
estavam em outro lugar, desconhecido. Encontrei seus nomes e retratos na
coleção de jornais locais, na coluna teatral, e soube que a apresentação
se coroara de certo sucesso. Com certeza haviam progredido desde a
última vez que nos víramos e, embora não estivessem entre os artistas
mais famosos, realizavam um número honesto, na primeira parte do
espetáculo, como prelúdio ao número principal. Olhei com muita
curiosidade as fotografias, e percebi que haviam simplificado bastante
os aparelhos, os movimentos, as cores.

--- Um dia esses rapazes vão chegar longe --- refleti.

\textls[-15]{--- Talvez --- respondeu Arabela, sem me aprovar ou contrariar.}

\textls[15]{Via-se claramente que, de uma maneira ou de outra, para ela dava no
mesmo.}

Continuei, insistente:

--- Falta-lhes só uma coisa. Você. Lá em cima, na tela de seda, fazendo
nada, esforçando-se apenas por manter um sorriso, você era a poesia do
trapézio deles. A flor inútil. Um diretor genial talvez não conseguisse
encontrar um detalhe de tamanho valor.

\textls[15]{Dizia-o para a aborrecer, ou para a desafiar, ou, simplesmente, para
exercitar meu velho instinto de maldade. Mas com certeza eu tinha razão
e, ao ver aquelas fotografias, percebi também que faltava naqueles
trapézios brancos a imagem de uma mulher, assim como uma pedra falta num
anel.}

Arabela me ouviu até o fim, risonha, e em seguida me pegou pelo braço e
me deu um beijo, com o ar de quem diz, repreendendo: ``E agora nos
comportemos e deixemos de besteira''.

\textls[20]{Por que estou me lembrando de todas essas ninharias agora, não sei ao
certo. Tento resgatá-las na memória para passar o tempo, assim como
reconstituo, por exemplo, uma partida de xadrez após terminá-la.}

\textls[-10]{Muito provavelmente elas não tiveram ligação alguma com o que aconteceu
depois, não foram esses pequenos incidentes que nos levaram à separação,
mas alguma outra coisa, mais simples e ao mesmo tempo mais inexplicável.
Outra coisa, que espantosamente se parecia com o início do nosso amor e
que se chamaria ``aventura'', caso essa palavra se adequasse à mente
pueril de Arabela.}

Muitas coisas surpreendentes passaram pelas nossas vidas e as deixamos
passar, amando-nos no último dia como no primeiro, com a mesma volúpia
suave, em que tudo era conhecido, como o gosto eterno do pão. Isso podia
durar um ano, dois, dez\ldots{} Assim como podia acabar a qualquer
momento.

A separação! Foi simples e, pensando nela muito sinceramente, parece-me
mais importante falar do casaco verde que Arabela usou no nosso primeiro
inverno juntos ou de seu vestido preto de gola amarela (vestido que a
tornava mais alta e encantadoramente pálida) do que sobre a separação.

Estávamos em Genebra, no início de setembro. Fomos para inaugurar a
temporada teatral na sala do cassino, enquanto Aristide Briand, a
duzentos passos de distância, na Liga das Nações, inaugurava a temporada
diplomática. Um outono passional, em que se discutiu, pela primeira vez,
o projeto de Estados Unidos da Europa,\footnote{Em 5 de setembro de 1929,
  em nome do governo francês, Aristide Briand anunciou o projeto de uma
  união europeia. {[}\textsc{n.\,t.}{]}} numa atmosfera festiva
infantil, para a qual, digo sem exagero, a presença de Arabela em muito
contribuiu, pois os ministros estrangeiros invariavelmente se
encontravam nas lojas, às nove da noite, durante o nosso concerto.

\textls[-10]{As manhãs brilhantes à beira do lago, os vestidos brancos esvoaçantes na
avenida Wilson, a invasão dos jornalistas diante do hotel des Bergues, a
corrida dos fotógrafos pela rua, atrás de instantâneos e
celebridades\ldots{} Era idílico e relaxante como uma opereta.}

Certo dia, à beira do lago, alguém nos chamou, um rapaz jovem, que tinha
acabado de descer de um bonde que passava por ali. Estávamos
acompanhando, distraídos, um jogo de polo aquático entre algumas jovens
inglesas. Arabela, com o rosto todo iluminado pelo sol, dava risadas
como uma criança.

Viramo-nos, surpresos, e, num primeiro momento, nem ela reconheceu Beb,
que nos chamara e ficara parado atrás de nós, embaraçado com o forte
entusiasmo do reencontro.

--- Olha só, Beb --- disse Arabela, sem levantar a voz. --- Como você
mudou, Beb: está bonito. Mas continua faltando um botão no seu colete.
Costure-o de noite, está me ouvindo? Costure-o.

Beb realmente mudara. Menos pálido do que quando o conheci, mais alto
talvez, mais esportivo. Estava usando um terno cinzento de verão e,
àquele sol branco de setembro, havia um quê de exageradamente juvenil na
sua surpresa e emoção.

\textls[15]{Explicou-nos, em poucas palavras, que se encontrava em Genebra apenas
por um dia, de passagem. Deveria partir naquela mesma noite rumo a
Montreux, onde Sam e Jef o estavam aguardando: um contrato excelente.}

--- E Dikki? --- perguntou Arabela.

\textls[15]{--- Nós o perdemos de vista quatro anos atrás em Argel e nunca mais o
vimos.}

--- E vocês?

--- Estamos bem. Sucesso, dinheiro. Se você soubesse, Arabela, como as
coisas correram bem para nós! Sempre te disse que a glória nos esperava.
Lembra quando você foi embora.

Falava com animação, rápido, com as mãos no bolso para não gesticular, e
caminhava com meio passo à nossa frente, para poder olhar nos olhos de
Arabela. Estava agitado como um estudante de liceu e gostei tanto de ver
aquilo que não pude evitar lhe perguntar, com simpatia, como a um velho
camarada:

--- Diga a verdade, Beb, você ainda ama Arabela?

\textls[15]{Respondeu na hora, brevemente, com um certo movimento pedante da cabeça,
mas bem-humorado.}

--- Sim.

\textls[15]{--- Vocês dois ficam falando besteira --- censurou Arabela. --- Melhor
irmos comer.}

À noite, Beb foi embora de trem para Montreux e, às nove, nós dois
fomos, como de costume, para o espetáculo. Vesti o fraque muito
tranquilamente. Pressentimentos são uma besteira. Desde o piano, olhava
para Arabela e dizia para mim mesmo, toda noite, que ela não era bonita
e nem sabia cantar, mas acompanhava com a mesma perplexidade e a mesma
tranquilidade profunda a sua voz úmida, que desencadeava em mim tantas
melancolias, parecendo revolver delicadamente lembranças e
esquecimentos.

\textls[20]{Mais tarde, depois do teatro, fomos caminhar à beira do lago. Soprava da
montanha um vento gelado anunciando tempo incerto, forte demais para uma
noite de verão, amistoso demais para uma de outono.}

--- Agora deve estar gostoso lá em cima, no nosso quarto --- disse
Arabela, apoiada no parapeito, de frente para o lago, apertando o meu
braço com força.

\textls[5]{Subimos devagar a escada do hotel, atrasando de propósito os passos,
pois sabíamos que noite agradável nos aguardava e, de fato, nos amamos,
sem pressa, cuidadosos, confiantes no momento do enlace e ouvindo
crescer ao nosso redor, na sombra, grandes círculos de silêncio. Creio
que, se até então entre os nossos corpos houvesse qualquer grão de
desentendimento, aquela noite consumiu tudo. No escuro, o sorriso de
Arabela era caloroso como um bichinho sonolento.}

\textls[-15]{Por isso, talvez, de manhã, não me assustei quando, esperando-a no
saguão e fitando-a descer a escada na minha direção, Arabela me fez de
longe um sinal para eu me aproximar e me perguntou, com naturalidade,
como se me perguntasse as horas:}

--- O que você diria, Ştefan, se eu fosse atrás do Beb?

\textls[15]{--- Sei lá, querida. Acho que seria complicado com o teatro daqui. Temos
um contrato.}

--- Eu daria um jeito.

--- Certo. Então vamos tentar.

À tarde, acompanhei-a até a estação de trem. Estava levando uma maleta
de mão, só uma. O resto iria depois.

\textls[-10]{Conversamos algumas ninharias até a chegada do trem. Apertamos as mãos,
sem nada de heroico no gesto, em perfeita sintonia.}

\textls[15]{--- Se fizer frio, Ştefan, vista o casaco à noite. Ainda mais na área do
lago, tem friagem.}

Entardecia, eram cinco horas. Fui a pé até o centro da cidade e, no
caminho, comprei os jornais para ver o que tinha acontecido de manhã na
Liga das Nações. Houvera debates ardorosos.
