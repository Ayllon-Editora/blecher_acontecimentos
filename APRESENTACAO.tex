\pagestyle{posf}  

\chapter*{Apresentação\smallskip\subtitulo{O mundo por trás do\break véu do visível}}
\addcontentsline{toc}{chapter}{Apresentação, \textit{por Mirella Botaro}}

\begin{flushright}
\textsc{mirella botaro}
\end{flushright}

\noindent{}\textls[15]{Mihail Sebastian foi um dos mais importantes romancistas e dramaturgos romenos do século \textsc{xx}. 
Intelectual afinado com as vanguardas europeias, com a música clássica
e, em particular, com a literatura francesa, despontou precocemente --- junto com outros autores como Emil Cioran,
Mircea Eliade e o franco"-romeno Eugène Ionesco.}
%\section{o contexto romeno}

Durante o auge de sua carreira, entre 1920 a 1930, a Romênia passou por um período de grande agitação política e cultural. Ao mesmo tempo que dialogava com outras potências europeias no campo da literatura, sobretudo a França, a juventude intelectual romena também apoiava a ascensão do fascismo --- representado, à época, pela Guarda de Ferro (em romeno, \textit{Garda de Fier}), organização que levou ao poder o ditador Ion Antonescu em 1940 e fomentou o \textit{pogrom} de Bucareste no ano seguinte.

A despeito de sua notoriedade no meio artístico local, o jovem escritor judeu 
se viu pouco a pouco excluído do cânone literário,
perseguido e ostracizado por um antissemitismo latente. Esse episódio foi detalhadamente relatado em seu
\emph{Diário}, escrito entre 1935 e 1944 mas publicado na Romênia somente em 1996, após o fim da censura
imposta pelo regime comunista entre 1945 e 1989. Tendo sobrevivido a
``tristes anos de humilhação e fracasso'' sob o regime romeno pró"-Hitler
e à Segunda Guerra Mundial, conforme descreve em seu diário, Sebastian 
morre tragicamente em 1945, aos 38 anos, atropelado por um caminhão
militar soviético. Professor recém"-contratado à época, ele se dirigia à
Universidade de Bucareste, onde daria sua primeira aula sobre Balzac.

Embora integrem sua obra globalmente, os
acontecimentos históricos que permearam a trajetória do autor não são
mencionados em \emph{Mulheres}, seu romance de estreia\footnote{Mesmo sendo o primeiro romance do autor, é sua segunda publicação. A primeira foi uma novela, \textit{Fragmentos de um diário encontrado} (1932). No Brasil, foi publicada pela Ayllon em 2020, vertida do romeno para o português por Fernando Klabin, também tradutor do volume que o leitor tem em mãos.} publicado
em 1933 pela Editura Națională S.\,Ciornei, de Bucareste. Tampouco é possível
obter um retrato fiel da realidade na leitura do livro --- da
geografia ou da sociologia de uma Romênia profunda, tradicional e
culturalmente marcada. Sebastian tem o cuidado de contornar espaços,
temáticas ou categorias que remetam a uma origem ou uma identidade, o
que não nos impede de identificar sutilezas narrativas sobre a época e a sociedade no olhar do escritor, que se
mantém fortemente arraigado em uma cultura francófona.

\section{galeria de retratos femininos}

\textls[5]{Não por acaso, a trama se desenvolve menos na Romênia do que na França,
mais precisamente entre os Alpes e Paris: a \textit{cidade"-protótipo} de uma certa
estética dos anos 1930 em que predominam, como herança da Belle Époque,
uma efervescência intelectual e artística vivida em salas de teatros,
cinemas e cabarés. Tendo morado na França entre 1929 e 1931, é provável
que Sebastian tenha circulado nesse ambiente artístico e boêmio que
será percorrido pelo personagem principal de \textit{Mulheres}, Ştefan Valeriu. É lá, por exemplo,
que ele conhece Arabela, intrigante dançarina de circo com quem passa a
viver junto --- mas como artista, deixando para trás sua carreira de médico. Ao final da
relação, de volta à sua realidade em Bucareste, Ştefan exprime a
falta que sente de Arabela e, com ela, da vida de liberdade representada por
aquela Paris do entre guerras: }

\begin{quote}
\emph{Que tu sens bon}, dizia"-lhe
com sinceridade, quando queria dizer o quanto a amava e teria
dificuldade em traduzir isso para o romeno.\footnote{Ver página \pageref{traduzir}.}
\end{quote}

Cada capítulo do romance leva o nome de uma ou mais mulheres que passaram
pela vida do protagonista. Ele retoma seus amores do passado,
descreve e analisa meticulosamente a natureza de cada uma de suas
amantes, além de seus próprios sentimentos.
Assim, por trás da galeria de retratos femininos que se constitui ao
longo do romance, é o próprio homem, Ştefan Valeriu, que se revela como
objeto de análise ao leitor. 

\textls[15]{O ponto de vista é sempre o de Ştefan. E isso permite ao leitor o acesso direto ao modo de
apreender e interpretar a realidade que o cerca, além de seus desejos
e contradições íntimas. É a sua voz que ouvimos ecoar quando
o narrador afirma, sobre uma de suas amantes: }

\begin{quote}
\textls[15]{após seu primeiro
momento de amor, lá em cima, no quarto conjugal, numa manhã imprevista,
Renée Rey se refugiara sem explicações numa pose perfeita de esposa
virtuosa. }
\end{quote}

\textls[10]{Comentário mordaz que nos dá a medida da distância entre
Ştefan e a moral burguesa da época e nos remete, em certa medida, a uma
certa \textit{ironia machadiana} familiar aos destinatários brasileiros do
romance.}

\section{sebastian, machado, proust}

\textls[10]{O diálogo franco e direto que Mihail Sebastian estabelece com um
hipotético leitor também o aproxima de Machado de Assis, escritor que em
toda a sua produção tampouco deixa de interpelar seu leitor, incluindo"-o
no jogo narrativo como se fosse um personagem. Assim o faz
Ştefan, o herói de \emph{Mulheres}, ao descrever e comentar, não sem
sarcasmo, o encontro insólito entre Irimia, um compatriota radicado em
Paris, e Emilie Vignon, mulher que ``mesmo feia, {[}\ldots{]} tinha por
vezes um ar de resignação que {[}o{]} atraía'':\footnote{Ver página \pageref{resignacao}.}}

\begin{quote}
Pediria perdão ao leitor por estes detalhes desavergonhados, mas, para
ser sincero, pouco me importa o leitor, e muito me importa Emilie
Vignon. Conto a vida dela em primeiro lugar porque eu quero chegar a
compreender alguma coisa da alma dessa moça, a quem no passado eu talvez
não tenha dado a devida atenção.\footnote{Ver página \pageref{atenção}.}
\end{quote}

\textls[5]{É possível entender esse diálogo com o leitor como uma estratégia do
escritor de interpretar sua própria obra, uma vez
que ele assume seu desejo pleno de dissecar a alma humana a partir
da análise de gestos, palavras e ações. A propósito, são
justamente as descrições pausadas e acuradas de situações corriqueiras
que, realizadas com uma lucidez implacável, tornam as personagens
(sobretudo as mulheres) tão interessantes, vivas e cheias de
personalidade. É o caso de Maria que, ao assumir a voz narrativa em uma
carta endereçada a Ştefan, depreende uma análise fina a respeito das
relações humanas a partir da observação de uma simples refeição de seu
amante Andrei:}

\begin{quote}
\textls[10]{Sempre gostei de olhar para ele enquanto comia, e creio que a avidez
seja a única coisa profundamente boa nele (talvez seja uma besteira o
que eu diga agora, mas acredito nisso e vou dizer do mesmo jeito), pois
um homem ávido tem um quê de criança, alguma coisa que diminui a sua
aspereza, a sua importância, o seu terror de macho. Se mulheres simples
e burras lograram viver a vida toda ao lado de homens grandiosos, reis,
generais, cientistas, talvez seja justamente porque comiam junto com
eles à mesa, tendo assim acesso àquela imagem de crianças bicudas e
esfomeadas, a única coisa que as protegia de suas majestades.\footnote{Ver página \pageref{majestades}.}}
\end{quote}

O mesmo ocorre quando Ştefan se recorda de Irimia em seus tempos vividos
no requintado liceu Lazăr, em Bucareste. Ao invocar uma certa Romênia
rural, arraigada em tradições ancestrais, Irimia apresenta em sua
rusticidade uma nobreza discreta e embaraçada, ``como se pedisse perdão
por um erro permanente''.\footnote{Ver página \pageref{erro}.} Ao observá"-lo junto ao tio, velho
camponês de Bărăgan em visita a Bucareste, Ştefan aprecia a delicadeza
de seus gestos rudimentares que, não obedecendo aos códigos da
civilização, estariam na contramão de qualquer performance social:

\begin{quote}
Eu,
que vivi num mundo de tradições falsas e leis falsas, tive a sensação de
uma espécie de eternidade que o meu colega Irimia \textsc{c}. Irimia
materializava ali, na rua, na minha frente, beijando a mão do parente
velho.\footnote{Ver página \pageref{velho}.}
\end{quote}

\textls[10]{\emph{Mulheres} é um romance carregado de sensibilidade. A atenção particular que Sebastian
dispensa às minúcias da vida social revelam, de fato, uma vida interior
agitada, mas que se esconde por trás das aparências e em
ângulos mortos do cotidiano. Assim, atos triviais como comer ou
cumprimentar alguém parecem atrelados a uma verdade profunda do sujeito,
que o romancista é capaz de captar e realçar, ao menos momentaneamente.
Como se, ao se ater às camadas externas de nossa socialização, outras
camadas de interioridade se revelassem, transpassando o \textit{véu do visível},
do prosaico, da vida cotidiana.}

\textls[5]{Não seria exagero afirmar que a intimidade dissecada
o aproxima de uma certa \textit{estética proustiana} em que predomina,
precisamente, a busca incessante pelo íntimo do sujeito: suas fantasias,
desejos, afetos. É conhecida e notória a admiração de
Sebastian por Marcel Proust, que publica o último tomo de \emph{Em
busca do tempo perdido} em 1927, quando o escritor romeno tinha apenas
20 anos. Seu profundo conhecimento da obra do autor francês o leva a mergulhar
na intimidade do próprio Proust, uma vez que Sebastian publica, em
1938, o primeiro ensaio a respeito de suas correspondências. ``Há na
presença de Proust junto aos outros uma intensa sede de intimidade, que
se torna enfim irrealizável, mas que pelo menos é satisfeita por
pequenos acordos momentâneos, passageiros, anestésicos'', afirma o
romancista romeno no referido ensaio.}

\section{alusões, sutilezas e não-ditos}

\emph{Mulheres} é um convite à interpretação das contradições do
sentimento amoroso e, sobretudo, dos discursos que são elaborados a seu
respeito. Mas o alcance poético do livro emana menos das histórias em
si do que do modo como são narradas. Longe de ser propriamente frio,
o dispositivo narrativo criado por Sebastian traduz um certo
desencantamento dos personagens. Sobretudo de Ştefan, conquistador
impenetrável, hermético aos sentimentos e ao \textit{transbordamento amoroso}, mas
que carrega uma dor íntima difícil de definir. Essa dor é sugerida por
alusões, sutilezas e não"-ditos que se sobrepõem à ``preguiça''
constantemente assumida tanto para si quanto para os outros: 

\begin{quote}
\textls[10]{jamais
encontrei uma mulher --- e algumas delas até mesmo amei --- jamais, que me
desse aquela sensação de volúpia calma que eu encontrava nos braços de
Arabela, sorvendo-lhe o cheiro de carne jovem, distendida na preguiça e
indiferença.\footnote{Ver página \pageref{indiferença}.}}
\end{quote}

Ao contrariar o ideal romântico e burguês do amor como uma experiência
que \textit{transborda e transcende}, Sebastian explora a falta, o vazio, as
contradições, tensões e desencontros que caracterizam igualmente o
sentimento amoroso e, mais amplamente, as relações humanas. Que sentido
poderíamos dar à ruptura de Ştefan e Arabela, que de tão repentina é
vivida como um ato absolutamente trivial, após anos de vida comum?
Trata-se de uma incompreensão comum, universal de certa forma, a medida
que se situa fora de um tempo, pertencendo, portanto, a todo o tempo, o
que dá justamente a dimensão atemporal da escrita do autor romeno. Ao
perscrutar a intimidade de seus personagens, seus jogos de sedução e
segredos de alcova, Sebastian propõe um exercício similar a seu
leitor, que se vê confrontado com seus desejos mais incômodos e
inconfessáveis.

\textls[-10]{Romance escrito em uma língua delicada e sugestiva,
de profunda maturidade, capaz de revolver nossas convicções e
experiências mais íntimas, \textit{Mulheres} desloca pontos de vista a partir de um
lugar ainda deveras longínquo e desconhecido do público brasileiro. Sua publicação no Brasil vem, 
portanto, em boa hora: é uma excelente oportunidade para estreitar os laços literários, 
hoje ainda relativamente parcos, entre o Brasil e a Romênia.}
